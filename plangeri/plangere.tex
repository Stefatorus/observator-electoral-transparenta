\documentclass[a4paper,12pt]{article}
\usepackage[romanian]{babel}
\usepackage[utf8]{inputenc}
\usepackage[T1]{fontenc}
\usepackage{geometry}
\usepackage{enumitem}
\usepackage{titlesec}
\usepackage{parskip}
\usepackage{microtype}
\usepackage[none]{hyphenat}
\usepackage{ragged2e}
\usepackage{url}
\usepackage{xurl}
\usepackage{hyperref}
\usepackage{fancyhdr}
\usepackage{listings}
\usepackage{tocloft}
\usepackage{lastpage}  % <-- Added lastpage package

% Set margins and spacing
\geometry{
    a4paper,
    margin=2.5cm,
    includehead
}

% Header and Footer
\pagestyle{fancy}
\fancyhf{}
\lhead{Plângere Contravențională}
\rhead{pagina \thepage/\pageref{LastPage}}  % <-- Updated header
\renewcommand{\headrulewidth}{0.4pt}

% Table of Contents formatting
\renewcommand{\contentsname}{Cuprins}
\setcounter{tocdepth}{2}  % Include up to subsections in TOC
\renewcommand{\cftsecfont}{\normalsize\bfseries}
\renewcommand{\cftsubsecfont}{\normalsize}
\renewcommand{\cftsecpagefont}{\normalsize}
\renewcommand{\cftsubsecpagefont}{\normalsize}
\setlength{\cftbeforesecskip}{5pt}
\setlength{\cftbeforesubsecskip}{2pt}

% Section formatting
\titleformat{\section}
  {\normalfont\Large\bfseries}{\thesection.}{0.5em}{}
\titleformat{\subsection}
  {\normalfont\large\bfseries}{\thesubsection.}{0.5em}{}

\begin{document}

\begin{center}
    \Large\textbf{PLÂNGERE CONTRAVENȚIONALĂ}
\end{center}

\vspace{1cm}

Subsemnatul Deleanu Ștefan-Lucian, domiciliat în Jud. Cluj, Cluj-Napoca, Str. Aurel Vlaicu, nr. 2, bloc 5A, Sc. I, etaj 7, ap. 28, în calitate de observator electoral acreditat de Funky Citizens, asociatie acreditata de Autoritatea Electorala Permanenta prin ACREDITAREA nr. 29743/07.10.2024, în temeiul prevederilor Legii 370/2004 privind alegerea Președintelui României, cu modificările și completările ulterioare, formulez următoarea:

\vspace{0.5cm}
\begin{center}
\textbf{\Large PLÂNGERE CONTRAVENȚIONALĂ}
\end{center}
\vspace{0.5cm}

prin care sesizez următoarele contraventii definite de art. 55 lit t), art 56, pg 1, pg 2 lit a, din Legea 370/2004 privind alegerea Președintelui României.

In analiza caracterului de propaganda electorala, s-a avut in vedere definitia prevazuta de Art 36, pct 7, din legea 334/2006, republicata, cu modificarile si completarile ulterioare.

Metodologia de analiza, mijloacele de mitigare a bias-urilor sunt detaliate aici:
https://github.com/stefatorus/observator-electoral-transparenta/

\newpage
\tableofcontents
\newpage

\section{Împotriva numiților}

\subsection{"Constanta MEA"}
Următoarele fapte contravenționale sunt sesizate împotriva acestei entități:

\begin{enumerate}[leftmargin=*, label=\arabic*.)]
    \item publicarea si promovarea unei reclame pe Facebook (ID: \href{https://www.facebook.com/ads/library/?id=3863248660623792}{3863248660623792}) dupa incheierea perioadei de campanie electorala, in data de 23.11.2024. Postarea constituie propaganda electorala prin faptul ca mentioneaza explicit candidatii Nicolae-Ionel Ciuca si Mircea Geoana, foloseste hashtag-uri cu caracter electoral ("\#PNLsub10laSuta" si "\#CIUCAesubVAL"), si urmareste in mod direct influentarea comportamentului electoral prin sugerarea ca sustinatorii PNL isi abandoneaza propriul candidat in favoarea lui Mircea Geoana. Efectul electoral urmarit este clar: descurajarea votantilor de a sustine candidatul PNL si incurajarea votului pentru candidatul independent Mircea Geoana.
\end{enumerate}

\vspace{0.5cm}

\subsection{"Opinii Independente"}
Următoarele fapte contravenționale sunt sesizate împotriva acestei entități:

\begin{enumerate}[leftmargin=*, label=\arabic*.)]
    \item promovarea unui material cu caracter de propaganda electorala dupa incheierea campaniei electorale, constand intr-un articol sponsorizat pe Facebook si Instagram (ID: \href{https://www.facebook.com/ads/library/?id=491454140618987}{491454140618987}) care promoveaza candidatul la presedintie Mircea Geoana si alianta sa politica. Materialul, publicat si promovat dupa ora 18:00 pe 23.11.2024, prezinta explicit o "alianta strategica" in contextul alegerilor, depasind caracterul informativ si avand un clar obiectiv electoral de influentare a votantilor prin prezentarea pozitiva a candidatului si a aliantelor sale politice. Efectul electoral urmarit este evident prin natura continutului si modalitatea de promovare platita catre un public larg, estimat intre 100.001 si 500.000 de persoane.
    \item promovarea unui material de propaganda electorala dupa incheierea campaniei electorale, constand intr-un articol sponsorizat pe Facebook si Instagram (ID: \href{https://www.facebook.com/ads/library/?id=824813186321358}{824813186321358}) care promoveaza candidatul la presedintie Mircea Geoana si alianta sa politica cu Cristian Lungu. Materialul, publicat si promovat dupa ora 18:00 pe 23.11.2024, are un caracter explicit electoral, vizand influentarea procesului electoral prin prezentarea unei aliante politice strategice si promovarea viziunii comune a candidatului, targetand un public intre 100.001 si 500.000 de persoane, depasind astfel sfera activitatii jurnalistice si incalcand perioada de liniste electorala.
\end{enumerate}

\vspace{0.5cm}

\subsection{"Salvam Romania"}
Următoarele fapte contravenționale sunt sesizate împotriva acestei entități:

\begin{enumerate}[leftmargin=*, label=\arabic*.)]
    \item difuzarea dupa data de 23.11.2024, ora 18:00, a unei reclame platite pe Facebook (ID: \href{https://www.facebook.com/ads/library/?id=933759081968191}{933759081968191}) care constituie propaganda electorala negativa la adresa candidatei Elena Lasconi. Postarea, care a avut un impact semnificativ (peste 150.000 afisari) si pentru care s-au cheltuit peste 4.500 RON, prezinta informatii despre sotul candidatei intr-un mod care urmareste sa ii ecteze imaginea si, implicit, rezultatul electoral. Continutul, desi bazat pe un articol de presa, este promovat strategic ca material de propaganda, depasind limitele activitatii jurnalistice de informare, cu scopul clar de a influenta negativ optiunea de vot a alegatorilor.
\end{enumerate}

\vspace{0.5cm}

\subsection{4Media.INFO}
Următoarele fapte contravenționale sunt sesizate împotriva acestei entități:

\begin{enumerate}[leftmargin=*, label=\arabic*.)]
    \item difuzarea unui mesaj de propaganda electorala dupa incheierea perioadei de campanie, concretizat intr-o postare platita pe Facebook (ID: \href{https://www.facebook.com/ads/library/?id=1106793317616811}{1106793317616811}) dupa ora 18:00 pe 23.11.2024. Postarea vizeaza in mod direct candidatul prezidential Marcel Ciolacu, folosind un limbaj tendentios ("pupe mana stapanului") si un indemn explicit de a actiona impotriva sa ("daca nu ii oprim"), avand un evident scop electoral de influentare negativa a votantilor. Mesajul depaseste limitele activitatii jurnalistice obiective, fiind distribuit ca reclama platita catre un public tinta de peste 1 milion de persoane, cu intentia clara de a influenta comportamentul electoral.
\end{enumerate}

\vspace{0.5cm}

\subsection{4media.INFO}
Următoarele fapte contravenționale sunt sesizate împotriva acestei entități:

\begin{enumerate}[leftmargin=*, label=\arabic*.)]
    \item promovarea unui material de propaganda electorala cu candidatul GEORGE-NICOLAE SIMION dupa incheierea perioadei de campanie. Materialul, cu ID-ul \href{https://www.facebook.com/ads/library/?id=1326414468524628}{1326414468524628} pe facebook, promovat dupa ora 18:00 pe 23.11.2024, prezinta continut electoral explicit referitor la candidatul la presedintie George Simion, sub forma unei reclame platite care a ajuns la 30.000-35.000 de persoane. Mesajul "George Simion explica cum se cucereste un popor" reprezinta un element de propaganda electorala care urmareste influentarea alegatorilor in perioada de interdictie, avand un evident obiectiv electoral prin asocierea candidatului cu un mesaj de leadership si putere politica.
    \item publicarea si promovarea unei reclame platite pe Facebook si Instagram (ID: \href{https://www.facebook.com/ads/library/?id=3519344485024756}{3519344485024756}) dupa ora 18:00 pe 23.11.2024, care vizeaza in mod direct un candidat la prezidentiale (Elena-Valerica Lasconi). Postarea, care contine textul "Sa ne amuzam cu Lasconi", reprezinta o forma de propaganda electorala negativa, avand ca scop influentarea opiniei publice prin ridiculizarea candidatului in perioada de restrictie electorala. Impactul postarii este semnificativ, atingand intre 35.000 si 40.000 de afisari, cu un buget substantial alocat pentru promovare, demonstrand intentia clara de a influenta un numar mare de alegatori prin continuarea propagandei electorale dupa incheierea perioadei legale.
    \item publicarea si promovarea unui mesaj electoral negativ impotriva candidatei Elena-Valerica Lasconi dupa incheierea perioadei de campanie electorala, respectiv dupa ora 18:00 pe 23.11.2024. Postarea cu ID-ul \href{https://www.facebook.com/ads/library/?id=452580617857882}{452580617857882} contine mesajul "CTP - Lasconi foloseste proceduri de sulfa politca veche", reprezentand propaganda electorala negativa, fiind promovata ca reclama platita pe Facebook si Instagram, cu un impact estimat intre 9.000 si 9.999 de afisari, targetand in mod direct un candidat la alegerile prezidentiale si avand ca scop influentarea opiniei publice in perioada de restrictie electorala.
    \item difuzarea unui material de propaganda electorala dupa incheierea perioadei legale de campanie, respectiv dupa ora 18:00 pe 23.11.2024. Materialul, cu ID-ul \href{https://www.facebook.com/ads/library/?id=902982898467615}{902982898467615}, il prezinta pe candidatul prezidential George Simion intr-un material video intitulat "George Simion explica cum se cucereste un popor", avand un caracter vadit electoral si potential de influentare a alegatorilor, fiind promovat activ prin reclama platita pe Facebook si Instagram, cu un impact semnificativ de peste 80.000 de afisari. Materialul continua sa fie distribuit activ in perioada de restrictie electorala, incalcand astfel prevederile legale privind interdictia propagandei electorale dupa incheierea campaniei.
    \item difuzarea unui mesaj de propaganda electorala dupa incheierea perioadei de campanie, la data de 23.11.2024, prin intermediul unei reclame platite pe Facebook si Instagram (ID: \href{https://www.facebook.com/ads/library/?id=9386410671387627}{9386410671387627}). Postarea contine un atac direct la adresa candidatei Elena Lasconi, prin afirmatia "Elena Lasconi ataca mereu fara probe", reprezentand propaganda electorala negativa menita sa influenteze opinia publica si intentiile de vot in perioada de interdictie. Mesajul a fost distribuit contra cost, cu un impact semnificativ (9000-9999 impresii), demonstrand intentia clara de a influenta procesul electoral prin denigrarea unui candidat in perioada de restrictie.
\end{enumerate}

\vspace{0.5cm}

\subsection{5news.RO}
Următoarele fapte contravenționale sunt sesizate împotriva acestei entități:

\begin{enumerate}[leftmargin=*, label=\arabic*.)]
    \item publicarea si promovarea unui mesaj de propaganda electorala negativa dupa incheierea perioadei de campanie, respectiv in data de 23.11.2024. Postarea cu ID-ul \href{https://www.facebook.com/ads/library/?id=1249273286195639}{1249273286195639} contine mesajul "LASCONI II FACE TRADATORI PE ROMANI", reprezentand propaganda electorala negativa la adresa candidatei ELENA-VALERICA LASCONI, fiind distribuita ca reclama platita pe Facebook si Instagram, cu un buget intre 200-299 RON si un impact intre 20.000-24.999 de afisari. Mesajul are caracter electoral evident, fiind conceput pentru a influenta negativ perceptia alegatorilor fata de candidata mentionata, depasind limitele activitatii jurnalistice prin lipsa oricarui context sau fundament factual.
    \item promovarea unui mesaj negativ platit pe platformele Facebook si Instagram despre candidata Elena-Valerica Lasconi ("Lasconi da dovada de aroganta"), dupa incheierea campaniei electorale, respectiv dupa ora 18:00 pe 23.11.2024. Mesajul are caracter electoral evident, fiind conceput sa influenteze negativ opinia publica despre candidata, atingand un public intre 6000-6999 de persoane, fiind astfel propaganda electorala dupa incheierea campaniei electorale. Postarea poate fi identificata pe Facebook cu ID-ul \href{https://www.facebook.com/ads/library/?id=1316851392647324}{1316851392647324}.
    \item publicarea si promovarea unui mesaj electoral platit dupa incheierea perioadei de campanie electorala (dupa ora 18:00 pe 23.11.2024), cu textul "Lasconi nu are nici un fel de pregatire pentru a deveni presedinte". Postarea, avand ID-ul \href{https://www.facebook.com/ads/library/?id=1529511724418845}{1529511724418845}, reprezinta propaganda electorala negativa indreptata impotriva candidatei Elena Lasconi, avand un efect electoral direct prin diminuarea increderii alegatorilor in capacitatea candidatei de a ocupa functia de presedinte, mesajul fiind distribuit ca reclama platita pe Facebook si Instagram, cu o audienta estimata de peste 1 milion de persoane, depasind astfel limitele unei simple opinii personale sau ale unei activitati jurnalistice legitime.
    \item publicarea unui articol sponsorizat pe Facebook si Instagram dupa ora 18:00 pe 23.11.2024, cu titlul "Cum pierde Lasconi voturi din cauza arogantei", care reprezinta propaganda electorala negativa la adresa candidatei Elena-Valerica Lasconi. Postarea are caracter electoral explicit, referindu-se direct la pierderea de voturi a candidatei, fiind difuzata in perioada de restrictie electorala cu un impact estimat intre 6000-6999 de afisari. Postarea cu ID-ul \href{https://www.facebook.com/ads/library/?id=1602086663727351}{1602086663727351} este conceputa si distribuita cu scopul clar de a influenta negativ optiunea de vot a alegatorilor in perioada de restrictie electorala.
    \item distribuirea de continut electoral negativ impotriva candidatei Elena Lasconi dupa incheierea perioadei de campanie electorala, la data de 23.11.2024. Postarea platita pe Facebook cu ID-ul \href{https://www.facebook.com/ads/library/?id=564170259552157}{564170259552157} contine mesajul "elena lasconi nu are facultate", reprezentand propaganda electorala negativa care urmareste influentarea votantilor prin discreditarea candidatului. Mesajul a fost distribuit ca reclama platita pe Facebook si Instagram, cu un buget semnificativ intre 300-399 RON si o audienta estimata intre 25.000-30.000 de persoane, demonstrand intentia clara de a influenta un numar mare de alegatori dupa perioada legala permisa pentru propaganda electorala.
    \item difuzarea unui mesaj de propaganda electorala negativa dupa incheierea perioadei de campanie, vizand in mod direct candidatul ELENA-VALERICA LASCONI. Postarea cu ID-ul \href{https://www.facebook.com/ads/library/?id=921254709937844}{921254709937844}, publicata dupa ora 18:00 pe 23.11.2024, foloseste un limbaj denigrator ("sotul parazit") cu scopul clar de a influenta negativ opinia publica si comportamentul electoral al votantilor. Caracterul de propaganda este demonstrat prin investitia semnificativa in promovare (200-299 RON), audienta tinta de peste 1 milion de persoane si distributia pe multiple platforme (Facebook si Instagram), rezultand intr-un impact substantial de 30.000-35.000 impresii.
\end{enumerate}

\vspace{0.5cm}

\subsection{60m.RO}
Următoarele fapte contravenționale sunt sesizate împotriva acestei entități:

\begin{enumerate}[leftmargin=*, label=\arabic*.)]
    \item publicarea si promovarea unui material de propaganda electorala (ID postare Facebook: \href{https://www.facebook.com/ads/library/?id=1564600244158214}{1564600244158214}) dupa incheierea perioadei de campanie electorala pentru alegerile prezidentiale. Materialul vizeaza direct un candidat la presedintie (Elena Lasconi), folosind un limbaj denigrator ("nepregatita, isterica si superficiala") cu scopul clar de a influenta negativ opinia publica si intentiile de vot. Postarea a fost promovata activ dupa data de 23.11.2024, ora 18:00, atingand un public de peste 100.000 de persoane prin intermediul unei campanii platite pe Facebook si Instagram, reprezentand astfel o incalcare clara a interdictiei de a continua propaganda electorala dupa incheierea campaniei.
    \item publicarea si promovarea unei reclame platite dupa ora 18:00 pe 23.11.2024 (ID postare Facebook: \href{https://www.facebook.com/ads/library/?id=1623065708567110}{1623065708567110}) care constituie propaganda electorala negativa la adresa candidatei Elena Lasconi. Postarea reprezinta propaganda electorala deoarece vizeaza direct un candidat la alegerile prezidentiale, are scop electoral evident prin modul in care critica si ridiculizeaza pozitia candidatei pe politica externa, este promovata ca reclama platita cu impact semnificativ (25.000-30.000 impresii estimate), si continutul sau depaseste limitele unei simple opinii jurnalistice, avand ca scop clar influentarea negativa a intentiei de vot a alegatorilor prin prezentarea candidatei intr-o lumina nefavorabila.
    \item continuarea propagandei electorale dupa incheierea campaniei, manifestata prin publicarea si mentinerea activa a unui anunt sponsorizat pe Facebook si Instagram (ID: \href{https://www.facebook.com/ads/library/?id=2048135028942292}{2048135028942292}) dupa ora 18:00 pe 23.11.2024. Postarea vizeaza in mod direct un candidat la presedintie (Elena-Valerica Lasconi), folosind caracterizari negative ("nepregatita, isterica si superficiala") in contextul unei dezbateri prezidentiale, cu scopul clar de a influenta opinia publica si comportamentul electoral. Anuntul a avut un impact semnificativ, atingand intre 80.000 si 90.000 de afisari, demonstrand intentia clara de a influenta un numar mare de alegatori prin cheltuirea de fonduri pentru promovare in perioada interzisa de lege.
    \item promovarea unui mesaj platit pe Facebook si Instagram dupa data de 23.11.2024, ora 18:00, care constituie propaganda electorala impotriva candidatei Elena Lasconi. Postarea cu ID-ul \href{https://www.facebook.com/ads/library/?id=2320121051682166}{2320121051682166} prezinta in mod critic si ironic pozitia candidatei pe politica externa, folosind un citat al lui Ion Cristoiu care o ridiculizeaza, avand un evident scop electoral de a-i diminua credibilitatea in fata alegatorilor. Caracterul de propaganda este demonstrat prin targetarea directa a candidatei, obiectivul electoral clar de influentare negativa a optiunii de vot, precum si prin distribuirea platita catre un public larg (peste 100.000 de impresii).
    \item publicarea si promovarea unei reclame platite pe Facebook si Instagram (ID: \href{https://www.facebook.com/ads/library/?id=3735627486698840}{3735627486698840}) dupa ora 18:00 pe 23.11.2024, care constituie propaganda electorala explicita impotriva candidatilor Elena Lasconi, Nicolae Ciuca si Mircea Geoana. Postarea utilizeaza un mesaj denigrator ("cea mai slaba dintre cei slabi") si face o comparatie defaimatoare intre candidati ("se lupta pentru medalia de bronz"), avand un evident scop electoral de influentare negativa a intentiei de vot. Materialul a fost distribuit ca reclama platita catre un public de peste 35.000 de persoane, demonstrand intentia clara de a influenta rezultatul alegerilor prezidentiale prin continuarea propagandei electorale dupa incheierea perioadei legale.
    \item promovarea unei reclame platite pe Facebook si Instagram (ID: \href{https://www.facebook.com/ads/library/?id=548556984620952}{548556984620952}) dupa ora 18:00 pe 23.11.2024, care constituie propaganda electorala in favoarea candidatului George Nicolae Simion si in defavoarea candidatei Elena Valerica Lasconi. Postarea, care a ajuns la un public de peste 80.000 de persoane, prezinta un comentariu selectiv si editorialiazat despre o dezbatere electorala, folosind expresii precum "Simion e mai gentleman" si criticand comportamentul contracandidatei ("nu lovesti apeland la jigniri"), cu scopul clar de a influenta preferintele electorale ale votantilor in perioada in care propaganda electorala este interzisa prin lege.
    \item continuarea propagandei electorale dupa incheierea campaniei, folosind o postare sponsorizata pe Facebook (ID: \href{https://www.facebook.com/ads/library/?id=596789496031396}{596789496031396}) care denigreaza in mod direct trei candidati la presedintie: Elena Lasconi, Nicolae Ciuca si Mircea Geoana. Postarea, care continua sa fie promovata dupa ora 18:00 pe 23.11.2024, foloseste un limbaj denigrator ("cea mai slaba dintre cei slabi") si creeaza o ierarhie negativa intre candidati ("se lupta pentru medalia de bronz"), avand un evident scop de influentare a votului. Impactul acestei postari este semnificativ, atingand intre 90.000 si 99.999 de afisari, reprezentand o incalcare clara a perioadei de liniste electorala.
    \item publicarea si promovarea unei reclame platite pe Facebook si Instagram (ID: \href{https://www.facebook.com/ads/library/?id=8705668632851069}{8705668632851069}) dupa ora 18:00 pe 23.11.2024, care reprezinta propaganda electorala negativa la adresa candidatei Elena Lasconi. Postarea, care compara candidata cu Viorica Dancila intr-un mod depreciativ ("Viorica Dancila 2.0"), are un evident scop electoral negativ, fiind promovata cu un buget substantial (400-499 RON) si atingand intre 50.000 si 59.999 de utilizatori. Aceasta comparatie denigratoare este menita sa influenteze negativ perceptia alegatorilor fata de candidata in perioada in care propaganda electorala este interzisa prin lege.
\end{enumerate}

\vspace{0.5cm}

\subsection{AUR - Salasu de Sus}
Următoarele fapte contravenționale sunt sesizate împotriva acestei entități:

\begin{enumerate}[leftmargin=*, label=\arabic*.)]
    \item promovarea unui mesaj de propaganda electorala pentru candidatul George Simion la functia de presedinte, dupa incheierea perioadei de campanie electorala. Postarea cu ID-ul \href{https://www.facebook.com/ads/library/?id=1089703459529652}{1089703459529652} contine un indemn explicit de vot ("il voi vota pe George Simion - presedinte"), fiind promovata ca reclama platita pe Facebook si Instagram, cu un buget intre 100-199 RON si un impact estimat intre 25.000-29.999 de afisari, dupa ora 18:00 pe 23.11.2024. Mesajul are un evident caracter de propaganda electorala, utilizand argumente persuasive pentru a influenta decizia de vot a alegatorilor in favoarea candidatului AUR la presedintie.
    \item publicarea si promovarea platita a unui mesaj de propaganda electorala dupa incheierea perioadei de campanie, folosind platforma Facebook pentru a influenta direct alegatorii sa voteze cu candidatul George Simion la presedintie. Postarea cu ID-ul \href{https://www.facebook.com/ads/library/?id=1121044549358758}{1121044549358758} contine indemnul explicit "la aceste alegeri, il voi vota pe George Simion - presedinte" si este distribuita ca reclama platita dupa ora 18:00 pe 23.11.2024, atingand un public de peste 1 milion de persoane, cu un buget intre 100-199 RON. Mesajul indeplineste toate criteriile propagandei electorale conform Art. 36(7): identifica direct candidatul, are scop electoral explicit, se adreseaza publicului larg si depaseste limitele activitatii jurnalistice.
    \item difuzarea de materiale de propaganda electorala pentru candidatul George Simion la alegerile prezidentiale dupa incheierea perioadei de campanie. Postarea sponsorizata cu ID \href{https://www.facebook.com/ads/library/?id=1231140648187092}{1231140648187092} promoveaza explicit candidatura la presedintie a lui George Simion prin titlul "De ce e nevoie de George Simion PRESEDINTE?" si textul care indeamna alegatorii sa "rupa cu partidele care au incalecat Romania", avand un evident obiectiv electoral de influentare a votului pentru alegerile prezidentiale. Postarea a fost activa si dupa ora 18:00 pe 23.11.2024, incalcand astfel prevederile legale privind perioada de campanie electorala.
    \item promovarea unui material de propaganda electorala pentru candidatul George Nicolae Simion la functia de presedinte al Romaniei dupa incheierea perioadei de campanie electorala. Materialul publicitar, cu ID-ul \href{https://www.facebook.com/ads/library/?id=2006851919778923}{2006851919778923}, promovat pe Facebook si Instagram dupa ora 18:00 pe 23.11.2024, contine mesaje directe de sustinere a candidaturii ("De ce e nevoie de George Simion PRESEDINTE?") si indeamna in mod explicit la schimbarea politica prin alegeri ("sansa unica de a schimba cu adevarat aceasta tara"), reprezentand astfel o continuare a propagandei electorale in afara perioadei legale permise. Efectul electoral este evident prin targetarea unui public larg (100,001-500,000 persoane) si utilizarea unui buget semnificativ pentru promovare (300-399 RON).
    \item continuarea propagandei electorale dupa incheierea perioadei legale de campanie, prin publicarea si mentinerea activa a unei reclame pe Facebook si Instagram (ID: \href{https://www.facebook.com/ads/library/?id=555150543924818}{555150543924818}) care promoveaza explicit candidatul George Simion la functia de presedinte. Postarea include numar CMF (11240014), contine indemnuri directe la vot ("Haideti sa dam impreuna un vot"), promoveaza explicit candidatul pentru functia de presedinte ("George Simion - PRESEDINTE"), si prezinta un mesaj electoral clar orientat spre influentarea votului ("Este sansa noastra de a pune capat unui sistem care a cultivat nesiguranta"). Publicitatea este activa si dupa ora 18:00 pe 23.11.2024, atingand un public intre 100,001 si 500,000 de persoane, cu un buget intre 300-399 RON, reprezentand astfel o incalcare clara a prevederilor legale privind incheierea campaniei electorale.
    \item promovarea unui mesaj de propaganda electorala pentru candidatul George Nicolae Simion la functia de presedinte al Romaniei, dupa incheierea perioadei de campanie electorala. Postarea cu ID-ul \href{https://www.facebook.com/ads/library/?id=932890622061028}{932890622061028}, publicata si promovata dupa ora 18:00 pe 23.11.2024, contine mesaje directe de sustinere electorala ("De ce e nevoie de George Simion PRESEDINTE?"), indeamna explicit la schimbarea conducerii tarii prin vot ("sansa unica de a schimba cu adevarat aceasta tara"), si a fost distribuita prin reclama platita pe Facebook si Instagram, atingand intre 45.000 si 50.000 de impresii, demonstrand astfel intentia clara de a influenta comportamentul electoral al alegatorilor in perioada in care propaganda electorala este interzisa prin lege.
    \item promovarea unui mesaj de propaganda electorala pentru candidatul George Nicolae Simion la alegerile prezidentiale, dupa incheierea perioadei de campanie electorala. Postarea cu ID-ul \href{https://www.facebook.com/ads/library/?id=979381740667749}{979381740667749} reprezinta propaganda electorala prin continutul sau explicit de indemn la vot ("il voi vota pe George Simion - presedinte"), fiind o reclama platita pe Facebook si Instagram, cu un impact estimat intre 10.000 si 15.000 de afisari, difuzata dupa ora 18:00 pe 23.11.2024. Mesajul are un clar caracter de propaganda electorala, utilizand narrative persuasive pentru influentarea votului si promovarea directa a unui candidat la functia de presedinte, depasind limitele unei simple exprimari de opinie personala prin natura sa de continut sponsorizat si distribuit la scara larga.
\end{enumerate}

\vspace{0.5cm}

\subsection{Actiunea.RO}
Următoarele fapte contravenționale sunt sesizate împotriva acestei entități:

\begin{enumerate}[leftmargin=*, label=\arabic*.)]
    \item postarea cu ID \href{https://www.facebook.com/ads/library/?id=1299186844846991}{1299186844846991} care constituie propaganda electorala dupa incheierea perioadei legale de campanie, publicata dupa ora 18:00 pe 23.11.2024. Postarea reprezinta propaganda electorala prin faptul ca promoveaza explicit candidatul George Simion la presedintie folosind hashtag-ul "\#GeorgeSimionPresedinte", discrediteaza candidata Elena Lasconi, si foloseste un mesaj electoral explicit cu scop de influentare a votului ("Romania merita o schimbare REALA!"). Postarea platita a avut un impact semnificativ, ajungand la 35,000-40,000 de persoane, folosind multiple platforme (Facebook si Instagram) pentru diseminare, cu un buget intre 300-399 RON, reprezentand astfel o incercare clara si sistematica de a influenta votul in perioada in care propaganda electorala este interzisa prin lege.
\end{enumerate}

\vspace{0.5cm}

\subsection{Adrian Axinia}
Următoarele fapte contravenționale sunt sesizate împotriva acestei entități:

\begin{enumerate}[leftmargin=*, label=\arabic*.)]
    \item continuarea propagandei electorale dupa incheierea campaniei electorale, postand pe Facebook (ID post: \href{https://www.facebook.com/ads/library/?id=1720075795393117}{1720075795393117}) un mesaj de propaganda electorala dupa ora 18:00 pe 23.11.2024. Postarea contine elementele definitorii ale propagandei electorale: numar de inregistrare CMF 11240014, indeamna explicit la vot pentru George Simion ca presedinte ("votati AUR si George Simion presedinte"), critica alti candidati si partide (PNL si PSD), si promoveaza mesaje de campanie ale AUR. Postarea este sponsorizata, avand un impact semnificativ (10,000-14,999 impresii), demonstrand intentia clara de a influenta votul in cadrul alegerilor prezidentiale dupa incheierea perioadei legale de campanie.
\end{enumerate}

\vspace{0.5cm}

\subsection{Adrian Costea}
Următoarele fapte contravenționale sunt sesizate împotriva acestei entități:

\begin{enumerate}[leftmargin=*, label=\arabic*.)]
    \item publicarea si promovarea unei reclame electorale platite pe Facebook (ID: \href{https://www.facebook.com/ads/library/?id=1649645469265920}{1649645469265920}) dupa incheierea perioadei de campanie electorala pentru alegerile prezidentiale, respectiv dupa ora 18:00 pe 23.11.2024. Postarea constituie propaganda electorala intrucat indeamna in mod explicit la votarea candidatei Elena Lasconi ("o iau pe Elena si o inghesui in turul 2"), face comparatii intre candidati cu scopul influentarii votului si foloseste argumentatie electorala pentru a convinge alegatorii sa voteze intr-un anumit fel. Mesajul este distribuit ca reclama platita, cu impact estimat intre 1000-1999 de afisari, demonstrand intentia clara de a influenta comportamentul electoral al unui public larg.
\end{enumerate}

\vspace{0.5cm}

\subsection{Agendazilei.ro}
Următoarele fapte contravenționale sunt sesizate împotriva acestei entități:

\begin{enumerate}[leftmargin=*, label=\arabic*.)]
    \item continuarea propagandei electorale dupa incheierea perioadei legale, prin publicarea si mentinerea activa a unei reclame platite pe Facebook (ID: \href{https://www.facebook.com/ads/library/?id=875282118090212}{875282118090212}) care il vizeaza direct pe candidatul prezidential Marcel Ciolacu, atacandu-l in mod explicit prin asocierea cu "vacante de lux" si presupuse legaturi cu compania Nordis, avand ca scop influentarea negativa a opiniei alegatorilor. Reclama, care a inceput pe 18 noiembrie si continua sa fie activa dupa ora 18:00 pe 23.11.2024, a avut un impact semnificativ, atingand intre 60.000 si 70.000 de afisari, reprezentand o incercare clara de a influenta negativ sansele electorale ale candidatului PSD la presedintie.
\end{enumerate}

\vspace{0.5cm}

\subsection{Agentia de publicitate Confort Media (Aldo Detail Direct SRL) pentru Partidul National Liberal - Filiala Hunedoara}
Următoarele fapte contravenționale sunt sesizate împotriva acestei entități:

\begin{enumerate}[leftmargin=*, label=\arabic*.)]
    \item continuarea propagandei electorale dupa incheierea campaniei pentru alegerile prezidentiale, prin postarea unui mesaj sponsorizat pe Facebook si Instagram (ID: \href{https://www.facebook.com/ads/library/?id=861447816196566}{861447816196566}) care promoveaza explicit candidatul Nicolae Ionel Ciuca ca "singurul care poate opri binomul PSD-AUR" si indeamna la "votul util", avand un caracter evident de propaganda electorala demonstrat prin prezenta numarului CMF 11240002, mesajul fiind difuzat dupa ora 18:00 pe 23.11.2024, cu o audienta estimata intre 100.001 si 500.000 de persoane si un buget substantial de promovare, scopul evident fiind influentarea comportamentului electoral al alegatorilor in favoarea candidatului PNL si in defavoarea candidatilor PSD si AUR.
\end{enumerate}

\vspace{0.5cm}

\subsection{AgentiadeInformatii.ro}
Următoarele fapte contravenționale sunt sesizate împotriva acestei entități:

\begin{enumerate}[leftmargin=*, label=\arabic*.)]
    \item promovarea dupa ora 18:00 pe 23.11.2024 a unui mesaj de propaganda electorala negativa care vizeaza un candidat la presedintie, folosind intrebarea retorica "Romania are nevoie de un presedinte mason?". Mesajul are un caracter electoral evident, fiind difuzat ca reclama platita pe Facebook si Instagram cu un buget intre 900-999 RON si atingand intre 200.000-250.000 de persoane, avand ca scop influentarea votului prin asocierea negativa a candidatului cu masoneria. Postarea cu ID-ul \href{https://www.facebook.com/ads/library/?id=1208628637065056}{1208628637065056} reprezinta o forma de propaganda electorala care continua dupa incheierea perioadei legale de campanie.
    \item difuzarea unui mesaj cu caracter electoral dupa incheierea perioadei de propaganda, folosind o intrebare retorica "Romania are nevoie de un presedinte mason?" care vizeaza in mod clar candidatul Mircea-Dan Geoana, cu scopul de a influenta negativ optiunea de vot a alegatorilor. Postarea cu ID-ul \href{https://www.facebook.com/ads/library/?id=1592309648050326}{1592309648050326} a fost promovata ca reclama platita pe Facebook si Instagram, atingand intre 200.000 si 250.000 de persoane, demonstrand intentia clara de a influenta opinia publica dupa incheierea perioadei legale de campanie electorala, respectiv dupa ora 18:00 pe 23.11.2024.
    \item promovarea unui mesaj de propaganda electorala dupa incheierea perioadei de campanie, referitor la candidatul George Simion. Postarea cu ID \href{https://www.facebook.com/ads/library/?id=3865348743725271}{3865348743725271} il prezinta pe George Simion ca "viitorul presedinte" si promoveaza o alianta electorala intre doi candidati, cu scopul clar de a influenta alegatorii. Mesajul a fost distribuit ca reclama platita pe Facebook si Instagram, cu un impact semnificativ (peste 250.000 de afisari), dupa ora 18:00 pe 23.11.2024, reprezentand astfel o forma clara de propaganda electorala in afara perioadei permise de lege.
\end{enumerate}

\vspace{0.5cm}

\subsection{Aladin Georgescu si PSD}
Următoarele fapte contravenționale sunt sesizate împotriva acestei entități:

\begin{enumerate}[leftmargin=*, label=\arabic*.)]
    \item continuarea propagandei electorale pentru alegerile prezidentiale dupa incheierea campaniei, prin intermediul unei postari sponsorizate pe Facebook (ID: \href{https://www.facebook.com/ads/library/?id=2229317524117504}{2229317524117504}) care indeamna in mod explicit la votarea candidatului Marcel Ciolacu in alegerile prezidentiale din 24 noiembrie. Postarea, difuzata dupa ora 18:00 pe 23.11.2024, foloseste un proiect de infrastructura medicala ca pretext pentru a promova candidatul la presedintie, continand indemnul direct "La fel de important este votul din data de 24 noiembrie pentru viitorul presedinte al Romaniei, Marcel Ciolacu!" si sloganuri de campanie, avand un evident caracter de propaganda electorala conform Art. 36(7) din Legea 334/2006, demonstrat si prin prezenta codului CMF 11240017.
\end{enumerate}

\vspace{0.5cm}

\subsection{Alianta pentru Unirea Romanilor (AUR)}
Următoarele fapte contravenționale sunt sesizate împotriva acestei entități:

\begin{enumerate}[leftmargin=*, label=\arabic*.)]
    \item promovarea unui material de propaganda electorala pentru candidatul prezidential George Nicolae Simion dupa incheierea perioadei de campanie electorala. Materialul, cu ID-ul facebook \href{https://www.facebook.com/ads/library/?id=562641266468758}{562641266468758}, postat dupa ora 18:00 pe 23.11.2024, constituie propaganda electorala prin faptul ca promoveaza explicit candidatul ("Sustine George Simion Judetul Constanta"), contine numar CMF (11240014), este platit pentru a ajunge la un public larg (25,000-29,999 impresii), si are obiectiv electoral explicit de sustinere a candidatului la presedintie. Materialul este comandat oficial de catre partid, conform mentiunii "Comandat de Alianta Pentru Unirea Romanilor".
\end{enumerate}

\vspace{0.5cm}

\subsection{Alianta pentru Unirea Romanilor Valcea}
Următoarele fapte contravenționale sunt sesizate împotriva acestei entități:

\begin{enumerate}[leftmargin=*, label=\arabic*.)]
    \item continuarea propagandei electorale dupa incheierea perioadei legale de campanie, intr-o postare platita pe Facebook (ID: \href{https://www.facebook.com/ads/library/?id=3847753942163982}{3847753942163982}) care promoveaza explicit candidatul George Simion si programul sau electoral "planul Simion". Postarea contine un indemn direct la vot ("vino la vot pentru un viitor mai bun!"), foloseste hashtag-uri electorale (\#VoteazaAUR) si promoveaza explicit programul candidatului pentru infrastructura, avand un evident scop electoral. Postarea a fost activa si dupa ora 18:00 pe 23.11.2024, incalcand astfel prevederile legale privind incheierea campaniei electorale.
    \item continuarea propagandei electorale dupa incheierea perioadei legale de campanie, prin postarea cu ID \href{https://www.facebook.com/ads/library/?id=914482113515376}{914482113515376} pe Facebook. Postarea promoveaza candidatul prezidential George Simion si programul sau electoral "Planul Simion", contine indemnuri directe la vot ("Vino la vot", "Voteaza AUR"), face promisiuni electorale privind educatia si salariile profesorilor, si foloseste hashtag-uri de campanie (\#VoteazaAUR). Postarea este sponsorizata si distribuita dupa ora 18:00 pe 23.11.2024, avand un impact semnificativ cu o audienta estimata intre 100,001 si 500,000 de persoane, constituind astfel propaganda electorala activa in afara perioadei legale de campanie.
\end{enumerate}

\vspace{0.5cm}

\subsection{Alin Calinescu}
Următoarele fapte contravenționale sunt sesizate împotriva acestei entități:

\begin{enumerate}[leftmargin=*, label=\arabic*.)]
    \item publicarea si promovarea unui mesaj electoral platit (ID postare Facebook: \href{https://www.facebook.com/ads/library/?id=1261220475211251}{1261220475211251}) dupa incheierea perioadei de campanie electorala, in data de 23.11.2024. Postarea constituie propaganda electorala conform tuturor criteriilor legale: contine numar CMF (11240002), face referire directa la candidati in alegerile prezidentiale (Nicolae Ciuca si Marcel Ciolacu), are obiectiv electoral explicit prin indemnul direct "Pe 24 noiembrie, votati Nicolae Ciuca", si depaseste limitele unei simple opinii personale prin natura sa de continut sponsorizat si distribuit catre un public larg. Efectul electoral este evident prin combinatia de promovare pozitiva a candidatului Nicolae Ciuca ("om integru, patriot adevarat") si denigrare a contracandidatului Marcel Ciolacu, cu intentia clara de a influenta votul in ziua alegerilor.
    \item continuarea propagandei electorale dupa incheierea campaniei, prin postarea cu ID \href{https://www.facebook.com/ads/library/?id=3743931592523454}{3743931592523454} pe Facebook dupa ora 18:00 pe 23.11.2024. Postarea constituie propaganda electorala conform art. 36(7) din Legea 334/2006, continand materiale care se refera direct la candidatii Nicolae Ciuca si Marcel Ciolacu, avand obiectiv electoral explicit prin indemnul "votati Nicolae Ciuca", adresandu-se publicului larg prin distributie platita pe Facebook si Instagram. Postarea include numar CMF (11240002), confirmand natura sa de material de propaganda electorala, si face referire directa la "aceasta duminica" in contextul alegerilor prezidentiale, incercand sa influenteze comportamentul electoral prin prezentarea contrastanta a celor doi candidati.
\end{enumerate}

\vspace{0.5cm}

\subsection{Alina Gorghiu (PNL)}
Următoarele fapte contravenționale sunt sesizate împotriva acestei entități:

\begin{enumerate}[leftmargin=*, label=\arabic*.)]
    \item publicarea si promovarea dupa data de 23.11.2024, ora 18:00, a unei postari cu caracter de propaganda electorala (ID postare: \href{https://www.facebook.com/ads/library/?id=1285438759162935}{1285438759162935}) care face referire directa la candidati in alegerile prezidentiale (Calin Georgescu, George Simion), discuta strategii electorale pentru turul doi si incearca sa influenteze optiunile de vot ale alegatorilor. Postarea include cod mandatar financiar (CMF 31240003), confirmand natura sa de propaganda electorala, si este promovata activ prin intermediul platformelor Facebook si Instagram, atingand un public tinta estimat intre 500.001 si 1.000.000 de persoane, demonstrand clar intentia de a influenta procesul electoral dupa incheierea perioadei legale de campanie.
\end{enumerate}

\vspace{0.5cm}

\subsection{Anchetatorii.RO}
Următoarele fapte contravenționale sunt sesizate împotriva acestei entități:

\begin{enumerate}[leftmargin=*, label=\arabic*.)]
    \item promovarea unui material cu caracter electoral negativ la adresa candidatei Elena Lasconi dupa incheierea perioadei de campanie electorala, prin intermediul unei reclame platite pe Facebook si Instagram (ID: \href{https://www.facebook.com/ads/library/?id=493537339697466}{493537339697466}). Materialul, difuzat dupa ora 18:00 pe 23.11.2024, prezinta un caracter vadit electoral prin atacul direct la adresa candidatei ("bani pentru condamnati si protejatii SRI"), folosind o investitie semnificativa in promovare (200-299 RON) pentru a atinge un public larg (100,000-124,999 impresii). Continutul depaseste limitele activitatii jurnalistice obiective, avand ca scop influentarea negativa a intentiei de vot a alegatorilor.
    \item publicarea si promovarea unui material de propaganda electorala negativa dupa incheierea perioadei de campanie electorala, vizand direct candidatul Elena Lasconi. Materialul, cu ID-ul \href{https://www.facebook.com/ads/library/?id=544782674990174}{544782674990174}, foloseste simboluri de avertizare () si asocieri negative ("condamnati si protejatii SRI") pentru a influenta negativ opinia alegatorilor, fiind distribuit ca reclama platita pe Facebook si Instagram, cu un impact intre 50.000 si 59.999 de afisari, dupa ora 18:00 pe 23.11.2024. Acest continut depaseste limitele activitatii jurnalistice de informare, avand un evident caracter de propaganda electorala negativa.
\end{enumerate}

\vspace{0.5cm}

\subsection{Andreka Adrian-Dan}
Următoarele fapte contravenționale sunt sesizate împotriva acestei entități:

\begin{enumerate}[leftmargin=*, label=\arabic*.)]
    \item publicarea si promovarea unui anunt electoral platit pe Facebook (ID: \href{https://www.facebook.com/ads/library/?id=1950318142131315}{1950318142131315}) dupa incheierea perioadei de campanie electorala pentru alegerile prezidentiale, respectiv dupa ora 18:00 pe 23.11.2024. Postarea contine mesaje directe de propaganda electorala pentru candidatul ION-MARCEL CIOLACU, cu indemnuri explicite de vot ("Duminica alegem", "alegeti calea sigura pentru Romania!") si promovarea directa a candidatului pentru functia de presedinte ("Marcel Ciolacu are toate calitatile pentru a fi Presedintele Romaniei"). Mesajul a fost promovat ca reclama platita, atingand intre 6.000 si 7.000 de afisari, avand un impact semnificativ si un obiectiv electoral clar de influentare a votului.
\end{enumerate}

\vspace{0.5cm}

\subsection{Antonel Tanase}
Următoarele fapte contravenționale sunt sesizate împotriva acestei entități:

\begin{enumerate}[leftmargin=*, label=\arabic*.)]
    \item promovarea unui mesaj de propaganda electorala pentru candidatul Ludovic Orban dupa incheierea perioadei de campanie electorala (dupa ora 18:00 pe 23.11.2024). Postarea cu ID-ul \href{https://www.facebook.com/ads/library/?id=568335808926391}{568335808926391} contine promisiuni electorale explicite ("Ne propunem ca obiectiv ca in decurs de 5 ani venitul mediu al cetatenilor romani sa atinga venitul mediu la nivel european"), prezinta candidatul in contextul alegerilor prezidentiale ("candidat la alegerile prezidentiale"), si foloseste hashtag-uri specifice campaniei (\#Prezidentiale2024). Mesajul a fost promovat ca reclama platita pe Facebook, cu un impact semnificativ de peste 450.000 de afisari, demonstrand intentia clara de a influenta comportamentul electoral al alegatorilor in perioada in care propaganda electorala este interzisa prin lege.
\end{enumerate}

\vspace{0.5cm}

\subsection{Arina Mos}
Următoarele fapte contravenționale sunt sesizate împotriva acestei entități:

\begin{enumerate}[leftmargin=*, label=\arabic*.)]
    \item distribuirea de materiale de propaganda electorala dupa incheierea perioadei de campanie, intr-o postare platita pe Facebook (ID: \href{https://www.facebook.com/ads/library/?id=3410615485741856}{3410615485741856}) cu impact intre 500.001 si 1.000.000 de persoane, dupa ora 18:00 pe 23.11.2024. Postarea contine elemente clare de propaganda electorala pentru candidatul Nicolae-Ionel Ciuca, inclusiv indemnuri directe la vot ("voi vota Nicolae Ionel Ciuca presedinte"), utilizeaza numar oficial de campanie (CMF 11240002), si incearca sa influenteze comportamentul electoral al alegatorilor prin mesaje precum "sa nu-si iroseasca votul de dreapta". Efectul electoral urmarit este clar orientat spre mobilizarea votantilor in favoarea candidatului PNL la presedintie.
\end{enumerate}

\vspace{0.5cm}

\subsection{Atelierul de Internet SRL}
Următoarele fapte contravenționale sunt sesizate împotriva acestei entități:

\begin{enumerate}[leftmargin=*, label=\arabic*.)]
    \item promovarea unui mesaj electoral platit pe Facebook (ID: \href{https://www.facebook.com/ads/library/?id=930212335660605}{930212335660605}) dupa incheierea campaniei electorale pentru alegerile prezidentiale (dupa ora 18:00 pe 23.11.2024). Postarea, desi aparent vizeaza alegerile parlamentare, contine elemente clare de propaganda electorala negativa la adresa candidatului prezidential Marcel Ciolacu, prin expresii precum "ia-i banii lui Ciolacu" si alte referinte negative, cu potential de a influenta votul la alegerile prezidentiale. Mesajul este platit, are numar CMF (11240046), si a atins intre 15.000 si 20.000 de persoane, demonstrand intentia clara de a influenta opinia publica in perioada in care propaganda electorala pentru alegerile prezidentiale este interzisa.
\end{enumerate}

\vspace{0.5cm}

\subsection{Atelierul de internet SRL}
Următoarele fapte contravenționale sunt sesizate împotriva acestei entități:

\begin{enumerate}[leftmargin=*, label=\arabic*.)]
    \item difuzarea de materiale de propaganda electorala dupa incheierea perioadei de campanie, respectiv dupa ora 18:00 pe 23.11.2024. Postarea cu ID-ul \href{https://www.facebook.com/ads/library/?id=1354983342574061}{1354983342574061} reprezinta propaganda electorala conform Art. 36(7), continand CMF 112400400, promovand explicit candidatura lui Mircea Geoana la presedintie si indemnand direct la vot ("Pe 1 decembrie votati candidatii Partidul Romania in Actiune"). Postarea este sponsorizata, targetand un public larg (45,000-50,000 impresii) si are obiectiv electoral explicit, sustinand un candidat la alegerile prezidentiale dupa incheierea perioadei legale de campanie.
    \item promovarea unui mesaj de propaganda electorala (ID Facebook: \href{https://www.facebook.com/ads/library/?id=541831591809413}{541831591809413}) dupa incheierea perioadei de campanie electorala pentru alegerile prezidentiale. Postarea, activa dupa ora 18:00 pe 23.11.2024, contine sustinere explicita pentru candidatul Mircea Geoana la presedintie ("sustin candidatura lui Mircea Geoana"), fiind insotita de un numar CMF (112400400) si prezentand un indemn direct la vot. Mesajul are caracter electoral evident, fiind distribuit prin intermediul unei reclame platite pe Facebook cu impact semnificativ (80,000-90,000 impresii), depasind astfel limitele comunicarii permise in aceasta perioada.
\end{enumerate}

\vspace{0.5cm}

\subsection{Bogdan Rodeanu si USR}
Următoarele fapte contravenționale sunt sesizate împotriva acestei entități:

\begin{enumerate}[leftmargin=*, label=\arabic*.)]
    \item continuarea propagandei electorale pentru alegerile prezidentiale dupa incheierea perioadei legale, prin postarea cu ID \href{https://www.facebook.com/ads/library/?id=1719350121971281}{1719350121971281} pe Facebook. Postarea, difuzata dupa ora 18:00 pe 23.11.2024, contine un indemn explicit de vot pentru "pozitia 1 pe toate buletinele de vot", atacuri directe la adresa candidatilor prezidentiali Marcel Ciolacu si Nicolae Ciuca, precum si promovarea indirecta a candidatului USR. Postarea include numar CMF (11240015), are caracter electoral evident prin mesajul sau, este distribuita ca reclama platita pe Facebook si Instagram, si vizeaza in mod direct influentarea votului pentru alegerile prezidentiale, incalcand astfel prevederile legale privind incheierea campaniei electorale.
\end{enumerate}

\vspace{0.5cm}

\subsection{Ciprian Paraschiv}
Următoarele fapte contravenționale sunt sesizate împotriva acestei entități:

\begin{enumerate}[leftmargin=*, label=\arabic*.)]
    \item difuzarea unui material de propaganda electorala pentru candidatul George Nicolae Simion la alegerile prezidentiale, dupa incheierea perioadei de campanie electorala. Postarea promovata pe Facebook si Instagram (ID: \href{https://www.facebook.com/ads/library/?id=429882539993488}{429882539993488}) contine mesajul explicit "Votez George Simion!" si "Presedintele tuturor romanilor", reprezentand un indemn clar la vot pentru un candidat specific, cu un impact potential asupra a peste 1 milion de utilizatori, fiind difuzata dupa ora 18:00 pe 23.11.2024. Materialul indeplineste toate criteriile propagandei electorale: referire directa la candidat, obiectiv electoral explicit, adresabilitate catre publicul larg si depasirea limitelor activitatii jurnalistice de informare.
    \item publicarea si promovarea unui mesaj de propaganda electorala (ID postare Facebook: \href{https://www.facebook.com/ads/library/?id=495679523503238}{495679523503238}) dupa incheierea perioadei de campanie electorala (dupa ora 18:00 pe 23.11.2024). Postarea contine un indemn explicit la vot pentru candidatul George Simion ("votezi George Simion presedinte"), atacuri la adresa contracandidatilor (Ciolacu, Ciuca, Lasconi), precum si indemnuri repetate de mobilizare la vot pentru data de 24 noiembrie, avand un evident caracter de propaganda electorala confirmat si de prezenta numarului CMF 11240014. Efectul electoral urmarit este influentarea directa a alegatorilor in favoarea candidatului AUR si in defavoarea celorlalti candidati mentionati, prin utilizarea unui limbaj emotional si mobilizator.
\end{enumerate}

\vspace{0.5cm}

\subsection{Comentatorii.RO}
Următoarele fapte contravenționale sunt sesizate împotriva acestei entități:

\begin{enumerate}[leftmargin=*, label=\arabic*.)]
    \item promovarea unui material de propaganda electorala pentru alegerile prezidentiale dupa incheierea perioadei de campanie, respectiv dupa ora 18:00 pe 23.11.2024. Materialul, cu ID-ul \href{https://www.facebook.com/ads/library/?id=8744995008916796}{8744995008916796} pe Facebook, prezinta explicit sustinerea unui candidat prezidential (George Simion) in detrimentul altui candidat (Elena Lasconi), folosind resurse financiare pentru promovare (600-699 RON) si obtinand un impact semnificativ (300,000-350,000 impresii). Continutul depaseste limitele activitatii jurnalistice obiective, prezentand explicit sustinerea unui candidat si respingerea altuia, cu scopul clar de a influenta preferintele electorale ale publicului larg.
\end{enumerate}

\vspace{0.5cm}

\subsection{Computer Fun SRL}
Următoarele fapte contravenționale sunt sesizate împotriva acestei entități:

\begin{enumerate}[leftmargin=*, label=\arabic*.)]
    \item difuzarea de materiale de propaganda electorala pentru candidatul ION-MARCEL CIOLACU dupa incheierea campaniei electorale. Postarea cu ID-ul \href{https://www.facebook.com/ads/library/?id=932359448775477}{932359448775477}, publicata dupa ora 18:00 pe 23.11.2024, contine un numar CMF (11240017) si foloseste sloganul "\#MALUCUFLORI" impreuna cu mesajul "CALEA SIGURA PENTRU", reprezentand in mod clar un indemn electoral care incerca sa influenteze comportamentul alegatorilor in favoarea candidatului PSD la presedintie. Materialul a fost difuzat ca reclama platita pe Facebook si Instagram, cu un impact estimat intre 1000 si 1999 de persoane, demonstrand intentia clara de a influenta votul prin continuarea propagandei electorale dupa incheierea perioadei legale.
\end{enumerate}

\vspace{0.5cm}

\subsection{Computer Fun SRL si PSD Dambovita}
Următoarele fapte contravenționale sunt sesizate împotriva acestei entități:

\begin{enumerate}[leftmargin=*, label=\arabic*.)]
    \item difuzarea unui mesaj de propaganda electorala dupa incheierea campaniei electorale, prin intermediul unei postari sponsorizate pe Facebook cu ID-ul \href{https://www.facebook.com/ads/library/?id=1103522491121382}{1103522491121382}. Postarea, care include mesajul "CALEA SIGURA PENTRU \#TATARANI" si numarul CMF 11240017, reprezinta o forma clara de propaganda electorala ce continua dupa ora 18:00 pe 23.11.2024, avand ca scop influentarea votului in favoarea candidatului PSD la presedintie. Materialul este distribuit ca reclama platita pe Facebook si Instagram, vizand specific comunitatea din Tatarani, depasind astfel cadrul legal permis pentru comunicarea politica in aceasta perioada.
    \item difuzarea de materiale de propaganda electorala dupa incheierea campaniei electorale, concretizata prin postarea cu ID \href{https://www.facebook.com/ads/library/?id=1752981645241567}{1752981645241567} pe facebook, care promoveaza mesajul "CALEA SIGURA PENTRU \#GURAFOII" cu numar CMF 11240017. Postarea, fiind activa si dupa ora 18:00 pe 23.11.2024, reprezinta continuare a propagandei electorale dupa incheierea acesteia, cu potential de influentare a votului in localitatea Gura Foii. Caracterul electoral este dovedit prin prezenta numarului CMF si natura mesajului promotional.
\end{enumerate}

\vspace{0.5cm}

\subsection{Comunitatea Liberala}
Următoarele fapte contravenționale sunt sesizate împotriva acestei entități:

\begin{enumerate}[leftmargin=*, label=\arabic*.)]
    \item publicarea si promovarea active dupa data de 23.11.2024, ora 18:00, a unei reclame pe Facebook (ID: \href{https://www.facebook.com/ads/library/?id=1598757470719860}{1598757470719860}) ce constituie propaganda electorala impotriva candidatului NICOLAE-IONEL CIUCA. Postarea contine un indemn direct de a nu-l vota pe candidat ("Nu va cer sa-l retrageti pe Ciuca, va cer sa nu-l votati!"), fiind promovata prin plata catre un public larg (peste 1 milion de utilizatori potentiali), cu un buget intre 100-199 RON. Efectul electoral este evident negativ, urmarind in mod direct descurajarea votarii candidatului PNL la alegerile prezidentiale, intr-o perioada in care propaganda electorala este interzisa prin lege.
\end{enumerate}

\vspace{0.5cm}

\subsection{Comunitatea Liberala 1848}
Următoarele fapte contravenționale sunt sesizate împotriva acestei entități:

\begin{enumerate}[leftmargin=*, label=\arabic*.)]
    \item promovarea unui material de propaganda electorala dupa incheierea perioadei de campanie, respectiv dupa ora 18:00 pe 23.11.2024. Materialul, cu ID-ul \href{https://www.facebook.com/ads/library/?id=506403022549368}{506403022549368} pe Facebook, promoveaza explicit candidatul Elena Lasconi pentru alegerile prezidentiale, prin articolul intitulat "De ce votez cu Lasconi", care face referire directa la scrutinul prezidential din 24 noiembrie si incearca sa influenteze decizia de vot a alegatorilor. Postarea este promovata ca reclama platita pe Facebook, avand o audienta tinta de peste 1 milion de persoane, depasind astfel sfera unei simple opinii personale si constituind propaganda electorala activa in perioada de interdictie.
\end{enumerate}

\vspace{0.5cm}

\subsection{Dan Cosma}
Următoarele fapte contravenționale sunt sesizate împotriva acestei entități:

\begin{enumerate}[leftmargin=*, label=\arabic*.)]
    \item promovarea unui mesaj electoral platit pe Facebook si Instagram (ID postare: \href{https://www.facebook.com/ads/library/?id=1490494721608433}{1490494721608433}) dupa incheierea perioadei de campanie electorala pentru alegerile prezidentiale. Postarea contine indemnuri directe de vot pentru candidatul George Simion la functia de presedinte ("Voteaza George Simion - pozitia 2 la prezidentiale"), material de propaganda electorala identificat prin cod CMF 11240014, si este distribuit in mod activ prin instrumente de promovare platita catre un public tinta estimat intre 100,001 si 500,000 de persoane, dupa ora 18:00 pe 23.11.2024, reprezentand astfel continuarea propagandei electorale dupa incheierea acesteia.
\end{enumerate}

\vspace{0.5cm}

\subsection{Dan Cosma si AUR}
Următoarele fapte contravenționale sunt sesizate împotriva acestei entități:

\begin{enumerate}[leftmargin=*, label=\arabic*.)]
    \item continuarea propagandei electorale pentru alegerile prezidentiale dupa incheierea campaniei electorale, prin postarea cu ID \href{https://www.facebook.com/ads/library/?id=1262654651439486}{1262654651439486} pe Facebook. Postarea contine indemnuri directe de vot pentru candidatul la presedintie George Simion ("Voteaza George Simion - pozitia 2 la prezidentiale"), fiind o reclama platita, activa dupa ora 18:00 pe 23.11.2024, cu un CMF valid (11240014), care isi propune sa influenteze in mod direct comportamentul electoral al alegatorilor pentru scrutinul prezidential. Postarea depaseste cadrul permis al campaniei parlamentare in derulare, constituind propaganda electorala explicita pentru alegerile prezidentiale in afara perioadei legale.
\end{enumerate}

\vspace{0.5cm}

\subsection{Dancus Ioan Doru}
Următoarele fapte contravenționale sunt sesizate împotriva acestei entități:

\begin{enumerate}[leftmargin=*, label=\arabic*.)]
    \item promovarea unui mesaj de propaganda electorala pentru candidatul ION-MARCEL CIOLACU dupa incheierea perioadei de campanie electorala, prin intermediul unei postari sponsorizate pe Facebook (ID: \href{https://www.facebook.com/ads/library/?id=1107137107522452}{1107137107522452}) dupa ora 18:00 pe 23.11.2024. Postarea promoveaza explicit candidatura la presedintie a lui Marcel Ciolacu, il prezinta ca fiind cel mai potrivit candidat pentru functia de presedinte ("este presedintele care intelege romanii si nevoile lor"), foloseste pozitia sa actuala de prim-ministru pentru a-i creste sansele electorale, si incearca sa influenteze alegatorii prin cheltuirea a 900-999 RON pentru o audienta estimata intre 30.000-35.000 de persoane. Mesajul are un evident caracter de propaganda electorala, depasind limitele unei simple opinii personale prin natura sa platita si distributia la scara larga.
    \item continuarea propagandei electorale dupa incheierea campaniei, prin intermediul unei postari sponsorizate pe Facebook (ID: \href{https://www.facebook.com/ads/library/?id=8333543073439536}{8333543073439536}) care ruleaza activ si dupa ora 18:00 pe 23.11.2024. Postarea face propaganda electorala explicita pentru candidatul ION-MARCEL CIOLACU, continand mesaje clare de indemn la vot ("Pe 24 noiembrie, votam \#CaleaSigura pentru Romania!"), promovarea realizarilor acestuia si declaratii cu caracter electoral ("Marcel Ciolacu este si va fi presedintele tuturor romanilor"). Efectul electoral este evident prin atingerea unei audiente estimate de peste 1 milion de persoane, cu un buget semnificativ alocat promovarii, intr-un moment in care legea interzice explicit continuarea propagandei electorale.
\end{enumerate}

\vspace{0.5cm}

\subsection{DigiPres.ro}
Următoarele fapte contravenționale sunt sesizate împotriva acestei entități:

\begin{enumerate}[leftmargin=*, label=\arabic*.)]
    \item promovarea unei reclame platite pe Facebook si Instagram (ID: \href{https://www.facebook.com/ads/library/?id=1565190404109410}{1565190404109410}) dupa ora 18:00 pe 23.11.2024, care constituie propaganda electorala pentru alegerile prezidentiale. Postarea promoveaza explicit mai multi candidati la presedintie, creand naratiuni despre aliante si opozitii intre acestia, cu un efect electoral clar: defavorizarea candidatei Elena Lasconi si favorizarea candidatilor George Simion si Calin Georgescu. Continutul depaseste limitele activitatii jurnalistice obiective, avand un obiectiv electoral explicit prin promovarea unor aliante politice si pozitionari fata de candidati, fiind distribuit contra cost catre un public larg de peste 300.000 de persoane.
\end{enumerate}

\vspace{0.5cm}

\subsection{Digidev Innotech}
Următoarele fapte contravenționale sunt sesizate împotriva acestei entități:

\begin{enumerate}[leftmargin=*, label=\arabic*.)]
    \item publicarea si promovarea dupa data de 23.11.2024 ora 18:00 a unui mesaj de propaganda electorala (ID postare: \href{https://www.facebook.com/ads/library/?id=1155401792993940}{1155401792993940}) care indeamna in mod explicit la votarea candidatei Elena Lasconi si prezinta in mod negativ candidatii Marcel Ciolacu si George Simion. Postarea contine numar CMF (11240022), face referiri directe la alegerile prezidentiale si include indemnuri explicite de vot ("trebuie dublat de prezenta noastra la vot pentru Lasconi"), avand un evident caracter de propaganda electorala. Mesajul este promovat ca reclama platita pe Facebook, atingand un public intre 10.000 si 14.999 de persoane, demonstrand intentia clara de influentare a votului in alegerile prezidentiale dupa incheierea campaniei electorale.
    \item publicarea si promovarea activa a unui mesaj de propaganda electorala (ID Facebook: \href{https://www.facebook.com/ads/library/?id=2631054730430413}{2631054730430413}) dupa incheierea perioadei de campanie electorala pentru alegerile prezidentiale. Postarea contine in mod explicit indemnuri la vot pentru candidatul Elena Lasconi la functia de presedinte ("Votati Elena Lasconi pentru presedinte"), precum si mesaje negative la adresa contracandidatilor Marcel Ciolacu si George Simion, fiind realizata cu intentie clara de influentare a votului. Materialul este marcat ca propaganda electorala prin prezenta codului CMF 11240022, are caracter platit si a fost difuzat dupa ora 18:00 pe 23.11.2024, incalcand astfel prevederile legale privind perioada de campanie electorala.
    \item publicarea si promovarea unui material de propaganda electorala (ID: \href{https://www.facebook.com/ads/library/?id=491308380618000}{491308380618000}) dupa incheierea perioadei de campanie electorala pentru alegerile prezidentiale. Materialul, difuzat ca reclama platita pe Facebook, face referire directa la candidatii la presedintie Marcel Ciolacu si George Simion, folosind expresia "sa spargem blatul dintre Ciolacu si Simion", avand un evident caracter electoral negativ fata de acestia. Postarea, care include si numarul CMF 11240022, demonstreaza clar intentia de a influenta comportamentul electoral al alegatorilor pentru scrutinul prezidential, fiind publicata dupa ora 18:00 pe 23.11.2024, in perioada de interdictie a propagandei electorale pentru alegerile prezidentiale.
\end{enumerate}

\vspace{0.5cm}

\subsection{Domnul Horia Constantinescu}
Următoarele fapte contravenționale sunt sesizate împotriva acestei entități:

\begin{enumerate}[leftmargin=*, label=\arabic*.)]
    \item publicarea si promovarea contra cost a unei postari pe Facebook (ID: \href{https://www.facebook.com/ads/library/?id=1289994562135510}{1289994562135510}) ce constituie propaganda electorala dupa incheierea perioadei de campanie, postare ce sustine explicit candidatura lui Marcel Ciolacu la presedintie. Postarea, activa si dupa ora 18:00 pe 23.11.2024, contine un indemn direct de sustinere ("Sustin candidatura lui Marcel Ciolacu!"), argumentatie electorala si mesaje ce vizeaza influentarea votului ("Trebuie sa aveti incredere intr-o decizie corecta", "Aceasta este calea sigura pentru Romania"), avand un impact semnificativ prin distribuirea platita ce a generat intre 10.000 si 14.999 de afisari.
\end{enumerate}

\vspace{0.5cm}

\subsection{Dumitru Rujan}
Următoarele fapte contravenționale sunt sesizate împotriva acestei entități:

\begin{enumerate}[leftmargin=*, label=\arabic*.)]
    \item continuarea propagandei electorale dupa incheierea campaniei electorale pentru alegerile prezidentiale, prin postarea cu ID \href{https://www.facebook.com/ads/library/?id=4411198482440103}{4411198482440103} pe Facebook/Instagram. Postarea, difuzata dupa ora 18:00 pe 23.11.2024, contine indemnuri directe la vot pentru candidatul Nicolae Ciuca ("il voi vota duminica"), atacuri la adresa contracandidatului Marcel Ciolacu, si promoveaza explicit un candidat la presedintie ("Nicolae Ciuca este singurul lider potrivit pentru Romania"). Postarea are scop electoral evident, fiind promovata prin plata catre un public larg (100,001-500,000 persoane estimate), incalcand astfel explicit prevederile legale privind incetarea propagandei electorale.
\end{enumerate}

\vspace{0.5cm}

\subsection{Eugen Tomac}
Următoarele fapte contravenționale sunt sesizate împotriva acestei entități:

\begin{enumerate}[leftmargin=*, label=\arabic*.)]
    \item publicarea si promovarea unui mesaj electoral dupa incheierea perioadei de campanie electorala, cu ID-ul postarii pe facebook \href{https://www.facebook.com/ads/library/?id=942804337906337}{942804337906337}. Postarea contine indemnuri directe la vot pentru candidata Elena Lasconi si impotriva candidatilor Marcel Ciolacu, Nicolae Ciuca si George Simion, avand un evident caracter de propaganda electorala demonstrat prin prezenta numarului CMF 11240022, mesaje directe de influentare a votului ("noi votam pentru un candidat de dreapta", "Un vot pentru Lasconi este o sansa pentru Dreapta!"), precum si indemnuri explicite la vot ("Iesiti la vot!"). Postarea a fost promovata dupa ora 18:00 pe 23.11.2024, atingand intre 40.000 si 45.000 de persoane, avand un impact semnificativ asupra procesului electoral.
\end{enumerate}

\vspace{0.5cm}

\subsection{FORTA DREPTEI ILFOV}
Următoarele fapte contravenționale sunt sesizate împotriva acestei entități:

\begin{enumerate}[leftmargin=*, label=\arabic*.)]
    \item continuarea propagandei electorale pentru candidatul la presedintie Elena Lasconi dupa incheierea campaniei electorale, prin postarea cu ID \href{https://www.facebook.com/ads/library/?id=2593648940822239}{2593648940822239} pe Facebook dupa ora 18:00 pe 23.11.2024. Postarea contine in mod explicit indemnul "Pe 24 noiembrie, votati Elena Lasconi Presedinte!", reprezentand propaganda electorala activa pentru alegerile prezidentiale. Postarea este sponsorizata si targetata catre un public larg (500,001-1,000,000 persoane), avand un caracter clar de propaganda electorala prin prezenta elementelor specifice: CUI, mesaje de mobilizare la vot si sustinere explicita a unui candidat la presedintie. Efectul electoral urmarit este influentarea directa a votului in favoarea candidatului Elena Lasconi in ziua alegerilor, incalcand astfel perioada de restrictie a propagandei electorale.
\end{enumerate}

\vspace{0.5cm}

\subsection{FORTA DREPTEI MURES}
Următoarele fapte contravenționale sunt sesizate împotriva acestei entități:

\begin{enumerate}[leftmargin=*, label=\arabic*.)]
    \item continuarea propagandei electorale dupa incheierea campaniei, prin postarea cu ID \href{https://www.facebook.com/ads/library/?id=1684741228923253}{1684741228923253} pe Facebook, dupa ora 18:00 pe 23.11.2024. Postarea constituie propaganda electorala deoarece contine numar CMF (11240022), face referiri directe la candidatii prezidentiali Ludovic Orban si Marcel Ciolacu, urmareste influentarea comportamentului electoral prin criticarea candidatului PSD si promovarea strategiei dreptei politice, fiind o comunicare platita ce a ajuns la minimum 15.000 de persoane. Postarea depaseste cadrul unei simple informari, avand un evident obiectiv electoral prin incercarea de a influenta votul impotriva candidatului PSD si in favoarea unui candidat al dreptei politice.
    \item continuarea propagandei electorale pentru alegerile prezidentiale dupa incheierea perioadei legale de campanie, prin postarea cu ID \href{https://www.facebook.com/ads/library/?id=932946168775192}{932946168775192} pe Facebook, dupa ora 18:00 pe 23.11.2024. Postarea constituie propaganda electorala deoarece contine numar CMF (11240022), promoveaza explicit candidati la presedintie (Elena Lasconi - pozitiv, Mircea Geoana - negativ), discuta strategii electorale si sondaje, si urmareste influentarea votului prin promovarea unei anumite optiuni electorale. Postarea este platita si are un reach semnificativ (15,000-19,999 impresii), demonstrand intent clar de a influenta corpul electoral.
\end{enumerate}

\vspace{0.5cm}

\subsection{Fluxdestiri}
Următoarele fapte contravenționale sunt sesizate împotriva acestei entități:

\begin{enumerate}[leftmargin=*, label=\arabic*.)]
    \item promovarea unei postari sponsorizate pe Facebook (ID: \href{https://www.facebook.com/ads/library/?id=1797273990811118}{1797273990811118}) dupa ora 18:00 pe 23.11.2024, care reprezinta propaganda electorala negativa impotriva candidatului Nicolae-Ionel Ciuca. Postarea, care a beneficiat de un buget semnificativ de promovare (1000-1499 RON) si a atins intre 80.000 si 90.000 de afisari, sustine in mod explicit ca "Ciuca manipuleaza sondajele de opinie", avand un evident obiectiv electoral de a influenta negativ opinia alegatorilor fata de candidat in perioada in care propaganda electorala este interzisa prin lege. Mesajul depaseste limitele unei simple opinii personale sau ale activitatii jurnalistice, fiind o actiune coordonata de comunicare electorala cu impact semnificativ.
\end{enumerate}

\vspace{0.5cm}

\subsection{Forta Dreptei}
Următoarele fapte contravenționale sunt sesizate împotriva acestei entități:

\begin{enumerate}[leftmargin=*, label=\arabic*.)]
    \item difuzarea unui mesaj electoral platit pe Facebook si Instagram (ID postare: \href{https://www.facebook.com/ads/library/?id=1232129874723325}{1232129874723325}) dupa ora 18:00 pe 23.11.2024, in care se promoveaza candidatul Elena Lasconi la presedintie prin intermediul unei declaratii de sustinere din partea fostului presedinte Traian Basescu. Postarea reprezinta propaganda electorala conform Art. 36(7) din Legea 334/2006, avand obiectiv electoral explicit, adresandu-se publicului larg prin reclama platita (reach 70.000-80.000 impresii), si facand referire directa la un candidat la presedintie. Efectul electoral urmarit este influentarea intentiei de vot in favoarea candidatului mentionat, utilizand autoritatea si notorietatea fostului presedinte pentru a convinge electoratul.
    \item promovarea unui mesaj electoral platit pe Facebook (ID: \href{https://www.facebook.com/ads/library/?id=515069744866032}{515069744866032}) dupa incheierea campaniei electorale, respectiv dupa ora 18:00 pe 23.11.2024. Postarea contine un mesaj explicit electoral care promoveaza candidatul Elena Lasconi prin intermediul unei declaratii a lui Traian Basescu, mentionand direct transferul de voturi catre aceasta din partea Fortei Dreptei. Efectul electoral este evident prin incercarea de influentare a votantilor prin promovarea unui transfer de voturi, utilizand hashtag-uri specifice alegerilor prezidentiale (\#prezidentiale) si avand o audienta estimata de peste 100.000 de persoane. Mesajul reprezinta propaganda electorala conform Art. 36(7) din Legea 334/2006, indeplinind toate criteriile: referire directa la candidat, obiectiv electoral si adresare catre publicul larg.
    \item publicarea si promovarea unei reclame platite (ID: \href{https://www.facebook.com/ads/library/?id=531701209853921}{531701209853921}) care continua propaganda electorala dupa incheierea perioadei legale de campanie pentru alegerile prezidentiale. Postarea contine un indemn explicit de vot pentru candidata Elena Lasconi ("Pe 24 noiembrie, votam Elena Lasconi"), foloseste cod mandatar electoral (11240022), si are un caracter vadit de propaganda electorala, fiind promovata dupa ora 18:00 pe 23.11.2024. Impactul electoral este direct si substantial, postarea avand o audienta estimata intre 15.000 si 20.000 de persoane, cu un buget de promovare intre 300-399 RON, reprezentand o incercare clara de influentare a votului in ziua alegerilor.
    \item difuzarea unei reclame platite pe Facebook (ID: \href{https://www.facebook.com/ads/library/?id=937805918197892}{937805918197892}) dupa ora 18:00 pe 23.11.2024, care constituie propaganda electorala pentru alegerile prezidentiale. Postarea contine un indemn direct la vot ("Votati verde, votati Forta Dreptei!"), foloseste metafore despre semafoare si Palatul Victoria pentru a influenta alegatorii, si are un obiectiv electoral clar de a promova candidatul Ludovic Orban. Mesajul a avut un impact semnificativ, atingand intre 200.000 si 250.000 de persoane, cu o investitie de aproximativ 2.250 RON, demonstrand caracterul sau de propaganda electorala sistematica si intentionata.
\end{enumerate}

\vspace{0.5cm}

\subsection{Forta Dreptei - Bacau si Marketing on Line and Business Advanced Communication - M.O.B.A.C.}
Următoarele fapte contravenționale sunt sesizate împotriva acestei entități:

\begin{enumerate}[leftmargin=*, label=\arabic*.)]
    \item promovarea unui material de propaganda electorala (ID postare Facebook: \href{https://www.facebook.com/ads/library/?id=1566318563996555}{1566318563996555}) dupa incheierea perioadei de campanie electorala pentru alegerile prezidentiale. Materialul promoveaza programul de guvernare al candidatului la presedintie Ludovic Orban, fiind distribuit ca reclama platita pe Facebook si Instagram dupa ora 18:00 pe 23.11.2024, cu un impact estimat intre 45.000 si 50.000 de persoane. Prezenta unui numar CMF (11240022) confirma natura de material electoral, iar promovarea programului de guvernare al unui candidat la presedintie in perioada de blackout electoral reprezinta o incalcare clara a prevederilor legale.
\end{enumerate}

\vspace{0.5cm}

\subsection{Forta Dreptei, prin intermediul Buyer Brain SRL}
Următoarele fapte contravenționale sunt sesizate împotriva acestei entități:

\begin{enumerate}[leftmargin=*, label=\arabic*.)]
    \item difuzarea unei reclame platite pe Facebook si Instagram (ID: \href{https://www.facebook.com/ads/library/?id=894985495946854}{894985495946854}) care continua propaganda electorala pentru alegerile prezidentiale dupa incheierea campaniei. Materialul promoveaza explicit candidatura Elenei Lasconi la presedintie, prezentand-o drept "singurul candidat de dreapta care poate sa il invinga pe Ciolacu", constituind astfel un mesaj cu clar obiectiv electoral. Postarea, difuzata dupa ora 18:00 pe 23.11.2024, reprezinta o continuare a propagandei electorale in afara perioadei permise legal, cu potential de a influenta alegatorii prin promovarea unui candidat si denigrarea altuia, atingand un public estimat intre 100.001 si 500.000 de persoane.
\end{enumerate}

\vspace{0.5cm}

\subsection{Gabriel Valer Zetea}
Următoarele fapte contravenționale sunt sesizate împotriva acestei entități:

\begin{enumerate}[leftmargin=*, label=\arabic*.)]
    \item promovarea unui material de propaganda electorala dupa incheierea perioadei de campanie, respectiv dupa ora 18:00 pe 23.11.2024, prin intermediul unei postari sponsorizate pe Facebook (ID: \href{https://www.facebook.com/ads/library/?id=1478717420197889}{1478717420197889}). Materialul promoveaza in mod explicit candidatul prezidential Marcel Ciolacu, folosind sloganuri de campanie precum "\#CaleaSigura", prezentandu-l ca "lider capabil" si indemnand indirect alegatorii sa il sustina prin fraze precum "Maramuresenii stiu care este \#CaleaSigura pentru Romania!". Desi aparent discuta despre aderarea la Schengen, postarea are un evident caracter electoral, fiind conceputa pentru a influenta optiunea de vot a alegatorilor in favoarea candidatului PSD la presedintie.
\end{enumerate}

\vspace{0.5cm}

\subsection{Ghergu Nicolae-Marius}
Următoarele fapte contravenționale sunt sesizate împotriva acestei entități:

\begin{enumerate}[leftmargin=*, label=\arabic*.)]
    \item difuzarea unui mesaj de propaganda electorala dupa incheierea perioadei de campanie, folosind o reclama platita pe Facebook (ID: \href{https://www.facebook.com/ads/library/?id=1266932291022514}{1266932291022514}) care continua sa ruleze dupa data de 23.11.2024, ora 18:00. Postarea contine numar CMF (11240060), are obiectiv electoral explicit prin indemnul "Schimba-i!" si criticarea administratiei actuale, si atinge un public larg (90.000-99.999 impresii). Mesajul are ca scop influentarea comportamentului electoral prin critica directa a actualilor politicieni si indeamna la schimbarea acestora, depasind sfera unei simple opinii personale prin natura sa de continut sponsorizat si prezenta elementelor specifice propagandei electorale.
\end{enumerate}

\vspace{0.5cm}

\subsection{I AM ONLINE SRL}
Următoarele fapte contravenționale sunt sesizate împotriva acestei entități:

\begin{enumerate}[leftmargin=*, label=\arabic*.)]
    \item promovarea unui material de propaganda electorala pentru candidatul George Nicolae Simion la alegerile prezidentiale, dupa incheierea perioadei de campanie electorala. Postarea cu ID \href{https://www.facebook.com/ads/library/?id=1324022541930513}{1324022541930513} contine un indemn explicit la vot ("votati George Simion Presedinte pentru o Romanie cinstita si unita!"), foloseste sustinerea mai multor primari pentru a influenta alegatorii, si include cod mandatar financiar (11240014), demonstrand caracterul sau oficial de propaganda electorala. Materialul este difuzat dupa ora 18:00 pe 23.11.2024, incalcand astfel prevederile legale privind perioada de campanie electorala, cu un impact semnificativ avand in vedere audienta estimata intre 30.000 si 35.000 de persoane.
    \item promovarea unui mesaj de propaganda electorala dupa incheierea perioadei de campanie, difuzat dupa ora 18:00 pe 23.11.2024 prin intermediul unei reclame platite pe Facebook si Instagram (ID: \href{https://www.facebook.com/ads/library/?id=1693913901167510}{1693913901167510}). Postarea promoveaza direct candidatura lui George Simion la presedintie ("Cu AUR la guvernare si George Simion presedinte"), face promisiuni electorale specifice privind agricultura, contine numar CMF (11240014) si foloseste hashtag-ul de campanie "\#AlbaVoteazaAUR", demonstrand astfel caracterul sau explicit de propaganda electorala pentru alegerile prezidentiale, cu intentia clara de a influenta votul in favoarea candidatului AUR.
\end{enumerate}

\vspace{0.5cm}

\subsection{Ilie Suciu si Partidul National Liberal}
Următoarele fapte contravenționale sunt sesizate împotriva acestei entități:

\begin{enumerate}[leftmargin=*, label=\arabic*.)]
    \item difuzarea unei reclame platite pe Facebook (ID: \href{https://www.facebook.com/ads/library/?id=2321738621330179}{2321738621330179}) care constituie propaganda electorala pentru alegerile prezidentiale dupa incheierea campaniei electorale. Postarea contine indemnuri directe de vot pentru candidatul CIUCA NICOLAE-IONEL la alegerile prezidentiale ("in 24 Noiembrie votam CIUCA NICOLAE-IONEL"), argumente explicite pentru sustinerea acestuia, si are un caracter vadit de propaganda electorala prin enumerarea calitatilor candidatului si comparatii cu alti candidati. Postarea are un impact estimat intre 2000-2999 de afisari si continua sa fie activa dupa ora 18:00 pe 23.11.2024, constituind astfel o incalcare clara a interdictiei de a face propaganda electorala dupa incheierea campaniei.
\end{enumerate}

\vspace{0.5cm}

\subsection{Ilie-Alin Colesa}
Următoarele fapte contravenționale sunt sesizate împotriva acestei entități:

\begin{enumerate}[leftmargin=*, label=\arabic*.)]
    \item publicarea si promovarea unei reclame platite pe Facebook/Instagram (ID: \href{https://www.facebook.com/ads/library/?id=8799652853454129}{8799652853454129}) dupa incheierea perioadei de campanie electorala pentru alegerile prezidentiale, postare ce vizeaza in mod direct candidatul Calin Georgescu, punand sub semnul intrebarii credintele sale religioase prin titlul "Calin Georgescu - un crestin sau un pagan?" si continutul ulterior. Postarea, difuzata dupa ora 18:00 pe 23.11.2024, reprezinta propaganda electorala intrucat indeplineste criteriile prevazute de lege: se refera direct la un candidat, are obiectiv electoral clar de influentare negativa a opiniei publice, se adreseaza publicului larg prin promovare platita (5000-5999 impresii), depasind cadrul unei simple opinii personale sau activitati jurnalistice.
\end{enumerate}

\vspace{0.5cm}

\subsection{InfoBraila}
Următoarele fapte contravenționale sunt sesizate împotriva acestei entități:

\begin{enumerate}[leftmargin=*, label=\arabic*.)]
    \item difuzarea unui mesaj de propaganda electorala platit (ID postare Facebook: \href{https://www.facebook.com/ads/library/?id=1058273652450836}{1058273652450836}) dupa ora 18:00 pe 23.11.2024, mesaj care il prezinta pe candidatul George Simion intr-o lumina pozitiva, ca luptator impotriva coruptiei. Postarea, care a beneficiat de promovare platita cu suma intre 200-299 RON si a ajuns la 30.000-35.000 de persoane, reprezinta o forma clara de propaganda electorala, avand obiectiv electoral explicit si adresandu-se publicului larg prin intermediul platformelor Facebook si Instagram, incercand sa influenteze perceptia publica despre candidatul la presedintie George Simion intr-un mod favorabil acestuia.
\end{enumerate}

\vspace{0.5cm}

\subsection{Infomed Pro SRL}
Următoarele fapte contravenționale sunt sesizate împotriva acestei entități:

\begin{enumerate}[leftmargin=*, label=\arabic*.)]
    \item difuzarea unei reclame platite pe Facebook si Instagram (ID: \href{https://www.facebook.com/ads/library/?id=1028956809243157}{1028956809243157}) dupa ora 18:00 pe 23.11.2024, care continua propaganda electorala pentru candidatul la presedintie Marcel Ciolacu. Postarea contine un indemn direct la vot pentru alegerile prezidentiale ("Duminica, 24 noiembrie, voteaza Marcel Ciolacu - presedinte!"), foloseste un cod CMF (11240017), si are un obiectiv electoral explicit, atingand un public estimat intre 100,001-500,000 de persoane. Materialul depaseste limitele comunicarii permise in aceasta perioada si reprezinta o continuare clara a propagandei electorale dupa incheierea campaniei pentru alegerile prezidentiale.
    \item difuzarea unei reclame platite pe Facebook si Instagram (ID: \href{https://www.facebook.com/ads/library/?id=1083249396324223}{1083249396324223}) dupa ora 18:00 pe 23.11.2024, continand propaganda electorala explicita pentru alegerile prezidentiale. Postarea contine indemnul direct "Duminica aceasta, votati Marcel Ciolacu presedinte", reprezinta propaganda electorala conform art. 36(7) prin prezenta numarului CMF 11240017, targetarea unui public larg prin reclama platita, si obiectivul electoral explicit de influentare a votului pentru candidatul Ion-Marcel Ciolacu la alegerile prezidentiale. Efectul electoral urmarit este cresterea intentiei de vot pentru candidatul PSD la presedintie, intr-o perioada in care campania electorala prezidentiala este incheiata legal.
    \item difuzarea de materiale de propaganda electorala pentru candidatul prezidential Marcel Ciolacu dupa incheierea perioadei de campanie electorala. Postarea cu ID \href{https://www.facebook.com/ads/library/?id=1087041495976896}{1087041495976896}, publicata dupa ora 18:00 pe 23.11.2024, promoveaza explicit "Planul de tara al presedintelui Marcel Ciolacu", foloseste numarul CMF 11240017, si are ca scop influentarea intentiei de vot prin prezentarea programului electoral si prin mesaje de mobilizare. Postarea platita pe Facebook si Instagram are un impact semnificativ, atingand intre 1000 si 1999 de persoane, reprezentand astfel o continuare clara a propagandei electorale in afara perioadei permise de lege.
    \item distribuirea unui material de propaganda electorala (ID postare Facebook: \href{https://www.facebook.com/ads/library/?id=1093387848852522}{1093387848852522}) dupa incheierea perioadei de propaganda pentru alegerile prezidentiale. Postarea, care include numarul CMF 11240017, promoveaza explicit realizarile candidatului prezidential Marcel Ciolacu, evidentiind masurile economice implementate sub conducerea sa ca premier, cu scopul clar de a influenta preferintele electoratului. Materialul, difuzat ca reclama platita pe Facebook si Instagram, vizeaza un public larg (100,001-500,000 persoane) si prezinta realizari concrete privind cresterea salariilor, cu intentia evidenta de a influenta optiunile de vot. Mentionam ca aceasta activitate continua dupa ora 18:00 din data de 23.11.2024, incalcand astfel prevederile legale privind perioada de campanie electorala.
    \item promovarea unui material de propaganda electorala pentru alegerile prezidentiale (ID postare Facebook: \href{https://www.facebook.com/ads/library/?id=1094718548873755}{1094718548873755}) dupa incheierea perioadei de campanie electorala (dupa ora 18:00 pe 23.11.2024). Materialul contine indemnuri directe de vot pentru candidatul Marcel Ciolacu la functia de presedinte ("votati Marcel Ciolacu presedinte pe 24 noiembrie"), fiind distribuit ca reclama platita pe Facebook si Instagram, cu identificator de campanie CMF 11240017, targetand un public de peste 100.000 de persoane. Efectul electoral este evident prin solicitarea explicita de vot pentru un candidat la functia de presedinte, depasind cadrul legal permis pentru perioada post-campanie.
    \item difuzarea de materiale de propaganda electorala pentru candidatul prezidential Marcel Ciolacu dupa incheierea perioadei de campanie electorala, prin intermediul unei reclame platite pe Facebook (ID: \href{https://www.facebook.com/ads/library/?id=1109886473308382}{1109886473308382}) difuzata dupa ora 18:00 pe 23.11.2024. Postarea contine un indemn direct la vot pentru candidatul prezidential Marcel Ciolacu ("Un vot pentru Marcel Ciolacu la alegerile prezidentiale"), foloseste numarul de inregistrare CMF11240017, si are ca scop influentarea intentiei de vot prin asocierea candidatului cu realizarile partidului si ale unor membri proeminenti, precum Roxana Minzatu. Efectul electoral urmarit este clar orientat spre obtinerea de voturi pentru candidatul PSD la alegerile prezidentiale.
    \item difuzarea de materiale de propaganda electorala dupa incheierea perioadei de campanie, concretizata prin postarea cu ID \href{https://www.facebook.com/ads/library/?id=1111404077659539}{1111404077659539} pe Facebook. Postarea, activa dupa ora 18:00 pe 23.11.2024, contine referinte directe la candidatul prezidential Marcel Ciolacu, foloseste numarul oficial de campanie CMF11240017, si promoveaza explicit mesaje electorale prin formulari precum "ne dorim sa continuam buna guvernare" si "PSD Bihor isi mentine angajamentul de a lupta alaturi de presedintele Marcel Ciolacu". Efectul electoral este evident prin promovarea pozitiva a candidatului si partidului sau in contextul alegerilor prezidentiale, fiind o reclama platita cu audienta estimata intre 100,001 si 500,000 de persoane.
    \item difuzarea de mesaje de propaganda electorala dupa incheierea perioadei legale de campanie, in data de 23.11.2024 dupa ora 18:00, prin postarea cu ID \href{https://www.facebook.com/ads/library/?id=1117507146617421}{1117507146617421} pe platformele Facebook si Instagram. Postarea contine indemnuri directe la vot pentru candidatul Marcel Ciolacu, folosind expresii precum "votam cu domnul presedinte Marcel Ciolacu" si "cu mic, cu mare, la votare!", avand un evident caracter de propaganda electorala si potential de influentare a votului pentru alegerile prezidentiale. Mesajul a ajuns la un numar semnificativ de persoane (45,000-49,999 impresii), fiind sponsorizat cu sume intre 200-299 RON.
    \item continuarea propagandei electorale pentru candidatul prezidential Marcel Ciolacu dupa incheierea perioadei legale, prin intermediul unei postari sponsorizate pe Facebook (ID: \href{https://www.facebook.com/ads/library/?id=1120710852905228}{1120710852905228}) ce promoveaza explicit candidatul la alegerile prezidentiale si indeamna la vot pentru PSD la alegerile prezidentiale. Postarea, facuta dupa ora 18:00 pe 23.11.2024, contine elemente clare de propaganda electorala, inclusiv numar CMF (11240017), slogan de campanie, si mesaje directe de sustinere electorala, mentionand explicit "alegerile prezidentiale" si promovand programul candidatului, avand un impact potential asupra unui numar mare de alegatori (intre 100,001-500,000 persoane) prin intermediul platformelor Facebook si Instagram.
    \item difuzarea de materiale de propaganda electorala pentru candidatul Ion-Marcel Ciolacu dupa incheierea perioadei de campanie electorala, prin intermediul unei reclame platite pe Facebook (ID: \href{https://www.facebook.com/ads/library/?id=1120850192933191}{1120850192933191}) difuzata dupa ora 18:00 pe 23.11.2024. Postarea contine un indemn direct la vot ("Voteaza Marcel Ciolacu, calea sigura pentru Romania!"), promoveaza realizarile candidatului in calitate de prim-ministru (aderarea la Schengen) si are un evident caracter de propaganda electorala, fiind o comunicare platita care vizeaza influentarea votului in favoarea candidatului PSD la alegerile prezidentiale.
    \item difuzarea de materiale de propaganda electorala pentru alegerile prezidentiale dupa incheierea perioadei de campanie, prin postarea cu ID \href{https://www.facebook.com/ads/library/?id=1126375062244385}{1126375062244385} pe Facebook. Postarea contine un indemn explicit de vot pentru candidatul Marcel Ciolacu la functia de presedinte ("Votati Marcel Ciolacu, presedinte!"), prezinta activitati de campanie electorala si este promovata ca reclama platita pe Facebook si Instagram dupa ora 18:00 pe 23.11.2024. Materialul este identificat ca propaganda electorala prin prezenta codului CMF 11240017, are obiectiv electoral explicit si se adreseaza publicului larg prin intermediul platformelor de social media, cu un impact estimat intre 15.000 si 19.999 de afisari.
    \item continuarea propagandei electorale pentru alegerile prezidentiale dupa incheierea campaniei electorale, prin postarea cu ID \href{https://www.facebook.com/ads/library/?id=1283780659398512}{1283780659398512} pe Facebook. Postarea, activa dupa ora 18:00 pe 23.11.2024, contine un indemn direct la vot pentru candidatul prezidential Marcel Ciolacu ("votati Marcel Ciolacu presedinte"), are numar CMF 11240017 care confirma natura sa de material de propaganda electorala, si este o reclama platita ce ajunge la un public de peste 100.000 de persoane. Mesajul are un clar obiectiv electoral, facand promisiuni electorale si solicitand explicit votul pentru un candidat la alegerile prezidentiale, reprezentand astfel continuarea propagandei electorale dupa incheierea perioadei legale de campanie.
    \item difuzarea unui mesaj de propaganda electorala pentru alegerile prezidentiale dupa incheierea perioadei de campanie, concretizat prin postarea cu ID \href{https://www.facebook.com/ads/library/?id=1306321060785441}{1306321060785441} pe Facebook. Postarea contine un indemn explicit la vot pentru candidatul Marcel Ciolacu la alegerile prezidentiale ("voteaza Marcel Ciolacu la alegerile prezidentiale"), fiind o reclama platita cu impact semnificativ (audienta estimata 100,001-500,000 persoane), difuzata dupa ora 18:00 pe 23.11.2024. Postarea include numarul CMF 11240017, confirmand natura sa de material de propaganda electorala, si are ca scop influentarea directa a votului pentru alegerile prezidentiale prin prezentarea candidatului ca "calea sigura pentru Romania".
    \item continuarea propagandei electorale dupa incheierea campaniei electorale pentru alegerile prezidentiale, prin intermediul unei reclame platite pe Facebook (ID: \href{https://www.facebook.com/ads/library/?id=1317717352743487}{1317717352743487}) care promoveaza explicit candidatura lui Marcel Ciolacu la presedintie. Postarea, care este activa dupa ora 18:00 pe 23.11.2024, contine mesaje directe de propaganda electorala precum "De ce il voteaza romanii pe Marcel Ciolacu ca presedinte al Romaniei" si "Marcel Ciolacu, calea sigura pentru Romania!", avand un evident obiectiv electoral de influentare a votului in favoarea candidatului PSD la presedintie, prin enumerarea motivelor pentru care ar trebui votat si prezentarea promisiunilor electorale.
    \item continuarea propagandei electorale pentru alegerile prezidentiale dupa incheierea perioadei legale de campanie, prin postarea cu ID \href{https://www.facebook.com/ads/library/?id=1332526384577298}{1332526384577298}. Postarea contine un indemn explicit de a vota candidatul Marcel Ciolacu la alegerile prezidentiale din 24 noiembrie ("votati Marcel Ciolacu presedinte pe 24 noiembrie"), fiind o reclama platita care a continuat sa ruleze dupa ora 18:00 pe 23.11.2024. Postarea include numar CMF (11240017), confirmand natura sa de propaganda electorala, si are un efect electoral direct prin indemnul explicit la vot pentru un candidat specific la alegerile prezidentiale.
    \item promovarea unui mesaj de propaganda electorala pentru candidatul prezidential Marcel Ciolacu dupa incheierea perioadei de campanie electorala. Postarea cu ID-ul \href{https://www.facebook.com/ads/library/?id=1354493685519333}{1354493685519333} contine un indemn direct la vot pentru alegerile prezidentiale ("Pe 24 noiembrie, voteaza Marcel Ciolacu la alegerile prezidentiale"), fiind publicata si promovata ca reclama platita pe Facebook si Instagram dupa ora 18:00 pe 23.11.2024. Mesajul are caracter electoral explicit, fiind insotit de cod CMF 11240017, si vizeaza influentarea directa a votului pentru alegerile prezidentiale prin solicitarea explicita de a vota un anumit candidat.
    \item continuarea propagandei electorale dupa incheierea campaniei electorale pentru alegerile prezidentiale, prin promovarea unui material electoral platit pe Facebook si Instagram (ID postare: \href{https://www.facebook.com/ads/library/?id=1481881965839933}{1481881965839933}) care prezinta explicit programul de guvernare si propunerile candidatului prezidential Marcel Ciolacu, folosind elemente specifice de campanie (CMF 11240017) si avand ca obiectiv influentarea votului in Bihor. Postarea, care este activa si dupa ora 18:00 pe 23.11.2024, reprezinta o continuare directa a activitatii de propaganda electorala, fiind adaptata specific pentru a influenta alegatorii din judetul Bihor in favoarea candidatului PSD la presedintie.
    \item publicarea si promovarea unei reclame platite pe Facebook (ID: \href{https://www.facebook.com/ads/library/?id=1501110307184723}{1501110307184723}) dupa ora 18:00 pe 23.11.2024, care continua propaganda electorala pentru alegerile prezidentiale dupa incheierea campaniei. Postarea contine un indemn direct de vot pentru candidatul Marcel Ciolacu la functia de presedinte ("votand Marcel Ciolacu presedinte in aceasta duminica"), fiind utilizata pentru influentarea alegatorilor in favoarea unui candidat la alegerile prezidentiale. Materialul include numar CMF (11240017), confirma intentia de propaganda electorala si are un impact semnificativ prin distribuirea platita pe platformele Facebook si Instagram, cu o audienta estimata intre 100,001 si 500,000 de persoane.
    \item difuzarea de materiale de propaganda electorala dupa incheierea perioadei de campanie, pentru candidatul ION-MARCEL CIOLACU la alegerile prezidentiale. Postarea cu ID \href{https://www.facebook.com/ads/library/?id=1520096789382440}{1520096789382440} contine indemnuri directe la vot ("votam cu domnul presedinte Marcel Ciolacu") si este inca activa dupa ora 18:00 pe 23.11.2024. Mesajul are caracter electoral explicit, fiind distribuit ca reclama platita pe Facebook si Instagram, cu un impact semnificativ (50,000-59,999 impresii), reprezentand astfel o forma clara de propaganda electorala care continua dupa incheierea campaniei electorale oficiale.
    \item difuzarea de propaganda electorala pentru alegerile prezidentiale dupa incheierea perioadei de campanie, prin postarea cu ID \href{https://www.facebook.com/ads/library/?id=1559839654674397}{1559839654674397} pe Facebook. Postarea contine un indemn direct de vot pentru candidatul Marcel Ciolacu la functia de presedinte ("votati Marcel Ciolacu presedinte in aceasta duminica"), fiind promovata ca reclama platita pe Facebook si Instagram dupa ora 18:00 pe 23.11.2024, cu un impact estimat intre 100.001 si 500.000 de persoane. Postarea include numarul CMF 11240017, confirmand natura sa de propaganda electorala, si are un obiectiv electoral explicit de influentare a votului pentru alegerile prezidentiale.
    \item difuzarea unei reclame platite pe Facebook (ID: \href{https://www.facebook.com/ads/library/?id=1563753904343301}{1563753904343301}) care continua propaganda electorala pentru candidatul prezidential Marcel Ciolacu dupa incheierea perioadei legale de campanie. Postarea contine un indemn direct de vot ("Votati Marcel Ciolacu, presedinte!"), descriere de activitati de campanie electorala, si foloseste un cod CMF (11240017), demonstrand caracterul sau de material de propaganda electorala. Postarea este activa si dupa ora 18:00 pe 23.11.2024, incalcand astfel prevederile legale privind perioada de campanie electorala pentru alegerile prezidentiale.
    \item difuzarea unui material de propaganda electorala dupa incheierea perioadei de campanie, prin intermediul unei reclame platite pe Facebook (ID: \href{https://www.facebook.com/ads/library/?id=1610696299520643}{1610696299520643}) care promoveaza candidatul PSD Marcel Ciolacu si denigreaza candidatul PNL Nicolae Ciuca. Materialul, publicat dupa ora 18:00 pe 23.11.2024, contine numar CMF (11240017) si are un caracter evident de propaganda electorala, criticand explicit activitatea lui Nicolae Ciuca si laudand realizarile lui Marcel Ciolacu, cu scopul clar de a influenta preferintele electorale ale votantilor in perioada de interdictie a propagandei electorale pentru alegerile prezidentiale.
    \item continuarea propagandei electorale dupa incheierea campaniei pentru alegerile prezidentiale, prin promovarea unui material electoral platit pe Facebook si Instagram (ID: \href{https://www.facebook.com/ads/library/?id=1696543720906867}{1696543720906867}) care il mentioneaza direct pe candidatul prezidential Marcel Ciolacu, promovand realizarile si promisiunile acestuia, folosind hashtag-ul de campanie "\#caleasigurapentruromania" si codul CMF11240017, avand un evident scop electoral prin prezentarea masurilor guvernarii si a promisiunilor viitoare, targetand un public larg estimat intre 100.001 si 500.000 de persoane, dupa ora 18:00 pe 23.11.2024. Materialul depaseste simpla informare si constituie propaganda electorala prin natura sa promotionala si scopul sau electoral explicit.
    \item publicarea si promovarea unei reclame platite pe Facebook (ID: \href{https://www.facebook.com/ads/library/?id=1702946183611560}{1702946183611560}) dupa ora 18:00 pe 23.11.2024, care contine propaganda electorala pentru candidatul prezidential Marcel Ciolacu si PSD. Postarea include numar CMF (11240017), face promisiuni electorale directe privind cresterea salariilor si dezvoltarea economica, si promoveaza explicit candidatul la presedintie Marcel Ciolacu, fiind astfel o continuare clara a propagandei electorale dupa incheierea campaniei. Efectul electoral urmarit este influentarea directa a votantilor prin promisiuni economice si prezentarea candidatului ca garantul acestor promisiuni.
    \item difuzarea unui mesaj de propaganda electorala pentru alegerile prezidentiale dupa incheierea perioadei de campanie, respectiv dupa ora 18:00 pe 23.11.2024. Postarea cu ID-ul \href{https://www.facebook.com/ads/library/?id=1742822496636862}{1742822496636862} contine un indemn explicit la vot pentru candidatul prezidential Marcel Ciolacu ("Pe 24 noiembrie, votati Marcel Ciolacu pentru un viitor sigur!"), fiind o reclama platita pe Facebook si Instagram cu impact potential intre 100,001 si 500,000 de persoane. Postarea include numar CMF (11240017), demonstrand natura sa de propaganda electorala, si are ca scop clar influentarea votului pentru alegerile prezidentiale din 24 noiembrie 2024.
    \item promovarea unui mesaj de propaganda electorala pentru alegerile prezidentiale dupa incheierea perioadei de campanie, la data de 23.11.2024 dupa ora 18:00. Postarea cu ID-ul \href{https://www.facebook.com/ads/library/?id=1938529673225729}{1938529673225729} promoveaza explicit candidatul ION-MARCEL CIOLACU la alegerile prezidentiale, contine numar CMF 11240017, si mentioneza explicit sustinerea pentru alegerile prezidentiale ("asigurandu-ne de sustinerea lor in alegerile prezidentiale"). Postarea este sponsorizata si are un impact semnificativ, atingand intre 15.000 si 19.999 de persoane, folosind slogan-ul de campanie "PSD, Calea Sigura pentru Romania!" si promovand explicit candidatul la presedintie Marcel Ciolacu.
    \item difuzarea de materiale de propaganda electorala dupa incheierea perioadei de campanie, respectiv dupa ora 18:00 pe 23.11.2024. Postarea cu ID-ul \href{https://www.facebook.com/ads/library/?id=2094189494329421}{2094189494329421} promoveaza direct candidatul prezidential Marcel Ciolacu, folosind materiale de campanie oficiale (cod CMF11240017), prezentand masuri guvernamentale si promisiuni electorale sub sloganul "\#caleasigurapentruromania". Postarea are caracter electoral evident, fiind o reclama platita ce vizeaza influentarea votului prin promovarea realizarilor guvernamentale si a promisiunilor electorale ale candidatului Marcel Ciolacu.
    \item promovarea platita pe Facebook si Instagram a unui mesaj de propaganda electorala (ID postare: \href{https://www.facebook.com/ads/library/?id=2312255482440591}{2312255482440591}) dupa incheierea perioadei de campanie electorala pentru alegerile prezidentiale (dupa ora 18:00 pe 23.11.2024). Postarea contine numar CMF (CMF11240017), promoveaza explicit realizarile guvernului condus de candidatul prezidential Marcel Ciolacu, foloseste hashtag-ul de campanie \#caleasigurapentruromania, si prezinta activitati de campanie electorala, cu un impact pozitiv direct asupra imaginii candidatului. Mesajul a avut o audienta semnificativa, atingand intre 15.000 si 19.999 de afisari, cu un buget de promovare intre 200 si 299 RON.
    \item continuarea propagandei electorale dupa incheierea campaniei, prin intermediul unei postari sponsorizate pe Facebook (ID: \href{https://www.facebook.com/ads/library/?id=376470698883242}{376470698883242}) care promoveaza candidatul ION-MARCEL CIOLACU la presedintie. Postarea, desi aparent informativa despre masuri guvernamentale, include elemente clare de propaganda electorala precum hashtag-ul "\#CiolacuPresedinte" si numarul CMF 11240017, avand un efect electoral direct prin asocierea masurilor sociale cu candidatul la presedintie. Postarea continua sa fie activa si dupa ora 18:00 pe 23.11.2024, incalcand astfel perioada legala de campanie electorala.
    \item promovarea unui material de propaganda electorala (ID postare Facebook: \href{https://www.facebook.com/ads/library/?id=3790873984507071}{3790873984507071}) dupa ora 18:00 pe 23.11.2024, material care promoveaza realizarile guvernamentale ale candidatului Ion-Marcel Ciolacu si foloseste hashtag-ul de campanie "\#caleasigurapentruromania". Postarea include numar CMF (11240017), are caracter electoral explicit prin promovarea realizarilor guvernamentale ale candidatului, mentionand programele StartUp si AFIR ca "masuri bune ale guvernului condus de Marcel Ciolacu", si este distribuita ca reclama platita pe Facebook si Instagram, cu un impact estimat intre 10.000-14.999 de afisari, demonstrand intentia clara de a influenta votul in favoarea candidatului PSD la alegerile prezidentiale.
    \item promovarea unui material de propaganda electorala (ID: \href{https://www.facebook.com/ads/library/?id=386986527742462}{386986527742462}) dupa incheierea perioadei de campanie electorala pentru alegerile prezidentiale, respectiv dupa ora 18:00 pe 23.11.2024. Materialul contine indemnuri directe la vot pentru candidatul prezidential Marcel Ciolacu ("Sustine-ne duminica aceasta, votand Marcel Ciolacu presedinte!"), fiind distribuit ca reclama platita pe Facebook si Instagram, cu un impact semnificativ (10.000-14.999 afisari). Materialul este marcat oficial ca propaganda electorala prin numarul CMF 11240017, confirmand astfel natura sa de comunicare electorala, si are ca obiectiv clar influentarea votului pentru alegerile prezidentiale prin promovarea activa a unui candidat specific.
    \item difuzarea de materiale de propaganda electorala pentru candidatul la presedintie Marcel Ciolacu dupa incheierea campaniei electorale. Postarea cu ID \href{https://www.facebook.com/ads/library/?id=3986809324924481}{3986809324924481}, publicata dupa ora 18:00 pe 23.11.2024, contine indemnuri directe la vot ("Duminica aceasta, votati Marcel Ciolacu presedinte"), promisiuni electorale privind pensiile (cresterea la 810 euro) si foloseste numarul oficial de campanie CMF 11240017, demonstrand caracterul sau de propaganda electorala. Postarea are efect electoral direct, fiind promovata activ prin reclame platite pe Facebook si Instagram, cu un impact estimat intre 6000-7000 de afisari, reprezentand o incercare clara de influentare a votului pentru alegerile prezidentiale in perioada in care propaganda electorala este interzisa prin lege.
    \item publicarea si promovarea unui mesaj de propaganda electorala (ID Facebook: \href{https://www.facebook.com/ads/library/?id=419482617925440}{419482617925440}) dupa incheierea perioadei de campanie electorala pentru alegerile prezidentiale. Postarea promoveaza explicit candidatul Marcel Ciolacu pentru functia de Presedinte al Romaniei, folosind expresii precum "Marcel Ciolacu va fi acel presedinte!" si "calea sigura pentru Romania", cu scopul clar de a influenta comportamentul electoral al alegatorilor. Mesajul este distribuit ca reclama platita pe Facebook si Instagram, vizand un public larg (100,001-500,000 persoane), dupa ora 18:00 pe 23.11.2024, cu intentie clara de a influenta decizia de vot a cetatenilor in perioada in care propaganda electorala este interzisa prin lege.
    \item difuzarea de materiale de propaganda electorala dupa incheierea perioadei de campanie, prin promovarea unui material electoral platit pe Facebook si Instagram (ID: \href{https://www.facebook.com/ads/library/?id=448675854942347}{448675854942347}) care il promoveaza pe candidatul prezidential Marcel Ciolacu, continand numar CMF (11240017), folosind sloganuri de campanie si facand promisiuni electorale despre cresterea veniturilor. Materialul are un evident caracter de propaganda electorala, fiind difuzat dupa ora 18:00 pe 23.11.2024, folosind hashtag-ul de campanie "\#caleasigurapentruromania" si promovand masurile guvernului condus de candidatul prezidential, cu scopul clar de a influenta alegatorii in favoarea acestuia.
    \item continuarea propagandei electorale dupa incheierea campaniei electorale pentru alegerile prezidentiale, prin publicarea si promovarea unei reclame platite pe Facebook (ID: \href{https://www.facebook.com/ads/library/?id=539951575522072}{539951575522072}) care indeamna in mod explicit la votarea candidatului Marcel Ciolacu in data de 24 noiembrie. Postarea contine un indemn direct la vot ("va asteptam sa-l votati pe presedintele Marcel Ciolacu"), fiind publicata si promovata dupa ora 18:00 pe 23.11.2024, cu un impact estimat intre 1000-1999 de afisari. Materialul include numar CMF (11240017), confirmand natura sa de propaganda electorala, si are un obiectiv electoral clar de influentare a votului pentru alegerile prezidentiale.
    \item continuarea propagandei electorale dupa incheierea campaniei electorale pentru alegerile prezidentiale, prin promovarea unui material electoral platit pe Facebook si Instagram (ID: \href{https://www.facebook.com/ads/library/?id=539970938841239}{539970938841239}) care il promoveaza pe candidatul prezidential Marcel Ciolacu, folosind numarul CMF11240017, continand mesaje electorale explicite precum "Vino la vot si sustine ajutorul oferit tinerilor de catre Partidul Social Democrat" si promovand programul electoral al candidatului. Materialul a fost difuzat dupa ora 18:00 pe 23.11.2024, incalcand astfel perioada legala de campanie electorala si avand ca efect electoral influentarea directa a votantilor in favoarea candidatului PSD la presedintie.
    \item difuzarea de materiale de propaganda electorala pentru candidatul prezidential Marcel Ciolacu dupa incheierea perioadei de campanie electorala. Postarea cu ID \href{https://www.facebook.com/ads/library/?id=540099652150122}{540099652150122} promoveaza in mod activ programul de guvernare si propunerile lui Marcel Ciolacu ca presedinte, folosind hashtag-ul de campanie \#caleasigurapentrubihor si continand numarul CMF 11240017, specific materialelor de campanie electorala. Efectul electoral al postarii este evident prin promovarea directa a candidatului si a programului sau prezidential, mesajul fiind difuzat ca reclama platita pe Facebook si Instagram dupa ora 18:00 pe 23.11.2024, atingand intre 5000-6000 de utilizatori.
    \item difuzarea unei reclame platite pe Facebook (ID: \href{https://www.facebook.com/ads/library/?id=576695268147731}{576695268147731}) dupa incheierea perioadei de campanie electorala, la data de 23.11.2024 dupa ora 18:00. Postarea contine propaganda electorala evidenta, incluzand numar CMF 11240017, promoveaza direct candidatul ION-MARCEL CIOLACU si programul sau prezidential, in timp ce denigreaza candidatul NICOLAE-IONEL CIUCA si PNL. Efectul electoral este evident prin prezentarea masurilor concrete ale candidatului PSD si criticarea directa a contracandidatilor, cu scopul clar de a influenta preferintele de vot ale publicului larg in perioada prohibita.
    \item difuzarea de materiale de propaganda electorala pentru alegerile prezidentiale dupa incheierea perioadei de campanie, prin postarea cu ID \href{https://www.facebook.com/ads/library/?id=579459671127548}{579459671127548} pe Facebook. Postarea, difuzata dupa ora 18:00 pe 23.11.2024, contine un indemn explicit de vot pentru candidatul Marcel Ciolacu la functia de presedinte ("votati Marcel Ciolacu presedinte"), constituind propaganda electorala conform art. 36(7) din Legea 334/2006, fiind o comunicare cu caracter electoral, adresata unui public larg prin intermediul unei reclame platite pe Facebook, care depaseste limitele activitatii jurnalistice si care face referire directa la un candidat la alegerile prezidentiale.
    \item promovarea unui mesaj de propaganda electorala pentru alegerile prezidentiale dupa incheierea perioadei legale, prin postarea cu ID \href{https://www.facebook.com/ads/library/?id=595398042978959}{595398042978959} pe Facebook dupa ora 18:00 pe 23.11.2024. Mesajul "castigam primul tur al alegerilor prezidentiale" reprezinta propaganda electorala explicita pentru candidatul PSD la prezidentiale, fiind o reclama platita care a ajuns la 4000-4999 de persoane. Postarea include numarul CMF 11240017, confirmand natura sa de material electoral, si are un obiectiv electoral clar de influentare a votului pentru alegerile prezidentiale, incalcand astfel prevederile legale privind perioada de campanie electorala.
    \item publicarea si promovarea unei reclame platite pe Facebook si Instagram (ID: \href{https://www.facebook.com/ads/library/?id=599690146053145}{599690146053145}) care continua propaganda electorala pentru candidatul prezidential Marcel Ciolacu dupa incheierea perioadei legale de campanie. Postarea contine un indemn direct la vot ("votati Marcel Ciolacu - presedinte!"), foloseste un numar CMF (11240017), si are caracter vadit propagandistic electoral, fiind promovata activ dupa ora 18:00 pe 23.11.2024. Impactul electoral este evident prin atingerea unui public de 2000-2999 persoane prin promovare platita, reprezentand o incercare clara de influentare a votului in perioada in care propaganda electorala este interzisa prin lege.
    \item continuarea difuzarii de materiale de propaganda electorala (ID postare Facebook: \href{https://www.facebook.com/ads/library/?id=792635229631100}{792635229631100}) dupa incheierea perioadei de campanie electorala. Materialul promovat contine numar CMF11240017, promoveaza realizarile guvernamentale ale candidatului ION-MARCEL CIOLACU, foloseste hashtag-ul de campanie "\#caleasigurapentruromania" si continua sa fie difuzat ca reclama platita pe Facebook si Instagram dupa ora 18:00 pe 23.11.2024, targetand un public de 25.000-29.999 persoane, cu un buget de 200-299 RON. Efectul electoral este evident prin asocierea pozitiva a candidatului cu reusitele programelor guvernamentale si folosirea elementelor specifice de campanie.
    \item publicarea si promovarea unei reclame electorale (ID Facebook: \href{https://www.facebook.com/ads/library/?id=862107419116899}{862107419116899}) dupa incheierea perioadei de campanie electorala pentru alegerile prezidentiale. Postarea contine un indemn explicit de vot pentru candidatul prezidential Marcel Ciolacu ("Pe 24 noiembrie, votati Marcel Ciolacu pentru un viitor sigur!"), fiind publicata si promovata dupa ora 18:00 pe 23.11.2024. Materialul include numar CMF (11240017), foloseste mesaje electorale clare si este distribuit ca reclama platita pe Facebook si Instagram, avand ca scop influentarea votului pentru alegerile prezidentiale intr-o perioada in care propaganda electorala este interzisa prin lege.
    \item continuarea propagandei electorale pentru alegerile prezidentiale dupa incheierea perioadei legale de campanie, prin promovarea unei reclame platite pe Facebook (ID: \href{https://www.facebook.com/ads/library/?id=891258106323887}{891258106323887}) care indeamna in mod explicit la votarea candidatului Marcel Ciolacu la alegerile prezidentiale din 24 noiembrie ("va asteptam sa-l votati pe presedintele Marcel Ciolacu"). Postarea include numar CMF (11240017), are caracter electoral explicit prin indemnul direct la vot, si este promovata dupa ora 18:00 pe 23.11.2024, incalcand astfel prevederile legale privind perioada de campanie electorala. Efectul electoral urmarit este influentarea directa a votantilor in favoarea candidatului PSD la presedintie, prin utilizarea resurselor financiare pentru promovarea mesajului electoral in afara perioadei legale de campanie.
    \item publicarea si promovarea unei reclame platite pe Facebook si Instagram (ID: \href{https://www.facebook.com/ads/library/?id=925257992403519}{925257992403519}) care constituie propaganda electorala pentru alegerile prezidentiale dupa incheierea perioadei legale de campanie. Postarea face referire directa la castigarea primului tur al alegerilor prezidentiale ("castigam primul tur al alegerilor prezidentiale"), foloseste numar CMF (11240017), si promoveaza explicit PSD in contextul alegerilor prezidentiale, avand un efect electoral direct prin incercarea de influentare a votului in favoarea candidatului PSD. Aceasta activitate continua si dupa ora 18:00 pe 23.11.2024, incalcand astfel prevederile legale privind incheierea campaniei electorale.
    \item promovarea unui mesaj de propaganda electorala (ID postare Facebook: \href{https://www.facebook.com/ads/library/?id=925418612429053}{925418612429053}) dupa incheierea perioadei de propaganda pentru alegerile prezidentiale. Postarea, care include numarul CMF11240017, promoveaza explicit candidatul Marcel Ciolacu si programul sau de guvernare, folosind hashtag-ul de campanie "\#caleasigurapentruromania". Mesajul are caracter electoral evident, promovand masurile si propunerile ce urmeaza a fi implementate "alaturi de presedintele Marcel Ciolacu", influentand astfel intentiile de vot ale publicului tinta (4000-4999 impresii) dupa ora 18:00 pe 23.11.2024.
    \item difuzarea unui material de propaganda electorala (ID Facebook: \href{https://www.facebook.com/ads/library/?id=937024215153673}{937024215153673}) dupa incheierea perioadei de campanie electorala, respectiv dupa ora 18:00 pe 23.11.2024. Materialul promovat contine un indemn explicit la vot pentru candidatul Marcel Ciolacu la alegerile prezidentiale din 24 noiembrie, folosind formulari precum "va asteptam sa-l votati pe presedintele Marcel Ciolacu" si include numar CMF 11240017, confirmand natura sa de material de propaganda electorala. Postarea este promovata activ prin instrumente de advertising pe Facebook si Instagram, targetand populatia din judetul Bihor, avand un impact estimat intre 2000 si 2999 de afisari, reprezentand astfel o incalcare clara a interdictiei de a continua propaganda electorala dupa incheierea campaniei.
\end{enumerate}

\vspace{0.5cm}

\subsection{Innen Analitiq}
Următoarele fapte contravenționale sunt sesizate împotriva acestei entități:

\begin{enumerate}[leftmargin=*, label=\arabic*.)]
    \item continuarea propagandei electorale dupa incheierea campaniei, prin intermediul unei reclame platite pe Facebook (ID: \href{https://www.facebook.com/ads/library/?id=1091320795959890}{1091320795959890}) care promoveaza explicit candidatul Nicolae Ciuca pentru functia de presedinte al Romaniei. Postarea, creata pe 22 noiembrie si inca activa dupa ora 18:00 pe 23.11.2024, contine numar CMF (11240002) si promoveaza direct candidatul cu mesajul "Nicolae Ciuca este alegerea Olteniei pentru Presedintia Romaniei", avand un evident caracter de propaganda electorala prin modul de prezentare, audienta tintita (500,001-1,000,000 persoane) si obiectivul electoral clar de influentare a votului in favoarea candidatului PNL la presedintie.
    \item promovarea unui mesaj de propaganda electorala pentru candidatul Nicolae-Ionel Ciuca dupa incheierea perioadei de campanie electorala, prin intermediul unei reclame platite pe Facebook si Instagram (ID: \href{https://www.facebook.com/ads/library/?id=3768559450053507}{3768559450053507}) difuzata dupa ora 18:00 pe 23.11.2024. Mesajul contine elemente clare de propaganda electorala, inclusiv numar CMF (11240002), sloganuri de campanie ("Nicolae Ciuca, presedinte!"), promisiuni electorale privind stabilitatea si dezvoltarea tarii, precum si argumentatie directa pentru sustinerea candidatului in functia de presedinte si comandant suprem al fortelor armate, avand un impact direct asupra intentiei de vot a unui public tinta estimat intre 500.001 si 1.000.000 de persoane.
    \item promovarea unui mesaj de propaganda electorala pentru candidatul Nicolae Ciuca la functia de presedinte dupa incheierea perioadei de campanie. Postarea cu ID-ul \href{https://www.facebook.com/ads/library/?id=529016233306349}{529016233306349} contine explicit mesajul "Nicolae Ciuca, presedinte!" insotit de numarul CMF 11240002, reprezentand propaganda electorala activa dupa ora 18:00 pe 23.11.2024. Mesajul are un evident caracter electoral, vizand mobilizarea unui grup demografic specific ("oltenii liberali") pentru sustinerea candidatului la presedintie, depasind astfel cadrul legal permis pentru comunicarea politica in aceasta perioada.
\end{enumerate}

\vspace{0.5cm}

\subsection{Ion Iordache}
Următoarele fapte contravenționale sunt sesizate împotriva acestei entități:

\begin{enumerate}[leftmargin=*, label=\arabic*.)]
    \item promovarea unui mesaj de propaganda electorala pentru candidatul Nicolae-Ionel Ciuca dupa incheierea perioadei de campanie electorala, folosind o reclama platita pe Facebook (ID: \href{https://www.facebook.com/ads/library/?id=582178580862377}{582178580862377}) care contine mesajul explicit "Nicolae Ciuca trebuie sa fie presedintele Romaniei!", avand un evident caracter de indemn electoral si folosind numar CMF 11240002. Mesajul are un efect electoral direct, fiind distribuit catre un public tinta de 10.000-15.000 persoane din zona Olteniei, cu un buget de aproximativ 250 RON, dupa ora 18:00 pe 23.11.2024, incalcand astfel prevederile legale privind incetarea campaniei electorale.
    \item promovarea unui mesaj electoral platit pe Facebook (ID: \href{https://www.facebook.com/ads/library/?id=878684511102919}{878684511102919}) care continua propaganda electorala dupa incheierea perioadei legale de campanie pentru alegerile prezidentiale. Postarea contine un indemn direct de sustinere a candidatului Nicolae Ciuca pentru functia de presedinte ("Nicolae Ciuca trebuie sa fie presedintele Romaniei!"), are numar CMF (11240002), si are un obiectiv electoral clar exprimat prin mesaje de sustinere si indemnuri la vot ("Hai la treaba! Tu decizi!"). Postarea, activa dupa ora 18:00 pe 23.11.2024, are un impact semnificativ, atingand intre 60.000 si 70.000 de afisari, fiind astfel o incercare clara de a influenta alegatorii in perioada in care propaganda electorala este interzisa prin lege.
\end{enumerate}

\vspace{0.5cm}

\subsection{Ion Iordache si PNL Gorj}
Următoarele fapte contravenționale sunt sesizate împotriva acestei entități:

\begin{enumerate}[leftmargin=*, label=\arabic*.)]
    \item continuarea propagandei electorale dupa incheierea perioadei legale de campanie, prin intermediul unei postari sponsorizate pe Facebook (ID: \href{https://www.facebook.com/ads/library/?id=1058089099448408}{1058089099448408}) dupa ora 18:00 pe 23.11.2024. Postarea promoveaza explicit candidatul Nicolae Ciuca pentru functia de presedinte, contine numar CMF (11240002), foloseste indemnuri directe catre alegatori ("Tu decizi!"), si include mesaje clare de sustinere electorala ("Nicolae Ciuca este cel mai bun si are sprijinul nostru!"). Postarea are un evident caracter de propaganda electorala, fiind difuzata catre un public larg (25.000-29.999 impresii) cu scopul explicit de a influenta decizia de vot pentru alegerile prezidentiale.
    \item continuarea propagandei electorale dupa incheierea perioadei legale de campanie, prin intermediul unei postari sponsorizate pe Facebook (ID: \href{https://www.facebook.com/ads/library/?id=1276702273486390}{1276702273486390}) dupa ora 18:00 pe 23.11.2024. Postarea contine materiale de propaganda electorala cu numar CMF 11240002, promovand explicit candidatul Nicolae Ciuca la alegerile prezidentiale prin mesaje precum "Nicolae Ciuca este cel mai bun si are sprijinul nostru!", "Tu decizi!" si "Mai sunt cateva zile pana la alegerile prezidentiale". Postarea are un evident caracter de propaganda electorala, fiind distribuita catre un public larg (70.000-80.000 impresii) prin intermediul unei reclame platite pe platformele Facebook si Instagram.
    \item continuarea propagandei electorale dupa incheierea campaniei, prin postarea cu ID \href{https://www.facebook.com/ads/library/?id=548069484753309}{548069484753309} pe Facebook dupa ora 18:00 pe 23.11.2024. Postarea contine elemente clare de propaganda electorala pentru candidatul Nicolae Ciuca la alegerile prezidentiale, avand CMF 11240002, folosind expresii precum "Nicolae Ciuca este cel mai bun si are sprijinul nostru!" si "Mai sunt cateva zile pana la alegerile prezidentiale", cu un impact semnificativ avand intre 30.000 si 35.000 de afisari, reprezentand o incercare clara de a influenta votul prin continuarea mesajelor de campanie in perioada interzisa prin lege.
\end{enumerate}

\vspace{0.5cm}

\subsection{Legacy Marketing SRL}
Următoarele fapte contravenționale sunt sesizate împotriva acestei entități:

\begin{enumerate}[leftmargin=*, label=\arabic*.)]
    \item continuarea propagandei electorale dupa incheierea campaniei electorale pentru alegerile prezidentiale, prin publicarea si mentinerea activa a unei reclame pe Facebook (ID: \href{https://www.facebook.com/ads/library/?id=2403627686650111}{2403627686650111}) dupa ora 18:00 pe 23.11.2024. Postarea contine elementele definitorii ale propagandei electorale conform Art. 36(7): identifica clar candidatul George Simion, foloseste numar CMF 11240014, are obiectiv electoral explicit prin indemnul "Duminica Romania voteaza \#GeorgeSimion\_presedinte" si afirmatia "vrea, stie si poate sa conduca Romania", se adreseaza publicului larg prin difuzare platita pe Facebook si Instagram, atingand intre 80.000 si 90.000 de afisari, depasind astfel cadrul activitatii jurnalistice de informare.
\end{enumerate}

\vspace{0.5cm}

\subsection{Lorand Toth}
Următoarele fapte contravenționale sunt sesizate împotriva acestei entități:

\begin{enumerate}[leftmargin=*, label=\arabic*.)]
    \item publicarea si promovarea unui mesaj de propaganda electorala (ID postare Facebook: \href{https://www.facebook.com/ads/library/?id=1575859769965869}{1575859769965869}) dupa ora 18:00 pe 23.11.2024, mesaj care promoveaza direct candidatul USR Elena Lasconi la presedintie. Postarea include numar CMF (11240015), foloseste indemnuri directe la vot ("Nu ne putem permite sa irosim niciun vot", "Turul 1 = Turul decisiv!"), promoveaza explicit candidatul ("Elena Lasconi este singurul candidat capabil") si este platita pentru a ajunge la un public larg (100,001-500,000 persoane). Efectul electoral urmarit este clar de influentare a votului in favoarea candidatului USR prin promovarea activa si directa a acestuia in perioada in care propaganda electorala este interzisa prin lege.
\end{enumerate}

\vspace{0.5cm}

\subsection{MEGASOFT SYSTEMS SRL}
Următoarele fapte contravenționale sunt sesizate împotriva acestei entități:

\begin{enumerate}[leftmargin=*, label=\arabic*.)]
    \item publicarea si promovarea unei reclame platite pe Facebook si Instagram (ID: \href{https://www.facebook.com/ads/library/?id=2459673824239734}{2459673824239734}) dupa ora 18:00 pe 23.11.2024, ce constituie propaganda electorala in favoarea candidatului ION-MARCEL CIOLACU. Postarea include numarul CMF 11240017, are un caracter electoral explicit prin afirmatia "Stabilitatea este cea mai importanta si singurul care a dovedit ca se poate acest lucru este Marcel Ciolacu", vizeaza un public larg (intre 500,001 si 1,000,000 de persoane) si are ca scop influentarea intentiei de vot a alegatorilor in perioada in care propaganda electorala pentru alegerile prezidentiale este interzisa prin lege.
    \item difuzarea de materiale de propaganda electorala dupa incheierea perioadei de campanie, prin intermediul unei reclame platite pe Facebook si Instagram (ID: \href{https://www.facebook.com/ads/library/?id=291752920699971}{291752920699971}) care promoveaza candidatul ION-MARCEL CIOLACU si partidul PSD, folosind un mesaj electoral explicit "Cu Marcel Ciolacu, suntem pe calea sigura pentru Cluj, calea sigura pentru Romania!" si numar CMF 11240017. Materialul are un caracter evident propagandistic, fiind difuzat dupa ora 18:00 pe 23.11.2024, targetand un public larg (500,001-1,000,000 persoane) si avand ca scop influentarea intentiei de vot prin asocierea realizarilor guvernamentale cu candidatul la presedintie.
    \item difuzarea unui mesaj de propaganda electorala dupa incheierea perioadei de campanie (dupa ora 18:00 pe 23.11.2024) pentru candidatul prezidential Marcel Ciolacu. Postarea cu ID-ul \href{https://www.facebook.com/ads/library/?id=575355425040710}{575355425040710} promoveaza explicit candidatul si partidul PSD, folosind mesaje precum "Marcel Ciolacu", "\#PSD" si indemnul "Haideti cu noi pe calea sigura!", prezentand un program economic si facand un apel direct catre alegatori de a sustine aceasta directie. Postarea include numar CMF (11240017), confirmand natura sa de propaganda electorala, si a fost distribuita ca reclama platita pe Facebook, targetand intre 100,001 si 500,000 de persoane, cu un buget alocat.
    \item difuzarea de materiale de propaganda electorala pentru candidatul prezidential Marcel Ciolacu dupa incheierea perioadei de campanie electorala (postare ID: \href{https://www.facebook.com/ads/library/?id=869919955125660}{869919955125660}). Postarea, difuzata dupa ora 18:00 pe 23.11.2024, promoveaza explicit candidatul prin asocierea cu realizari guvernamentale (modernizarea caii ferate Oradea) si contine mesajul electoral explicit "PSD si Marcel Ciolacu sunt, fara echivoc, calea sigura pentru Romania!", fiind marcata cu numarul CMF 11240017. Materialul are un impact semnificativ, fiind promovat pe Facebook si Instagram cu peste 200.000 de afisari, reprezentand o clara intentie de influentare a votului in perioada in care propaganda electorala este interzisa prin lege.
    \item continuarea propagandei electorale dupa incheierea campaniei electorale pentru alegerile prezidentiale, prin publicarea si mentinerea activa a unei reclame pe Facebook (ID: \href{https://www.facebook.com/ads/library/?id=977130791118334}{977130791118334}) dupa ora 18:00 pe 23.11.2024. Postarea constituie propaganda electorala conform Art. 36(7) deoarece: mentioneaza explicit candidatul prezidential Marcel Ciolacu, contine numar CMF (11240017), prezinta promisiuni electorale privind reindustrializarea Romaniei, si face apel direct la sustinerea publicului ("Haideti cu noi pe calea sigura!"). Efectul electoral este evident prin asocierea pozitiva a candidatului cu un plan economic si prin incercarea de a influenta decizia de vot prin prezentarea unei "cai sigure" sub conducerea candidatului mentionat.
\end{enumerate}

\vspace{0.5cm}

\subsection{Maramures INFO News}
Următoarele fapte contravenționale sunt sesizate împotriva acestei entități:

\begin{enumerate}[leftmargin=*, label=\arabic*.)]
    \item promovarea unui mesaj de propaganda electorala pentru candidatul USR la prezidentiale Elena Lasconi, dupa incheierea perioadei de campanie electorala, prin intermediul unei reclame platite pe Facebook si Instagram (ID: \href{https://www.facebook.com/ads/library/?id=1077556713676876}{1077556713676876}). Postarea promoveaza in mod explicit candidatul ("Elena Lasconi nu doar ca este o femeie cu o viziune clara pentru viitorul acestei tari"), prezentand caracteristici relevante pentru functia de presedinte si urmarind influentarea intentiei de vot a publicului larg, dupa ora 18:00 pe 23.11.2024, prin intermediul unei reclame cu impact semnificativ (4000-4999 impresii). Caracterul electoral al mesajului este evident prin referirea directa la calitatile de conducator si la viitorul tarii, in contextul alegerilor prezidentiale.
    \item promovarea unui mesaj cu caracter electoral in perioada de restrictie, dupa ora 18:00 pe 23.11.2024. Postarea cu ID-ul \href{https://www.facebook.com/ads/library/?id=1323418585516701}{1323418585516701} contine un mesaj explicit referitor la alegerile prezidentiale ("Pe 24 noiembrie, Romania are sansa de a alege un presedinte"), promovand indirect candidatul USR prin caracterizari precum "presedinte onest, care nu face jocurile murdare ale sistemului corupt". Mesajul este distribuit ca reclama platita pe Facebook si Instagram, cu un impact estimat intre 100,001 si 500,000 de persoane, avand un clar obiectiv electoral de influentare a votului in ziua alegerilor. Postarea depaseste limitele activitatii jurnalistice si reprezinta propaganda electorala activa in perioada de restrictie.
\end{enumerate}

\vspace{0.5cm}

\subsection{Marian Ciofica}
Următoarele fapte contravenționale sunt sesizate împotriva acestei entități:

\begin{enumerate}[leftmargin=*, label=\arabic*.)]
    \item publicarea si promovarea unei reclame platite pe Facebook (ID: \href{https://www.facebook.com/ads/library/?id=1303880497450662}{1303880497450662}) dupa incheierea perioadei de campanie electorala, in care promoveaza explicit candidatura lui Calin Georgescu la presedintie. Postarea, difuzata dupa ora 18:00 pe 23.11.2024, reprezinta propaganda electorala prin faptul ca il descrie pe candidat ca fiind "profesionist, experimentat din punct de vedere politic" si declara explicit "imi declar sustinerea pentru domnul Calin Georgescu", avand ca scop influentarea comportamentului electoral al unui public tinta estimat intre 500.001 si 1.000.000 de persoane, fiind astfel o incercare clara de a influenta decizia de vot a alegatorilor in perioada in care propaganda electorala este interzisa prin lege.
    \item publicarea si promovarea unei reclame platite pe Facebook (ID: \href{https://www.facebook.com/ads/library/?id=1334206434372922}{1334206434372922}) dupa ora 18:00 pe 23.11.2024, prin care face in mod explicit propaganda electorala pentru candidatul la presedintie Calin Georgescu. Postarea constituie propaganda electorala deoarece indeplineste toate criteriile prevazute de lege: se refera direct la un candidat la presedintie, are obiectiv electoral explicit prin indemnul direct la vot ("Votul pentru Calin Georgescu este un vot pentru independenta"), se adreseaza publicului larg prin promovare platita pe Facebook si Instagram (3000-3999 impresii), si depaseste limitele unei simple opinii personale prin utilizarea unui limbaj persuasiv de campanie si prin natura sa de continut sponsorizat.
    \item publicarea si promovarea activa prin plata pe Facebook si Instagram a unui mesaj de propaganda electorala (ID postare: \href{https://www.facebook.com/ads/library/?id=1622851758269156}{1622851758269156}) dupa incheierea campaniei electorale, in data de 23.11.2024. Postarea contine indemnuri directe de vot pentru candidatul Calin Georgescu la alegerile prezidentiale ("Eu votez Calin Georgescu, iar daca va regasiti in propunerea mea haideti va rog sa sustinem pe 24 Noiembrie 2024 candidatura dansului"), fiind marcata cu numar CMF 11240022, ceea ce confirma natura sa de material de propaganda electorala. Mesajul are un evident caracter de influentare a votului prin comparatii cu alegeri anterioare si indemn explicit la vot pentru un anumit candidat, fiind promovat activ prin plata catre un public tinta estimat intre 500.001 si 1.000.000 de persoane.
    \item publicarea si promovarea unei reclame platite pe Facebook si Instagram (ID: \href{https://www.facebook.com/ads/library/?id=1624012695133033}{1624012695133033}) dupa data de 23.11.2024, ora 18:00, continand propaganda electorala explicita in favoarea candidatului Calin Georgescu la functia de Presedinte al Romaniei. Postarea include sustinere directa ("Sustin Calin Georgescu!") si un mesaj de influentare a opiniei publice prin sugerarea unei cresteri in sondaje, avand un impact potential asupra a peste 500.000 de utilizatori. Caracterul sau propagandistic este evident prin natura platita a materialului si prin obiectivul sau electoral explicit de influentare a votului.
    \item continuarea propagandei electorale dupa incheierea campaniei, prin postarea cu ID \href{https://www.facebook.com/ads/library/?id=8823369257728083}{8823369257728083} pe Facebook. Postarea, publicata si promovata dupa ora 18:00 pe 23.11.2024, contine elemente clare de propaganda electorala pentru candidatul Calin Georgescu, incluzand indemnuri directe la vot ("Iesiti la vot", "Alegeti Calin Georgescu!"), elemente de mobilizare electorala ("Mobilizati-va familia, prietenii si vecinii"), precum si cod CMF (11240022), demonstrand natura sa de material electoral. Postarea are un evident caracter de propaganda electorala, vizand influentarea directa a votului in favoarea unui candidat la presedintie, intr-o perioada in care asemenea activitati sunt interzise prin lege.
\end{enumerate}

\vspace{0.5cm}

\subsection{Matei Stefanescu}
Următoarele fapte contravenționale sunt sesizate împotriva acestei entități:

\begin{enumerate}[leftmargin=*, label=\arabic*.)]
    \item publicarea si promovarea unei reclame platite pe Facebook si Instagram (ID: \href{https://www.facebook.com/ads/library/?id=1404871247307885}{1404871247307885}) dupa ora 18:00 pe 23.11.2024, care constituie propaganda electorala explicita pentru candidatul George Simion la functia de presedinte. Postarea contine numar CMF (11240014), indeamna direct la vot folosind expresii precum "sa votam primul Presedinte al romanilor" si face referire directa la candidat ("George Simion Presedinte"), avand un evident caracter de propaganda electorala prin mesajul si modul de promovare, atingand un public estimat de peste 1 milion de persoane prin publicitate platita, intr-o perioada in care campania electorala este incheiata legal.
\end{enumerate}

\vspace{0.5cm}

\subsection{Media Flux Dambovita SRL}
Următoarele fapte contravenționale sunt sesizate împotriva acestei entități:

\begin{enumerate}[leftmargin=*, label=\arabic*.)]
    \item promovarea unui material de propaganda electorala pentru candidatul la presedintie Nicolae Ciuca dupa incheierea perioadei de campanie electorala. Postarea cu ID-ul \href{https://www.facebook.com/ads/library/?id=551750367733446}{551750367733446}, publicata dupa ora 18:00 pe 23.11.2024, promoveaza in mod direct candidatul pentru functia de presedinte, folosind un mesaj electoral explicit ("Romania are nevoie de un presedinte pregatit cum este Nicolae Ciuca"), indeamna la vot pentru date specifice, si foloseste elementele specifice campaniei electorale, inclusiv numar CMF (11240002). Efectul electoral urmarit este influentarea alegatorilor in favoarea candidatului PNL la presedintie, prin prezentarea acestuia ca fiind cea mai buna optiune pentru functia de presedinte.
\end{enumerate}

\vspace{0.5cm}

\subsection{Media New Strategy}
Următoarele fapte contravenționale sunt sesizate împotriva acestei entități:

\begin{enumerate}[leftmargin=*, label=\arabic*.)]
    \item promovarea unui material de propaganda electorala pentru alegerile prezidentiale (ID postare: \href{https://www.facebook.com/ads/library/?id=470040132292081}{470040132292081}) dupa incheierea perioadei de campanie electorala. Materialul platit, distribuit pe Facebook si Instagram dupa ora 18:00 pe 23.11.2024, contine un indemn explicit de vot pentru candidatul Elena Lasconi la alegerile prezidentiale ("Votati cu candidatul sustinut de Forta Dreptei la alegerile prezidentiale, Elena Lasconi"), fiind insotit de numar CMF 11240022, care confirma natura sa de material de propaganda electorala. Postarea are un evident caracter de propaganda electorala, adresandu-se unui public larg estimat intre 10.001 si 50.000 de persoane, cu scopul explicit de a influenta votul in favoarea unui candidat la alegerile prezidentiale.
\end{enumerate}

\vspace{0.5cm}

\subsection{Media Smart SRL}
Următoarele fapte contravenționale sunt sesizate împotriva acestei entități:

\begin{enumerate}[leftmargin=*, label=\arabic*.)]
    \item publicarea si promovarea unei reclame platite pe Facebook si Instagram (ID: \href{https://www.facebook.com/ads/library/?id=1546185322672477}{1546185322672477}) dupa ora 18:00 pe 23.11.2024, care constituie propaganda electorala in favoarea candidatului ION-MARCEL CIOLACU. Postarea utilizeaza rezultatele negocierilor pentru Schengen pentru a promova calitatile de lider ale candidatului, laudand explicit "premierul Marcel Ciolacu" si PSD pentru "puterea si priceperea" lor, avand un evident scop electoral. Prezenta codului CMF (11240017) confirma caracterul electoral al mesajului. Postarea platita vizeaza un public tinta de 100.001-500.000 persoane, demonstrand intentia clara de a influenta comportamentul electoral al unui numar semnificativ de alegatori in perioada in care campania electorala pentru alegerile prezidentiale este inchisa.
\end{enumerate}

\vspace{0.5cm}

\subsection{Moldova Invest}
Următoarele fapte contravenționale sunt sesizate împotriva acestei entități:

\begin{enumerate}[leftmargin=*, label=\arabic*.)]
    \item publicarea si promovarea unei reclame platite pe Facebook (ID: \href{https://www.facebook.com/ads/library/?id=590483866782246}{590483866782246}) dupa ora 18:00 pe 23.11.2024, care continua propaganda electorala pentru candidatul prezidential George Nicolae Simion. Postarea sugereaza explicit ca succesul candidatilor parlamentari AUR depinde de intrarea lui Simion in turul II al alegerilor prezidentiale, creand astfel o presiune electorala si inducand publicul sa voteze cu acesta. Mesajul "daca Simion intra in turul II al alegerilor prezidentiale atunci si sansele AUR la alegerile parlamentare din 1 decembrie cresc simtitor" reprezinta o forma clara de propaganda electorala care urmareste influentarea votului pentru alegerile prezidentiale, incalcand astfel perioada legala de campanie.
\end{enumerate}

\vspace{0.5cm}

\subsection{Moza Costel - Consilier Local Oradea}
Următoarele fapte contravenționale sunt sesizate împotriva acestei entități:

\begin{enumerate}[leftmargin=*, label=\arabic*.)]
    \item difuzarea de propaganda electorala pentru alegerile prezidentiale dupa incheierea perioadei de campanie, respectiv dupa ora 18:00 pe 23.11.2024. Postarea cu ID-ul \href{https://www.facebook.com/ads/library/?id=492771710468418}{492771710468418} contine un indemn direct la vot pentru candidatul George Simion la functia de presedinte ("Pe 24 noiembrie, voteaza George Simion Presedinte!"), foloseste numar CMF (11240014), si are un obiectiv electoral explicit pentru alegerile prezidentiale. Postarea este promovata prin reclama platita pe Facebook, atingand intre 35.000 si 39.999 de persoane, reprezentand astfel o incercare clara de influentare a votului in perioada in care propaganda electorala este interzisa prin lege.
\end{enumerate}

\vspace{0.5cm}

\subsection{Neata Eugen}
Următoarele fapte contravenționale sunt sesizate împotriva acestei entități:

\begin{enumerate}[leftmargin=*, label=\arabic*.)]
    \item continuarea propagandei electorale dupa incheierea campaniei electorale, prin intermediul unei postari sponsorizate pe Facebook (ID: \href{https://www.facebook.com/ads/library/?id=587640377105732}{587640377105732}) dupa ora 18:00 pe 23.11.2024. Postarea promoveaza explicit candidatul ION-MARCEL CIOLACU pentru functia de presedinte, contine numar CMF (11240017), indeamna direct la vot ("Duminica mergem cu totii la vot") si sugereaza explicit optiunea de vot ("Nu ma indoiesc ca aceasta decizie este Marcel Ciolacu, presedinte al romanilor"). Materialul are caracter evident de propaganda electorala, fiind distribuit contra cost pe platformele Facebook si Instagram, cu un potential de audienta estimat la peste 1 milion de persoane, reprezentand astfel o incalcare clara a prevederilor legale privind incheierea campaniei electorale.
    \item continuarea propagandei electorale dupa incheierea campaniei oficiale, prin intermediul unei postari sponsorizate pe Facebook (ID: \href{https://www.facebook.com/ads/library/?id=667140289053461}{667140289053461}) care promoveaza in mod explicit candidatura lui Marcel Ciolacu la functia de presedinte. Postarea, care include numar CMF 11240017, contine indemnuri directe de vot ("Ne pregatim cu totii pentru a-l alege pe Marcel Ciolacu in functia de presedinte al tarii"), prezentarea activitatilor de campanie si a proiectului electoral, fiind difuzata dupa ora 18:00 pe 23.11.2024, cu scopul clar de a influenta optiunea de vot a alegatorilor in favoarea candidatului PSD la presedintie.
\end{enumerate}

\vspace{0.5cm}

\subsection{Organizatia PNL Santana}
Următoarele fapte contravenționale sunt sesizate împotriva acestei entități:

\begin{enumerate}[leftmargin=*, label=\arabic*.)]
    \item publicarea si promovarea unui mesaj de propaganda electorala (ID postare Facebook: \href{https://www.facebook.com/ads/library/?id=599969435814755}{599969435814755}) dupa incheierea perioadei de campanie electorala pentru alegerile prezidentiale. Postarea, facuta dupa ora 18:00 pe 23.11.2024, contine un indemn explicit de vot pentru candidatul Nicolae Ciuca la pozitia 4 pe buletinul de vot din 24 noiembrie 2024, folosind realizari administrative pentru a influenta alegatorii si incluzand un cod de mandatar (11240002). Mesajul combina in mod strategic realizari administrative cu solicitarea explicita de vot pentru candidatul la presedintie, avand un evident caracter de propaganda electorala prin indemnul direct "veniti la vot si sa ma ajutati, pe mine si localitatea noastra, votand, pe toate buletinele de vot, Partidul National Liberal - In 24 noiembrie 2024 pozitia 4."
\end{enumerate}

\vspace{0.5cm}

\subsection{PNL Arad, prin intermediul paginii "Ucu Dima"}
Următoarele fapte contravenționale sunt sesizate împotriva acestei entități:

\begin{enumerate}[leftmargin=*, label=\arabic*.)]
    \item continuarea propagandei electorale dupa incheierea campaniei electorale pentru alegerile prezidentiale, prin promovarea activa si explicita a candidatului Nicolae Ciuca la functia de presedinte al Romaniei. Postarea cu ID-ul \href{https://www.facebook.com/ads/library/?id=554639790609638}{554639790609638}, publicata si promovata dupa ora 18:00 pe 23.11.2024, contine elemente clare de propaganda electorala, incluzand numar CMF 11240002, indemnuri directe de sustinere a candidatului ("il sustinem cu incredere pe Nicolae Ciuca"), caracterizari laudative in context electoral ("un lider de exceptie", "alegerea ideala pentru functia de Presedinte al Romaniei") si utilizarea de sloganuri electorale, toate acestea avand ca scop influentarea votului in favoarea candidatului PNL la presedintie.
\end{enumerate}

\vspace{0.5cm}

\subsection{PNL Arad, prin reprezentant Ucu Dima}
Următoarele fapte contravenționale sunt sesizate împotriva acestei entități:

\begin{enumerate}[leftmargin=*, label=\arabic*.)]
    \item continuarea propagandei electorale pentru alegerile prezidentiale dupa incheierea perioadei legale de campanie, intr-o postare sponsorizata pe Facebook (ID: \href{https://www.facebook.com/ads/library/?id=836994191768944}{836994191768944}) publicata dupa ora 18:00 pe 23.11.2024. Postarea promoveaza in mod explicit candidatul Nicolae Ciuca pentru functia de presedinte, folosind expresii precum "il sustinem cu incredere pe Nicolae Ciuca" si "alegerea ideala pentru functia de Presedinte al Romaniei", constituind astfel propaganda electorala conform Art. 36(7), fapt confirmat si de prezenta numarului CMF 11240002. Efectul electoral urmarit este influentarea alegatorilor in favoarea candidatului Nicolae Ciuca la alegerile prezidentiale, intr-o perioada in care propaganda electorala este interzisa prin lege.
\end{enumerate}

\vspace{0.5cm}

\subsection{PNL Arad, prin reprezentantul Ucu Dima}
Următoarele fapte contravenționale sunt sesizate împotriva acestei entități:

\begin{enumerate}[leftmargin=*, label=\arabic*.)]
    \item promovarea unui mesaj electoral platit pe Facebook (ID: \href{https://www.facebook.com/ads/library/?id=1069524538193933}{1069524538193933}) dupa incheierea perioadei de propaganda electorala pentru alegerile prezidentiale. Postarea, activa dupa ora 18:00 pe 23.11.2024, promoveaza explicit candidatura lui Nicolae Ciuca la presedintie, folosind expresii precum "il sustinem cu incredere pe Nicolae Ciuca - un lider de exceptie" si "alegerea ideala pentru functia de Presedinte al Romaniei", avand un evident caracter de propaganda electorala confirmat si prin prezenta codului CMF 11240002. Mesajul are ca scop influentarea intentiei de vot a alegatorilor pentru alegerile prezidentiale, depasind cadrul legal permis pentru comunicarea electorala in aceasta perioada.
    \item difuzarea unui material de propaganda electorala (ID Facebook: \href{https://www.facebook.com/ads/library/?id=487741760321294}{487741760321294}) dupa incheierea campaniei electorale pentru alegerile prezidentiale, respectiv dupa ora 18:00 pe 23.11.2024. Materialul promovat pe Facebook si Instagram contine un indemn explicit de sustinere a candidatului Nicolae Ciuca la functia de Presedinte al Romaniei, descriindu-l ca "lider de exceptie" si "alegerea ideala pentru functia de Presedinte al Romaniei", avand un evident caracter de propaganda electorala confirmat si prin prezenta numarului CMF 11240002. Mesajul are ca scop influentarea intentiei de vot a alegatorilor pentru alegerile prezidentiale, fiind distribuit contra cost catre un public tinta de 1001-5000 persoane.
\end{enumerate}

\vspace{0.5cm}

\subsection{PNL Arges}
Următoarele fapte contravenționale sunt sesizate împotriva acestei entități:

\begin{enumerate}[leftmargin=*, label=\arabic*.)]
    \item continuarea propagandei electorale dupa incheierea campaniei electorale pentru alegerile prezidentiale, prin postarea cu ID \href{https://www.facebook.com/ads/library/?id=1092751075297851}{1092751075297851} pe Facebook, publicata si promovata dupa ora 18:00 pe 23.11.2024. Postarea constituie propaganda electorala conform criteriilor din Art. 36(7), avand numar CMF (11240002), promovand direct candidatul Nicolae Ciuca pentru functia de presedinte, mentionand explicit alegerile prezidentiale si listand motive pentru care acesta ar trebui votat ("optiunea mea... este Nicolae Ciuca"), cu un impact electoral semnificativ demonstrat prin bugetul alocat promovarii (200-299 RON) si audienta targetata (10,000-14,999 impresii).
    \item publicarea si promovarea unui mesaj de propaganda electorala (ID postare Facebook: \href{https://www.facebook.com/ads/library/?id=597485686033312}{597485686033312}) dupa incheierea perioadei de campanie electorala (dupa ora 18:00 pe 23.11.2024). Postarea contine un atac direct la adresa candidatului prezidential Marcel Ciolacu, foloseste numar CMF (11240002), include indemnuri explicite de vot impotriva candidatului ("Trimite acasa Ciolacul mincinos!"), si are un evident scop electoral demonstrat prin cheltuirea de fonduri pentru promovare si atingerea unui public tinta de 30.000-35.000 de persoane. Mesajul depaseste limitele unei simple opinii personale, fiind o actiune coordonata de campanie cu scop explicit de influentare a votului in alegerile prezidentiale.
    \item publicarea si promovarea dupa data de 23.11.2024, ora 18:00, a unui material de propaganda electorala pe platforma Instagram (ID postare: \href{https://www.facebook.com/ads/library/?id=607248631642642}{607248631642642}) care vizeaza in mod direct candidatul ION-MARCEL CIOLACU, cu intentia clara de a influenta votul in sens negativ. Materialul contine indemnuri explicite de vot impotriva candidatului ("Trimite acasa Ciolacul mincinos!"), atacuri la persoana si incercari de discreditare, fiind marcat cu numar CMF 11240002, ceea ce confirma natura sa de propaganda electorala. Postarea este promovata cu bani (paid advertising) si are un impact estimat intre 50,001 si 100,000 de persoane, demonstrand intentia clara de a influenta electoratul intr-un mod organizat si sistematic.
\end{enumerate}

\vspace{0.5cm}

\subsection{PNL Calarasi si Datablitz SRL}
Următoarele fapte contravenționale sunt sesizate împotriva acestei entități:

\begin{enumerate}[leftmargin=*, label=\arabic*.)]
    \item publicarea si promovarea unui anunt sponsorizat pe Facebook (ID: \href{https://www.facebook.com/ads/library/?id=1259843452016145}{1259843452016145}) care face propaganda electorala pentru candidatul prezidential Nicolae Ciuca dupa incheierea perioadei de campanie. Postarea contine in mod explicit pozitia pe buletinul de vot si indeamna la votarea candidatului ("Nicolae Ciuca, pozitia 4 pe buletinul de vot pe 24 noiembrie"), avand un efect electoral direct prin promovarea platita catre un public tinta de 100,001-500,000 de persoane din zona Calarasi, dupa ora 18:00 pe 23.11.2024. Postarea include si numar CMF (11240002), confirmand natura sa de material de propaganda electorala.
\end{enumerate}

\vspace{0.5cm}

\subsection{PNL DOLJ}
Următoarele fapte contravenționale sunt sesizate împotriva acestei entități:

\begin{enumerate}[leftmargin=*, label=\arabic*.)]
    \item continuarea propagandei electorale dupa incheierea perioadei legale de campanie pentru alegerile prezidentiale. Postarea cu ID-ul \href{https://www.facebook.com/ads/library/?id=903805185049021}{903805185049021}, publicata si promovata dupa ora 18:00 pe 23.11.2024, contine un indemn explicit la vot pentru candidatul prezidential Nicolae Ciuca ("va invit sa votati listele Partidului National Liberal si pe domnul Presedinte Nicolae Ciuca"), foloseste infrastructura de campanie (cod AEP 11240002), si are un caracter clar propagandistic electoral, fiind promovata cu bani catre un public larg (8000-8999 impresii). Mesajul are un efect electoral direct, incercand sa influenteze alegatorii in favoarea unui candidat specific la presedintie intr-o perioada in care propaganda electorala este interzisa prin lege.
\end{enumerate}

\vspace{0.5cm}

\subsection{PNL Galati}
Următoarele fapte contravenționale sunt sesizate împotriva acestei entități:

\begin{enumerate}[leftmargin=*, label=\arabic*.)]
    \item publicarea si promovarea unei reclame pe Facebook (ID: \href{https://www.facebook.com/ads/library/?id=539157538894856}{539157538894856}) care constituie propaganda electorala activa dupa ora 18:00 pe 23.11.2024. Postarea contine un numar CMF (11240002), face apel direct la vot ("iesi la vot"), ataca direct contracandidatii (Simion si Ciolacu) si promoveaza pozitia PNL pentru alegerile prezidentiale, folosind mesaje precum "singurul partid care poate garanta un viitor european". Postarea are un impact semnificativ, fiind promovata cu un buget intre 900-999 RON si atingand intre 100.000-124.999 impresii, reprezentand astfel o incercare clara de a influenta votul pentru alegerile prezidentiale din 1 decembrie 2024.
    \item continuarea propagandei electorale dupa incheierea campaniei electorale pentru alegerile prezidentiale, prin postarea cu ID \href{https://www.facebook.com/ads/library/?id=542353765276315}{542353765276315} pe Facebook dupa ora 18:00 pe 23.11.2024. Postarea contine elemente clare de propaganda electorala, incluzand indemnuri directe la vot ("Pe \#24Noiembrie votati pentru continuitate, responsabilitate si dezvoltare! Votati Partidul National Liberal si Nicolae Ionel Ciuca, viitorul presedinte al Romaniei!"), foloseste numar CMF (11240002), si promoveaza explicit candidatul Nicolae Ciuca pentru functia de presedinte al Romaniei, avand ca efect electoral influentarea directa a alegatorilor in favoarea candidatului PNL la presedintie.
\end{enumerate}

\vspace{0.5cm}

\subsection{PNL IASI}
Următoarele fapte contravenționale sunt sesizate împotriva acestei entități:

\begin{enumerate}[leftmargin=*, label=\arabic*.)]
    \item continuarea propagandei electorale dupa incheierea perioadei legale de campanie, prin intermediul unei postari sponsorizate pe Facebook (ID: \href{https://www.facebook.com/ads/library/?id=591241616682140}{591241616682140}) dupa ora 18:00 pe 23.11.2024. Postarea contine un indemn direct la vot pentru candidatul Nicolae Ciuca ("Pe \#24noiembrie VOTAM Nicolae Ciuca Presedinte!"), precum si mesaje de pozitionare electorala care il prezinta drept solutia pentru "dezvoltarea comunitatilor locale" si alternativa la "extremism si populism". Postarea are caracter de propaganda electorala conform Art. 36(7), fiind adresata publicului larg prin intermediul unei reclame platite pe platformele Facebook si Instagram, cu un impact estimat intre 1001-5000 de persoane.
\end{enumerate}

\vspace{0.5cm}

\subsection{PNL Iasi}
Următoarele fapte contravenționale sunt sesizate împotriva acestei entități:

\begin{enumerate}[leftmargin=*, label=\arabic*.)]
    \item difuzarea unui mesaj de propaganda electorala pentru candidatul la presedintie Nicolae Ionel Ciuca dupa incheierea perioadei de campanie electorala, prin intermediul unei reclame platite pe Facebook si Instagram (ID: \href{https://www.facebook.com/ads/library/?id=586462180413319}{586462180413319}). Postarea contine un indemn explicit de vot ("sa-l votam pe Nicolae Ionel Ciuca presedinte") si leaga acest vot de beneficii concrete pentru cetateni, fiind difuzata dupa ora 18:00 pe 23.11.2024. Mesajul are caracter electoral evident, fiind promovat prin intermediul unei reclame platite cu impact semnificativ (60.000-70.000 afisari), reprezentand astfel o incalcare clara a prevederilor legale privind incetarea propagandei electorale.
\end{enumerate}

\vspace{0.5cm}

\subsection{PNL Maramures}
Următoarele fapte contravenționale sunt sesizate împotriva acestei entități:

\begin{enumerate}[leftmargin=*, label=\arabic*.)]
    \item promovarea unui material de propaganda electorala (ID Facebook: \href{https://www.facebook.com/ads/library/?id=1080939136916586}{1080939136916586}) dupa incheierea perioadei de campanie electorala pentru alegerile prezidentiale. Materialul, publicat si promovat dupa ora 18:00 pe 23.11.2024, contine un indemn direct de vot pentru candidatul Nicolae Ionel Ciuca ("votul util este pentru Nicolae Ionel Ciuca"), fiind insotit de CMF 11240002, ceea ce confirma natura sa de material de propaganda electorala. Postarea are un evident caracter electoral, adresandu-se unui public larg prin intermediul unei reclame platite pe Facebook, cu un mesaj care vizeaza influentarea votului in alegerile prezidentiale ("vom decide viitorul Romaniei pentru urmatorii 10 ani"), depasind astfel limitele comunicarii publice permise in aceasta perioada.
    \item promovarea unui mesaj electoral platit (ID Facebook: \href{https://www.facebook.com/ads/library/?id=1628682881410360}{1628682881410360}) ce contine indemnuri explicite la vot pentru candidatul Nicolae Ciuca la alegerile prezidentiale ("Pe 24 noiembrie, votam impreuna Nicolae Ciuca!"), dupa incheierea perioadei de campanie electorala pentru alegerile prezidentiale. Postarea, care include numar CMF (11240002) si este difuzata ca reclama platita pe Facebook cu un impact estimat intre 10.000 si 14.999 de afisari, reprezinta o continuare a propagandei electorale dupa incheierea acesteia, avand un efect electoral direct prin indemnul explicit la vot pentru un candidat specific la alegerile prezidentiale. Mentionam ca aceasta activitate continua si dupa ora 18:00 pe 23.11.2024, in perioada de prohibitie electorala.
    \item publicarea si promovarea unei reclame pe Facebook si Instagram (ID: \href{https://www.facebook.com/ads/library/?id=3326577030806855}{3326577030806855}) dupa incheierea perioadei de campanie electorala pentru alegerile prezidentiale. Postarea, difuzata dupa ora 18:00 pe 23.11.2024, contine propaganda electorala explicita pentru candidatul Nicolae-Ionel Ciuca, prin afirmatii precum "Sighetenii il sustin si pe candidatul nostru la alegerile prezidentiale" si "experienta si viziunea sa vor aduce stabilitate si prosperitate Romaniei". Mesajul este marcat cu cod CMF (11240002), confirmand natura sa de propaganda electorala, si a fost difuzat catre un public larg (20.000-25.000 impresii), cu un buget semnificativ (300-399 RON), avand ca scop clar influentarea votului in favoarea candidatului PNL la presedintie.
    \item publicarea si promovarea unei reclame pe Facebook (ID: \href{https://www.facebook.com/ads/library/?id=458881507235742}{458881507235742}) ce contine propaganda electorala explicita pentru candidatul la presedintie Nicolae Ionel Ciuca dupa incheierea perioadei de campanie electorala. Materialul, publicat si promovat dupa ora 18:00 pe 23.11.2024, contine referinte directe la candidatura prezidentiala, exprimand "sprijinul puternic pentru candidatura lui Nicolae Ionel Ciuca la presedintia Romaniei" si promovand calitatile sale de lider prezidential. Postarea include numar CMF (11240002), confirmand natura sa de propaganda electorala, si a fost distribuita catre un public larg prin intermediul unei reclame platite pe Facebook, cu un impact estimat intre 15.000 si 19.999 de afisari.
\end{enumerate}

\vspace{0.5cm}

\subsection{PNL Mures}
Următoarele fapte contravenționale sunt sesizate împotriva acestei entități:

\begin{enumerate}[leftmargin=*, label=\arabic*.)]
    \item difuzarea unui mesaj de propaganda electorala pentru candidatul la presedintie Nicolae Ciuca dupa incheierea perioadei de campanie electorala, prin intermediul unei postari sponsorizate pe Facebook si Instagram (ID: \href{https://www.facebook.com/ads/library/?id=489373544122065}{489373544122065}). Postarea afirma explicit ca "Nicolae Ciuca este alegerea rationala pentru viitorul Romaniei", reprezentand un mesaj clar de influentare a votului pentru alegerile prezidentiale, difuzat dupa ora 18:00 pe 23.11.2024. Mesajul are un evident caracter de propaganda electorala, fiind marcat inclusiv cu numar CMF (11240002), si a fost distribuit catre un public larg, avand intre 60.000 si 70.000 de afisari.
\end{enumerate}

\vspace{0.5cm}

\subsection{PNL Olt}
Următoarele fapte contravenționale sunt sesizate împotriva acestei entități:

\begin{enumerate}[leftmargin=*, label=\arabic*.)]
    \item publicarea si promovarea unei reclame platite pe Facebook si Instagram (ID: \href{https://www.facebook.com/ads/library/?id=2483117645229321}{2483117645229321}) ce contine propaganda electorala explicita pentru candidatul la presedintie Nicolae Ciuca dupa incheierea perioadei de campanie electorala. Postarea, care include un indemn direct de vot ("Votati Nicolae Ciuca, Presedinte al Romaniei!"), a fost promovata dupa ora 18:00 pe 23.11.2024, atingand intre 125.000 si 150.000 de persoane, avand un buget de promovare intre 1.000 si 1.499 RON. Materialul este confirmat ca fiind propaganda electorala prin prezenta codului CMF:11240002 si prin natura sa de indemn explicit la vot pentru un candidat la presedintie, depasind limitele comunicarii permise in aceasta perioada.
    \item difuzarea unui mesaj de propaganda electorala pentru candidatul la presedintie Nicolae Ciuca dupa incheierea perioadei de campanie electorala prezidentiala. Postarea cu ID \href{https://www.facebook.com/ads/library/?id=479625595132799}{479625595132799}, publicata si promovata dupa ora 18:00 pe 23.11.2024, contine un indemn direct la vot pentru candidatul la presedintie ("Votati Nicolae Ciuca, Presedinte al Romaniei!"), fiind distribuita platit catre un public larg (peste 100.000 de afisari). Materialul include numar CMF (11240002), confirmand natura sa de propaganda electorala, si are un obiectiv electoral clar de influentare a votului pentru alegerile prezidentiale, fiind astfel o continuare nepermisa a propagandei electorale dupa incheierea perioadei legale de campanie.
\end{enumerate}

\vspace{0.5cm}

\subsection{PNL Pascani}
Următoarele fapte contravenționale sunt sesizate împotriva acestei entități:

\begin{enumerate}[leftmargin=*, label=\arabic*.)]
    \item continuarea propagandei electorale pentru alegerile prezidentiale dupa incheierea perioadei legale de campanie, prin postarea cu ID \href{https://www.facebook.com/ads/library/?id=1292348225285327}{1292348225285327} pe Facebook. Postarea, publicata si promovata dupa ora 18:00 pe 23.11.2024, contine un indemn explicit de a vota candidatul Nicolae Ciuca la alegerile prezidentiale ("sa votam pe 24 noiembrie pe Nicolae Nicolae Ionel Ciuca presedinte"), prezinta realizari si promisiuni electorale, foloseste numar CMF (11240002), si are un evident scop electoral de influentare a votului pentru alegerile prezidentiale. Postarea a ajuns la un numar semnificativ de persoane (40,000-45,000 impresii), fiind promovata cu bani pe Facebook si Instagram.
\end{enumerate}

\vspace{0.5cm}

\subsection{PNL SALAJ}
Următoarele fapte contravenționale sunt sesizate împotriva acestei entități:

\begin{enumerate}[leftmargin=*, label=\arabic*.)]
    \item promovarea unui material de propaganda electorala pentru candidatul la presedintie Nicolae Ciuca dupa incheierea perioadei de campanie electorala, prin intermediul unei postari sponsorizate pe Facebook (ID: \href{https://www.facebook.com/ads/library/?id=1287775152348244}{1287775152348244}) difuzata dupa ora 18:00 pe 23.11.2024. Materialul promovat contine in mod explicit indemnul la sustinerea candidatului ("importanta sustinerii lui Nicolae Ciuca pentru functia de presedinte al Romaniei"), asociaza candidatul cu beneficii si stabilitate economica, si a fost distribuit ca reclama platita catre un public de 35.000-40.000 persoane, demonstrand astfel intentia clara de a influenta comportamentul electoral al alegatorilor in perioada in care campania electorala pentru alegerile prezidentiale este inchisa.
    \item difuzarea unui material de propaganda electorala (ID postare Facebook: \href{https://www.facebook.com/ads/library/?id=510922691951131}{510922691951131}) dupa incheierea perioadei de campanie electorala pentru alegerile prezidentiale. Materialul, publicat si promovat dupa ora 18:00 pe 23.11.2024, contine mesaje explicite de sustinere a candidatului Nicolae-Ionel Ciuca la presedintie, incluzand indemnuri directe la vot ("Pe 24 noiembrie, ne aflam in fata unui moment crucial pentru viitorul Romaniei!"), prezentarea candidatului ca "singura optiune" si utilizarea numarului CMF 11240002, specific materialelor de campanie electorala. Postarea are un evident caracter de propaganda electorala, vizand influentarea directa a votului prin promovarea calitatilor candidatului si prin utilizarea unei strategii de comunicare care depaseste simpla informare, avand ca scop mobilizarea alegatorilor pentru sustinerea unui anumit candidat in ziua votului.
\end{enumerate}

\vspace{0.5cm}

\subsection{PNL Tecuci}
Următoarele fapte contravenționale sunt sesizate împotriva acestei entități:

\begin{enumerate}[leftmargin=*, label=\arabic*.)]
    \item publicarea si promovarea unei reclame pe Facebook (ID: \href{https://www.facebook.com/ads/library/?id=972415984698063}{972415984698063}) dupa ora 18:00 pe 23.11.2024, continand propaganda electorala explicita pentru alegerile prezidentiale. Postarea include un numar CMF (11240002), face referiri directe la candidati prezidentiali (Ciolacu si Simion), contine indemnuri directe la vot ("Pe 1 decembrie, viitorul Romaniei depinde de votul tau"), si ataca explicit alti candidati prezidentiali in incercarea de a influenta votul ("PSD, alaturi de AUR, isi arata adevaratele intentii"). Postarea a avut un impact semnificativ, atingand intre 40.000 si 45.000 de afisari, demonstrand intentia clara de a influenta votul in perioada in care campania electorala era inchisa.
\end{enumerate}

\vspace{0.5cm}

\subsection{PNL Valcea}
Următoarele fapte contravenționale sunt sesizate împotriva acestei entități:

\begin{enumerate}[leftmargin=*, label=\arabic*.)]
    \item difuzarea unui mesaj de propaganda electorala (ID postare Facebook: \href{https://www.facebook.com/ads/library/?id=486201574473068}{486201574473068}) dupa incheierea campaniei electorale, respectiv dupa ora 18:00 pe 23.11.2024. Postarea reprezinta propaganda electorala clara prin promovarea directa a candidatului Nicolae Ciuca pentru functia de presedinte, folosind expresii precum "Nicolae Ciuca este liderul care poate garanta o astfel de abordare" si "Impreuna pentru Nicolae Ciuca, viitorul presedinte al Romaniei", continand si cod AEP (11240002), avand un efect electoral direct prin indemnul la vot pentru data de 24 noiembrie si promovarea explicita a candidatului. Mesajul este distribuit ca reclama platita pe Facebook si Instagram, cu un impact estimat intre 500,001 si 1,000,000 de persoane, depasind astfel sfera comunicarii obisnuite si constituind propaganda electorala activa dupa incheierea perioadei legale de campanie.
\end{enumerate}

\vspace{0.5cm}

\subsection{PNL, prin reprezentantul Ucu Dima}
Următoarele fapte contravenționale sunt sesizate împotriva acestei entități:

\begin{enumerate}[leftmargin=*, label=\arabic*.)]
    \item promovarea unui mesaj electoral platit pe Facebook (ID: \href{https://www.facebook.com/ads/library/?id=594402326490969}{594402326490969}) dupa incheierea perioadei de campanie electorala pentru alegerile prezidentiale, respectiv dupa ora 18:00 pe 23.11.2024. Postarea mentioneaza explicit "Cu Nicolae Ciuca Presedinte", promovand astfel candidatura la presedintie a acestuia intr-o perioada in care propaganda electorala pentru alegerile prezidentiale este interzisa. Efectul electoral este evident prin asocierea directa a candidatului cu functia prezidentiala si prin indemnul explicit la vot, depasind simpla informare si constituind propaganda electorala conform criteriilor stabilite in Art. 36(7) din Legea 334/2006.
\end{enumerate}

\vspace{0.5cm}

\subsection{POD TV}
Următoarele fapte contravenționale sunt sesizate împotriva acestei entități:

\begin{enumerate}[leftmargin=*, label=\arabic*.)]
    \item continuarea propagandei electorale dupa incheierea campaniei, prin publicarea si promovarea unei reclame platite (ID: \href{https://www.facebook.com/ads/library/?id=854887666859658}{854887666859658}) ce vizeaza in mod direct candidatul la presedintie George Nicolae Simion si partidul sau, utilizand un limbaj emotional si narativ menit sa influenteze opinia alegatorilor, continand numar CMF (11240027) specific materialelor de campanie, cu un impact semnificativ demonstrat prin reach-ul de 60.000-70.000 impresii. Materialul depaseste limitele jurnalismului obiectiv, avand un evident scop electoral negativ, fiind difuzat dupa ora 18:00 pe 23.11.2024, incalcand astfel prevederile legale privind incheierea campaniei electorale.
\end{enumerate}

\vspace{0.5cm}

\subsection{PS News}
Următoarele fapte contravenționale sunt sesizate împotriva acestei entități:

\begin{enumerate}[leftmargin=*, label=\arabic*.)]
    \item promovarea unui articol platit pe Facebook (ID: \href{https://www.facebook.com/ads/library/?id=1089620366052762}{1089620366052762}) dupa ora 18:00 pe 23.11.2024, care indeamna in mod explicit romanii sa voteze cu candidatul Nicolae Ciuca la alegerile prezidentiale. Postarea reprezinta propaganda electorala conform Art. 36(7), continand un indemn direct de vot ("indeamna romanii sa voteze cu Nicolae Ciuca"), adresandu-se publicului larg prin distributie platita pe Facebook cu un reach estimat de peste 1 milion de persoane, avand un obiectiv electoral clar de influentare a votului in favoarea unui candidat specific la alegerile prezidentiale, depasind astfel limitele activitatii jurnalistice de informare a publicului.
    \item promovarea unui material cu caracter electoral dupa incheierea perioadei de campanie, respectiv dupa ora 18:00 pe 23.11.2024. Materialul platit promoveaza un sondaj electoral care prezinta sansele candidatilor Nicolae Ciuca si George Simion pentru turul doi al alegerilor prezidentiale, avand un efect electoral direct prin informarea si potentiala influentare a votantilor cu privire la optiunile electorale. Postarea cu ID-ul \href{https://www.facebook.com/ads/library/?id=1730750311038529}{1730750311038529} reprezinta propaganda electorala prin natura sa, fiind promovata prin publicitate platita pe Facebook, cu un reach estimat de peste 1 milion de persoane, depasind astfel limitele unei simple informari jurnalistice.
    \item promovarea unui material platit pe Facebook (ID \href{https://www.facebook.com/ads/library/?id=1882700792138874}{1882700792138874}) dupa ora 18:00 pe 23.11.2024, care promoveaza candidatul Nicolae Ciuca prin evidentierea sprijinului international primit de la familia PPE si Manfred Weber. Materialul, prezentat sub forma de stire, depaseste limitele activitatii jurnalistice obiective, avand ca scop influentarea preferintelor electoratului prin sublinierea sustinerii internationale a candidatului. Efectul electoral este amplificat prin targetarea unui public larg (peste 1 milion de persoane potential expuse) si prin utilizarea unei platforme de social media cu plata pentru promovare.
    \item difuzarea unei reclame platite pe Facebook (ID: \href{https://www.facebook.com/ads/library/?id=444559491737765}{444559491737765}) dupa incheierea perioadei de campanie electorala, continand propaganda electorala negativa la adresa candidatului George-Nicolae Simion. Postarea, care il compara pe candidat cu Alexandr Stoianoglo, are un evident caracter electoral negativ, fiind menita sa influenteze decizia de vot a alegatorilor prin asocierea candidatului cu o figura controversata din Republica Moldova. Materialul este difuzat ca reclama platita dupa ora 18:00 pe 23.11.2024, cu un reach estimat de peste 1 milion de persoane, demonstrand intentia clara de a influenta comportamentul electoral al unui numar cat mai mare de alegatori.
    \item promovarea unui material platit pe Facebook (ID: \href{https://www.facebook.com/ads/library/?id=916375703921699}{916375703921699}) dupa data de 23.11.2024, ora 18:00, care face referire directa la candidatul ION-MARCEL CIOLACU si campania sa prezidentiala. Materialul, intitulat "VIDEO Prezidentiale 2024, R. Pricopie: Ciolacu - campanie structurata, dialog extern consistent, lovituri contracarate", prezinta in mod favorabil strategia de campanie a candidatului, analizand performanta acestuia si promovand imaginea sa in contextul alegerilor prezidentiale, reprezentand astfel continuarea propagandei electorale dupa incheierea perioadei legale de campanie.
\end{enumerate}

\vspace{0.5cm}

\subsection{PS News si Partidul National Liberal}
Următoarele fapte contravenționale sunt sesizate împotriva acestei entități:

\begin{enumerate}[leftmargin=*, label=\arabic*.)]
    \item difuzarea unui material de propaganda electorala dupa incheierea perioadei de campanie, constand intr-o postare platita pe Facebook (ID: \href{https://www.facebook.com/ads/library/?id=532419989618931}{532419989618931}) care promoveaza candidatul Nicolae Ciuca la functia de presedinte al Romaniei. Postarea, difuzata dupa ora 18:00 pe 23.11.2024, contine mesaje cu caracter electoral explicit ("Oltenia poate da viitorul presedinte al Romaniei - pe Nicolae Ciuca"), avand ca efect electoral influentarea directa a alegatorilor prin sugerarea votarii unui anumit candidat. Materialul depaseste sfera jurnalismului fiind o reclama platita cu scop electoral clar, care a ajuns la un numar semnificativ de utilizatori (3000-3999 impresii).
\end{enumerate}

\vspace{0.5cm}

\subsection{PSD BN}
Următoarele fapte contravenționale sunt sesizate împotriva acestei entități:

\begin{enumerate}[leftmargin=*, label=\arabic*.)]
    \item distribuirea unui mesaj de propaganda electorala platit pe Facebook si Instagram (ID postare: \href{https://www.facebook.com/ads/library/?id=2310944462575110}{2310944462575110}) dupa incheierea perioadei de campanie electorala pentru alegerile prezidentiale. Postarea contine un indemn direct la vot pentru candidatul Marcel Ciolacu ("VOTAM Marcel Ciolacu presedinte!"), este marcata cu coduri CMF (31240010, 11240017), are caracter explicit electoral si a fost distribuita ca reclama platita dupa ora 18:00 pe 23.11.2024, continuand sa fie activa si sa influenteze alegatorii in perioada de interdictie a propagandei electorale.
\end{enumerate}

\vspace{0.5cm}

\subsection{PSD Bistrita-Nasaud}
Următoarele fapte contravenționale sunt sesizate împotriva acestei entități:

\begin{enumerate}[leftmargin=*, label=\arabic*.)]
    \item continuarea propagandei electorale pentru alegerile prezidentiale dupa incheierea campaniei electorale, prin postarea cu ID \href{https://www.facebook.com/ads/library/?id=603296632079168}{603296632079168} pe Facebook. Postarea, publicata si promovata dupa ora 18:00 pe 23.11.2024, contine un indemn explicit de a vota pentru Marcel Ciolacu la alegerile prezidentiale ("Pe 24 noiembrie si 8 decembrie votam  Marcel Ciolacu, Presedinte al Romaniei!"), reprezentand astfel propaganda electorala activa pentru candidatul prezidential intr-o perioada in care acest lucru este interzis prin lege. Postarea are caracter electoral evident, fiind marcata cu numar CMF (11240017), si a fost distribuita ca reclama platita catre un public tinta de peste 100.000 de persoane, avand astfel potentialul de a influenta procesul electoral dupa incheierea oficiala a campaniei.
\end{enumerate}

\vspace{0.5cm}

\subsection{PSD Brasov}
Următoarele fapte contravenționale sunt sesizate împotriva acestei entități:

\begin{enumerate}[leftmargin=*, label=\arabic*.)]
    \item publicarea si promovarea unei reclame electorale (ID Facebook: \href{https://www.facebook.com/ads/library/?id=1744268173063704}{1744268173063704}) dupa incheierea perioadei de campanie electorala prezidentiala, respectiv dupa ora 18:00 pe 23.11.2024. Postarea promovata prezinta candidatul prezidential Marcel Ciolacu intr-o lumina pozitiva, asociindu-l cu dezvoltarea regionala a Brasovului si transformarea acestuia intr-un HUB industrial si turistic. Materialul contine numar CMF (11240017), confirmand natura sa de propaganda electorala, si a fost distribuit catre un public larg (80.000-89.999 impresii), cu un buget semnificativ (700-799 RON), avand ca efect electoral imbunatatirea imaginii candidatului in perioada alegerilor prezidentiale.
    \item difuzarea unui mesaj de propaganda electorala pentru candidatul prezidential Marcel Ciolacu dupa incheierea perioadei de campanie. Postarea cu ID \href{https://www.facebook.com/ads/library/?id=492335780485449}{492335780485449} contine un indemn explicit la vot pentru data de 24 noiembrie ("sa alegem Calea sigura pentru Brasov, pentru Romania, duminica, 24 noiembrie"), promoveaza candidatul prezidential Marcel Ciolacu, si include numar CMF (11240017), fiind difuzata ca reclama platita pe Facebook si Instagram dupa ora 18:00 pe 23.11.2024. Mesajul are un efect electoral direct, promovand candidatul si solicitand explicit votul pentru acesta in ziua alegerilor.
    \item continuarea propagandei electorale dupa incheierea campaniei electorale pentru alegerile prezidentiale, prin intermediul unei postari sponsorizate pe Facebook (ID: \href{https://www.facebook.com/ads/library/?id=550408210951641}{550408210951641}) care promoveaza explicit candidatul ION-MARCEL CIOLACU si victoria acestuia la alegerile prezidentiale. Postarea, care include numar CMF 11240017, reprezinta propaganda electorala conform Art. 36(7) prin referirea directa la candidat, obiectivul electoral explicit ("va aduce victoria lui Marcel Ciolacu in aceste alegeri"), si adresarea catre publicul larg prin distributie platita cu impact intre 100.000 si 124.999 de afisari, dupa ora 18:00 pe 23.11.2024.
    \item promovarea unui material de propaganda electorala (ID Facebook: \href{https://www.facebook.com/ads/library/?id=551155220867298}{551155220867298}) dupa incheierea perioadei de campanie electorala pentru alegerile prezidentiale. Materialul promovat dupa ora 18:00 pe 23.11.2024 face referire directa la Marcel Ciolacu ca "candidatul PSD la presedintia Romaniei", prezinta programul sau electoral, foloseste sloganuri de campanie ("CALEA SIGURA PENTRU BRASOV SI PENTRU ROMANIA"), si include numar CMF (11240017), demonstrand natura sa de propaganda electorala. Postarea are un efect electoral direct, promovand candidatul si programul sau prezidential, fiind distribuita ca reclama platita pe Facebook si Instagram cu un impact semnificativ (peste 175,000 afisari).
    \item continuarea propagandei electorale dupa incheierea campaniei electorale pentru alegerile prezidentiale, prin difuzarea unei reclame platite pe Facebook (ID: \href{https://www.facebook.com/ads/library/?id=567368295837608}{567368295837608}) care il promoveaza explicit pe Marcel Ciolacu in calitate de candidat la presedintie, prezentand viziunea sa pentru dezvoltarea sportului romanesc. Materialul contine numar CMF (11240017), are caracter explicit electoral prin prezentarea candidatului si a promisiunilor electorale, si a continuat sa fie difuzat dupa ora 18:00 pe 23.11.2024, incalcand astfel perioada legala de campanie electorala. Efectul electoral este evident prin promovarea explicita a candidatului si a programului sau prezidential in domeniul sportului.
\end{enumerate}

\vspace{0.5cm}

\subsection{PSD Cluj-Organizatia Judeteana si WEDEV NOVAT SRL}
Următoarele fapte contravenționale sunt sesizate împotriva acestei entități:

\begin{enumerate}[leftmargin=*, label=\arabic*.)]
    \item publicarea si promovarea dupa data de 23.11.2024, ora 18:00, a unei reclame electorale (ID Facebook: \href{https://www.facebook.com/ads/library/?id=1655170358449919}{1655170358449919}) ce il prezinta pe candidatul prezidential Marcel Ciolacu intr-un context electoral pozitiv, asociindu-l cu promisiuni economice si planuri de reindustrializare. Materialul foloseste un limbaj specific campaniei electorale ("Haideti cu noi pe calea sigura!"), are numar CMF (11240017) si promoveaza explicit candidatul la presedintie in perioada de interdictie, avand un efect electoral direct prin asocierea acestuia cu solutii pentru problemele economice ale tarii.
\end{enumerate}

\vspace{0.5cm}

\subsection{PSD Dolj}
Următoarele fapte contravenționale sunt sesizate împotriva acestei entități:

\begin{enumerate}[leftmargin=*, label=\arabic*.)]
    \item promovarea unui material de propaganda electorala pentru candidatul ION-MARCEL CIOLACU dupa incheierea perioadei de campanie electorala. Postarea cu ID-ul \href{https://www.facebook.com/ads/library/?id=1076877734172969}{1076877734172969} contine mesajul "De ce Marcel Ciolacu?" insotit de hashtag-urile \#MarcelCiolacu \#Echilibru \#Stabilitate, reprezentand propaganda electorala explicita, fiind o reclama platita cu un reach estimat intre 100,001 si 500,000 de persoane, difuzata dupa ora 18:00 pe 23.11.2024. Materialul indeplineste toate criteriile propagandei electorale conform Art. 36(7), referindu-se direct la candidat, avand obiectiv electoral si adresandu-se publicului larg prin intermediul unei reclame platite pe Facebook si Instagram.
    \item publicarea si promovarea unei reclame platite pe Facebook cu ID \href{https://www.facebook.com/ads/library/?id=1086980712697748}{1086980712697748} care constituie propaganda electorala negativa la adresa candidatului prezidential Nicolae-Ionel Ciuca dupa incheierea perioadei de campanie electorala. Postarea, care afirma "Cum sa pierzi la tine acasa? Ciuca a fost invins chiar in localitatea natala la alegerile locale. Se pare ca nici vecinii nu-l mai voteaza!", are un evident scop electoral negativ, incercand sa influenteze opinia alegatorilor prin sublinierea unui esec electoral anterior al candidatului. Actiunea a fost realizata dupa ora 18:00 pe 23.11.2024, intr-un moment in care propaganda electorala este interzisa prin lege, atingand un public intre 15.000 si 19.999 de persoane prin intermediul unei reclame platite.
    \item promovarea unui mesaj electoral platit (ID postare \href{https://www.facebook.com/ads/library/?id=1108531926873914}{1108531926873914}) dupa incheierea perioadei de campanie electorala (dupa ora 18:00 pe 23.11.2024). Postarea contine elemente clare de propaganda electorala, incluzand numar CMF (11240017), mentionarea directa a candidatului Marcel Ciolacu, utilizarea hashtag-urilor de campanie (\#caleasigurapentruRomania), precum si sloganuri de partid specifice campaniei prezidentiale. Mesajul, desi porneste de la realizari locale, este construit pentru a influenta intentiile de vot prin asocierea succesului administrativ local cu partidul si candidatul la presedintie, fiind distribuit contra cost catre un public larg estimat intre 5001-10000 de persoane.
    \item difuzarea de materiale de propaganda electorala dupa incheierea perioadei de campanie, prin postarea cu ID \href{https://www.facebook.com/ads/library/?id=1247733936473803}{1247733936473803} pe Facebook. Postarea, platita si difuzata dupa ora 18:00 pe 23.11.2024, promoveaza explicit candidatul Marcel Ciolacu pentru functia de presedinte, folosind sloganul de campanie "\#CaleaSiguraPentruRomania" si continand numar CMF 11240017. Mesajul are un evident caracter electoral, asociind candidatul cu conducerea tarii ("in fruntea tarii") si promovand imaginea acestuia prin asocierea cu competentele de conducere, atingand un public larg de 45.000-50.000 de persoane prin intermediul platformelor Facebook si Instagram.
    \item difuzarea unui material de propaganda electorala (ID postare Facebook: \href{https://www.facebook.com/ads/library/?id=1334442494200293}{1334442494200293}) dupa incheierea perioadei de campanie electorala pentru alegerile prezidentiale. Postarea, difuzata dupa ora 18:00 pe 23.11.2024, contine un indemn explicit la vot pentru candidatul Marcel Ciolacu ("duminica, 24 noiembrie, votam Marcel Ciolacu"), face promisiuni electorale si include numar CMF (11240017), demonstrand natura sa de propaganda electorala. Materialul a fost distribuit ca reclama platita pe Facebook si Instagram, cu un impact estimat intre 10.000 si 14.999 de afisari, reprezentand o incercare clara de a influenta votul in ziua precedenta alegerilor prezidentiale.
    \item promovarea unui material de propaganda electorala (ID Facebook: \href{https://www.facebook.com/ads/library/?id=1355346872396754}{1355346872396754}) dupa incheierea perioadei de campanie electorala pentru alegerile prezidentiale. Materialul, publicat ca reclama platita dupa ora 18:00 pe 23.11.2024, promoveaza explicit candidatul prezidential Marcel Ciolacu si proiectul sau de tara, folosind sloganuri de campanie ("calea sigura"), prezentand promisiuni electorale in domeniul sanatatii si continand numar CMF (11240017), elementele care confirma natura sa de propaganda electorala. Efectul electoral urmarit este clar de influentare a votului in favoarea candidatului PSD la presedintie, prin prezentarea realizarilor si promisiunilor acestuia in domeniul sanatatii.
    \item publicarea si promovarea unui mesaj de propaganda electorala dupa incheierea perioadei de campanie, respectiv dupa ora 18:00 pe 23.11.2024. Postarea cu ID-ul \href{https://www.facebook.com/ads/library/?id=1736664547156289}{1736664547156289} il promoveaza direct pe candidatul prezidential Marcel Ciolacu, folosind termeni laudativi precum "Lider, strateg, antrenor" si analogii sportive ("Champions League") pentru a-i construi o imagine pozitiva si a influenta preferintele electorale ale votantilor. Postarea este sponsorizata, avand un buget intre 100-199 RON si un impact semnificativ de 10.000-14.999 impresii, demonstrand intentia clara de a influenta procesul electoral in afara perioadei legale de campanie.
    \item promovarea unui material de propaganda electorala pentru candidatul ION-MARCEL CIOLACU dupa incheierea campaniei electorale. Postarea cu ID-ul \href{https://www.facebook.com/ads/library/?id=1778143106258366}{1778143106258366}, publicata dupa ora 18:00 pe 23.11.2024, promoveaza realizarile candidatului folosind hashtag-uri precum \#MarcelCiolacu si \#psdscrieistorie, cu scopul clar de a influenta alegatorii prin evidentierea realizarilor acestuia in calitate de prim-ministru, in special aderarea la Schengen. Postarea foloseste un ton laudativ si exclamativ ("O victorie bine meritata pentru romani!", "Cu diplomatie si ambitie, Marcel Ciolacu indeplineste un mare vis al romanilor!"), depasind simpla informare si constituind propaganda electorala platita, cu un impact estimat intre 100.001 si 500.000 de persoane.
    \item promovarea unui material de propaganda electorala pentru candidatul ION-MARCEL CIOLACU dupa incheierea perioadei de campanie electorala. Materialul publicitar cu ID-ul \href{https://www.facebook.com/ads/library/?id=379816521808234}{379816521808234}, postat dupa ora 18:00 pe 23.11.2024, foloseste realizarile guvernamentale privind aderarea la Schengen pentru a promova candidatura la presedintie a lui Marcel Ciolacu, continand explicit mesajul "Calea sigura pentru Romania este \#MarcelCiolacuPresedinte". Postarea include numar CMF (11240017), are obiectiv electoral explicit si vizeaza influentarea votului prin asocierea realizarilor guvernamentale cu candidatura la presedintie, fiind promovata ca reclama platita cu un reach estimat intre 100.001 si 500.000 de persoane.
    \item promovarea unui material publicitar platit pe Facebook si Instagram (ID: \href{https://www.facebook.com/ads/library/?id=419365714570211}{419365714570211}) dupa ora 18:00 pe 23.11.2024, care face referire directa la candidatul prezidential Marcel Ciolacu. Postarea, desi aparent inofensiva, reprezinta o forma de propaganda electorala prin mentinerea vizibilitatii candidatului in perioada restrictionata, utilizand hashtag-ul \#MarcelCiolacu si fiind sponsorizata de catre partidul sau. Efectul electoral este evident prin umanizarea candidatului si mentinerea prezentei sale in spatiul public prin continut platit in perioada in care propaganda electorala este interzisa.
    \item publicarea si promovarea unei reclame platite (ID: \href{https://www.facebook.com/ads/library/?id=536199915962104}{536199915962104}) dupa ora 18:00 pe 23.11.2024, care constituie propaganda electorala negativa la adresa candidatului Nicolae-Ionel Ciuca. Postarea, care a avut un reach de peste 60.000 de afisari si un buget de aproximativ 350 RON, sugereaza in mod explicit ca candidatul nu ar fi demn de incredere, utilizand formulari precum "Cum sa pierzi la tine acasa?" si "nici vecinii nu-l mai voteaza", cu scopul clar de a influenta negativ intentiile de vot ale alegatorilor in perioada in care propaganda electorala este interzisa prin lege.
    \item continuarea propagandei electorale dupa incheierea acesteia, prin postarea cu ID \href{https://www.facebook.com/ads/library/?id=554122070651046}{554122070651046} pe Facebook dupa ora 18:00 pe 23.11.2024. Postarea reprezinta propaganda electorala negativa la adresa candidatului Nicolae-Ionel Ciuca, folosind acuzatii directe privind cheltuirea banilor publici, sugerand retragerea sa din cursa prezidentiala si punand la indoiala integritatea sa. Postarea include numar CMF (1124001), demonstrand natura sa de propaganda electorala, iar prin distributia platita pe Facebook si Instagram catre un public tinta de peste 100.000 de persoane, urmareste in mod clar influentarea votului in alegerile prezidentiale. Efectul electoral urmarit este diminuarea increderii alegatorilor in candidatul PNL si descurajarea votarii acestuia.
    \item publicarea si promovarea unei reclame platite pe Facebook (ID: \href{https://www.facebook.com/ads/library/?id=596284366168045}{596284366168045}) dupa ora 18:00 pe 23.11.2024, in care se face in mod explicit propaganda electorala pentru candidatul la presedintie Marcel Ciolacu, prin indemnul direct "Votati Marcel Ciolacu, presedintele Romaniei!". Postarea, care a ajuns la un public de peste 1000 de persoane, reprezinta o continuare clara a propagandei electorale pentru alegerile prezidentiale dupa incheierea perioadei legale de campanie, avand un evident obiectiv electoral de influentare a votului pentru functia de presedinte al Romaniei. Caracterul de propaganda este demonstrat prin mesajul explicit de indemn la vot pentru un candidat specific la functia de presedinte, precum si prin natura platita a comunicarii, care ii mareste substantial impactul si audienta.
    \item continuarea propagandei electorale dupa incheierea campaniei, prin intermediul unei postari sponsorizate pe Facebook (ID: \href{https://www.facebook.com/ads/library/?id=873431361659277}{873431361659277}) care indeamna explicit la vot pentru data de 24 noiembrie si promoveaza sloganul campaniei prezidentiale "Calea Sigura Pentru Romania". Postarea contine numar CMF (11240017), face referire directa la alegerea presedintelui si include indemnuri explicite la vot pentru datele de 24 noiembrie, 1 si 8 decembrie, fiind difuzata dupa ora 18:00 pe 23.11.2024. Efectul electoral urmarit este influentarea alegatorilor in favoarea candidatului PSD la presedintie, prin utilizarea unei comunicari electorale platite in perioada in care acest lucru este interzis de lege.
    \item publicarea si promovarea unei reclame platite pe Facebook/Instagram (ID: \href{https://www.facebook.com/ads/library/?id=925092865738975}{925092865738975}) dupa incheierea perioadei de campanie electorala, ce contine un indemn explicit de a vota candidatul Marcel Ciolacu in data de 24 noiembrie, folosind materiale de campanie oficiale (avand cod CMF:11240017). Postarea constituie propaganda electorala prin prezenta unui indemn direct de vot ("Duminica, 24 noiembrie, votam Marcel Ciolacu"), folosirea sloganurilor de campanie si promisiunilor electorale ("Calea Sigura catre o Romanie mai verde, mai curata, mai sanatoasa"), fiind distribuita dupa ora 18:00 pe 23.11.2024, cu un impact semnificativ demonstrat de numarul mare de impresii (4000-4999) si audienta estimata (100,001-500,000 persoane).
    \item continuarea propagandei electorale dupa incheierea campaniei electorale pentru alegerile prezidentiale, prin promovarea candidatului Marcel Ciolacu in postarea cu ID \href{https://www.facebook.com/ads/library/?id=936257145070434}{936257145070434}. Postarea, efectuata dupa ora 18:00 pe 23.11.2024, foloseste realizarile guvernamentale privind aderarea la Schengen pentru a promova candidatul la presedintie, folosind explicit hashtag-ul \#MarcelCiolacuPresedinte si avand numar CMF (11240017), demonstrand clar intentia de propaganda electorala. Efectul electoral urmarit este de a influenta alegatorii prin asocierea realizarilor administrative cu candidatura la presedintie, depasind astfel cadrul legal permis pentru comunicare dupa incheierea campaniei electorale.
    \item difuzarea unui mesaj de propaganda electorala dupa incheierea perioadei de campanie, folosind postarea cu ID-ul \href{https://www.facebook.com/ads/library/?id=957521712887209}{957521712887209} pe Facebook. Postarea, realizata dupa ora 18:00 pe 23.11.2024, promoveaza explicit candidatul prezidential Marcel Ciolacu si PSD, folosind sloganuri de campanie precum "\#CaleaSiguraPentruRomania" si continand numarul oficial de material electoral CMF 11240017. Mesajul are un efect electoral clar, prezentand candidatul ca parte a "echipei complete in fruntea tarii", cu intentia evidenta de a influenta alegatorii. Postarea este platita si a ajuns la un public larg de 15.000-19.999 persoane, demonstrand caracterul sau de propaganda electorala.
\end{enumerate}

\vspace{0.5cm}

\subsection{PSD Dolj, prin consilierul local Ionut Mlagiu}
Următoarele fapte contravenționale sunt sesizate împotriva acestei entități:

\begin{enumerate}[leftmargin=*, label=\arabic*.)]
    \item continuarea propagandei electorale dupa incheierea campaniei pentru alegerile prezidentiale, folosind o postare sponsorizata pe Facebook (ID: \href{https://www.facebook.com/ads/library/?id=4246401465586841}{4246401465586841}) dupa data de 23.11.2024, ora 18:00. Postarea indeamna in mod direct cetatenii sa voteze candidatul PSD Marcel Ciolacu ca "viitorul Presedinte al Romaniei", avand un efect electoral direct si explicit prin folosirea unor indemnuri clare la vot ("veniti la vot si alegeti candidatul PSD Marcel Ciolacu"). Postarea are caracter de propaganda electorala conform Art. 36(7), fiind folosita pentru promovarea unui candidat specific la presedintie, adresandu-se publicului larg prin intermediul unei reclame platite pe platformele de social media, depasind astfel limitele comunicarii informative.
\end{enumerate}

\vspace{0.5cm}

\subsection{PSD Giurgiu}
Următoarele fapte contravenționale sunt sesizate împotriva acestei entități:

\begin{enumerate}[leftmargin=*, label=\arabic*.)]
    \item publicarea si promovarea unei reclame platite pe Facebook si Instagram (ID: \href{https://www.facebook.com/ads/library/?id=1241527783630357}{1241527783630357}) dupa incheierea perioadei de campanie electorala pentru alegerile prezidentiale, respectiv dupa ora 18:00 pe 23.11.2024. Postarea constituie propaganda electorala evidenta pentru candidatul prezidential Marcel Ciolacu, continand indemnuri directe la vot ("E timpul pentru Marcel Ciolacu - presedinte"), utilizand un buget semnificativ de promovare (1000-1499 RON) pentru a atinge un public larg (125,000-150,000 impresii), si folosind un mesaj electoral explicit care vizeaza influentarea votului pentru alegerile prezidentiale. Postarea include si numarul CMF 11240017, confirmand natura sa de propaganda electorala.
    \item continuarea propagandei electorale dupa incheierea campaniei electorale pentru alegerile prezidentiale, prin distribuirea dupa ora 18:00 pe 23.11.2024 a unei postari sponsorizate (ID: \href{https://www.facebook.com/ads/library/?id=1411711393139703}{1411711393139703}) care il prezinta pe candidatul Marcel Ciolacu ca "Noul presedinte" si face promisiuni electorale specifice despre sistemul de irigatii. Postarea are scop electoral explicit, foloseste numar CMF (11240017), include sloganuri de campanie ("PSD este calea sigura pentru Romania"), si face promisiuni directe catre alegatori despre actiunile post-electorale, avand un impact direct asupra procesului electoral prezidential prin influentarea votului in favoarea candidatului PSD.
    \item promovarea unui mesaj electoral platit pe Facebook (ID \href{https://www.facebook.com/ads/library/?id=2394020644263136}{2394020644263136}) care face referire directa la alegerile prezidentiale dupa incheierea perioadei de campanie, prezentandu-l pe Marcel Ciolacu drept "viitorul presedinte". Postarea, difuzata dupa ora 18:00 pe 23.11.2024, reprezinta o forma de propaganda electorala pentru alegerile prezidentiale, avand un impact semnificativ cu peste 250.000 de afisari si un buget de aproximativ 2.750 RON, influentand in mod direct optiunile electoratului prin prezentarea candidatului Marcel Ciolacu ca viitor castigator al alegerilor prezidentiale, depasind astfel cadrul legal al campaniei electorale pentru Senat.
\end{enumerate}

\vspace{0.5cm}

\subsection{PSD Gorj}
Următoarele fapte contravenționale sunt sesizate împotriva acestei entități:

\begin{enumerate}[leftmargin=*, label=\arabic*.)]
    \item publicarea si promovarea unei reclame platite pe Facebook (ID: \href{https://www.facebook.com/ads/library/?id=1068055818117576}{1068055818117576}) dupa data de 23.11.2024, ora 18:00, care contine propaganda electorala explicita pentru candidatul prezidential Marcel Ciolacu. Postarea include indemnuri directe la vot ("Votul tau conteaza"), promoveaza explicit candidatura lui Marcel Ciolacu la presedintie ("Marcel Ciolacu pentru Presedintia Romaniei"), si foloseste un limbaj specific campaniei electorale pentru a influenta alegatorii. Efectul electoral este evident prin sustinerea directa a candidatului si solicitarea explicita de voturi, depasind sfera comunicarii obisnuite si constituind propaganda electorala in perioada in care aceasta este interzisa prin lege.
    \item continuarea propagandei electorale dupa incheierea campaniei pentru alegerile prezidentiale, prin postarea platita pe Facebook cu ID \href{https://www.facebook.com/ads/library/?id=3845124615805481}{3845124615805481}, publicata dupa ora 18:00 pe 23.11.2024. Postarea promoveaza explicit candidatul Marcel Ciolacu pentru functia de presedinte, folosind hashtag-uri electorale (\#MarcelCiolacu\_PRESEDINTE), indemnand direct la vot pentru acesta, avand un efect electoral direct prin atingerea unui public tinta de peste 7000 de persoane, constituind astfel propaganda electorala conform Art. 36(7) din Legea 334/2006, fiind realizata si distribuita dupa incheierea perioadei legale de campanie.
    \item promovarea unui mesaj de propaganda electorala pentru candidatul la presedintie Marcel Ciolacu dupa incheierea perioadei de campanie, prin intermediul unei reclame platite pe Facebook (ID: \href{https://www.facebook.com/ads/library/?id=888133213528318}{888133213528318}) ce continua sa fie activa dupa ora 18:00 pe 23.11.2024. Postarea contine in mod explicit indemnul de sustinere a candidatului la presedintie Marcel Ciolacu ("vom sustine... pe domnul presedinte Marcel Ciolacu la alegerile prezidentiale"), reprezinta propaganda electorala conform Art. 36(7) prin prezenta CMF-ului, targetarea publicului larg prin reclama platita, si obiectivul electoral explicit de influentare a votului pentru alegerile prezidentiale.
\end{enumerate}

\vspace{0.5cm}

\subsection{PSD Iasi}
Următoarele fapte contravenționale sunt sesizate împotriva acestei entități:

\begin{enumerate}[leftmargin=*, label=\arabic*.)]
    \item promovarea unui material de propaganda electorala (ID Facebook: \href{https://www.facebook.com/ads/library/?id=1795323061006990}{1795323061006990}) dupa incheierea perioadei de campanie electorala pentru alegerile prezidentiale. Materialul, difuzat dupa ora 18:00 pe 23.11.2024, il promoveaza explicit pe candidatul prezidential Marcel Ciolacu, continand elemente specifice de propaganda electorala (numar CMF 11240017), mesaje de sustinere directa ("NOI suntem calea sigura pentru Iasi"), si realizari ale acestuia in calitate de premier, cu scopul clar de a influenta alegatorii in favoarea sa. Postarea a fost promovata ca reclama platita pe Facebook si Instagram, cu un impact estimat intre 100.000 si 124.999 de afisari, demonstrand intentia clara de a influenta un numar cat mai mare de alegatori.
    \item promovarea unui material de propaganda electorala (ID Facebook: \href{https://www.facebook.com/ads/library/?id=586428570613663}{586428570613663}) dupa incheierea perioadei de campanie electorala pentru alegerile prezidentiale. Materialul, publicat ca reclama platita pe Facebook si Instagram, il promoveaza direct pe candidatul prezidential Marcel Ciolacu, evidentiind realizarile sale ca premier si pozitionandu-l favorabil in fata electoratului. Postarea contine numar CMF (11240017), are caracter electoral explicit, si a fost difuzata ca reclama platita dupa ora 18:00 pe 23.11.2024, atingand intre 200.000 si 250.000 de cetateni, cu un buget de aproximativ 850 RON.
    \item promovarea unui mesaj electoral platit pe Facebook (ID: \href{https://www.facebook.com/ads/library/?id=8590905931028294}{8590905931028294}) care continua propaganda electorala pentru candidatul la presedintie Marcel Ciolacu dupa incheierea perioadei legale de campanie. Postarea, care include numar CMF 11240017, contine explicit sustinerea candidatului la presedintie ("Genoveva Farcas il sustine pe Marcel Ciolacu pentru functia de presedinte al Romaniei"), fiind difuzata ca reclama platita pe Facebook si Instagram, cu un impact semnificativ (60.000-70.000 impresii). Efectul electoral este evident prin incercarea de a influenta votul prin promovarea si sustinerea explicita a candidatului PSD la presedintie dupa ora 18:00 pe 23.11.2024.
\end{enumerate}

\vspace{0.5cm}

\subsection{PSD Maramures}
Următoarele fapte contravenționale sunt sesizate împotriva acestei entități:

\begin{enumerate}[leftmargin=*, label=\arabic*.)]
    \item continuarea propagandei electorale pentru candidatul prezidential Marcel Ciolacu dupa incheierea perioadei de campanie, prin postarea cu ID \href{https://www.facebook.com/ads/library/?id=1120036146358587}{1120036146358587} promovata pe Facebook si Instagram dupa ora 18:00 pe 23.11.2024. Postarea promoveaza "proiectele de tara propuse de Marcel Ciolacu" si foloseste sloganul campaniei prezidentiale "\#CaleaSiguraPentruRomania", avand un efect electoral direct prin promovarea imaginii si proiectelor candidatului la presedintie. Postarea este marcata cu cod CMF 11240017, confirmand natura sa de propaganda electorala, si a fost promovata platit catre un public larg de 45.000-50.000 persoane.
    \item publicarea si promovarea platita a postarii cu ID \href{https://www.facebook.com/ads/library/?id=1667935190435016}{1667935190435016} dupa incheierea perioadei de campanie electorala pentru alegerile prezidentiale. Postarea promoveaza realizarile si actiunile candidatului prezidential Marcel Ciolacu, folosind pozitia sa actuala de premier pentru a genera capital electoral, si include un indemn direct catre alegatori de a-i sustine pe "oamenii care au dovedit ca sunt alaturi de noi". Postarea include numar CMF (11240017), confirmand natura sa de propaganda electorala, si a fost difuzata dupa ora 18:00 pe 23.11.2024, incalcand astfel prevederile legale privind incheierea campaniei electorale.
    \item continuarea propagandei electorale dupa incheierea campaniei electorale pentru alegerile prezidentiale, prin intermediul unei postari sponsorizate pe Facebook (ID: \href{https://www.facebook.com/ads/library/?id=3855023434716826}{3855023434716826}) care promoveaza explicit candidatura lui Marcel Ciolacu la presedintie, utilizand hashtag-ul "\#MarcelCiolacupresedinte" si facand promisiuni electorale legate de dezvoltarea economica sub presedintia sa. Postarea, care include numarul CMF 11240017, reprezinta in mod clar material de propaganda electorala conform Art. 36(7), avand obiectiv electoral explicit si fiind difuzata dupa ora 18:00 pe 23.11.2024, cand propaganda electorala pentru alegerile prezidentiale trebuia sa inceteze.
\end{enumerate}

\vspace{0.5cm}

\subsection{PSD Maramures, prin intermediul paginii "Dancus Ioan Doru"}
Următoarele fapte contravenționale sunt sesizate împotriva acestei entități:

\begin{enumerate}[leftmargin=*, label=\arabic*.)]
    \item continuarea propagandei electorale pentru candidatul la presedintie Marcel Ciolacu dupa incheierea perioadei legale de campanie, intr-o postare sponsorizata cu ID \href{https://www.facebook.com/ads/library/?id=1307053713792365}{1307053713792365}. Postarea prezinta explicit "proiectul pentru Romania propus de Marcel Ciolacu, candidatul nostru la Presedintia Romaniei" si descrie programul sau electoral, avand efect de influentare a votului pentru alegerile prezidentiale. Postarea contine numar CMF (11240017) si este distribuita dupa ora 18:00 pe 23.11.2024, avand un impact semnificativ prin atingerea unui public de peste 125.000 de persoane prin plata pe platformele Facebook si Instagram.
\end{enumerate}

\vspace{0.5cm}

\subsection{PSD Neamt, prin Tifui Dumitru Bogdan}
Următoarele fapte contravenționale sunt sesizate împotriva acestei entități:

\begin{enumerate}[leftmargin=*, label=\arabic*.)]
    \item continuarea propagandei electorale dupa incheierea campaniei oficiale, manifestata prin postarea cu ID \href{https://www.facebook.com/ads/library/?id=1274716527004296}{1274716527004296} pe platforma Facebook. Postarea, publicata si promovata dupa ora 18:00 pe 23.11.2024, contine un mesaj explicit de sustinere si indemn la vot pentru candidatul Marcel Ciolacu, folosind formulari precum "va invit sa votati Marcel Ciolacu presedinte" si "alegeti Calea sigura pentru judetul Neamt si Romania". Postarea include numar CMF (CMF-11240017), confirmand natura sa de propaganda electorala, si vizeaza influentarea directa a votului prin enumerarea realizarilor si promisiunilor electorale, depasind cadrul legal permis pentru perioada post-campanie.
\end{enumerate}

\vspace{0.5cm}

\subsection{PSD OLT}
Următoarele fapte contravenționale sunt sesizate împotriva acestei entități:

\begin{enumerate}[leftmargin=*, label=\arabic*.)]
    \item continuarea propagandei electorale dupa incheierea campaniei, prin postarea cu ID \href{https://www.facebook.com/ads/library/?id=928493602047350}{928493602047350} pe Facebook, promovata dupa ora 18:00 pe 23.11.2024. Postarea contine un indemn explicit la vot pentru candidatul Marcel Ciolacu la functia de presedinte, utilizand mesaje precum "duminica, 24 noiembrie ALEGEM CALEA SIGURA PENTRU DEZVOLTAREA AGRICULTURII ROMANESTI, MARCEL CIOLACU, PRESEDINTE", fiind o forma clara de propaganda electorala care vizeaza influentarea votului in ziua alegerilor, adresandu-se unui public larg format din fermieri si tineri din agricultura, cu impact electoral direct prin promovarea platita pe retelele de socializare Facebook si Instagram.
\end{enumerate}

\vspace{0.5cm}

\subsection{PSD Pascani}
Următoarele fapte contravenționale sunt sesizate împotriva acestei entități:

\begin{enumerate}[leftmargin=*, label=\arabic*.)]
    \item publicarea si promovarea unei reclame platite pe Facebook (ID: \href{https://www.facebook.com/ads/library/?id=581864764347888}{581864764347888}) dupa ora 18:00 pe 23.11.2024, care constituie propaganda electorala pentru candidatul prezidential Marcel Ciolacu. Postarea, care include cod CMF 11240017, promoveaza explicit imaginea lui Marcel Ciolacu ca viitor conducator al tarii prin afirmatia "Cu o guvernare social-democrata si Marcel Ciolacu in fruntea tarii, Romania are sansa unui progres pe termen lung", reprezentand astfel un mesaj cu caracter electoral explicit pentru alegerile prezidentiale, in perioada in care campania electorala este inchisa.
    \item continuarea propagandei electorale pentru candidatul ION-MARCEL CIOLACU dupa incheierea perioadei legale de campanie, prin postarea cu ID \href{https://www.facebook.com/ads/library/?id=920139349642495}{920139349642495} pe Facebook. Postarea, publicata dupa ora 18:00 pe 23.11.2024, contine elemente clare de propaganda electorala, incluzand numar CMF (11240017), indemnuri directe la vot ("Votam Marcel Ciolacu Presedinte!"), promisiuni electorale si promovarea realizarilor candidatului, fiind distribuita ca reclama platita catre un public larg. Postarea are un evident caracter de propaganda electorala, vizand influentarea intentionata a votului pentru alegerile prezidentiale din 24 noiembrie 2024.
\end{enumerate}

\vspace{0.5cm}

\subsection{PSD Tulcea}
Următoarele fapte contravenționale sunt sesizate împotriva acestei entități:

\begin{enumerate}[leftmargin=*, label=\arabic*.)]
    \item difuzarea unui material de propaganda electorala (ID postare Facebook: \href{https://www.facebook.com/ads/library/?id=873014781330956}{873014781330956}) dupa incheierea perioadei de campanie electorala pentru alegerile prezidentiale. Materialul, marcat cu cod CMF 11240017, promoveaza realizarile guvernelor PSD, inclusiv cel condus de candidatul la presedintie Marcel Ciolacu, prezentand intr-o lumina pozitiva activitatea acestuia si a partidului sau, cu scopul evident de a influenta optiunile electorale ale cetatenilor din judetul Tulcea si nu numai. Postarea a fost promovata dupa ora 18:00 pe 23.11.2024, incalcand astfel prevederile legale privind incetarea campaniei electorale.
\end{enumerate}

\vspace{0.5cm}

\subsection{PSDMM}
Următoarele fapte contravenționale sunt sesizate împotriva acestei entități:

\begin{enumerate}[leftmargin=*, label=\arabic*.)]
    \item continuarea propagandei electorale dupa incheierea campaniei electorale, prin intermediul unei postari sponsorizate pe Facebook (ID: \href{https://www.facebook.com/ads/library/?id=1650154845845951}{1650154845845951}) care promoveaza explicit candidatul ION-MARCEL CIOLACU la functia de presedinte, folosind hashtag-ul "\#MarcelCiolacupresedinte" si materiale de campanie oficiale (CMF 11240017). Postarea, care a avut un impact semnificativ (35,000-40,000 impresii) si continua sa fie activa dupa ora 18:00 pe 23.11.2024, reprezinta o incalcare clara a legislatiei electorale, avand un evident obiectiv electoral de influentare a votului in favoarea candidatului PSD la presedintie.
    \item promovarea unui mesaj de propaganda electorala pentru candidatul ION-MARCEL CIOLACU la functia de presedinte al Romaniei, dupa incheierea perioadei de campanie electorala. Postarea cu ID-ul \href{https://www.facebook.com/ads/library/?id=1986893058390371}{1986893058390371} contine elementele specifice propagandei electorale, inclusiv numarul CMF 11240017, sloganul "Calea sigura pentru Romania!" si mentiunea explicita "Marcel Ciolacu, presedinte!", fiind difuzata ca reclama platita pe Facebook si Instagram dupa ora 18:00 pe 23.11.2024, cu un impact semnificativ (50,000-59,999 impresii). Mesajul combina in mod deliberat teme de politici publice cu promovarea electorala a candidatului la presedintie, avand un clar obiectiv electoral de influentare a votului.
\end{enumerate}

\vspace{0.5cm}

\subsection{Pagina "Alegerea de AUR"}
Următoarele fapte contravenționale sunt sesizate împotriva acestei entități:

\begin{enumerate}[leftmargin=*, label=\arabic*.)]
    \item continuarea propagandei electorale dupa incheierea perioadei legale de campanie, prin promovarea unui mesaj electoral explicit pentru candidatul George Nicolae Simion la functia de presedinte. Postarea cu ID \href{https://www.facebook.com/ads/library/?id=1123473639412728}{1123473639412728}, publicata si promovata dupa ora 18:00 pe 23.11.2024, foloseste hashtag-uri precum "\#VoteazaAUR" si promoveaza "Planul Simion", reprezentand o incercare clara de a influenta alegatorii in favoarea candidatului AUR la presedintie, intr-o perioada in care propaganda electorala este interzisa prin lege.
\end{enumerate}

\vspace{0.5cm}

\subsection{Pagina "Fruncea"}
Următoarele fapte contravenționale sunt sesizate împotriva acestei entități:

\begin{enumerate}[leftmargin=*, label=\arabic*.)]
    \item continuarea propagandei electorale dupa incheierea perioadei legale de campanie, distribuind un mesaj platit pe Facebook si Instagram (ID postare: \href{https://www.facebook.com/ads/library/?id=1087378399405279}{1087378399405279}) care vizeaza in mod direct candidatii la presedintie Mircea Geoana si Cristian Diaconescu, incercand sa ii discrediteze si sa influenteze negativ optiunea de vot a alegatorilor prin asocierea acestora cu PSD si punerea la indoiala a statutului lor de independenti, folosind indemnul "sa nu uitam trecutul". Postarea, difuzata dupa ora 18:00 pe 23.11.2024, reprezinta propaganda electorala conform Art. 36(7), avand obiectiv electoral clar, adresandu-se publicului larg prin intermediul unei reclame platite cu impact semnificativ (peste 50.000 de afisari) si depasind limitele activitatii jurnalistice de informare a publicului.
    \item publicarea si promovarea unei reclame platite pe Facebook si Instagram (ID: \href{https://www.facebook.com/ads/library/?id=386674257771104}{386674257771104}) dupa ora 18:00 pe 23.11.2024, ce constituie propaganda electorala pentru alegerile prezidentiale. Postarea face referire directa la candidatul George Nicolae Simion si strategii electorale pentru turul 2 al alegerilor prezidentiale, utilizand un mesaj de natura sa influenteze opinia publica ("INCREDIBIL CE FACE PSD pentru a-l avea pe G. Simion in turul 2"), atingand un public tinta de peste 15.000 de persoane prin promovare platita, depasind astfel limitele comunicarii permise in perioada de restrictie electorala.
\end{enumerate}

\vspace{0.5cm}

\subsection{Pagina "Lasa-i acasa"}
Următoarele fapte contravenționale sunt sesizate împotriva acestei entități:

\begin{enumerate}[leftmargin=*, label=\arabic*.)]
    \item difuzarea de continut de propaganda electorala dupa incheierea perioadei de campanie, pe data de 23.11.2024. Postarea cu ID-ul \href{https://www.facebook.com/ads/library/?id=889814966684713}{889814966684713} reprezinta propaganda electorala negativa la adresa candidatului George Nicolae Simion, facand acuzatii de evaziune fiscala si folosind un ton denigrator, cu scopul clar de a influenta negativ intentiile de vot ale alegatorilor. Postarea este sponsorizata, targetand intre 100.001 si 500.000 de persoane pe platformele Facebook si Instagram, demonstrand intentia clara de a influenta un numar mare de alegatori prin continut negativ la adresa unui candidat la prezidentiale, dupa ora 18:00 pe 23.11.2024.
    \item continuarea propagandei electorale dupa incheierea perioadei legale, manifestata prin publicarea si promovarea activa a unei reclame negative (ID: \href{https://www.facebook.com/ads/library/?id=927543008740969}{927543008740969}) ce vizeaza direct candidatul la presedintie George-Nicolae Simion, dupa ora 18:00 pe 23.11.2024. Postarea foloseste caracterizari negative si comparative pentru a influenta comportamentul electoral al publicului larg, avand un reach estimat intre 400.000 si 450.000 de afisari, cu o investitie substantiala in promovare de peste 2.500 RON, demonstrand intentia clara de a influenta procesul electoral prin denigrarea candidatului si sugerarea indirect a modului in care ar trebui sa voteze alegatorii.
\end{enumerate}

\vspace{0.5cm}

\subsection{Partidul AUR}
Următoarele fapte contravenționale sunt sesizate împotriva acestei entități:

\begin{enumerate}[leftmargin=*, label=\arabic*.)]
    \item publicarea si promovarea unei reclame platite pe Facebook (ID: \href{https://www.facebook.com/ads/library/?id=1261056901921116}{1261056901921116}) dupa data de 23.11.2024, ora 18:00, care contine elemente clare de propaganda electorala pentru candidatul la presedintie George Simion. Postarea include hashtag-ul "\#GeorgeSimionPresedinte", promoveaza explicit candidatura sa la presedintie, si contine numarul CMF 11240014, specific materialelor de propaganda electorala. Efectul electoral al postarii este evident prin promovarea directa a candidatului AUR la presedintie si criticarea adversarilor politici, depasind sfera permisa a campaniei parlamentare si intrand in sfera propagandei electorale prezidentiale interzise dupa incheierea campaniei.
\end{enumerate}

\vspace{0.5cm}

\subsection{Partidul Alianta pentru Unirea Romanilor}
Următoarele fapte contravenționale sunt sesizate împotriva acestei entități:

\begin{enumerate}[leftmargin=*, label=\arabic*.)]
    \item promovarea unui material de propaganda electorala pentru candidatul la prezidentiale George Nicolae Simion dupa incheierea perioadei de campanie, prin intermediul unei reclame platite pe Facebook (ID: \href{https://www.facebook.com/ads/library/?id=1135231278123402}{1135231278123402}) care promoveaza "Planul Simion pentru Ilfov" si face referire directa la viziunea si programul candidatului la presedintie. Postarea, activa dupa ora 18:00 pe 23.11.2024, are un caracter electoral evident, fiind distribuita ca reclama platita catre un public larg (peste 1 milion de utilizatori potentiali), continand si numar CMF (11240014), specific materialelor de campanie electorala. Efectul electoral este evident prin asocierea pozitiva a numelui candidatului cu dezvoltarea regiunii si promisiunea ridicarii Romaniei la nivel european.
\end{enumerate}

\vspace{0.5cm}

\subsection{Partidul Alianta pentru Unirea Romanilor (AUR)}
Următoarele fapte contravenționale sunt sesizate împotriva acestei entități:

\begin{enumerate}[leftmargin=*, label=\arabic*.)]
    \item difuzarea unui mesaj de propaganda electorala dupa incheierea perioadei de campanie, respectiv dupa ora 18:00 pe 23.11.2024. Postarea cu ID-ul \href{https://www.facebook.com/ads/library/?id=533765789578462}{533765789578462} contine elementele clare ale propagandei electorale: foloseste un numar CMF (11240014), promoveaza direct candidatul George Simion pentru functia de presedinte ("George Simion presedinte!"), utilizeaza simboluri nationale in context electoral, si este difuzata ca reclama platita cu impact semnificativ (30.000-35.000 impresii). Mesajul are un obiectiv electoral clar si direct, vizand influentarea votului in favoarea candidatului AUR la alegerile prezidentiale.
\end{enumerate}

\vspace{0.5cm}

\subsection{Partidul National Liberal}
Următoarele fapte contravenționale sunt sesizate împotriva acestei entități:

\begin{enumerate}[leftmargin=*, label=\arabic*.)]
    \item promovarea unui material de propaganda electorala (ID post Facebook: \href{https://www.facebook.com/ads/library/?id=1857303024677817}{1857303024677817}) dupa incheierea campaniei electorale prezidentiale, material ce promoveaza explicit candidatul Nicolae Ciuca in calitate de presedinte ("Cu Nicolae Ciuca Presedinte"), avand efect electoral direct prin prezentarea unei viziuni si a unor promisiuni cu acesta in functia de presedinte. Materialul, publicat si promovat dupa ora 18:00 pe 23.11.2024, contine numar CMF (11240002) si reprezinta o continuare nepermisa a propagandei electorale pentru alegerile prezidentiale intr-o perioada in care acest lucru este interzis prin lege.
    \item promovarea candidatului la presedintie Nicolae Ciuca intr-o postare sponsorizata pe Facebook (ID: \href{https://www.facebook.com/ads/library/?id=516328028047409}{516328028047409}) dupa incheierea perioadei de campanie electorala. Postarea contine referinta directa "Cu Nicolae Ciuca Presedinte" intr-un context electoral explicit, cu promisiuni si indemnuri la vot, fiind in mod clar destinata sa influenteze optiunea de vot a alegatorilor pentru alegerile prezidentiale. Postarea include numar CMF (11240002), confirmand natura sa de propaganda electorala, si a fost difuzata dupa ora 18:00 pe 23.11.2024, incalcand astfel prevederile legale privind incetarea propagandei electorale.
\end{enumerate}

\vspace{0.5cm}

\subsection{Partidul National Liberal (PNL)}
Următoarele fapte contravenționale sunt sesizate împotriva acestei entități:

\begin{enumerate}[leftmargin=*, label=\arabic*.)]
    \item promovarea unui material de propaganda electorala (ID postare Facebook: \href{https://www.facebook.com/ads/library/?id=581946374325171}{581946374325171}) dupa incheierea perioadei de campanie electorala pentru alegerile prezidentiale. Materialul, publicat si promovat dupa ora 18:00 pe 23.11.2024, face referire directa la candidatul Nicolae Ciuca in calitate de presedinte ("Cu Nicolae Ciuca Presedinte"), avand scop electoral explicit si fiind distribuit catre public larg prin intermediul unei reclame platite pe Facebook. Materialul este marcat cu cod CMF 11240002, confirmand natura sa de propaganda electorala, si urmareste influentarea optiunilor de vot pentru functia de presedinte al Romaniei, incalcand astfel restrictiile legale privind perioada de campanie electorala.
\end{enumerate}

\vspace{0.5cm}

\subsection{Partidul National Liberal (PNL) si Ucu Dima}
Următoarele fapte contravenționale sunt sesizate împotriva acestei entități:

\begin{enumerate}[leftmargin=*, label=\arabic*.)]
    \item difuzarea de materiale de propaganda electorala dupa incheierea perioadei de campanie pentru alegerile prezidentiale. Postarea cu ID-ul \href{https://www.facebook.com/ads/library/?id=1290694521930956}{1290694521930956} promoveaza explicit candidatul Nicolae Ciuca pentru functia de presedinte, prezentandu-l ca "lider de exceptie" si "alegerea ideala pentru functia de Presedinte al Romaniei", avand un evident caracter de propaganda electorala demonstrat prin prezenta numarului CMF 11240002, natura platita a postarii si distribuirea acesteia dupa ora 18:00 pe 23.11.2024. Mesajul are un efect electoral direct, incercand sa influenteze alegatorii in favoarea candidatului PNL la presedintie, depasind limitele comunicarii permise in aceasta perioada.
\end{enumerate}

\vspace{0.5cm}

\subsection{Partidul National Liberal (PNL), prin intermediul paginii "Ucu Dima"}
Următoarele fapte contravenționale sunt sesizate împotriva acestei entități:

\begin{enumerate}[leftmargin=*, label=\arabic*.)]
    \item difuzarea unei reclame platite pe Facebook (ID: \href{https://www.facebook.com/ads/library/?id=546432674845737}{546432674845737}) dupa ora 18:00 pe 23.11.2024, care continua propaganda electorala pentru candidatul la presedintie Nicolae Ciuca. Postarea, care include numarul CMF 11240002, promoveaza explicit candidatul la presedintie prin mentiunea "Cu Nicolae Ciuca Presedinte", constituind astfel continuarea propagandei electorale pentru alegerile prezidentiale dupa incheierea campaniei. Efectul electoral al acestei comunicari este evident prin asocierea directa a candidatului cu functia de Presedinte si prin indemnul explicit la vot, incalcand astfel prevederile legale privind perioada de campanie electorala pentru alegerile prezidentiale.
    \item continuarea propagandei electorale dupa incheierea campaniei, prin publicarea si promovarea unui mesaj sponsorizat pe Facebook (ID: \href{https://www.facebook.com/ads/library/?id=589233960303601}{589233960303601}) care face referire directa la candidatul prezidential Nicolae Ciuca, prezentandu-l explicit in contextul alegerilor prezidentiale ("Cu Nicolae Ciuca Presedinte"). Postarea, care contine numar CMF 11240002, reprezinta propaganda electorala conform Art. 36(7) prin faptul ca se refera direct la un candidat, are obiectiv electoral explicit prin indemnul la vot, si depaseste limitele activitatii jurnalistice. Aceasta activitate continua dupa ora 18:00 pe 23.11.2024, incalcand astfel prevederile legale privind incheierea campaniei electorale pentru alegerile prezidentiale.
\end{enumerate}

\vspace{0.5cm}

\subsection{Partidul National Liberal Braila}
Următoarele fapte contravenționale sunt sesizate împotriva acestei entități:

\begin{enumerate}[leftmargin=*, label=\arabic*.)]
    \item publicarea si promovarea unui mesaj de propaganda electorala (ID Facebook: \href{https://www.facebook.com/ads/library/?id=397977850062132}{397977850062132}) dupa incheierea perioadei de campanie electorala (dupa ora 18:00 pe 23.11.2024). Postarea contine elementele definitorii ale propagandei electorale, incluzand cod CMF 11240002, indeamna explicit la vot pentru candidatul Nicolae Ciuca - pozitia 4, denigreaza candidatul Marcel Ciolacu, si foloseste hashtag-uri specifice campaniei prezidentiale. Mesajul are un evident obiectiv electoral, fiind promovat pe Facebook si Instagram cu un buget semnificativ (800-899 RON) si o audienta estimata intre 45.000-50.000 de persoane, reprezentand o clara tentativa de influentare a votului in afara perioadei legale de campanie.
\end{enumerate}

\vspace{0.5cm}

\subsection{Partidul National Liberal Teleorman}
Următoarele fapte contravenționale sunt sesizate împotriva acestei entități:

\begin{enumerate}[leftmargin=*, label=\arabic*.)]
    \item publicarea si promovarea unui mesaj cu caracter electoral (ID postare: \href{https://www.facebook.com/ads/library/?id=605488808500280}{605488808500280}) dupa incheierea perioadei de campanie electorala, respectiv dupa ora 18:00 pe 23.11.2024. Mesajul "Viitorul Romaniei nu se decide in sondaje, ci la vot!" reprezinta in mod clar o forma de propaganda electorala, fiind insotit de numar CMF (11240002), platit pentru distribuire pe Facebook si Instagram, cu un impact estimat intre 15.000 si 19.999 de afisari, avand ca scop influentarea comportamentului electoral al cetatenilor in perioada de interdictie legala.
\end{enumerate}

\vspace{0.5cm}

\subsection{Partidul National Liberal, prin intermediul paginii "Raluca Turcan"}
Următoarele fapte contravenționale sunt sesizate împotriva acestei entități:

\begin{enumerate}[leftmargin=*, label=\arabic*.)]
    \item continuarea propagandei electorale dupa incheierea campaniei electorale pentru alegerile prezidentiale, prin postarea cu ID \href{https://www.facebook.com/ads/library/?id=552165887544658}{552165887544658} pe Instagram. Postarea, difuzata dupa ora 18:00 pe 23.11.2024, contine elementele constitutive ale propagandei electorale: numar unic CMF (11240002), indeamna explicit la vot pentru candidatul Nicolae Ciuca la presedintie ("Mediasul voteaza presedintele liberal Nicolae Ciuca!"), foloseste hashtag-ul \#NicolaeCiucaPresedinte, si este o reclama platita ce ajunge la un public larg (intre 100,001 si 500,000 persoane). Efectul electoral este evident prin incercarea de a influenta alegatorii in favoarea candidatului PNL la presedintie, folosind realizari administrative locale pentru a sustine mesajul electoral prezidential.
\end{enumerate}

\vspace{0.5cm}

\subsection{Partidul National Liberal, prin intermediul paginii "Raluca Turcan" (ID: \href{https://www.facebook.com/ads/library/?id=360184320712594}{360184320712594})}
Următoarele fapte contravenționale sunt sesizate împotriva acestei entități:

\begin{enumerate}[leftmargin=*, label=\arabic*.)]
    \item promovarea unui material de propaganda electorala (ID post: \href{https://www.facebook.com/ads/library/?id=930222319052589}{930222319052589}) dupa incheierea perioadei de campanie electorala pentru alegerile prezidentiale. Materialul, publicat si promovat dupa ora 18:00 pe 23.11.2024, contine elementele specifice unei comunicari electorale (numar CMF 11240002), promoveaza explicit candidatul Nicolae Ciuca la presedintie prin hashtag-ul "\#NicolaeCiucaPresedinte", si indeamna direct la vot folosind expresii precum "Votul liberal este votul nostru natural". Efectul electoral al postarii este amplificat prin distribuirea platita care a atins intre 4000-4999 de persoane, reprezentand o clara tentativa de influentare a votului in perioada in care propaganda electorala este interzisa prin lege.
\end{enumerate}

\vspace{0.5cm}

\subsection{Partidul National Liberal, prin intermediul paginii "Ucu Dima"}
Următoarele fapte contravenționale sunt sesizate împotriva acestei entități:

\begin{enumerate}[leftmargin=*, label=\arabic*.)]
    \item difuzarea unei reclame platite pe Facebook si Instagram (ID postare: \href{https://www.facebook.com/ads/library/?id=1242890290351940}{1242890290351940}) dupa incheierea perioadei de campanie electorala pentru alegerile prezidentiale (dupa ora 18:00 pe 23.11.2024). Postarea constituie propaganda electorala intrucat promoveaza explicit candidatul Nicolae Ciuca pentru functia de presedinte, contine numar CMF (11240002), include indemnuri directe la vot ("Votati PNL - Votati pentru Viitorul Romaniei!"), si prezinta un program electoral detaliat. Efectul electoral urmarit este influentarea directa a votantilor prin prezentarea unui program politic si promovarea explicita a candidatului la presedintie, intr-o perioada in care propaganda electorala este interzisa prin lege.
    \item promovarea unui mesaj electoral platit (ID postare Facebook: \href{https://www.facebook.com/ads/library/?id=1306460553865972}{1306460553865972}) care face referire directa la candidatul prezidential Nicolae Ciuca in contextul pozitiei de Presedinte ("Cu Nicolae Ciuca Presedinte"), continuand astfel propaganda electorala pentru alegerile prezidentiale dupa incheierea perioadei legale de campanie, respectiv dupa ora 18:00 pe 23.11.2024. Mesajul, desi aparent focusat pe alegerile parlamentare, include in mod explicit promovarea candidatului la presedintie, avand efect electoral direct asupra alegerilor prezidentiale prin prezentarea acestuia intr-o pozitie prezidentiala si influentand astfel optiunea de vot a alegatorilor.
    \item continuarea propagandei electorale dupa incheierea campaniei electorale pentru alegerile prezidentiale, prin publicarea si promovarea unei reclame platite pe Facebook (ID: \href{https://www.facebook.com/ads/library/?id=1652235849023143}{1652235849023143}) dupa ora 18:00 pe 23.11.2024. Postarea contine in mod explicit indemnul de a-l sustine pe Nicolae Ciuca pentru functia de Presedinte al Romaniei, folosind formulari precum "il sustinem cu incredere pe Nicolae Ciuca" si "alegerea ideala pentru functia de Presedinte al Romaniei", avand scop electoral evident si fiind marcata cu numar CMF 11240002, ceea ce confirma natura sa de material de propaganda electorala.
    \item continuarea propagandei electorale pentru alegerile prezidentiale dupa incheierea campaniei oficiale, prin publicarea si promovarea unei reclame platite pe Facebook (ID: \href{https://www.facebook.com/ads/library/?id=1812516469285879}{1812516469285879}) dupa ora 18:00 pe 23.11.2024. Postarea contine in mod explicit propaganda electorala pentru candidatul Nicolae Ciuca la functia de presedinte, prezentandu-l ca "alegerea ideala pentru functia de Presedinte al Romaniei" si folosind expresii menite sa influenteze decizia de vot precum "lider de exceptie" si "caracterizat prin integritate si devotament". Postarea include numar CMF (11240002) si este distribuita ca reclama platita pe platformele Facebook si Instagram, avand un impact estimat intre 1001 si 5000 de persoane.
    \item difuzarea unui material de propaganda electorala pentru alegerile prezidentiale (ID postare Facebook: \href{https://www.facebook.com/ads/library/?id=445998178606699}{445998178606699}) dupa incheierea perioadei de campanie electorala (dupa ora 18:00 pe 23.11.2024). Materialul promovat include in mod explicit mesajul "Cu Nicolae Ciuca Presedinte", promovand astfel candidatura sa la functia de presedinte al Romaniei, avand un evident caracter de propaganda electorala pentru alegerile prezidentiale, fapt demonstrat si prin prezenta codului unic CMF 11240002. Efectul electoral urmarit este influentarea intentiei de vot a alegatorilor in favoarea candidatului Nicolae Ciuca la functia de presedinte, prin asocierea acestuia cu un program de guvernare si promisiuni electorale.
\end{enumerate}

\vspace{0.5cm}

\subsection{Partidul National Liberal, prin intermediul paginii deputatului Sorin Nacuta}
Următoarele fapte contravenționale sunt sesizate împotriva acestei entități:

\begin{enumerate}[leftmargin=*, label=\arabic*.)]
    \item difuzarea unui mesaj de propaganda electorala dupa incheierea campaniei electorale pentru alegerile prezidentiale. Postarea cu ID-ul \href{https://www.facebook.com/ads/library/?id=501390856272523}{501390856272523} constituie propaganda electorala directa pentru candidatul Nicolae Ionel Ciuca la alegerile prezidentiale, continand indemnul explicit "pe data de 24 noiembrie, il sustinem pe Nicolae Ionel Ciuca la alegerile prezidentiale", intr-o reclama platita cu impact semnificativ (900.000-999.999 impresii), difuzata dupa ora 18:00 pe 23.11.2024, reprezentand o incalcare clara a perioadei de restrictie electorala si avand ca efect electoral influentarea directa a alegatorilor in favoarea candidatului PNL la presedintie.
\end{enumerate}

\vspace{0.5cm}

\subsection{Partidul National Liberal, prin reprezentantul Alin Calinescu}
Următoarele fapte contravenționale sunt sesizate împotriva acestei entități:

\begin{enumerate}[leftmargin=*, label=\arabic*.)]
    \item continuarea propagandei electorale dupa incheierea campaniei electorale pentru alegerile prezidentiale, prin intermediul unei postari sponsorizate pe Facebook (ID: \href{https://www.facebook.com/ads/library/?id=1827516994652644}{1827516994652644}) dupa ora 18:00 pe 23.11.2024. Postarea contine elemente clare de propaganda electorala, inclusiv indemnuri directe la vot ("Haideti la vot duminica"), sloganuri de campanie ("Argesul voteaza Ciuca"), si indemnuri explicite de sustinere a candidatului ("votati un candidat cu sanse"), fiind marcata cu numar CMF 11240002, demonstrand caracterul sau de material electoral. Efectul electoral urmarit este influentarea directa a votantilor in favoarea candidatului Nicolae Ciuca, incalcand astfel perioada de liniste electorala.
\end{enumerate}

\vspace{0.5cm}

\subsection{Partidul Politic ALIANTA PENTRU UNIREA ROMANILOR}
Următoarele fapte contravenționale sunt sesizate împotriva acestei entități:

\begin{enumerate}[leftmargin=*, label=\arabic*.)]
    \item difuzarea unei reclame platite pe Facebook (ID: \href{https://www.facebook.com/ads/library/?id=1714538599122156}{1714538599122156}) dupa ora 18:00 pe 23.11.2024, care contine propaganda electorala explicita pentru candidatul prezidential George Simion. Postarea include indemnuri directe la vot ("votati"), promovare directa a candidatului ("votez George Simion!"), promisiuni electorale despre institutiile statului, si foloseste numar CMF (11240014), demonstrand caracterul sau de propaganda electorala. Impactul electoral este evident prin audienta tintita (10,000-14,999 impresii) si mesajul care incearca sa influenteze comportamentul electoral al votantilor prin promisiuni despre institutiile statului.
    \item publicarea si promovarea dupa ora 18:00 pe 23.11.2024 a unui mesaj cu caracter de propaganda electorala pe platforma Facebook (ID postare: \href{https://www.facebook.com/ads/library/?id=466425329793976}{466425329793976}). Postarea, care a fost promovata cu un buget semnificativ (1000-1499 RON) si a ajuns la 250,000-300,000 de persoane, contine mesaje care sugereaza nereguli electorale ("buletine stampilate dinainte") si indeamna direct la vot ("Mergeti la vot!"), reprezentand astfel o continuare a propagandei electorale dupa incheierea campaniei si in ziua votului, cu potential de influentare a comportamentului electoral al alegatorilor.
\end{enumerate}

\vspace{0.5cm}

\subsection{Partidul Social Democrat - Organizatia Judeteana Bistrita-Nasaud}
Următoarele fapte contravenționale sunt sesizate împotriva acestei entități:

\begin{enumerate}[leftmargin=*, label=\arabic*.)]
    \item publicarea si mentinerea activa a unei reclame pe Facebook (ID: \href{https://www.facebook.com/ads/library/?id=595862286340426}{595862286340426}) care contine propaganda electorala explicita pentru candidatul la presedintie Marcel Ciolacu dupa incheierea perioadei de campanie electorala. Postarea, care include textul "Pe 24 noiembrie si 8 decembrie votam  Marcel Ciolacu, Presedinte al Romaniei!", reprezinta o incercare clara de a influenta votul pentru alegerile prezidentiale, fiind difuzata si dupa ora 18:00 pe 23.11.2024. Materialul este marcat cu CMF 11240017, confirmand natura sa de propaganda electorala, si a fost distribuit ca reclama platita pe platformele Facebook si Instagram, avand un impact semnificativ asupra alegatorilor.
\end{enumerate}

\vspace{0.5cm}

\subsection{Partidul Social Democrat - Organizatia Judeteana Hunedoara}
Următoarele fapte contravenționale sunt sesizate împotriva acestei entități:

\begin{enumerate}[leftmargin=*, label=\arabic*.)]
    \item publicarea si promovarea dupa data de 23.11.2024, ora 18:00, a unei reclame platite pe Facebook (ID: \href{https://www.facebook.com/ads/library/?id=1540641106581455}{1540641106581455}) ce contine indemnuri directe de vot pentru candidatul la presedintie Marcel Ciolacu ("votam Marcel Ciolacu - presedinte al Romaniei"). Postarea constituie propaganda electorala conform Art. 36(7) intrucat se refera direct la un candidat, este utilizata in afara perioadei permise de campanie, are obiectiv electoral explicit prin indemnul la vot, si se adreseaza publicului larg prin intermediul unei reclame platite pe platformele de social media, cu un reach estimat intre 100,001 si 500,000 de persoane.
\end{enumerate}

\vspace{0.5cm}

\subsection{Partidul Social Democrat Bistrita-Nasaud}
Următoarele fapte contravenționale sunt sesizate împotriva acestei entități:

\begin{enumerate}[leftmargin=*, label=\arabic*.)]
    \item continuarea propagandei electorale pentru candidatul prezidential Marcel Ciolacu dupa incheierea perioadei de campanie, prin intermediul unei reclame platite pe Facebook (ID: \href{https://www.facebook.com/ads/library/?id=1626528474611407}{1626528474611407}) care indeamna explicit la vot pentru Marcel Ciolacu ca presedinte ("Pe 24 noiembrie si 8 decembrie votam  Marcel Ciolacu, Presedinte al Romaniei!"). Materialul, care include numar CMF (11240017) specific materialelor de propaganda electorala, a continuat sa fie difuzat si dupa ora 18:00 pe 23.11.2024, avand un impact semnificativ cu peste 10.000 de afisari, reprezentand astfel o incalcare clara a prevederilor legale privind incheierea campaniei electorale pentru alegerile prezidentiale.
    \item continuarea propagandei electorale pentru candidatul la presedintie Marcel Ciolacu dupa incheierea perioadei de campanie, prin postarea cu ID \href{https://www.facebook.com/ads/library/?id=447243754691508}{447243754691508} pe Facebook. Postarea contine in mod explicit indemnul "Pe 24 noiembrie si 8 decembrie votam  Marcel Ciolacu, Presedinte al Romaniei!", reprezentand propaganda electorala dupa incheierea campaniei prezidentiale (dupa ora 18:00 pe 23.11.2024). Postarea include numar CMF (11240017), este sponsorizata pentru a ajunge la un public larg (100,001-500,000 persoane) si are un obiectiv electoral explicit de influentare a votului pentru alegerile prezidentiale.
    \item difuzarea unui material de propaganda electorala (ID Facebook: \href{https://www.facebook.com/ads/library/?id=480709598330522}{480709598330522}) dupa incheierea perioadei de campanie electorala pentru alegerile prezidentiale. Materialul, publicat dupa ora 18:00 pe 23.11.2024, contine in mod explicit indemnul la vot pentru candidatul Marcel Ciolacu la functia de presedinte ("Pe 24 noiembrie si 8 decembrie votam  Marcel Ciolacu, Presedinte al Romaniei!"), fiind o reclama platita cu impact semnificativ (25,000-29,999 impresii). Postarea include numarul CMF 11240017, confirmand natura sa de material electoral, si reprezinta o continuare nepermisa a propagandei electorale pentru alegerile prezidentiale dupa incheierea perioadei legale de campanie.
\end{enumerate}

\vspace{0.5cm}

\subsection{Partidul Social Democrat Bucuresti}
Următoarele fapte contravenționale sunt sesizate împotriva acestei entități:

\begin{enumerate}[leftmargin=*, label=\arabic*.)]
    \item publicarea si promovarea activa a unei reclame pe Facebook (ID: \href{https://www.facebook.com/ads/library/?id=1116163450109715}{1116163450109715}) dupa incheierea perioadei de campanie electorala pentru alegerile prezidentiale. Postarea, care include explicit hashtag-ul \#AlegeriPrezidentiale si face referire la alegerea presedintelui ("problemele din Bucuresti si din tara isi vor gasi o rezolvare mai rapida prin implicarea guvernului, parlamentului si presedintelui"), reprezinta propaganda electorala conform Art. 36(7), fiind marcata cu numar CMF 11240017, adresandu-se publicului larg si avand obiectiv electoral explicit. Postarea a fost promovata dupa ora 18:00 pe 23.11.2024, incalcand astfel prevederile legale privind incheierea campaniei electorale prezidentiale.
    \item publicarea si promovarea unei reclame pe Facebook (ID: \href{https://www.facebook.com/ads/library/?id=887823363487514}{887823363487514}) dupa incheierea perioadei de campanie electorala prezidentiala, in care se face propaganda electorala explicita pentru candidatul la presedintie Marcel Ciolacu. Postarea contine afirmatia "Echipa social-democrata, prin Marcel Ciolacu la presedintia Romaniei si candidatii la Parlament, sunt calea sigura pentru Bucuresti si Romania", reprezentand un mesaj electoral explicit care vizeaza influentarea votului pentru alegerile prezidentiale. Postarea este una platita, cu un CMF alocat (11240017), distribuita dupa ora 18:00 pe 23.11.2024, atingand intre 25.000 si 30.000 de persoane, avand astfel un impact electoral semnificativ.
\end{enumerate}

\vspace{0.5cm}

\subsection{Partidul Social Democrat Buzau}
Următoarele fapte contravenționale sunt sesizate împotriva acestei entități:

\begin{enumerate}[leftmargin=*, label=\arabic*.)]
    \item publicarea si promovarea unei reclame platite pe Facebook (ID: \href{https://www.facebook.com/ads/library/?id=541801878737165}{541801878737165}) dupa ora 18:00 pe 23.11.2024, ce reprezinta propaganda electorala in favoarea candidatului prezidential Marcel Ciolacu. Postarea contine elemente clare de propaganda electorala, inclusiv numar CMF (11240017), simboluri de partid, si prezinta candidatul ca "garantie a dezvoltarii", cu un mesaj electoral explicit "Calea Sigura Pentru Buzau". Reclama a fost distribuita activ catre 40.000-45.000 de persoane, cu un buget semnificativ de peste 1.000 RON, reprezentand o incercare clara de influentare a votului in perioada de restrictie electorala.
\end{enumerate}

\vspace{0.5cm}

\subsection{Partidul Social Democrat Buzau prin MRC SHOW LED SRL}
Următoarele fapte contravenționale sunt sesizate împotriva acestei entități:

\begin{enumerate}[leftmargin=*, label=\arabic*.)]
    \item publicarea si promovarea unei postari cu caracter electoral (ID Facebook: \href{https://www.facebook.com/ads/library/?id=580083777794224}{580083777794224}) dupa incheierea perioadei de campanie electorala, la data de 23.11.2024. Postarea constituie propaganda electorala intrucat contine numar CMF (11240017), are obiectiv electoral explicit vizand influentarea votului prin denigrarea candidatului Mircea-Dan Geoana, folosind expresii precum "Cine vrea inca un presedinte pe genul absent nemotivat" si facand referiri directe la activitatea si veniturile acestuia cu scopul de a-l discredita in fata alegatorilor. Postarea este promovata ca reclama platita pe Facebook si Instagram, avand o audienta estimata intre 100,001 si 500,000 de persoane, demonstrand intentia clara de a influenta comportamentul electoral al unui public larg.
\end{enumerate}

\vspace{0.5cm}

\subsection{Partidul Social Democrat Buzau si MRC SHOW LED SRL}
Următoarele fapte contravenționale sunt sesizate împotriva acestei entități:

\begin{enumerate}[leftmargin=*, label=\arabic*.)]
    \item difuzarea de materiale de propaganda electorala dupa incheierea perioadei de campanie, prin intermediul unei postari sponsorizate pe Facebook si Instagram (ID: \href{https://www.facebook.com/ads/library/?id=2274374652938316}{2274374652938316}) care il promoveaza pe candidatul prezidential Marcel Ciolacu. Postarea, difuzata dupa ora 18:00 pe 23.11.2024, contine elementele definitorii ale propagandei electorale conform Art. 36(7): foloseste identificator CMF (11240017), promoveaza direct candidatul, foloseste simbolurile partidului (trandafirul PSD) si sloganuri de campanie ("Calea Sigura Pentru Buzau"), fiind in mod clar destinata influentarii votului prin promovarea pozitiva a candidatului PSD la prezidentiale.
\end{enumerate}

\vspace{0.5cm}

\subsection{Partidul Social Democrat Org. Jud. Botosani}
Următoarele fapte contravenționale sunt sesizate împotriva acestei entități:

\begin{enumerate}[leftmargin=*, label=\arabic*.)]
    \item distribuirea unui mesaj de propaganda electorala (ID postare Facebook: \href{https://www.facebook.com/ads/library/?id=1101884608608594}{1101884608608594}) dupa incheierea perioadei de campanie electorala pentru alegerile prezidentiale. Postarea, care include numarul CMF 11240017, contine mesaje directe de influentare a votului impotriva candidatului Nicolae-Ionel Ciuca, folosind afirmatii precum "Fiecare vot pentru Ciuca si PNL este un vot pentru Guvernul Iohannis" si "Voi decideti daca il vreti sau nu pe Iohannis premier!", reprezentand astfel o forma clara de propaganda electorala activa dupa ora 18:00 pe 23.11.2024, cu intentia vadita de a influenta comportamentul electoral al alegatorilor in ziua votului.
    \item publicarea si promovarea unei reclame platite pe Facebook (ID: \href{https://www.facebook.com/ads/library/?id=1651224968820842}{1651224968820842}) dupa ora 18:00 pe 23.11.2024, care constituie propaganda electorala negativa la adresa candidatului prezidential Nicolae Ciuca. Postarea, care include numarul CMF 11240017, face acuzatii directe privind cheltuirea a 4 milioane de euro din bani publici pentru promovarea unei carti, avand un evident scop electoral de a influenta negativ opinia publica fata de candidat. Mesajul este distribuit prin reclama platita pe Facebook si Instagram, cu un impact estimat intre 15.000 si 19.999 de afisari, demonstrand intentia clara de a influenta un numar cat mai mare de alegatori.
    \item publicarea si mentinerea activa dupa data de 23.11.2024 ora 18:00 a unei reclame electorale (ID Facebook: \href{https://www.facebook.com/ads/library/?id=550648517783371}{550648517783371}) ce contine propaganda electorala explicita pentru candidatul prezidential Marcel Ciolacu si partidul PSD. Postarea contine numar CMF (11240017), face promisiuni electorale directe legate de infrastructura, si include un indemn explicit la vot ("Votati \#PSD pe 1 decembrie"), avand ca efect electoral influentarea directa a alegatorilor in perioada in care propaganda electorala pentru alegerile prezidentiale este interzisa prin lege.
\end{enumerate}

\vspace{0.5cm}

\subsection{Partidul Uniunea Salvati Romania - Buzau}
Următoarele fapte contravenționale sunt sesizate împotriva acestei entități:

\begin{enumerate}[leftmargin=*, label=\arabic*.)]
    \item continuarea propagandei electorale dupa incheierea campaniei electorale pentru alegerile prezidentiale, prin intermediul unei postari sponsorizate pe Facebook (ID: \href{https://www.facebook.com/ads/library/?id=2283471848692348}{2283471848692348}) care promoveaza in mod direct candidatura Elenei Lasconi la presedintie. Postarea contine elemente clare de propaganda electorala, inclusiv prezentarea programului electoral, indemnuri directe la vot ("Pe 24 noiembrie, haideti sa alegem o cale noua!"), si utilizeaza coduri AEP oficiale de campanie (11240015, 31240009). Efectul electoral urmarit este influentarea directa a alegatorilor in favoarea candidatei USR la presedintie, dupa ora 18:00 pe 23.11.2024, cand propaganda electorala pentru alegerile prezidentiale este interzisa prin lege.
\end{enumerate}

\vspace{0.5cm}

\subsection{Patridul National Liberal Teleorman}
Următoarele fapte contravenționale sunt sesizate împotriva acestei entități:

\begin{enumerate}[leftmargin=*, label=\arabic*.)]
    \item difuzarea unui mesaj de propaganda electorala (ID Facebook: \href{https://www.facebook.com/ads/library/?id=1294693224868664}{1294693224868664}) dupa incheierea perioadei de campanie electorala pentru alegerile prezidentiale (dupa ora 18:00 pe 23.11.2024). Postarea contine un indemn direct la vot pentru candidatul Nicolae Ciuca ("Pe 24 noiembrie votam Nicolae Ciuca presedinte!"), este marcata cu cod CMF 31240003, si are caracter de propaganda electorala fiind promovata ca reclama platita pe Facebook si Instagram, cu un impact estimat intre 3.000 si 3.999 de afisari. Mesajul are un obiectiv electoral clar, promovand candidatul ca "cel mai bun" si solicitand explicit votul pentru acesta in ziua alegerilor.
\end{enumerate}

\vspace{0.5cm}

\subsection{Peter Costea}
Următoarele fapte contravenționale sunt sesizate împotriva acestei entități:

\begin{enumerate}[leftmargin=*, label=\arabic*.)]
    \item difuzarea unui material de propaganda electorala pentru alegerile prezidentiale dupa incheierea perioadei de campanie, respectiv dupa ora 18:00 pe 23.11.2024. Materialul, cu ID-ul \href{https://www.facebook.com/ads/library/?id=3859478547647884}{3859478547647884} pe Facebook, promoveaza explicit votul pentru candidatul George Simion ("De ce si eu voi vota pentru GEORGE SIMION") si contine mesaje negative despre candidatul Cristian Terhes, fiind distribuit prin reclama platita pe Facebook si Instagram catre un public tinta de peste 100.000 de persoane, cu scopul clar de a influenta comportamentul electoral al alegatorilor in perioada de restrictie electorala.
\end{enumerate}

\vspace{0.5cm}

\subsection{Platforma "60m.RO"}
Următoarele fapte contravenționale sunt sesizate împotriva acestei entități:

\begin{enumerate}[leftmargin=*, label=\arabic*.)]
    \item promovarea unui material de propaganda electorala (ID postare Facebook: \href{https://www.facebook.com/ads/library/?id=1743529059779832}{1743529059779832}) dupa incheierea perioadei de campanie electorala, respectiv dupa ora 18:00 pe 23.11.2024. Materialul face o comparatie directa intre candidatii George-Nicolae Simion si Elena-Valerica Lasconi, prezentandu-l pe primul intr-o lumina favorabila ("mai gentleman") si criticand-o pe cea de-a doua pentru presupuse "jigniri", avand astfel un clar obiectiv electoral de influentare a votului. Postarea a fost promovata ca reclama platita pe Facebook si Instagram, atingand un public estimat intre 15.000 si 19.999 de persoane, demonstrand intentia clara de a influenta opinia publica in perioada in care propaganda electorala este interzisa prin lege.
\end{enumerate}

\vspace{0.5cm}

\subsection{PresaLibera.ro}
Următoarele fapte contravenționale sunt sesizate împotriva acestei entități:

\begin{enumerate}[leftmargin=*, label=\arabic*.)]
    \item promovarea dupa incheierea campaniei electorale a unui material de propaganda electorala platit (ID postare Facebook: \href{https://www.facebook.com/ads/library/?id=1374414093939771}{1374414093939771}) care promoveaza sansele electorale ale candidatului George Simion la alegerile prezidentiale. Materialul, difuzat dupa ora 18:00 pe 23.11.2024, constituie propaganda electorala prin faptul ca promoveaza direct sansele electorale ale candidatului in cursa prezidentiala, atingand un public de peste 35.000 de persoane prin publicitate platita pe Facebook si Instagram, depasind astfel limitele unei simple informari jurnalistice si avand un clar obiectiv electoral de influentare a votului.
    \item publicarea si promovarea unei reclame platite pe Facebook (ID: \href{https://www.facebook.com/ads/library/?id=1635942760687821}{1635942760687821}) dupa ora 18:00 pe 23.11.2024, care constituie propaganda electorala directa. Postarea face referire explicita la candidatul prezidential Elena Lasconi si incearca sa influenteze comportamentul electoral prin afirmatii despre rezultatele votului ("Lasconi va pune premier un liberal - Votezi USR iese PNL - Votezi PNL iese PSD"). Mesajul a avut un impact semnificativ, fiind distribuit ca reclama platita cu un buget substantial si o audienta estimata de peste 50.000 de persoane, depasind astfel limitele comunicarii jurnalistice obiective si constituind propaganda electorala activa in perioada restrictionata.
    \item publicarea si promovarea activa a unui material de propaganda electorala dupa incheierea perioadei de campanie, referitor la sansele candidatului George Simion in alegerile prezidentiale. Materialul, cu ID-ul \href{https://www.facebook.com/ads/library/?id=2062388527550424}{2062388527550424}, publicat si promovat dupa ora 18:00 pe 23.11.2024, reprezinta o incalcare clara a legii prin faptul ca discuta explicit despre sansele electorale ale unui candidat la presedintie intr-un context care vizeaza influentarea opiniei publice, avand un caracter propagandistic evident prin promovarea platita pe platformele Facebook si Instagram, cu o audienta estimata intre 80.000 si 89.999 de impresii.
\end{enumerate}

\vspace{0.5cm}

\subsection{REPER Vrancea, prin intermediul Atelierul de Internet SRL}
Următoarele fapte contravenționale sunt sesizate împotriva acestei entități:

\begin{enumerate}[leftmargin=*, label=\arabic*.)]
    \item promovarea unui material electoral (ID: \href{https://www.facebook.com/ads/library/?id=457080787409031}{457080787409031}) care face referire directa la campania prezidentiala a Elenei Lasconi ("Campania Si Presedintele e om" s-a potrivit ca o manusa Elenei Lasconi") dupa incheierea perioadei de campanie electorala pentru alegerile prezidentiale, respectiv dupa ora 18:00 pe 23.11.2024. Desi continutul principal al postarii se refera la campania parlamentara, includerea unui titlu de link care promoveaza explicit campania prezidentiala reprezinta o continuare a propagandei electorale pentru alegerile prezidentiale dupa incheierea acesteia, avand efect electoral prin mentinerea vizibilitatii si promovarea candidatului la presedintie dupa perioada legal permisa.
\end{enumerate}

\vspace{0.5cm}

\subsection{Rafael Nichita}
Următoarele fapte contravenționale sunt sesizate împotriva acestei entități:

\begin{enumerate}[leftmargin=*, label=\arabic*.)]
    \item publicarea si promovarea activa a unui mesaj de propaganda electorala (ID post Facebook: \href{https://www.facebook.com/ads/library/?id=445644034932892}{445644034932892}) dupa incheierea perioadei de campanie electorala, respectiv dupa ora 18:00 pe 23.11.2024. Postarea contine indemnuri directe de vot pentru candidatul Nicolae Ciuca la presedintie ("va indemn sa sustineti candidatul nostru la presedintia Romaniei"), foloseste hashtag-ul de campanie "\#NicolaeCiucaPresedinte", include numar CMF oficial de campanie (11240002), si este promovata activ prin plata pe platformele Facebook si Instagram, avand un impact semnificativ cu 10.000-15.000 de afisari, constituind astfel o continuare clara a propagandei electorale dupa incheierea perioadei legale de campanie.
\end{enumerate}

\vspace{0.5cm}

\subsection{Raluca Giorgiana Dumitrescu}
Următoarele fapte contravenționale sunt sesizate împotriva acestei entități:

\begin{enumerate}[leftmargin=*, label=\arabic*.)]
    \item publicarea si promovarea unui anunt electoral platit pe Facebook (ID: \href{https://www.facebook.com/ads/library/?id=562233786388115}{562233786388115}) dupa incheierea perioadei de campanie electorala pentru alegerile prezidentiale, continuand propaganda electorala prin promovarea explicita a candidatului ION-MARCEL CIOLACU la functia de presedinte al Romaniei. Postarea contine elementele definitorii ale propagandei electorale conform Art. 36(7), incluzand indemnul direct "Alegem Marcel Ciolacu - Presedintele Romaniei!", foloseste sloganuri de campanie si este distribuita ca reclama platita cu impact asupra unui public larg (50,001-100,000 persoane). Postarea include numar de mandatar financiar (CUI 11240017), confirmand natura sa de comunicare electorala oficiala. Aceasta activitate continua dupa ora 18:00 pe 23.11.2024, incalcand explicit prevederile legale privind incheierea campaniei electorale.
\end{enumerate}

\vspace{0.5cm}

\subsection{Razvan Biro}
Următoarele fapte contravenționale sunt sesizate împotriva acestei entități:

\begin{enumerate}[leftmargin=*, label=\arabic*.)]
    \item difuzarea unei reclame platite pe Facebook (ID: \href{https://www.facebook.com/ads/library/?id=921810482745028}{921810482745028}) dupa ora 18:00 pe 23.11.2024, care continua propaganda electorala pentru candidatul prezidential George Simion. Postarea contine un indemn direct la vot ("Duminica votam George Simion presedinte!"), foloseste fonduri de campanie (dovada prin prezenta CMF 11240014), si are un impact semnificativ fiind o reclama platita cu 200-299 RON si un reach de 20,000-24,999 impresii. Mesajul are un evident caracter de propaganda electorala, fiind conceput si distribuit cu scopul explicit de a influenta alegatorii in favoarea unui candidat specific in perioada de restrictie electorala.
\end{enumerate}

\vspace{0.5cm}

\subsection{Razvan Biro si AUR}
Următoarele fapte contravenționale sunt sesizate împotriva acestei entități:

\begin{enumerate}[leftmargin=*, label=\arabic*.)]
    \item continuarea propagandei electorale dupa incheierea campaniei electorale pentru alegerile prezidentiale, prin publicarea si mentinerea activa dupa ora 18:00 pe 23.11.2024 a unei reclame Facebook (ID: \href{https://www.facebook.com/ads/library/?id=1289695989147162}{1289695989147162}) ce promoveaza candidatura lui George Simion la presedintie. Postarea contine promisiuni electorale specifice ("Reducerea drastica a taxelor"), foloseste hashtag-ul \#GeorgeSimionPresedinte, include numar CMF (11240014), si face referire directa la data alegerilor prezidentiale, avand un evident scop electoral de influentare a votului pentru candidatul AUR la presedintie.
\end{enumerate}

\vspace{0.5cm}

\subsection{RedNews}
Următoarele fapte contravenționale sunt sesizate împotriva acestei entități:

\begin{enumerate}[leftmargin=*, label=\arabic*.)]
    \item publicarea si promovarea unui mesaj de propaganda electorala dupa incheierea perioadei de campanie, respectiv dupa ora 18:00 pe 23.11.2024. Postarea cu ID-ul \href{https://www.facebook.com/ads/library/?id=1251511855972536}{1251511855972536} contine un indemn direct la vot pentru candidatul George Simion ("Voteaza \#SIMION"), promoveaza programul electoral AUR si este distribuit ca reclama platita pe Facebook, cu un impact estimat intre 5000-6000 de afisari. Mesajul are un clar caracter de propaganda electorala, fiind conceput si distribuit cu scopul de a influenta optiunea de vot a alegatorilor pentru candidatul AUR la alegerile prezidentiale.
\end{enumerate}

\vspace{0.5cm}

\subsection{Ropres.ro}
Următoarele fapte contravenționale sunt sesizate împotriva acestei entități:

\begin{enumerate}[leftmargin=*, label=\arabic*.)]
    \item publicarea si promovarea contra cost a unui material de propaganda electorala dupa incheierea perioadei de campanie, cu ID-ul postarii pe facebook \href{https://www.facebook.com/ads/library/?id=1003904621501720}{1003904621501720}. Materialul, publicat dupa ora 18:00 pe 23.11.2024, vizeaza direct doi candidati la presedintie (Elena Lasconi si Nicolae Ciuca), prezentandu-i intr-o lumina negativa si sugerand infrangerea lor inevitabila, folosind expresii precum "rezultate devastatoare" si "scoruri dezastruoase". Postarea are un evident caracter de propaganda electorala, depasind limitele activitatii jurnalistice prin limbajul tendentios si prezentarea subiectiva, cu scopul clar de a influenta negativ optiunea de vot a alegatorilor.
\end{enumerate}

\vspace{0.5cm}

\subsection{Rus Marinel}
Următoarele fapte contravenționale sunt sesizate împotriva acestei entități:

\begin{enumerate}[leftmargin=*, label=\arabic*.)]
    \item continuarea propagandei electorale pentru candidatul USR la presedintie Elena Lasconi dupa incheierea perioadei legale de campanie, prin postarea cu ID 28450776881176277 pe Facebook dupa ora 18:00 pe 23.11.2024. Postarea constituie propaganda electorala conform Art. 36(7) prin prezenta numarului CMF 11240015, referirea directa la candidat, obiectivul electoral explicit ("Turul 1 = Turul decisiv!"), si indemnul direct la vot ("Nu ne putem permite sa irosim niciun vot in aceste alegeri!"). Postarea este una platita, cu audienta estimata intre 100,001-500,000 persoane, amplificand astfel impactul mesajului electoral dupa incheierea campaniei.
\end{enumerate}

\vspace{0.5cm}

\subsection{SALAMINA PRINT SRL}
Următoarele fapte contravenționale sunt sesizate împotriva acestei entități:

\begin{enumerate}[leftmargin=*, label=\arabic*.)]
    \item difuzarea de materiale de propaganda electorala dupa incheierea perioadei de campanie, in data de 23.11.2024 dupa ora 18:00, pentru postarea cu ID-ul \href{https://www.facebook.com/ads/library/?id=952853273557827}{952853273557827}. Materialul promovat pe Facebook contine un indemn explicit la vot pentru candidatul Ludovic Orban si partidul Forta Dreptei ("Pe 1 decembrie votati Forta Dreptei!"), fiind insotit de numar CMF (11240022), ceea ce demonstreaza natura sa de propaganda electorala. Postarea are un obiectiv electoral clar, adresandu-se publicului larg prin intermediul unei reclame platite pe Facebook, cu un impact estimat intre 40.000 si 45.000 de afisari, depasind astfel limitele comunicarii obiective sau jurnalistice.
\end{enumerate}

\vspace{0.5cm}

\subsection{SC EURO MEDIA BIS SRL}
Următoarele fapte contravenționale sunt sesizate împotriva acestei entități:

\begin{enumerate}[leftmargin=*, label=\arabic*.)]
    \item difuzarea de materiale de propaganda electorala dupa incheierea campaniei electorale pentru alegerile prezidentiale, respectiv dupa ora 18:00 pe 23.11.2024. Postarea cu ID-ul \href{https://www.facebook.com/ads/library/?id=1080818846619500}{1080818846619500} reprezinta propaganda electorala clara, continand numar CMF (11240002), indeamna explicit la vot ("Duminica vom face primul pas pe acest drum dand un VOT\_UTIL pentru Romania!"), promoveaza candidatul PNL si partidul sau, si este difuzata ca reclama platita pe Facebook si Instagram, cu un impact estimat intre 15.000 si 19.999 de afisari. Mesajul are scop electoral explicit, promovand ideea ca liberalii sunt "o forta capabila sa conduca Romania" si face referire directa la votul de duminica, constituind astfel propaganda electorala activa in perioada in care aceasta este interzisa prin lege.
    \item promovarea unui mesaj de propaganda electorala (ID postare Facebook: \href{https://www.facebook.com/ads/library/?id=1806660900143980}{1806660900143980}) dupa incheierea perioadei de campanie electorala (dupa ora 18:00 pe 23.11.2024). Postarea contine elementele definitorii ale propagandei electorale: are numar CMF (11240002), face referire directa la candidati la presedintie (atat negativ catre Ciolacu, Lasconi si Simion, cat si pozitiv catre Nicolae Ciuca), indeamna explicit la vot ("Sa iesim duminica la vot"), si are scop electoral evident prin promovarea calitatilor unui candidat ("Generalul Nicolae Ciuca este familist, este onest, corect, educat") si denigrarea contracandidatilor. Mesajul este platit si targetat catre un public larg (100,001-500,000 persoane), depasind astfel sfera unei simple opinii personale.
    \item promovarea unui mesaj de propaganda electorala (ID postare Facebook: \href{https://www.facebook.com/ads/library/?id=3997343703873698}{3997343703873698}) dupa ora 18:00 pe 23.11.2024, cu referire directa la candidatul GEORGE-NICOLAE SIMION si sansele acestuia de a intra in turul doi al alegerilor prezidentiale. Postarea contine numar CMF (11240022), este platita pentru distributie pe Facebook si Instagram, cu un impact intre 30.000 si 35.000 de afisari, si discuta explicit despre perspectivele electorale ale unui candidat la prezidentiale ("cei de la putere se vaita ca intra Simion in turul doi"), reprezentand astfel o forma clara de propaganda electorala in afara perioadei permise de lege.
    \item continuarea propagandei electorale dupa incheierea acesteia, prin promovarea unui mesaj electoral explicit in favoarea candidatului Nicolae Ciuca si in defavoarea candidatului Marcel Ciolacu. Postarea cu ID-ul \href{https://www.facebook.com/ads/library/?id=752286993767980}{752286993767980}, publicata si promovata dupa ora 18:00 pe 23.11.2024, contine un mesaj electoral clar ("Romania are nevoie de un presedinte de dreapta" si "Vom da Romaniei un presedinte liberal"), foloseste numar CMF (11240002), si are ca obiectiv influentarea votului prin prezentarea explicita a candidatului PNL ca fiind singura optiune viabila pentru presedintie, in timp ce denigreaza contracandidatii. Postarea a avut un impact semnificativ, atingand intre 15.000 si 19.999 de persoane prin promovare platita pe Facebook si Instagram.
\end{enumerate}

\vspace{0.5cm}

\subsection{SC VIDEO DANCRYS-STUDIO S.R.L}
Următoarele fapte contravenționale sunt sesizate împotriva acestei entități:

\begin{enumerate}[leftmargin=*, label=\arabic*.)]
    \item difuzarea de materiale de propaganda electorala dupa incheierea campaniei electorale pentru alegerile prezidentiale, prin postarea cu ID \href{https://www.facebook.com/ads/library/?id=1328124568153532}{1328124568153532} pe Facebook. Postarea contine un indemn direct de a vota pentru candidatul George Simion la alegerile prezidentiale ("La alegerile prezidentiale, votam George Simion"), fiind promovata ca reclama platita dupa ora 18:00 pe 23.11.2024, cu un impact estimat intre 4000 si 4999 de persoane. Mesajul include elemente clare de propaganda electorala, inclusiv hashtag-ul "\#GeorgeSimionPresedinte" si sloganuri de campanie, avand ca obiectiv clar influentarea votului in favoarea candidatului AUR la alegerile prezidentiale.
    \item difuzarea de material de propaganda electorala dupa incheierea campaniei electorale pentru alegerile prezidentiale, constand intr-o postare platita pe Facebook (ID: \href{https://www.facebook.com/ads/library/?id=3822009734713369}{3822009734713369}) care promoveaza explicit candidatul George Simion la presedintie. Postarea, difuzata dupa ora 18:00 pe 23.11.2024, foloseste hashtag-uri electorale precum "\#GeorgeSimionPresedinte" si "\#alegeri2024", prezinta mesaje de sustinere directa a candidatului, si are un evident scop electoral, targetand un public larg estimat intre 500.001 si 1.000.000 de persoane, constituind astfel propaganda electorala in perioada de restrictie.
    \item difuzarea de materiale de propaganda electorala dupa incheierea campaniei electorale pentru alegerile prezidentiale, respectiv dupa ora 18:00 pe 23.11.2024. Postarea cu ID-ul \href{https://www.facebook.com/ads/library/?id=528827646645998}{528827646645998} promoveaza explicit candidatul George Simion pentru functia de presedinte, folosind expresii precum "El va fi presedintele care va reconstrui Romania" si hashtag-ul "\#GeorgeSimionPresedinte", avand un evident caracter de propaganda electorala. Postarea este promovata ca reclama platita pe Facebook si Instagram, cu o audienta estimata intre 35.000 si 40.000 de afisari, demonstrand intentia clara de a influenta comportamentul electoral al votantilor in perioada in care propaganda electorala este interzisa prin lege.
    \item promovarea unui material de propaganda electorala pentru candidatul prezidential George Nicolae Simion dupa incheierea perioadei de campanie electorala (dupa ora 18:00 pe 23.11.2024). Postarea cu ID-ul \href{https://www.facebook.com/ads/library/?id=546433941508702}{546433941508702} promoveaza un mars de sustinere pentru candidatul George Simion, utilizand hashtag-uri specifice campaniei prezidentiale (\#georgesimionpresedinte, \#alegeri2024), si are un obiectiv electoral clar de mobilizare si influentare a alegatorilor prin organizarea unui mars de sustinere. Materialul este distribuit ca reclama platita pe Facebook si Instagram, cu un reach estimat intre 500.001 si 1.000.000 de persoane, demonstrand intentia clara de a influenta comportamentul electoral al unui numar semnificativ de alegatori dupa incheierea perioadei legale de campanie.
\end{enumerate}

\vspace{0.5cm}

\subsection{SC VIDEO DANCRYS-STUDIO S.R.L.}
Următoarele fapte contravenționale sunt sesizate împotriva acestei entități:

\begin{enumerate}[leftmargin=*, label=\arabic*.)]
    \item continuarea propagandei electorale dupa incheierea campaniei pentru alegerile prezidentiale, prin promovarea activa si platita a mesajului "George Simion Presedinte" pe platformele Facebook si Instagram, cu hashtag-uri specifice campaniei prezidentiale (\#GeorgeSimionPresedinte \#Alegeri2024), avand un CMF alocat (31240006), ceea ce demonstreaza clar intentia de propaganda electorala. Postarea, cu ID-ul \href{https://www.facebook.com/ads/library/?id=451750557962158}{451750557962158}, a continuat sa fie difuzata si dupa ora 18:00 pe 23.11.2024, atingand intre 10.000 si 14.999 de persoane, avand un efect electoral direct prin indemnul explicit de a-l sustine pe George Simion la functia de presedinte.
\end{enumerate}

\vspace{0.5cm}

\subsection{SC VIDEO DANCRYS-STUDIO SRL}
Următoarele fapte contravenționale sunt sesizate împotriva acestei entități:

\begin{enumerate}[leftmargin=*, label=\arabic*.)]
    \item difuzarea de materiale de propaganda electorala dupa incheierea campaniei electorale pentru alegerile prezidentiale. Postarea cu ID-ul \href{https://www.facebook.com/ads/library/?id=1136207171352012}{1136207171352012} promoveaza explicit candidatul George Simion pentru functia de presedinte, folosind sloganul "George Simion Presedinte!" insotit de simboluri nationale si hashtag-uri electorale (\#GeorgeSimionPresedinte, \#Algeri2024). Materialul este marcat cu cod CMF 31240006, confirmand natura sa de propaganda electorala, si a fost distribuit ca reclama platita pe Facebook si Instagram dupa ora 18:00 pe 23.11.2024, avand un impact semnificativ cu o audienta estimata intre 500.001 si 1.000.000 de persoane, fiind astfel o incercare clara de a influenta votul in perioada in care propaganda electorala este interzisa prin lege.
    \item promovarea unui mars de sustinere pentru candidatul la presedintie George Simion dupa incheierea perioadei de campanie electorala. Postarea cu ID-ul \href{https://www.facebook.com/ads/library/?id=1146734783545268}{1146734783545268}, publicata dupa ora 18:00 pe 23.11.2024, promoveaza explicit un eveniment de sustinere electorala, folosind hashtag-uri precum \#georgesimionpresedinte si \#alegeri2024, si mobilizeaza cetatenii la un mars de sustinere pentru candidat, reprezentand o forma clara de propaganda electorala in afara perioadei legale de campanie. Postarea a fost promovata ca reclama platita pe Facebook si Instagram, atingand intre 5000 si 6000 de afisari, demonstrand intentia clara de a influenta comportamentul electoral al cetatenilor in perioada in care acest lucru este interzis prin lege.
    \item difuzarea de materiale de propaganda electorala dupa incheierea campaniei electorale, respectiv postarea cu ID \href{https://www.facebook.com/ads/library/?id=3801991510054994}{3801991510054994} pe Facebook care promoveaza explicit candidatul George Simion la functia de presedinte. Materialul contine elementele specifice ale propagandei electorale, incluzand cod unic CMF 31240006, sloganul "George Simion Presedinte" si hashtag-uri electorale, fiind difuzat ca reclama platita pe Facebook si Instagram dupa ora 18:00 pe 23.11.2024, cu un impact semnificativ (intre 10.000 si 14.999 afisari). Efectul electoral este evident prin promovarea directa a candidatului la functia de presedinte si indemnul implicit la vot pentru acesta.
    \item promovarea unui mars de sustinere pentru candidatul George Simion la presedintie dupa incheierea perioadei de campanie electorala, in data de 22.11.2024. Postarea cu ID-ul \href{https://www.facebook.com/ads/library/?id=568354982553353}{568354982553353} reprezinta propaganda electorala explicita prin organizarea unui eveniment public de sustinere a candidatului, utilizarea hashtag-urilor de campanie (\#georgesimionpresedinte), si promovarea platita pe Facebook si Instagram, cu un impact estimat intre 4000-4999 de afisari, dupa ora 18:00 pe 23.11.2024. Actiunea are un evident caracter electoral si urmareste influentarea votului prin mobilizarea sustinatorilor intr-o actiune publica de sustinere a candidatului la presedintie.
    \item difuzarea de materiale de propaganda electorala dupa incheierea campaniei electorale pentru alegerile prezidentiale, prin intermediul unei postari sponsorizate pe Facebook si Instagram (ID: \href{https://www.facebook.com/ads/library/?id=579583984614765}{579583984614765}) care indeamna in mod explicit la votarea candidatului George Simion la alegerile prezidentiale ("La alegerile prezidentiale, votam George Simion, presedintele care va reda Romania romanilor!"). Postarea, difuzata dupa ora 18:00 pe 23.11.2024, reprezinta propaganda electorala avand toate elementele constitutive: identifica clar candidatul, are obiectiv electoral explicit, se adreseaza publicului larg prin platforme de social media si depaseste limitele activitatii jurnalistice, incalcand astfel prevederile legale privind incheierea campaniei electorale.
    \item promovarea unui material de propaganda electorala pentru candidatul George Simion dupa incheierea perioadei de campanie electorala, prin intermediul unei reclame platite pe Facebook (ID: \href{https://www.facebook.com/ads/library/?id=584647100592642}{584647100592642}) care promoveaza un mars de sustinere pentru candidatul la presedintie si foloseste hashtag-ul explicit \#georgesimionpresedinte. Materialul are un clar obiectiv electoral, fiind difuzat dupa ora 18:00 pe 23.11.2024, adresandu-se unui public larg estimat intre 500.001 si 1.000.000 de persoane, iar prin continutul sau explicit indeamna la sustinerea unui candidat la functia de presedinte al Romaniei in perioada in care propaganda electorala este interzisa prin lege.
    \item difuzarea de continut propagandistic electoral dupa incheierea perioadei de campanie electorala, prin postarea cu ID \href{https://www.facebook.com/ads/library/?id=908952204223735}{908952204223735} pe Facebook si Instagram. Postarea promoveaza explicit candidatul George Simion la functia de presedinte, folosind sloganul "George Simion Presedinte!" insotit de simboluri nationale si hashtag-uri electorale, precum si numarul CMF 31240006, confirmand natura sa de propaganda electorala. Postarea a fost promovata ca reclama platita, cu un impact semnificativ (intre 10.000 si 14.999 de afisari), dupa ora 18:00 pe 23.11.2024, incalcand astfel prevederile legale privind perioada de campanie electorala.
    \item promovarea unui mesaj de propaganda electorala pentru candidatul George Simion la alegerile prezidentiale, dupa incheierea perioadei de campanie electorala (postare ID: \href{https://www.facebook.com/ads/library/?id=943728441010897}{943728441010897}). Postarea contine elementele definitorii ale propagandei electorale conform Art. 36(7): identificarea clara a candidatului ("George Simion"), obiectiv electoral explicit ("il sustin pe George Simion pentru alegerile prezidentiale din 2024"), adresabilitate catre publicul larg (postare platita pe Facebook si Instagram), si depaseste limitele activitatii jurnalistice. Postarea include cod CMF (11240014), confirmand natura sa de material de propaganda electorala. Aceasta activitate continua dupa ora 18:00 pe 23.11.2024, incalcand astfel prevederile legale privind perioada permisa pentru propaganda electorala.
\end{enumerate}

\vspace{0.5cm}

\subsection{Sabin Sarmas}
Următoarele fapte contravenționale sunt sesizate împotriva acestei entități:

\begin{enumerate}[leftmargin=*, label=\arabic*.)]
    \item publicarea si promovarea unui mesaj de propaganda electorala dupa incheierea perioadei de campanie pentru alegerile prezidentiale, in data de 23.11.2024 dupa ora 18:00. Postarea cu ID-ul \href{https://www.facebook.com/ads/library/?id=1789161281900055}{1789161281900055} face referire directa la candidatul prezidential Elena Lasconi si la strategia electorala pentru turul doi al alegerilor prezidentiale ("intra in turul 2"), constituind astfel propaganda electorala conform Art. 36(7). Mesajul este distribuit ca reclama platita pe Facebook si Instagram, avand un impact semnificativ (6000-7000 impresii), si contine numar CMF (14240010), confirmand natura sa de material de propaganda electorala. Efectul electoral este evident prin discutarea explicita a strategiei electorale pentru alegerile prezidentiale si mentionarea directa a unui candidat prezidential intr-un context electoral.
\end{enumerate}

\vspace{0.5cm}

\subsection{Senatorul Artene Singeorzan}
Următoarele fapte contravenționale sunt sesizate împotriva acestei entități:

\begin{enumerate}[leftmargin=*, label=\arabic*.)]
    \item publicarea si promovarea unei reclame electorale (ID Facebook: \href{https://www.facebook.com/ads/library/?id=1534691147214423}{1534691147214423}) dupa ora 18:00 pe 23.11.2024, in perioada in care propaganda electorala pentru alegerile prezidentiale este interzisa. Postarea constituie propaganda electorala conform Art. 36(7) deoarece: promoveaza direct candidatul USR Elena Lasconi, contine numar CMF (11240015), foloseste indemnuri explicite la vot ("Nu ne putem permite sa irosim niciun vot", "Turul 1 = Turul decisiv!"), si are ca obiectiv clar influentarea votului prin prezentarea candidatului ca "singurul candidat capabil sa rupa vechile jocuri politice". Efectul electoral este evident prin incercarea de a convinge alegatorii sa voteze cu candidatul USR in primul tur al alegerilor prezidentiale.
\end{enumerate}

\vspace{0.5cm}

\subsection{Senatorul Sorin Cristian Alic}
Următoarele fapte contravenționale sunt sesizate împotriva acestei entități:

\begin{enumerate}[leftmargin=*, label=\arabic*.)]
    \item continuarea propagandei electorale dupa incheierea perioadei legale, intr-o postare sponsorizata pe Facebook (ID: \href{https://www.facebook.com/ads/library/?id=497335456648147}{497335456648147}) distribuita dupa ora 18:00 pe 23.11.2024. Postarea contine un indemn explicit de a vota candidatul George Simion, foloseste numar CMF ( - ), si are un evident caracter de propaganda electorala prin mesajul "sa iesim masiv la vot si sa votam George Simion", reprezentand o incercare directa de a influenta comportamentul electoral al cetatenilor in perioada in care propaganda electorala este interzisa prin lege.
\end{enumerate}

\vspace{0.5cm}

\subsection{Serban Todorica-Constantin}
Următoarele fapte contravenționale sunt sesizate împotriva acestei entități:

\begin{enumerate}[leftmargin=*, label=\arabic*.)]
    \item promovarea unui mesaj electoral platit pe Facebook si Instagram (ID postare: \href{https://www.facebook.com/ads/library/?id=1745223982905054}{1745223982905054}) dupa incheierea perioadei de campanie electorala pentru alegerile prezidentiale, respectiv dupa ora 18:00 pe 23.11.2024. Mesajul "Votez Nicolae Ionel Ciuca presedinte pentru un viitor stabil, pragmatic si dedicat Romaniei!" reprezinta in mod clar propaganda electorala, fiind o incercare directa de a influenta alegatorii sa voteze un anumit candidat la presedintie, atingand un public de 15.000-19.999 persoane prin intermediul unei reclame platite, cu un buget intre 200-299 RON.
\end{enumerate}

\vspace{0.5cm}

\subsection{TOHANEANU NICOLAE-VLADUT PERSOANA FIZICA AUTORIZATA}
Următoarele fapte contravenționale sunt sesizate împotriva acestei entități:

\begin{enumerate}[leftmargin=*, label=\arabic*.)]
    \item difuzarea de materiale de propaganda electorala dupa incheierea perioadei de campanie, respectiv dupa ora 18:00 pe 23.11.2024. Postarea cu ID-ul \href{https://www.facebook.com/ads/library/?id=1255773598984272}{1255773598984272} reprezinta propaganda electorala evidenta, continand numere CMF (11240003 si 31240004), promovand direct candidatul Cristian-Vasile Terhes prin hashtag-ul \#TerhesPresedinte si atacand contracandidati precum Marcel Ciolacu. Postarea este platita si targetata catre un public larg (500,001-1,000,000 persoane), cu scop electoral explicit, utilizand hashtag-ul \#alegeriprezidentiale2024, reprezentand astfel o incercare clara de influentare a votului in perioada in care propaganda electorala este interzisa prin lege.
    \item publicarea si promovarea unui mesaj de propaganda electorala dupa incheierea campaniei electorale pentru alegerile prezidentiale. Postarea cu ID-ul \href{https://www.facebook.com/ads/library/?id=3928901367359145}{3928901367359145} contine elemente clare de propaganda electorala, inclusiv numere CMF (11240003 si 31240004), critica directa la adresa candidatului Marcel Ciolacu si promovarea candidatului Cristian Terhes prin hashtag-ul \#TerhesPresedinte, avand ca scop influentarea votului in randul diasporei. Postarea este promovata dupa ora 18:00 pe 23.11.2024, incalcand astfel perioada de restrictie electorala si avand un efect electoral direct prin mesajele care vizeaza comunitatea romanilor din diaspora.
\end{enumerate}

\vspace{0.5cm}

\subsection{TOP DUO CONSULTING S.R.L.}
Următoarele fapte contravenționale sunt sesizate împotriva acestei entități:

\begin{enumerate}[leftmargin=*, label=\arabic*.)]
    \item difuzarea de materiale de propaganda electorala dupa incheierea campaniei electorale pentru alegerile prezidentiale, prin promovarea unui mesaj platit pe Facebook si Instagram (ID: \href{https://www.facebook.com/ads/library/?id=1195555961926959}{1195555961926959}) dupa ora 18:00 pe 23.11.2024. Postarea contine material de propaganda electorala explicit, incluzand numar CMF 11240014, promovand candidatul George Simion, facand promisiuni electorale directe ("voi lupta pentru dreptate, transparenta si respect pentru lege"), si folosind hashtag-uri de campanie. Mesajul are scop electoral clar, fiind dirijat catre un public larg (500,001-1,000,000 persoane) si depasind limitele comunicarii obisnuite sau jurnalistice.
    \item difuzarea de materiale de propaganda electorala pentru candidatul prezidential George Nicolae Simion dupa incheierea perioadei de campanie electorala. Postarea cu ID-ul \href{https://www.facebook.com/ads/library/?id=538309045777464}{538309045777464} contine elemente clare de propaganda electorala, inclusiv numar CMF (11240014), indemnuri directe la vot ("Sustine-ne la alegeri"), prezentarea candidatului intr-o lumina pozitiva ("lideri integri"), si folosirea hashtag-urilor electorale. Postarea are un efect electoral direct, vizand influentarea intentiei de vot a alegatorilor prin promovarea candidatului si a partidului sau, fiind difuzata dupa ora 18:00 pe 23.11.2024, incalcand astfel prevederile legale privind perioada de campanie electorala.
    \item promovarea unui material electoral (ID Facebook: \href{https://www.facebook.com/ads/library/?id=914473200271350}{914473200271350}) care face referire directa la candidatul la presedintie George Simion, ulterior incheierii perioadei de campanie electorala prezidentiala (dupa ora 18:00 pe 23.11.2024). Materialul promovat prezinta sustinerea explicita pentru George Simion in contextul alegerilor prezidentiale prin afirmatia "lupt alaturi de George Simion pentru un viitor curat si liber de coruptie", avand un evident efect electoral de influentare a votului pentru alegerile prezidentiale, desi este mascat sub forma unei campanii parlamentare. Prezenta codului CMF 11240014 confirma natura de material de propaganda electorala a acestei comunicari.
\end{enumerate}

\vspace{0.5cm}

\subsection{Total Impact}
Următoarele fapte contravenționale sunt sesizate împotriva acestei entități:

\begin{enumerate}[leftmargin=*, label=\arabic*.)]
    \item promovarea unei reclame platite pe Facebook si Instagram (ID: \href{https://www.facebook.com/ads/library/?id=606526731946679}{606526731946679}) dupa ora 18:00 pe 23.11.2024, care face referire directa la candidatul prezidential George Nicolae Simion, asociindu-l cu conducerea pro-rusa din Republica Moldova. Postarea, prin natura sa platita si distributia larga (reach potential 100,001-500,000 persoane), reprezinta propaganda electorala activa in perioada de restrictie, avand ca obiectiv influentarea negativa a intentiei de vot prin asocierea candidatului cu sentimente pro-ruse, intr-un context geopolitic sensibil.
    \item promovarea unui material de propaganda electorala pentru candidatul Nicolae Ciuca la alegerile prezidentiale, dupa incheierea perioadei de campanie electorala. Postarea cu ID \href{https://www.facebook.com/ads/library/?id=972207998081030}{972207998081030} contine elemente clare de propaganda electorala, precum mentiunea directa a candidatului in contextul "luptei stranse pentru presedintia Romaniei" si afirmatii precum "Romania are nevoie de un presedinte cu buna credinta", avand ca scop influentarea intentiei de vot a alegatorilor. Postarea sponsorizata, difuzata dupa ora 18:00 pe 23.11.2024, constituie o continuare a propagandei electorale dupa incheierea perioadei legale de campanie.
\end{enumerate}

\vspace{0.5cm}

\subsection{UNIWORLD - MEDIA SRL}
Următoarele fapte contravenționale sunt sesizate împotriva acestei entități:

\begin{enumerate}[leftmargin=*, label=\arabic*.)]
    \item difuzarea unui mesaj de propaganda electorala pentru alegerile prezidentiale dupa incheierea perioadei de campanie, prin postarea cu ID \href{https://www.facebook.com/ads/library/?id=1074079691186801}{1074079691186801} pe Facebook, dupa ora 18:00 pe 23.11.2024. Postarea face o promisiune electorala legata de o baza olimpica de canotaj si solicita explicit votul pentru candidatul George Simion la functia de presedinte ("Votati AUR in Parlamentul Romaniei si pe George Simion presedinte pentru dezvoltarea acestui proiect!"), reprezentand propaganda electorala activa care influenteaza decizia de vot a alegatorilor in perioada in care acest lucru este interzis de lege.
    \item continuarea propagandei electorale dupa incheierea campaniei pentru alegerile prezidentiale, prin intermediul unei postari sponsorizate pe Facebook (ID: \href{https://www.facebook.com/ads/library/?id=1309103433586021}{1309103433586021}) care indeamna in mod explicit la votarea candidatului George Simion la alegerile prezidentiale. Postarea, care continua sa fie activa dupa ora 18:00 pe 23.11.2024, contine indemnul direct "Pe 24 noiembrie, 1 decembrie si 8 decembrie, va invit sa votati George Simion si A.U.R pentru o schimbare adevarata", reprezentand astfel propaganda electorala care incalca prevederile legale privind incheierea campaniei electorale. Efectul electoral este evident prin promovarea explicita a candidatului la presedintie si asocierea acestuia cu programul politic denumit "Planul Simion", intr-un context care vizeaza influentarea directa a votului in favoarea sa.
    \item promovarea dupa incheierea campaniei electorale (dupa ora 18:00 pe 23.11.2024) a unui material de propaganda electorala ce promoveaza candidatul la presedintie George Nicolae Simion. Postarea cu ID-ul \href{https://www.facebook.com/ads/library/?id=1313713093326681}{1313713093326681} prezinta "viziunea unui lider hotarat sa redea demnitatea romanilor" si promoveaza "Planul Simion", folosind un limbaj promotional care il descrie pe candidat ca fiind "singurul" cu un plan "concret, bine gandit", avand astfel un efect electoral direct prin influentarea alegatorilor in perioada in care propaganda electorala este interzisa prin lege.
    \item continuarea propagandei electorale dupa incheierea campaniei pentru alegerile prezidentiale, prin postarea cu ID \href{https://www.facebook.com/ads/library/?id=1354010492236303}{1354010492236303} pe Facebook. Postarea reprezinta propaganda electorala explicita pentru candidatul la presedintie George Simion, continand indemnuri directe de vot ("Ii indemn pe toti cei care isi doresc o schimbare reala sa iasa la vot pe 24 noiembrie si sa-l sustina pe George Simion!"), fiind difuzata dupa ora 18:00 pe 23.11.2024. Mesajul are caracter electoral evident, fiind o reclama platita ce a ajuns la minimum 10.000 de persoane, cu scopul clar de a influenta comportamentul electoral in favoarea unui candidat la presedintie.
    \item difuzarea unui mesaj de propaganda electorala dupa incheierea perioadei de campanie electorala, la data de 23.11.2024 dupa ora 18:00. Postarea cu ID-ul \href{https://www.facebook.com/ads/library/?id=1611606109739658}{1611606109739658} contine un indemn direct la vot pentru candidatul AUR la presedintie, George Simion, utilizand formularea explicita "Votati AUR si George Simion, presedintele Romaniei!", alaturi de hashtag-uri de campanie. Mesajul are un evident caracter de propaganda electorala, fiind o reclama platita care a ajuns la 20.000-24.999 de persoane, cu un buget intre 200-299 RON, avand ca scop influentarea directa a intentiei de vot a alegatorilor pentru candidatul AUR la functia de presedinte al Romaniei.
    \item difuzarea de continut promotional electoral pentru candidatul AUR la presedintie George Nicolae Simion dupa incheierea perioadei de campanie electorala. Postarea cu ID \href{https://www.facebook.com/ads/library/?id=1652036598729276}{1652036598729276} contine un indemn explicit la vot pentru data de 24 noiembrie, promoveaza candidatul si partidul sau, si face referire directa la "Planul pentru o Romanie dreapta" al acestuia. Efectul electoral este evident prin indemnul direct "Pe 24 noiembrie, 1 decembrie si 8 decembrie, votati AUR pentru o Romanie a dreptatii!" difuzat dupa ora 18:00 pe 23.11.2024, reprezentand o continuare clara a propagandei electorale dupa incheierea perioadei legale de campanie.
    \item difuzarea unui material de propaganda electorala pentru candidatul prezidential George Simion dupa incheierea perioadei de campanie electorala. Materialul, cu ID-ul 27593234510292262 pe Facebook, contine un indemn explicit la vot pentru data de 24 noiembrie, mentionand "Cred in planul lui George Simion" si "ii invit pe toti cei care vor o schimbare autentica sa voteze AUR pe 24 noiembrie", reprezentand propaganda electorala activa dupa ora 18:00 pe 23.11.2024. Postarea, promovata ca reclama platita cu un impact estimat intre 150.000-175.000 de afisari, are un obiectiv electoral clar si vizeaza influentarea votului pentru alegerile prezidentiale.
    \item difuzarea de materiale de propaganda electorala dupa incheierea campaniei electorale prezidentiale, prin intermediul unei reclame platite pe Facebook (ID: \href{https://www.facebook.com/ads/library/?id=2795941963927464}{2795941963927464}) care continua sa fie activa dupa ora 18:00 pe 23.11.2024. Postarea include un indemn direct de a-l vota pe candidatul George Simion la alegerile prezidentiale din 24 noiembrie ("Va indemn sa-l sustineti pe George Simion pe 24 noiembrie"), constituind astfel propaganda electorala explicita pentru alegerile prezidentiale, in perioada in care acest lucru este interzis prin lege. Efectul electoral este amplificat de faptul ca este o reclama platita cu reach estimat intre 100,000 si 124,999 de afisari.
    \item promovarea unui material de propaganda electorala (ID postare Facebook: \href{https://www.facebook.com/ads/library/?id=2963929023764464}{2963929023764464}) dupa incheierea perioadei de campanie electorala pentru alegerile prezidentiale. Materialul, difuzat dupa ora 18:00 pe 23.11.2024, promoveaza explicit candidatul George Simion la presedintie, folosind hashtag-ul "\#GeorgeSimionPresedinte" si continand numarul oficial de material electoral CMF11240014. Postarea este structurata pentru a influenta intentiile de vot ale alegatorilor, combinand o poveste personala de business cu un mesaj politic explicit de sustinere, atingand un public larg prin promovare platita (60,000-70,000 impresii). Efectul electoral este evident prin asocierea pozitiva a candidatului cu sustinerea antreprenoriatului local si folosirea hashtag-urilor de campanie.
    \item continuarea propagandei electorale pentru candidatul prezidential George Simion dupa incheierea perioadei legale de campanie, prin intermediul unei postari sponsorizate pe Facebook (ID: \href{https://www.facebook.com/ads/library/?id=3623273844640710}{3623273844640710}) care promoveaza explicit candidatura prezidentiala prin utilizarea hashtag-urilor \#GeorgeSimionPresedinte si a mesajelor ce il prezinta pe George Simion ca fiind cel mai potrivit pentru conducerea tarii, dupa ora 18:00 pe 23.11.2024. Postarea utilizeaza un CMF (11240014) si are un efect electoral direct, fiind distribuita catre un public larg estimat intre 100.001 si 500.000 de persoane, cu un buget semnificativ de promovare, reprezentand astfel o incalcare clara a prevederilor legale privind incheierea campaniei electorale prezidentiale.
    \item promovarea unui mesaj de propaganda electorala pentru candidatul George Simion dupa incheierea perioadei de campanie electorala, la data de 23.11.2024 dupa ora 18:00. Postarea cu ID-ul \href{https://www.facebook.com/ads/library/?id=3925497371059759}{3925497371059759} contine elementele specifice propagandei electorale: cod unic CMF11240014, hashtag-uri de campanie (\#GeorgeSimionPresedinte), promisiuni electorale privind constructia de locuinte si mesaje directe de sustinere a candidatului ("Am decis sa fac tot ce-mi sta in puteri alaturi de el"), fiind o reclama platita cu impact semnificativ (20.000-24.999 impresii). Mesajul are obiectiv electoral clar, vizand influentarea intentiei de vot a alegatorilor prin asocierea candidatului cu promisiuni concrete de politici publice si folosind elemente emotionale pentru mobilizarea alegatorilor.
    \item difuzarea unui material de propaganda electorala dupa incheierea perioadei de campanie, respectiv dupa ora 18:00 pe 23.11.2024. Materialul, cu ID-ul \href{https://www.facebook.com/ads/library/?id=457836440239634}{457836440239634} pe Facebook, promoveaza explicit candidatul prezidential George Simion si partidul AUR, prezentand realizarile acestora in domeniul sanatatii, inclusiv crearea unui spital mobil, cu scopul clar de a influenta preferintele electorale ale votantilor. Mesajul are caracter electoral evident, fiind o reclama platita care ajunge la un public larg (20.000-24.999 impresii), depasind sfera unei simple informari si constituind propaganda electorala prin promovarea realizarilor si initiativelor candidatului in perioada post-campanie.
    \item difuzarea unui material de propaganda electorala pentru candidatul George Simion dupa incheierea perioadei de campanie electorala, respectiv dupa ora 18:00 pe 23.11.2024. Materialul, cu ID-ul \href{https://www.facebook.com/ads/library/?id=612675074424929}{612675074424929} pe facebook, reprezinta propaganda electorala evidenta prin indemnul direct la vot ("Eu il votez pe George Simion si va indemn sa faceti la fel"), fiind platit pentru a ajunge la un public tinta de 100,001-500,000 persoane, cu un mesaj clar electoral care promoveaza candidatul la presedintie si partidul AUR, depasind limitele unei simple opinii personale prin utilizarea resurselor financiare pentru promovare si prin folosirea unui indemn explicit la vot pentru un candidat specific.
    \item difuzarea dupa data de 23.11.2024, ora 18:00, a unei reclame electorale platite pe platforma Facebook (ID: \href{https://www.facebook.com/ads/library/?id=871483411851862}{871483411851862}) ce contine propaganda electorala explicita pentru candidatul la presedintie George Simion. Postarea include un indemn direct la vot ("Votati AUR si pe George Simion, presedintele Romaniei!"), depasind perioada legala permisa pentru propaganda electorala. Efectul electoral este evident prin mesajul care indeamna explicit la votarea unui candidat specific la presedintie, atingand un public tinta de 25.000-30.000 de persoane, cu un buget de aproximativ 250 RON.
    \item difuzarea de materiale de propaganda electorala dupa incheierea perioadei de campanie electorala pentru alegerile prezidentiale, prin intermediul unei reclame platite pe Facebook (ID: \href{https://www.facebook.com/ads/library/?id=959848279534435}{959848279534435}) care continua sa ruleze dupa data de 23.11.2024, ora 18:00. Postarea contine un indemn explicit de vot pentru candidatul George Simion la functia de presedinte ("Votati AUR pentru o Romanie puternica si pe George Simion presedinte"), avand un evident caracter de propaganda electorala prin promisiuni electorale si indemnuri directe la vot, atingand un public intre 40.000 si 45.000 de persoane, cu un buget de aproximativ 250 RON, reprezentand astfel o incercare clara de influentare a comportamentului electoral al alegatorilor dupa incheierea perioadei legale de campanie.
\end{enumerate}

\vspace{0.5cm}

\subsection{UNIWORLD - MEDIA SRL si AUR}
Următoarele fapte contravenționale sunt sesizate împotriva acestei entități:

\begin{enumerate}[leftmargin=*, label=\arabic*.)]
    \item continuarea propagandei electorale dupa incheierea perioadei legale de campanie, distribuind un anunt platit pe Facebook (ID: \href{https://www.facebook.com/ads/library/?id=3554420868191864}{3554420868191864}) care promoveaza candidatul George Simion la presedintie, dupa ora 18:00 pe 23.11.2024. Postarea include distribuirea de afise electorale cu candidatul, foloseste hashtag-uri de campanie (\#VoteazaAUR), si face apel direct la sustinerea candidatului prin mesajul "Impreuna putem aduce schimbarea pe care o merita Romania", avand un evident caracter de propaganda electorala si vizand influentarea votului prin atingerea unui public tinta de pana la 500.000 de persoane in regiunea Timis.
\end{enumerate}

\vspace{0.5cm}

\subsection{USR Arad}
Următoarele fapte contravenționale sunt sesizate împotriva acestei entități:

\begin{enumerate}[leftmargin=*, label=\arabic*.)]
    \item publicarea si promovarea unei reclame platite pe Facebook (ID: \href{https://www.facebook.com/ads/library/?id=3739588549685648}{3739588549685648}) dupa data de 23.11.2024, ora 18:00, care face referire directa la candidatii la prezidentiale Ciolacu, Simion si Lasconi. Mesajul "Aceste alegeri nu sunt doar despre Ciolacu, Simion sau Lasconi. Aceste alegeri sunt despre Romania pe care ne-o dorim" reprezinta propaganda electorala prin mentionarea explicita a candidatilor si incercarea de influentare a perceptiei publice asupra alegerilor prezidentiale, avand un impact substantial prin targetarea unei audiente estimate intre 100,001 si 500,000 de persoane, cu un buget intre 200-299 RON.
    \item difuzarea unui mesaj de propaganda electorala dupa incheierea perioadei de campanie, pe data de 23.11.2024 dupa ora 18:00, prin intermediul unei reclame platite pe Facebook (ID: \href{https://www.facebook.com/ads/library/?id=931967841627574}{931967841627574}) care promoveaza explicit candidatul Elena Lasconi la functia de presedinte. Mesajul "Elena Lasconi nu este raul cel mai mic, ci singura care poate invinge raul cel mare" reprezinta in mod evident propaganda electorala, avand ca scop influentarea intentiei de vot a alegatorilor prin pozitionarea candidatului ca singura optiune viabila, depasind limitele unei simple informari si constituind un indemn direct la sustinerea candidatului.
\end{enumerate}

\vspace{0.5cm}

\subsection{USR Arges}
Următoarele fapte contravenționale sunt sesizate împotriva acestei entități:

\begin{enumerate}[leftmargin=*, label=\arabic*.)]
    \item promovarea unui material de propaganda electorala (ID Facebook: \href{https://www.facebook.com/ads/library/?id=3525561504410923}{3525561504410923}) dupa incheierea perioadei de campanie electorala pentru alegerile prezidentiale. Materialul, publicat dupa ora 18:00 pe 23.11.2024, contine un indemn direct la vot pentru candidatul Elena Lasconi la functia de presedinte ("\#ElenaLasconiPresedinte"), critici la adresa contracandidatilor Marcel Ciolacu si George Simion, precum si prezentarea unui program electoral. Materialul este marcat cu cod CMF 11240015, confirmand natura sa de propaganda electorala, si a fost distribuit ca reclama platita pe Facebook, cu un impact semnificativ (35,000-40,000 impresii), demonstrand intentia clara de a influenta votul in alegerile prezidentiale dupa incheierea perioadei legale de campanie.
\end{enumerate}

\vspace{0.5cm}

\subsection{USR Bistrita-Nasaud}
Următoarele fapte contravenționale sunt sesizate împotriva acestei entități:

\begin{enumerate}[leftmargin=*, label=\arabic*.)]
    \item continuarea propagandei electorale dupa incheierea campaniei electorale, in data de 23.11.2024 dupa ora 18:00, prin postarea cu ID \href{https://www.facebook.com/ads/library/?id=895329446034288}{895329446034288} pe Facebook. Postarea constituie propaganda electorala conform Art. 36(7) prin urmatoarele elemente: contine numar CMF (11240015), promoveaza direct candidatul Elena Lasconi ca "singurul candidat capabil", face apel explicit la vot prin mesaje precum "Nu ne putem permite sa irosim niciun vot in aceste alegeri!" si "Turul 1 = Turul decisiv!", are obiectiv electoral clar de influentare a votului si se adreseaza publicului larg prin intermediul unei reclame platite pe Facebook si Instagram. Efectul electoral urmarit este cresterea numarului de voturi pentru candidatul USR prin prezentarea acestuia ca unica optiune viabila pentru schimbare.
\end{enumerate}

\vspace{0.5cm}

\subsection{USR Braila}
Următoarele fapte contravenționale sunt sesizate împotriva acestei entități:

\begin{enumerate}[leftmargin=*, label=\arabic*.)]
    \item publicarea si promovarea dupa data de 23.11.2024, ora 18:00, a unei reclame pe Facebook (ID: \href{https://www.facebook.com/ads/library/?id=572976648757324}{572976648757324}) ce constituie propaganda electorala impotriva candidatului prezidential Marcel Ciolacu. Postarea contine elemente clare de propaganda electorala, inclusiv numar CMF (11240015), atacuri directe la adresa candidatului ("Nu tinerii sunt problema, domnule Ciolacu. Problema sunteti voi"), indemn explicit de a nu vota PSD prin hashtag-ul "\#NuVotatiPSD", si prezentarea USR ca alternativa politica. Efectul electoral urmarit este clar negativ fata de candidatul PSD la presedintie, incercand sa influenteze decizia de vot a tinerilor prin asocierea directa a candidatului cu probleme precum coruptia si nepotismul.
\end{enumerate}

\vspace{0.5cm}

\subsection{USR Bucuresti}
Următoarele fapte contravenționale sunt sesizate împotriva acestei entități:

\begin{enumerate}[leftmargin=*, label=\arabic*.)]
    \item difuzarea unui mesaj electoral platit pe Instagram (ID postare: \href{https://www.facebook.com/ads/library/?id=1078508163752941}{1078508163752941}) dupa incheierea perioadei de campanie electorala pentru alegerile prezidentiale, respectiv dupa ora 18:00 din 23.11.2024. Mesajul contine indemnul explicit "Votam Elena Lasconi duminica aceasta", fiind distribuit catre un public tinta de peste 100.000 de persoane, cu un buget de aproximativ 450 RON si avand coduri CMF specifice materialelor de propaganda electorala (11240015 si 31240009). Postarea are un efect electoral direct, indemnand in mod explicit la votarea unui candidat specific pentru functia de presedinte, depasind astfel cadrul legal permis pentru comunicarea in aceasta perioada.
\end{enumerate}

\vspace{0.5cm}

\subsection{USR Caras-Severin}
Următoarele fapte contravenționale sunt sesizate împotriva acestei entități:

\begin{enumerate}[leftmargin=*, label=\arabic*.)]
    \item difuzarea de materiale de propaganda electorala dupa incheierea campaniei electorale pentru alegerile prezidentiale, concretizata prin postarea cu ID \href{https://www.facebook.com/ads/library/?id=1247985159866179}{1247985159866179} pe Facebook. Postarea, difuzata dupa ora 18:00 pe 23.11.2024, promoveaza explicit programul electoral al candidatei Elena Lasconi, prezentand masuri economice specifice precum "reducerea taxelor pentru antreprenori si PFA" si alte promisiuni electorale. Materialul include numarul CMF 11240015, confirmand natura sa de propaganda electorala, si are ca obiectiv clar influentarea intentiei de vot prin prezentarea detaliata a masurilor economice propuse de candidata in campania prezidentiala.
\end{enumerate}

\vspace{0.5cm}

\subsection{USR Constanta}
Următoarele fapte contravenționale sunt sesizate împotriva acestei entități:

\begin{enumerate}[leftmargin=*, label=\arabic*.)]
    \item publicarea si promovarea unui mesaj de propaganda electorala dupa incheierea perioadei de campanie, pe data de 22.11.2024, cu ID-ul postarii pe facebook 27606559912324043. Postarea contine atacuri directe la adresa candidatilor prezidentiali Marcel Ciolacu si Nicolae Ciuca, face referire explicita la alegerile prezidentiale ("ca tot sunt alegeri peste 3 zile"), si indeamna in mod direct la actiune electorala ("Ai ocazia sa-i lasi fara putere si fara privilegii. Vino la vot!"). Mesajul este marcat cu cod CMF (11240015), confirmand natura sa de propaganda electorala, si a fost promovat dupa ora 18:00 pe 23.11.2024, incalcand astfel prevederile legale privind incetarea campaniei electorale.
    \item publicarea si promovarea unei reclame platite pe Facebook (ID: \href{https://www.facebook.com/ads/library/?id=439252415885874}{439252415885874}) dupa ora 18:00 pe 23.11.2024, care constituie propaganda electorala pentru alegerile prezidentiale. Postarea promoveaza explicit candidatul Elena Lasconi la presedintie prin indemnul direct "Elena Lasconi Presedinte", in timp ce denigreaza candidatii Marcel Ciolacu si George Simion. Postarea include elemente clare de propaganda electorala, cum ar fi numarul CMF 11240015, foloseste un mesaj de indemn la vot pentru un candidat specific la prezidentiale si ataca alti candidati, depasind astfel cadrul permis al comunicarii politice in perioada post-campanie pentru alegerile prezidentiale.
    \item publicarea si promovarea unei reclame pe Facebook (ID: \href{https://www.facebook.com/ads/library/?id=8481490908566311}{8481490908566311}) dupa incheierea perioadei de campanie electorala pentru alegerile prezidentiale. Postarea constituie propaganda electorala intrucat contine atacuri directe la adresa candidatului Marcel Ciolacu, Nicolae Ciuca si George Simion, promoveaza candidatul Elena Lasconi (prin hashtag \#ElenaLasconi), include numar CMF (11240015), si indeamna explicit alegatorii sa voteze impotriva anumitor candidati. Efectul electoral este evident prin indemnul "Haideti sa-i oprim, sa-l trimitem acasa si pe Ciolacu, sa-i trimitem si pe Simion si pe pesedistul vopsit in galben, omul lui Iohannis si omul sistemului, Ciuca". Postarea este activa si dupa ora 18:00 pe 23.11.2024, incalcand astfel prevederile legale privind incheierea campaniei electorale.
    \item publicarea si promovarea unei reclame pe Facebook (ID: \href{https://www.facebook.com/ads/library/?id=914310827466619}{914310827466619}) dupa ora 18:00 pe 23.11.2024, in care se face propaganda electorala negativa la adresa candidatilor prezidentiali Marcel Ciolacu si Nicolae Ciuca. Postarea contine elemente clare de propaganda electorala, inclusiv numar CMF (11240015), referinte directe la alegerile prezidentiale ("ca tot sunt alegeri peste 3 zile"), critici directe la adresa candidatilor si indemnuri explicite de vot ("Ai ocazia sa-i lasi fara putere si fara privilegii. Vino la vot!"). Efectul electoral urmarit este diminuarea increderii in candidatii PSD si PNL si influentarea comportamentului electoral al cititorilor.
\end{enumerate}

\vspace{0.5cm}

\subsection{USR Galati}
Următoarele fapte contravenționale sunt sesizate împotriva acestei entități:

\begin{enumerate}[leftmargin=*, label=\arabic*.)]
    \item publicarea si promovarea dupa data de 23.11.2024, ora 18:00, a unei reclame electorale (ID Facebook: \href{https://www.facebook.com/ads/library/?id=1102673304560545}{1102673304560545}) ce contine propaganda electorala explicita pentru alegerile prezidentiale. Postarea contine un atac direct la adresa candidatului prezidential Marcel Ciolacu, folosind naratiunea zborurilor private Nordis pentru a-i diminua credibilitatea, culminand cu un indemn explicit de vot pentru USR ("Votam USR  Pozitia 1 pe TOATE buletinele de vot!"). Postarea include numar CMF (11240015), confirmand natura sa de propaganda electorala, si este promovata ca reclama platita pe Facebook si Instagram catre un public tinta de 500.001-1.000.000 de persoane, avand un efect electoral semnificativ in perioada de interdictie.
\end{enumerate}

\vspace{0.5cm}

\subsection{USR Ilfov}
Următoarele fapte contravenționale sunt sesizate împotriva acestei entități:

\begin{enumerate}[leftmargin=*, label=\arabic*.)]
    \item continuarea propagandei electorale dupa incheierea perioadei legale de campanie, prin postarea cu ID \href{https://www.facebook.com/ads/library/?id=449144147850872}{449144147850872} pe Facebook. Postarea, difuzata dupa ora 18:00 pe 23.11.2024, contine elementele constitutive ale propagandei electorale: are numar CMF (11240015), vizeaza direct un candidat la presedintie (Nicolae Ciuca), face o comparatie negativa cu regimul Ceausescu, si indeamna explicit la vot impotriva unui candidat prezidential pentru data de 24 noiembrie. Postarea este sponsorizata si are un impact semnificativ, atingand intre 90.000 si 99.999 de persoane, demonstrand intentia clara de a influenta comportamentul electoral al alegatorilor in ziua votului.
    \item continuarea propagandei electorale pentru alegerile prezidentiale dupa incheierea perioadei legale, prin postarea cu ID \href{https://www.facebook.com/ads/library/?id=571435878604456}{571435878604456} pe Facebook/Instagram. Postarea, publicata si promovata dupa ora 18:00 pe 23.11.2024, contine un mesaj electoral explicit care il vizeaza negativ pe candidatul ION-MARCEL CIOLACU, foloseste numar CMF (11240015), face referire directa la alegerile din 1 decembrie si indeamna alegatorii sa faca o alegere specifica ("Politicieni imbuibati sau oameni ca mine si tine"), constituind astfel propaganda electorala conform Art. 36(7) din Legea 334/2006. Postarea a avut un impact semnificativ, atingand intre 100.000 si 124.999 de persoane, cu o investitie de peste 2.000 RON in promovare.
\end{enumerate}

\vspace{0.5cm}

\subsection{USR Maramures}
Următoarele fapte contravenționale sunt sesizate împotriva acestei entități:

\begin{enumerate}[leftmargin=*, label=\arabic*.)]
    \item continuarea propagandei electorale dupa incheierea campaniei electorale pentru alegerile prezidentiale, prin postarea cu ID \href{https://www.facebook.com/ads/library/?id=828153992620006}{828153992620006} pe Facebook. Postarea, difuzata si dupa ora 18:00 din 23.11.2024, promoveaza explicit candidatura Elenei Lasconi la presedintie, prezentand-o ca "viitor presedinte al Romaniei" si detaliind programul sau electoral, incluzand promisiuni specifice campaniei. Postarea include coduri CMF (11240015, 31240009), confirmand natura sa de propaganda electorala, si a fost promovata ca reclama platita, atingand intre 15.000 si 19.999 de persoane, avand astfel un impact electoral semnificativ in perioada in care propaganda electorala este interzisa prin lege.
\end{enumerate}

\vspace{0.5cm}

\subsection{USR, prin intermediul paginii "Ionut Mosteanu"}
Următoarele fapte contravenționale sunt sesizate împotriva acestei entități:

\begin{enumerate}[leftmargin=*, label=\arabic*.)]
    \item publicarea si promovarea dupa data de 23.11.2024, ora 18:00, a unui material de propaganda electorala (ID postare Facebook: \href{https://www.facebook.com/ads/library/?id=571074518803045}{571074518803045}) care contine un mesaj explicit de influentare a votului pentru alegerile prezidentiale. Postarea include indemnuri directe la vot ("Veniti la vot!"), promoveaza candidatul Elena Lasconi prin hashtag-ul "\#ElenaLasconiPresedinte", ataca direct contracandidatul Marcel Ciolacu, si contine numar CMF (11240015), demonstrand natura sa de propaganda electorala. Efectul electoral urmarit este clar: descurajarea votului pentru Marcel Ciolacu si incurajarea votului pentru Elena Lasconi, incalcand astfel explicit prevederile legale privind incheierea campaniei electorale.
\end{enumerate}

\vspace{0.5cm}

\subsection{USR, prin reprezentantul Narcis Mircescu}
Următoarele fapte contravenționale sunt sesizate împotriva acestei entități:

\begin{enumerate}[leftmargin=*, label=\arabic*.)]
    \item continuarea propagandei electorale pentru candidatul la presedintie Elena Lasconi dupa incheierea perioadei legale de campanie, prin intermediul unei postari sponsorizate pe Facebook (ID: \href{https://www.facebook.com/ads/library/?id=569180649096733}{569180649096733}) care solicita explicit sustinerea candidatului ("Elena Lasconi la Presedintie") si promite actiuni viitoare in calitate de presedinte. Postarea, care include numar CMF 11240015, a ajuns la peste 200.000 de utilizatori dupa data de 23.11.2024, ora 18:00, avand un efect electoral direct prin solicitarea explicita de voturi si prezentarea candidatului ca "sansa reala de a avea un presedinte care sa fie cu adevarat aproape de romani", constituind astfel propaganda electorala activa in afara perioadei legale de campanie.
\end{enumerate}

\vspace{0.5cm}

\subsection{Ucu Dima si PNL Arad}
Următoarele fapte contravenționale sunt sesizate împotriva acestei entități:

\begin{enumerate}[leftmargin=*, label=\arabic*.)]
    \item publicarea si promovarea unui material de propaganda electorala (ID postare Facebook: \href{https://www.facebook.com/ads/library/?id=1256654532239498}{1256654532239498}) dupa incheierea perioadei de campanie electorala pentru alegerile prezidentiale. Postarea, realizata si promovata dupa ora 18:00 pe 23.11.2024, contine mesaje explicite de sustinere a candidatului Nicolae Ciuca la presedintie, descriindu-l ca "un lider de exceptie" si "alegerea ideala pentru functia de Presedinte al Romaniei", avand un evident scop electoral de influentare a votului. Materialul include numar CMF (11240002), confirmand natura sa de propaganda electorala, si este difuzat ca reclama platita pe platformele Facebook si Instagram, maximizand impactul si audienta mesajului electoral dupa incheierea perioadei legale de campanie.
\end{enumerate}

\vspace{0.5cm}

\subsection{Ucu Dima si Partidul National Liberal}
Următoarele fapte contravenționale sunt sesizate împotriva acestei entități:

\begin{enumerate}[leftmargin=*, label=\arabic*.)]
    \item continuarea propagandei electorale dupa incheierea campaniei electorale pentru alegerile prezidentiale, prin postarea cu ID \href{https://www.facebook.com/ads/library/?id=482379781056772}{482379781056772} pe Facebook. Postarea, sponsorizata si activa dupa ora 18:00 pe 23.11.2024, promoveaza explicit candidatura lui Nicolae Ciuca la presedintie, folosind expresii precum "il sustinem cu incredere pe Nicolae Ciuca - un lider de exceptie" si "alegerea ideala pentru functia de Presedinte al Romaniei". Materialul este marcat clar ca propaganda electorala prin prezenta numarului CMF 11240002, are obiectiv electoral explicit si vizeaza influentarea directa a votului pentru alegerile prezidentiale, depasind cadrul legal al campaniei electorale.
\end{enumerate}

\vspace{0.5cm}

\subsection{Uniunea Salvati Romania}
Următoarele fapte contravenționale sunt sesizate împotriva acestei entități:

\begin{enumerate}[leftmargin=*, label=\arabic*.)]
    \item distribuirea unui material de propaganda electorala (ID: \href{https://www.facebook.com/ads/library/?id=1099811228424088}{1099811228424088}) dupa data de 23.11.2024, ora 18:00. Materialul contine referiri directe la candidatii la presedintie Marcel Ciolacu si Nicolae Ciuca, folosind un ton denigrator ("fratii Cici"), are scop electoral explicit vizibil prin indemnul la vot si critica directa a candidatilor prezidentiali, fiind distribuit ca reclama platita pe Facebook cu un impact semnificativ (60.000-70.000 afisari). Materialul contine numarul CMF 11240015, confirmand natura sa de propaganda electorala, si urmareste influentarea comportamentului electoral prin criticarea directa a candidatilor prezidentiali si indemnul de a vota impotriva acestora.
    \item continuarea propagandei electorale dupa incheierea campaniei electorale, prin intermediul unei postari sponsorizate pe Facebook (ID: \href{https://www.facebook.com/ads/library/?id=884481050533794}{884481050533794}) care promoveaza candidatul Elena Lasconi si ataca direct contracandidatii Ciolacu, Simion, Ciuca si Geoana. Postarea contine numar CMF (11240015), mesaje clare de sustinere electorala si indemnuri directe impotriva altor candidati, avand un efect electoral direct prin atingerea unui public tinta de peste 100.000 de persoane. Mesajul este distribuit dupa ora 18:00 pe 23.11.2024, incalcand astfel perioada legala de campanie electorala si constituind propaganda electorala explicit interzisa in aceasta perioada.
\end{enumerate}

\vspace{0.5cm}

\subsection{Uniunea Salvati Romania (USR)}
Următoarele fapte contravenționale sunt sesizate împotriva acestei entități:

\begin{enumerate}[leftmargin=*, label=\arabic*.)]
    \item publicarea si promovarea unei reclame platite pe Facebook (ID: \href{https://www.facebook.com/ads/library/?id=463364796334558}{463364796334558}) dupa ora 18:00 pe 23.11.2024, continand propaganda electorala explicita pentru alegerile prezidentiale. Postarea contine un indemn direct la vot pentru candidatul Elena Lasconi ("\#ElenaLasconiPresedinte"), precum si atacuri la adresa contracandidatului Marcel Ciolacu, fiind marcata cu numar CMF 11240015. Postarea a fost distribuita ca reclama platita pe Facebook si Instagram, atingand intre 60.000 si 70.000 de afisari, demonstrand intentia clara de a influenta votul in cadrul alegerilor prezidentiale prin continuarea propagandei electorale dupa incheierea perioadei legale.
\end{enumerate}

\vspace{0.5cm}

\subsection{VESTEA MEDIA}
Următoarele fapte contravenționale sunt sesizate împotriva acestei entități:

\begin{enumerate}[leftmargin=*, label=\arabic*.)]
    \item promovarea unui articol cu caracter electoral negativ la adresa candidatului ION-MARCEL CIOLACU dupa incheierea perioadei de campanie electorala (dupa ora 18:00 pe 23.11.2024). Materialul publicitar cu ID-ul \href{https://www.facebook.com/ads/library/?id=1516528592335743}{1516528592335743} pe facebook reprezinta propaganda electorala negativa, utilizand termeni precum "aroganta" si facand asocieri negative despre viata personala a candidatului, cu scopul clar de a influenta negativ opinia alegatorilor. Articolul depaseste limitele jurnalismului obiectiv prin ton si limbaj, fiind promovat cu bani pe retelele sociale pentru a atinge un public cat mai larg, demonstrand intentia clara de a influenta rezultatul electoral.
\end{enumerate}

\vspace{0.5cm}

\subsection{VIVINET Agency (SC NANOPRO ARL)}
Următoarele fapte contravenționale sunt sesizate împotriva acestei entități:

\begin{enumerate}[leftmargin=*, label=\arabic*.)]
    \item continuarea propagandei electorale dupa incheierea campaniei pentru alegerile prezidentiale, prin publicarea si mentinerea activa a unei reclame pe Facebook si Instagram (ID: \href{https://www.facebook.com/ads/library/?id=833832162042609}{833832162042609}) care promoveaza explicit candidatul Nicolae-Ionel Ciuca la presedintie, folosind hashtag-ul "\#ciucapresedinte" si asociind imaginea acestuia cu realizari administrative locale, avand un impact direct asupra intentiei de vot. Postarea, fiind inca activa dupa ora 18:00 pe 23.11.2024, reprezinta o continuare clara a propagandei electorale in perioada interzisa, atingand un public intre 30.000 si 35.000 de persoane, cu un buget de aproximativ 250 RON.
\end{enumerate}

\vspace{0.5cm}

\subsection{VOCEA TV}
Următoarele fapte contravenționale sunt sesizate împotriva acestei entități:

\begin{enumerate}[leftmargin=*, label=\arabic*.)]
    \item difuzarea unui anunt platit pe Facebook (ID: \href{https://www.facebook.com/ads/library/?id=1615145879409512}{1615145879409512}) care constituie propaganda electorala dupa incheierea perioadei legale de campanie, respectiv dupa ora 18:00 pe 23.11.2024. Postarea prezinta un mesaj cu impact electoral direct, anuntand retragerea candidatului Ludovic Orban si redirectionarea sprijinului sau catre Elena Lasconi, fapt ce poate influenta direct comportamentul electoral al votantilor. Materialul are caracter de propaganda electorala conform Art. 36(7), identificand clar candidatii, adresandu-se publicului larg prin publicitate platita, si avand un obiectiv electoral explicit de influentare a votului in favoarea unui candidat.
\end{enumerate}

\vspace{0.5cm}

\subsection{Vladut Purja}
Următoarele fapte contravenționale sunt sesizate împotriva acestei entități:

\begin{enumerate}[leftmargin=*, label=\arabic*.)]
    \item publicarea si promovarea unei reclame electorale platite pe Facebook si Instagram (ID: \href{https://www.facebook.com/ads/library/?id=1266112257969679}{1266112257969679}) dupa incheierea perioadei de campanie electorala pentru alegerile prezidentiale. Postarea contine un indemn explicit de a vota candidatul Marcel Ciolacu la alegerile prezidentiale din 24.11.2024, folosind formulari precum "haideti sa fim alaturi de el, duminica" si "este omul potrivit pentru a ne reprezenta tara in urmatorii 4 ani". Materialul foloseste realizari administrative locale pentru a influenta decizia de vot a cetatenilor din Comuna Tarlisua, depasind astfel cadrul legal permis dupa ora 18:00 pe 23.11.2024. Efectul electoral este evident prin asocierea directa a realizarilor locale cu candidatul si solicitarea explicita de sustinere la vot.
\end{enumerate}

\vspace{0.5cm}

\subsection{Vocea Botosani}
Următoarele fapte contravenționale sunt sesizate împotriva acestei entități:

\begin{enumerate}[leftmargin=*, label=\arabic*.)]
    \item promovarea unui material cu caracter electoral dupa incheierea perioadei de propaganda, respectiv dupa ora 18:00 pe 23.11.2024. Materialul publicitar platit pe Facebook (ID: \href{https://www.facebook.com/ads/library/?id=1273462180683780}{1273462180683780}) prezinta in mod explicit un candidat la alegerile prezidentiale (George Simion), intr-un context negativ, prin publicarea unei convorbiri controversate si utilizarea unui limbaj care poate influenta decizia alegatorilor. Continutul depaseste sfera jurnalistica obiectiva, avand un caracter vadit electoral prin modul de prezentare si contextul temporal al promovarii, cu intentia clara de a influenta negativ opinia publica fata de candidatul mentionat in perioada restrictionata legal.
    \item publicarea si promovarea activa a unui material de propaganda electorala dupa incheierea perioadei legale de campanie, la data de 23.11.2024 dupa ora 18:00. Materialul, avand codul CMF31240003, promoveaza explicit candidatul Nicolae-Ionel Ciuca la functia de presedinte, folosind declaratiile fostului ambasador Adrian Zuckerman pentru a influenta preferintele electorale ale votantilor. Postarea include mesaje clare de sustinere electorala precum "Nicolae Ciuca, liderul de care Romania are nevoie pentru a atinge prosperitatea si siguranta" si promoveaza programul sau electoral, avand un evident scop de influentare a votului. ID postare Facebook: \href{https://www.facebook.com/ads/library/?id=1631655967707245}{1631655967707245}.
\end{enumerate}

\vspace{0.5cm}

\subsection{Voga Design \& Architecture}
Următoarele fapte contravenționale sunt sesizate împotriva acestei entități:

\begin{enumerate}[leftmargin=*, label=\arabic*.)]
    \item distribuirea unui material de propaganda electorala (ID Facebook: \href{https://www.facebook.com/ads/library/?id=1922465028238746}{1922465028238746}) dupa ora 18:00 pe 23.11.2024, continand un indemn explicit de vot pentru candidatul George Simion la alegerile prezidentiale ("Duminica votez altceva: un familist, cinstit, patriot  pe George Simion!"). Materialul include numar CMF (11200014), foloseste hashtag-uri electorale (\#GeorgeSimion \#Presedinte), si contine mesaje negative la adresa altor candidati prin criticarea partidelor acestora. Postarea este promovata ca reclama platita pe Facebook, cu un impact estimat intre 20.000 si 24.999 de afisari, reprezentand o incercare clara de influentare a votului in perioada in care propaganda electorala este interzisa prin lege.
    \item continuarea propagandei electorale dupa incheierea campaniei, prin intermediul unei postari sponsorizate pe Facebook (ID: \href{https://www.facebook.com/ads/library/?id=4091020691135777}{4091020691135777}) care promoveaza candidatul George Simion la alegerile prezidentiale. Postarea, activa si dupa ora 18:00 pe 23.11.2024, contine elemente clare de propaganda electorala, inclusiv numar CMF (11240014), naratiune personala menita sa influenteze electoratul, si indemn direct la vot ("Alegeti curajul, demnitatea, omul care lupta pentru limba noastra"). Impactul electoral este evident prin constructia narativa care il prezinta pe candidat intr-o lumina favorabila si prin apelul explicit la sustinerea sa electorala, targetand o audienta estimata intre 500,001 si 1,000,000 de persoane.
\end{enumerate}

\vspace{0.5cm}

\subsection{WEBASSIST CONSULTING SRL}
Următoarele fapte contravenționale sunt sesizate împotriva acestei entități:

\begin{enumerate}[leftmargin=*, label=\arabic*.)]
    \item promovarea unui mesaj de propaganda electorala (ID postare Facebook: \href{https://www.facebook.com/ads/library/?id=923407932626906}{923407932626906}) dupa incheierea perioadei de campanie electorala, respectiv dupa ora 18:00 pe 23.11.2024. Mesajul promovat contine elemente clare de propaganda electorala, incluzand indemnuri directe de vot pentru candidatul Elena Lasconi, descurajarea votului pentru Nicolae Ciuca, precum si avertizari negative despre Marcel Ciolacu si George Simion. Postarea, promovata prin plata, a ajuns la un public de 125.000-150.000 persoane, demonstrand caracterul sau de comunicare electorala in masa. Efectul electoral urmarit este clar: redirectionarea voturilor de la un candidat la altul si influentarea rezultatului alegerilor prezidentiale prin sugestii explicite de vot.
\end{enumerate}

\vspace{0.5cm}

\subsection{WEBSITESDESIGN SRL}
Următoarele fapte contravenționale sunt sesizate împotriva acestei entități:

\begin{enumerate}[leftmargin=*, label=\arabic*.)]
    \item difuzarea de materiale de propaganda electorala dupa incheierea campaniei electorale pentru alegerile prezidentiale, prin intermediul unei reclame platite pe Facebook (ID: \href{https://www.facebook.com/ads/library/?id=559852190367969}{559852190367969}) care promoveaza direct candidatul de pe pozitia 13 la alegerile prezidentiale. Postarea, difuzata dupa ora 18:00 pe 23.11.2024, contine un indemn explicit la vot ("Pe 1 decembrie votam DREPT. Pozitia 13 pe Buletinul de vot!"), are numar CMF (11240046), si promoveaza un program electoral legat de birocratia si legislatia fiscala, cu scopul clar de a influenta alegatorii in favoarea candidatului de pe pozitia 13, incalcand astfel prevederile legale privind perioada de campanie electorala.
\end{enumerate}

\vspace{0.5cm}

\subsection{WEDEV NOVAT SRL}
Următoarele fapte contravenționale sunt sesizate împotriva acestei entități:

\begin{enumerate}[leftmargin=*, label=\arabic*.)]
    \item difuzarea de materiale de propaganda electorala dupa incheierea campaniei electorale pentru alegerile prezidentiale, prin postarea cu ID \href{https://www.facebook.com/ads/library/?id=1102376751503672}{1102376751503672} pe facebook. Postarea, activa dupa ora 18:00 pe 23.11.2024, promoveaza candidatul Marcel Ciolacu prin evidentierea realizarilor guvernamentale si contine un indemn direct la vot pentru data de 24 noiembrie ("Pe 24 noiembrie, stiti cu cine sa votati pentru a continua aceste proiecte!"). Materialul este marcat cu cod CMF (11240017), confirmat ca material de propaganda electorala, si a avut un impact semnificativ, ajungand la 25.000-29.999 de persoane prin promovare platita.
\end{enumerate}

\vspace{0.5cm}

\subsection{Zarnescu Samuel, reprezentant AUR}
Următoarele fapte contravenționale sunt sesizate împotriva acestei entități:

\begin{enumerate}[leftmargin=*, label=\arabic*.)]
    \item continuarea propagandei electorale dupa incheierea campaniei pentru alegerile prezidentiale, promovand prin postare platita pe Facebook (ID: \href{https://www.facebook.com/ads/library/?id=1535094047370846}{1535094047370846}) un mesaj explicit de sustinere si indemn la vot pentru candidatul George Simion la presedintie. Postarea contine indemnuri directe la vot ("Votam pentru George Simion presedintele Romaniei"), fiind difuzata dupa ora 18:00 pe 23.11.2024, cu un impact semnificativ demonstrat prin datele de reach (10,000-14,999 impresii). Mesajul constituie propaganda electorala conform Art. 36(7) din Legea 334/2006, indeplinind toate criteriile: referire directa la candidat, obiectiv electoral explicit si adresare catre publicul larg prin intermediul platformelor de social media.
\end{enumerate}

\vspace{0.5cm}

\subsection{ZiarulEconomic}
Următoarele fapte contravenționale sunt sesizate împotriva acestei entități:

\begin{enumerate}[leftmargin=*, label=\arabic*.)]
    \item promovarea unui mesaj de propaganda electorala platit (ID postare \href{https://www.facebook.com/ads/library/?id=2008908872902760}{2008908872902760}) dupa incheierea perioadei de campanie electorala, respectiv dupa ora 18:00 pe 23.11.2024. Postarea reprezinta propaganda electorala evidenta prin atacarea directa a candidatei Elena Lasconi (folosind termeni precum "neomarxista" si "incompetenta") si promovarea explicita a candidatului George Simion (prin hashtag-ul \#GeorgeSimionPresedinte si declaratii precum "singurul lider care poate readuce demnitatea"). Mesajul a fost distribuit contra cost pe platformele Facebook si Instagram, cu un impact estimat intre 40.000 si 45.000 de afisari, demonstrand intentia clara de a influenta comportamentul electoral al unui public larg.
\end{enumerate}

\vspace{0.5cm}

\section{Solicitări}

Față de cele de mai sus, solicit:

\begin{enumerate}[leftmargin=*, label=\arabic*.]
    \item Constatarea contravențiilor săvârșite;
    \item Identificarea persoanelor vinovate;
    \item Aplicarea sancțiunilor prevăzute de lege.
\end{enumerate}

\section{Anexe}

Anexez prezentei plângeri următoarele dovezi:

\begin{enumerate}[leftmargin=*, label=\arabic*.]
    \item Capturi de ecran ale postărilor care fac obiectul sesizării;
    \item Dovada calității de observator electoral.
\end{enumerate}

\vspace{1cm}
\noindent Data: \today

\vspace{1.5cm}
\noindent Observator electoral,\\[0.3cm]
Deleanu Ștefan-Lucian

\vspace{1cm}
\noindent Semnătura: [SEMNAT ELECTRONIC]

\end{document}
