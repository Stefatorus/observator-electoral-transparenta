\documentclass[a4paper,12pt]{article}
\usepackage[romanian]{babel}
\usepackage[utf8]{inputenc}
\usepackage[T1]{fontenc}
\usepackage{geometry}
\usepackage{enumitem}
\usepackage{titlesec}
\usepackage{parskip}
\usepackage{microtype}
\usepackage[none]{hyphenat}
\usepackage{ragged2e}
\usepackage{url}
\usepackage{xurl}
\usepackage{hyperref}
\usepackage{fancyhdr}
\usepackage{listings}
\usepackage{tocloft}
\usepackage{lastpage}  % <-- Added lastpage package

% Set margins and spacing
\geometry{
    a4paper,
    margin=2.5cm,
    includehead
}

% Header and Footer
\pagestyle{fancy}
\fancyhf{}
\lhead{Plângere Contravențională}
\rhead{pagina \thepage/\pageref{LastPage}}  % <-- Updated header
\renewcommand{\headrulewidth}{0.4pt}

% Table of Contents formatting
\renewcommand{\contentsname}{Cuprins}
\setcounter{tocdepth}{2}  % Include up to subsections in TOC
\renewcommand{\cftsecfont}{\normalsize\bfseries}
\renewcommand{\cftsubsecfont}{\normalsize}
\renewcommand{\cftsecpagefont}{\normalsize}
\renewcommand{\cftsubsecpagefont}{\normalsize}
\setlength{\cftbeforesecskip}{5pt}
\setlength{\cftbeforesubsecskip}{2pt}

% Section formatting
\titleformat{\section}
  {\normalfont\Large\bfseries}{\thesection.}{0.5em}{}
\titleformat{\subsection}
  {\normalfont\large\bfseries}{\thesubsection.}{0.5em}{}

\begin{document}

\begin{center}
    \Large\textbf{PLÂNGERE CONTRAVENȚIONALĂ}
\end{center}

\vspace{1cm}

Subsemnatul Deleanu Ștefan-Lucian, domiciliat în Jud. Cluj, Cluj-Napoca, Str. Aurel Vlaicu, nr. 2, bloc 5A, Sc. I, etaj 7, ap. 28, în calitate de observator electoral acreditat de Funky Citizens, asociatie acreditata de Autoritatea Electorala Permanenta prin ACREDITAREA nr. 29743/07.10.2024, în temeiul prevederilor Legii 370/2004 privind alegerea Președintelui României, cu modificările și completările ulterioare, formulez următoarea:

\vspace{0.5cm}
\begin{center}
\textbf{\Large PLÂNGERE CONTRAVENȚIONALĂ}
\end{center}
\vspace{0.5cm}

prin care sesizez următoarele contraventii definite de art. 55 lit t), art 56, pg 1, pg 2 lit a, din Legea 370/2004 privind alegerea Președintelui României.

In analiza caracterului de propaganda electorala, s-a avut in vedere definitia prevazuta de Art 36, pct 7, din legea 334/2006, republicata, cu modificarile si completarile ulterioare.


\newpage
\tableofcontents
\newpage

\section{Împotriva numiților}

\subsection{"Curierul Romanesc"}
Următoarele fapte contravenționale sunt sesizate împotriva acestei entități:

\begin{enumerate}[leftmargin=*, label=\arabic*.)]
    \item difuzarea de materiale de propaganda electorala dupa incheierea campaniei electorale, constand intr-o postare sponsorizata pe Facebook (ID: \href{https://www.facebook.com/ads/library/?id=2904984826317098}{2904984826317098}) care transmite mesaje negative despre candidatul prezidential Marcel Ciolacu, avand ca scop influentarea votului. Postarea a fost promovata dupa ora 18:00 pe 23.11.2024, incalcand explicit prevederile legale privind incetarea campaniei electorale. Materialul indeplineste toate conditiile prevazute in Art. 36(7) pentru a fi considerat propaganda electorala, fiind direct tintit impotriva unui candidat, avand obiectiv electoral clar si adresandu-se publicului larg prin intermediul unei platforme de social media cu plata.
    \item promovarea cu plata a unui continut legat de alegerile prezidentiale dupa incheierea campaniei electorale. Postarea cu ID-ul \href{https://www.facebook.com/ads/library/?id=892001846396135}{892001846396135} promoveaza previziuni electorale despre alegerile prezidentiale prin intermediul platformelor Facebook si Instagram, cu un impact semnificativ (35,000-40,000 impresii), dupa ora 18:00 pe 23.11.2024. Aceasta actiune constituie propaganda electorala continuata dupa incheierea perioadei legale de campanie.
\end{enumerate}

\vspace{0.5cm}

\subsection{"Opinii Independente"}
Următoarele fapte contravenționale sunt sesizate împotriva acestei entități:

\begin{enumerate}[leftmargin=*, label=\arabic*.)]
    \item difuzarea de materiale de propaganda electorala dupa incheierea campaniei electorale pentru alegerile prezidentiale. Postarea cu ID-ul \href{https://www.facebook.com/ads/library/?id=491454140618987}{491454140618987} promoveaza candidatul Mircea-Dan Geoana si aliantele sale politice, fiind o reclama platita pe Facebook si Instagram, cu un buget intre 100-199 RON, care a ajuns la 4,000-4,999 de persoane. Postarea este activa si dupa ora 18:00 pe 23.11.2024, incalcand astfel prevederile legale privind incheierea campaniei electorale pentru alegerile prezidentiale.
\end{enumerate}

\vspace{0.5cm}

\subsection{"Salvam Romania"}
Următoarele fapte contravenționale sunt sesizate împotriva acestei entități:

\begin{enumerate}[leftmargin=*, label=\arabic*.)]
    \item promovarea unui material de propaganda electorala dupa incheierea campaniei electorale, constand intr-o postare sponsorizata pe Facebook (ID: \href{https://www.facebook.com/ads/library/?id=933759081968191}{933759081968191}) care vizeaza in mod direct si negativ candidatul ELENA-VALERICA LASCONI. Postarea a fost promovata dupa ora 18:00 pe 23.11.2024, in perioada de restrictie, cu un buget semnificativ (4500-5000 RON) si o audienta estimata de peste 150.000 de persoane, avand un evident caracter de propaganda electorala negativa prin prezentarea unor informatii menite sa influenteze optiunea de vot a alegatorilor.
\end{enumerate}

\vspace{0.5cm}

\subsection{4Media.INFO}
Următoarele fapte contravenționale sunt sesizate împotriva acestei entități:

\begin{enumerate}[leftmargin=*, label=\arabic*.)]
    \item promovarea unui material cu caracter electoral negativ la adresa candidatului ION-MARCEL CIOLACU dupa incheierea perioadei de campanie electorala. Materialul, cu ID-ul postarii \href{https://www.facebook.com/ads/library/?id=1086682602723380}{1086682602723380}, contine caracterizari negative directe ("minciuna de Pinocchio", "premier fara busola") si este promovat prin reclama platita pe Facebook si Instagram dupa ora 18:00 pe 23.11.2024, avand un impact estimat de peste 1 milion de persoane. Caracterul electoral este evident prin modul in care ataca credibilitatea candidatului in perioada pre-electorala.
    \item difuzarea de materiale de propaganda electorala dupa incheierea campaniei electorale, constand intr-o postare sponsorizata pe Facebook si Instagram (ID: \href{https://www.facebook.com/ads/library/?id=1106793317616811}{1106793317616811}) care vizeaza in mod direct candidatul ION-MARCEL CIOLACU, prezentandu-l intr-o lumina negativa si incercand sa influenteze decizia de vot a alegatorilor. Postarea a fost promovata dupa ora 18:00 pe 23.11.2024, in perioada de restrictie electorala, atingand intre 6000 si 6999 de persoane, reprezentand o incalcare clara a prevederilor legale privind incetarea propagandei electorale.
    \item promovarea cu plata a unui material de propaganda electorala negativa impotriva candidatului NICOLAE-IONEL CIUCA dupa incheierea perioadei de campanie. Materialul, cu ID-ul postarii \href{https://www.facebook.com/ads/library/?id=458388087285719}{458388087285719}, prezinta in mod tendentios informatii despre un presupus dosar penal, cu scopul clar de a influenta negativ opinia alegatorilor. Postarea a fost promovata activ dupa ora 18:00 pe 23.11.2024, incalcand astfel prevederile legale privind incheierea campaniei electorale. Impactul este demonstrat de numarul mare de afisari (peste 7000) si audienta tintita de peste 1 milion de persoane.
    \item promovarea unui material cu caracter electoral dupa incheierea perioadei de campanie, referitor la candidatul ELENA-VALERICA LASCONI. Materialul promovat (ID Facebook: \href{https://www.facebook.com/ads/library/?id=762479362731326}{762479362731326}) vizeaza in mod direct un candidat la presedintie, prezentand declaratii controversate despre avort, cu potential de influentare a votantilor. Postarea a fost promovata dupa ora 18:00 pe 23.11.2024, atingand intre 5000-6000 de persoane prin intermediul unei reclame platite pe Facebook si Instagram, depasind astfel limitele activitatii jurnalistice de informare si constituind propaganda electorala.
\end{enumerate}

\vspace{0.5cm}

\subsection{4media.INFO}
Următoarele fapte contravenționale sunt sesizate împotriva acestei entități:

\begin{enumerate}[leftmargin=*, label=\arabic*.)]
    \item promovarea unui material de propaganda electorala (ID postare Facebook: \href{https://www.facebook.com/ads/library/?id=1326414468524628}{1326414468524628}) dupa incheierea perioadei de campanie electorala. Materialul promovat face referire directa la candidatul prezidential George Simion, avand un caracter electoral evident prin titlul "George Simion explica cum se cucereste un popor", fiind difuzat dupa ora 18:00 pe 23.11.2024, cu un impact semnificativ demonstrat prin reach-ul de peste 30.000 de afisari si o investitie de 200-299 RON in promovare pe platformele Facebook si Instagram.
    \item publicarea si promovarea unei reclame platite pe Facebook si Instagram (ID: \href{https://www.facebook.com/ads/library/?id=3519344485024756}{3519344485024756}) dupa ora 18:00 pe 23.11.2024, care constituie propaganda electorala negativa la adresa candidatei ELENA-VALERICA LASCONI. Postarea, cu textul "Sa ne amuzam cu Lasconi", a fost promovata cu un buget intre 100-199 RON, atingand intre 35,000-39,999 de afisari, reprezentand o incercare clara de influentare a opiniei publice in perioada de restrictie electorala. Caracterul sau propagandistic este demonstrat de natura platita a postarii, audienta larga tintita si continutul sau denigrator la adresa unui candidat prezidential.
    \item publicarea si promovarea unui mesaj de propaganda electorala dupa incheierea campaniei electorale (ID postare Facebook: \href{https://www.facebook.com/ads/library/?id=452580617857882}{452580617857882}). Postarea, publicata si promovata dupa ora 18:00 pe 23.11.2024, contine un atac direct la adresa candidatei ELENA-VALERICA LASCONI, avand un caracter electoral evident negativ, cu intentia de a influenta optiunea alegatorilor. Mesajul a fost difuzat ca reclama platita pe Facebook si Instagram, cu o audienta estimata de peste 1 milion de persoane si intre 9.000-9.999 de afisari efective, demonstrand caracterul sau de propaganda electorala si nu de simpla opinie personala.
    \item difuzarea unui material de propaganda electorala dupa incheierea campaniei electorale, respectiv postarea cu ID-ul \href{https://www.facebook.com/ads/library/?id=902982898467615}{902982898467615} pe facebook, care promoveaza candidatul George Simion. Materialul este o reclama platita, cu un buget substantial (600-699 RON), care a generat intre 80,000 si 89,999 de afisari, fiind difuzata dupa ora 18:00 pe 23.11.2024. Postarea are caracter electoral evident, prezentand mesaje despre candidat intr-un mod care poate influenta decizia de vot, incalcand astfel prevederile legale privind incetarea campaniei electorale.
    \item difuzarea de materiale de propaganda electorala dupa incheierea campaniei electorale, constand intr-un anunt platit pe Facebook si Instagram (ID: \href{https://www.facebook.com/ads/library/?id=9386410671387627}{9386410671387627}) care o vizeaza direct si negativ pe candidata ELENA-VALERICA LASCONI. Postarea, difuzata dupa ora 18:00 pe 23.11.2024, constituie propaganda electorala prin faptul ca face afirmatii tendentioase despre un candidat ("ataca mereu fara probe"), cu intentia vadita de a influenta negativ optiunea alegatorilor. Mesajul a avut un impact semnificativ, ajungand la aproximativ 9.500 de utilizatori si fiind directionat catre un public potential de peste 1 milion de persoane.
\end{enumerate}

\vspace{0.5cm}

\subsection{5news.RO}
Următoarele fapte contravenționale sunt sesizate împotriva acestei entități:

\begin{enumerate}[leftmargin=*, label=\arabic*.)]
    \item difuzarea unui mesaj de propaganda electorala dupa incheierea perioadei de campanie, constand in postarea cu ID \href{https://www.facebook.com/ads/library/?id=1529511724418845}{1529511724418845} pe platformele Facebook si Instagram, care afirma ca "Lasconi nu are nici un fel de pregatire pentru a deveni presedinte". Mesajul constituie propaganda electorala negativa impotriva candidatei ELENA-VALERICA LASCONI, fiind difuzat dupa ora 18:00 pe 23.11.2024, avand un impact estimat intre 6000-7000 de afisari si fiind promovat cu bani (advertising platit). Mesajul are scop electoral clar, vizeaza influentarea votantilor si a fost distribuit in mod deliberat dupa incheierea campaniei electorale.
    \item publicarea si promovarea unui mesaj electoral negativ dupa incheierea perioadei de campanie electorala. Postarea cu ID-ul \href{https://www.facebook.com/ads/library/?id=1602086663727351}{1602086663727351} pe facebook constituie propaganda electorala prin faptul ca vizeaza direct candidatul ELENA-VALERICA LASCONI intr-un mod negativ ("pierde voturi din cauza arogantei"), avand un impact direct asupra procesului electoral. Mesajul a fost promovat ca reclama platita pe Facebook si Instagram dupa ora 18:00 pe 23.11.2024, atingand intre 6.000 si 6.999 de persoane, cu un public tinta potential de peste 1 milion de utilizatori.
    \item difuzarea unui mesaj de propaganda electorala negativa dupa incheierea perioadei de campanie, vizand candidatii Nicolae-Ionel Ciuca si Ion-Marcel Ciolacu. Materialul publicitar platit, cu ID-ul \href{https://www.facebook.com/ads/library/?id=1637704603755589}{1637704603755589}, a fost difuzat pe Facebook si Instagram dupa ora 18:00 pe 23.11.2024, atingand intre 80.000 si 90.000 de persoane, cu un mesaj care critica explicit activitatea guvernamentala a celor doi candidati la presedintie, avand un evident caracter de propaganda electorala negativa in perioada in care acest lucru este interzis prin lege.
    \item promovarea unui material cu continut electoral referitor la candidatul George-Nicolae Simion dupa incheierea campaniei electorale. Materialul, un articol sponsorizat pe Facebook si Instagram (ID: \href{https://www.facebook.com/ads/library/?id=2887411748100565}{2887411748100565}), a fost difuzat dupa ora 18:00 pe 23.11.2024, in perioada de liniste electorala. Impactul materialului este semnificativ, avand intre 15.000 si 19.999 de afisari, cu o investitie intre 100 si 199 RON, reprezentand o incalcare clara a prevederilor legale privind interdictia propagandei electorale dupa incheierea campaniei.
    \item difuzarea unui mesaj electoral negativ despre candidata ELENA-VALERICA LASCONI dupa incheierea campaniei electorale. Postarea cu ID-ul \href{https://www.facebook.com/ads/library/?id=564170259552157}{564170259552157} pe facebook reprezinta propaganda electorala prin faptul ca ataca direct un candidat la presedintie, fiind promovata ca reclama platita, cu un buget intre 300-399 RON si un impact de 25.000-29.999 afisari, dupa ora 18:00 pe 23.11.2024. Mesajul "elena lasconi nu are facultate" este conceput si distribuit cu scopul clar de a influenta negativ opinia publica fata de candidat in perioada post-campanie.
    \item publicarea si promovarea unei postari cu caracter de propaganda electorala (ID postare Facebook: \href{https://www.facebook.com/ads/library/?id=921254709937844}{921254709937844}) dupa ora 18:00 pe 23.11.2024. Postarea vizeaza in mod direct un candidat la alegerile prezidentiale (Elena Lasconi), folosind limbaj denigrator si avand scop electoral evident prin atingerea adusa imaginii candidatului. Postarea a fost promovata cu sume intre 200-299 RON, atingand intre 30.000-35.000 de persoane, demonstrand intentia clara de influentare a votului prin propaganda negativa dupa incheierea campaniei electorale.
\end{enumerate}

\vspace{0.5cm}

\subsection{60m.RO}
Următoarele fapte contravenționale sunt sesizate împotriva acestei entități:

\begin{enumerate}[leftmargin=*, label=\arabic*.)]
    \item difuzarea unui mesaj de propaganda electorala platit pe platforma Facebook/Instagram (ID postare: \href{https://www.facebook.com/ads/library/?id=1564600244158214}{1564600244158214}) dupa ora 18:00 pe 23.11.2024. Materialul vizeaza in mod direct candidatul ELENA-VALERICA LASCONI, folosind caracterizari negative ("nepregatita, isterica si superficiala") cu scopul clar de a influenta decizia de vot a alegatorilor. Impactul este demonstrat prin reach-ul substantial (100,000-124,999 impresii) si bugetul alocat (200-299 RON), reprezentand o incercare clara si sistematica de a influenta procesul electoral in afara perioadei legale de campanie.
    \item continuarea propagandei electorale dupa incheierea campaniei, prin publicarea si mentinerea activa a unui anunt sponsorizat pe Facebook si Instagram (ID: \href{https://www.facebook.com/ads/library/?id=1623065708567110}{1623065708567110}) care vizeaza in mod direct si negativ candidatul prezidential Elena Lasconi, dupa ora 18:00 pe 23.11.2024. Materialul, care a atins intre 25.000 si 30.000 de afisari, reprezinta propaganda electorala prin criticarea directa a competentei candidatului in domeniul politicii externe, cu intentia clara de a influenta opinia alegatorilor in perioada in care propaganda electorala este interzisa prin lege.
    \item promovarea unei reclame platite pe Facebook si Instagram (ID: \href{https://www.facebook.com/ads/library/?id=1743529059779832}{1743529059779832}) dupa ora 18:00 pe 23.11.2024, in perioada de liniste electorala. Continutul promovat face comparatii directe intre candidatii GEORGE-NICOLAE SIMION si ELENA-VALERICA LASCONI, prezentand unul dintre candidati intr-o lumina favorabila ("mai gentleman") si pe celalalt intr-o lumina nefavorabila, avand astfel un clar caracter de propaganda electorala. Reclama a fost difuzata catre un public estimat de peste 1 milion de persoane, demonstrand intentia clara de a influenta votul in perioada de liniste electorala.
    \item difuzarea unui mesaj de propaganda electorala negativa la adresa candidatei ELENA-VALERICA LASCONI dupa incheierea perioadei de campanie. Postarea cu ID-ul \href{https://www.facebook.com/ads/library/?id=2048135028942292}{2048135028942292} pe facebook contine atacuri la persoana si caracterizari negative ("nepregatita, isterica si superficiala"), avand un evident scop electoral negativ. Mesajul a fost promovat ca reclama platita, cu un impact intre 80.000-89.999 de afisari, continuand sa fie activ si dupa ora 18:00 pe 23.11.2024, incalcand astfel in mod direct prevederile legale privind incetarea propagandei electorale.
    \item promovarea unei reclame platite pe Facebook si Instagram (ID: \href{https://www.facebook.com/ads/library/?id=2320121051682166}{2320121051682166}) dupa ora 18:00 pe 23.11.2024, cu continut de propaganda electorala negativa la adresa candidatei Elena Lasconi. Postarea a avut un impact semnificativ, ajungand la peste 100.000 de afisari, fiind platita cu 200-299 RON. Continutul prezinta in mod denigrator pozitia candidatei pe politica externa, cu scopul clar de a influenta alegatorii in perioada de interdictie a propagandei electorale.
    \item publicarea si promovarea platita a unui mesaj de propaganda electorala dupa incheierea perioadei legale de campanie. Postarea cu ID-ul \href{https://www.facebook.com/ads/library/?id=3735627486698840}{3735627486698840} contine atacuri directe la adresa candidatilor Elena Lasconi, Nicolae Ciuca si Mircea Geoana, cu intentia clara de a influenta negativ optiunea de vot a alegatorilor. Materialul a fost promovat dupa ora 18:00 pe 23.11.2024, incalcand astfel prevederile legale privind incheierea campaniei electorale. Impactul a fost semnificativ, atingand intre 35.000 si 40.000 de afisari pe platformele Facebook si Instagram.
    \item difuzarea de materiale de propaganda electorala dupa incheierea campaniei electorale, constand intr-un anunt platit pe Facebook si Instagram (ID: \href{https://www.facebook.com/ads/library/?id=548556984620952}{548556984620952}) care promoveaza o comparatie intre candidatii GEORGE-NICOLAE SIMION si ELENA-VALERICA LASCONI, favorizandu-l pe primul si defavorizand-o pe cea de-a doua. Materialul a fost distribuit dupa ora 18:00 pe 23.11.2024, atingand intre 80.000 si 90.000 de impresii, cu un buget intre 200 si 299 RON, reprezentand astfel o incalcare clara a prevederilor legale privind incetarea propagandei electorale.
    \item promovarea unui mesaj de propaganda electorala dupa incheierea perioadei legale de campanie, respectiv dupa ora 18:00 pe 23.11.2024. Materialul promotional, cu ID-ul \href{https://www.facebook.com/ads/library/?id=596789496031396}{596789496031396} pe Facebook, contine atacuri directe la adresa candidatilor prezidentiali Elena Lasconi, Nicolae Ciuca si Mircea Geoana, cu scopul vadit de a influenta negativ opinia alegatorilor. Mesajul a fost promovat ca reclama platita, cu un impact estimat intre 90.000 si 100.000 de afisari, demonstrand intentia clara de a influenta un numar mare de alegatori in perioada restrictionata legal.
    \item publicarea si promovarea unei postari platite pe Facebook si Instagram (ID: \href{https://www.facebook.com/ads/library/?id=8705668632851069}{8705668632851069}) care constituie propaganda electorala negativa impotriva candidatei ELENA-VALERICA LASCONI dupa incheierea perioadei legale de campanie. Postarea, care a fost promovata dupa ora 18:00 pe 23.11.2024, are un evident caracter electoral negativ, comparand-o pe candidata in mod denigrator cu Viorica Dancila, atingand un public tinta de peste 50.000 de persoane prin intermediul unei reclame platite cu suma de aproximativ 450 RON, cu scopul clar de a influenta negativ optiunea de vot a alegatorilor.
\end{enumerate}

\vspace{0.5cm}

\subsection{AUR - Salasu de Sus}
Următoarele fapte contravenționale sunt sesizate împotriva acestei entități:

\begin{enumerate}[leftmargin=*, label=\arabic*.)]
    \item difuzarea de materiale de propaganda electorala dupa incheierea campaniei electorale pentru alegerile prezidentiale. Postarea cu ID-ul \href{https://www.facebook.com/ads/library/?id=1089703459529652}{1089703459529652} contine un indemn direct de a vota candidatul George Simion la functia de presedinte, fiind promovata ca reclama platita pe Facebook si Instagram, cu un impact estimat intre 25.000 si 30.000 de afisari. Materialul a fost difuzat si mentinut activ dupa ora 18:00 pe 23.11.2024, incalcand astfel prevederile legale privind incheierea campaniei electorale.
    \item publicarea si promovarea unei reclame platite pe Facebook (ID: \href{https://www.facebook.com/ads/library/?id=1121044549358758}{1121044549358758}) ce constituie propaganda electorala explicita pentru candidatul George Simion la functia de presedinte, dupa incheierea perioadei de campanie electorala. Postarea, care a avut un impact semnificativ (9000-9999 afisari), indeamna in mod direct la votarea candidatului AUR, folosind fonduri pentru promovare si targetand peste 1 milion de persoane, dupa ora 18:00 pe 23.11.2024, incalcand astfel prevederile legale privind incheierea campaniei electorale.
    \item continuarea propagandei electorale dupa incheierea campaniei, prin promovarea unui material electoral platit pe Facebook si Instagram (ID postare: \href{https://www.facebook.com/ads/library/?id=1231140648187092}{1231140648187092}) care il promoveaza explicit pe candidatul George Simion la functia de presedinte. Materialul este activ si dupa ora 18:00 pe 23.11.2024, incalcand astfel perioada legala de campanie. Postarea are un clear caracter de propaganda electorala, avand un impact intre 35.000-40.000 de afisari si un buget de 300-399 RON, promovand direct candidatul la presedintie si solicitand explicit sustinerea acestuia.
    \item continuarea propagandei electorale dupa incheierea perioadei legale de campanie, prin promovarea unui mesaj electoral explicit in favoarea candidatului George Simion si in defavoarea PSD, intr-o postare sponsorizata pe Facebook si Instagram (ID: \href{https://www.facebook.com/ads/library/?id=555150543924818}{555150543924818}). Postarea contine numar CMF (11240014), indemn direct la vot si promisiuni electorale, fiind difuzata dupa ora 18:00 pe 23.11.2024, incalcand astfel prevederile legale privind incetarea campaniei electorale.
    \item continuarea propagandei electorale dupa incheierea campaniei, manifestata prin publicarea si mentinerea activa a unei reclame platite pe Facebook (ID: \href{https://www.facebook.com/ads/library/?id=932890622061028}{932890622061028}) care promoveaza candidatul George Simion la functia de presedinte. Materialul, care a ajuns la 45.000-50.000 de afisari, reprezinta in mod clar propaganda electorala prin mesajul sau explicit "De ce e nevoie de George Simion PRESEDINTE?" si continua sa fie activ dupa ora 18:00 pe 23.11.2024, incalcand astfel perioada de interdictie a propagandei electorale.
    \item difuzarea de propaganda electorala dupa incheierea perioadei de campanie, constand intr-o postare sponsorizata pe Facebook (ID: \href{https://www.facebook.com/ads/library/?id=979381740667749}{979381740667749}) care indeamna in mod explicit la votarea candidatului George Simion la functia de presedinte. Postarea, difuzata dupa ora 18:00 pe 23.11.2024, contine mesaje de sustinere directa si indemn la vot ("il voi vota pe George Simion - presedinte"), avand un impact semnificativ prin atingerea a 10,000-14,999 de persoane prin intermediul unei cheltuieli publicitare de aproximativ 150 RON.
\end{enumerate}

\vspace{0.5cm}

\subsection{AUR Mures si Razvan Daniel BIRO}
Următoarele fapte contravenționale sunt sesizate împotriva acestei entități:

\begin{enumerate}[leftmargin=*, label=\arabic*.)]
    \item promovarea candidatului prezidential George Simion intr-o postare sponsorizata pe Facebook (ID: \href{https://www.facebook.com/ads/library/?id=916525717157198}{916525717157198}) dupa incheierea perioadei de campanie electorala prezidentiala. Postarea, activa dupa ora 18:00 pe 23.11.2024, contine mesaje electorale explicite si promoveaza imaginea candidatului prezidential intr-un mod care incalca prevederile legale privind incheierea campaniei electorale. Materialul include CMF 11240014, confirmand natura sa de propaganda electorala, si a fost distribuit catre un public estimat intre 100,001 si 500,000 de persoane.
\end{enumerate}

\vspace{0.5cm}

\subsection{Actiunea.RO}
Următoarele fapte contravenționale sunt sesizate împotriva acestei entități:

\begin{enumerate}[leftmargin=*, label=\arabic*.)]
    \item distribuirea de continut electoral platit (ID postare Facebook: \href{https://www.facebook.com/ads/library/?id=1299186844846991}{1299186844846991}) dupa incheierea perioadei de campanie electorala. Postarea promoveaza in mod direct candidatul George Simion si denigreaza candidata Elena Lasconi, avand un impact asupra procesului electoral prezidential. Materialul a fost distribuit dupa ora 18:00 pe 23.11.2024, incalcand explicit prevederile legale privind incheierea campaniei electorale. Postarea a avut un impact semnificativ, atingand intre 35.000 si 40.000 de persoane, fiind distribuita pe multiple platforme (Facebook si Instagram) si avand un buget de promovare intre 300-399 RON.
\end{enumerate}

\vspace{0.5cm}

\subsection{Adrian Axinia}
Următoarele fapte contravenționale sunt sesizate împotriva acestei entități:

\begin{enumerate}[leftmargin=*, label=\arabic*.)]
    \item promovarea unui mesaj de propaganda electorala (ID Facebook: \href{https://www.facebook.com/ads/library/?id=1720075795393117}{1720075795393117}) dupa ora 18:00 pe 23.11.2024. Materialul contine indemnuri explicite de vot pentru candidatul George Simion la presedintie, criticarea directa a contracandidatilor, si foloseste numar oficial de material electoral CMF 11240014. Postarea este promovata activ prin plata pe Facebook, cu un buget intre 100-199 RON si o audienta estimata de peste 1 milion de persoane, reprezentand astfel o incalcare clara a prevederilor legale privind incheierea campaniei electorale.
\end{enumerate}

\vspace{0.5cm}

\subsection{Adrian Costea}
Următoarele fapte contravenționale sunt sesizate împotriva acestei entități:

\begin{enumerate}[leftmargin=*, label=\arabic*.)]
    \item promovarea unui mesaj electoral platit pe Facebook (ID post: \href{https://www.facebook.com/ads/library/?id=1649645469265920}{1649645469265920}) dupa data de 23.11.2024, ora 18:00. Postarea promoveaza explicit candidatul Elena Lasconi pentru functia de presedinte, folosind argumentatie electorala si indemnuri directe la vot ("o iau pe Elena si o inghesui in turul 2"), constituind astfel propaganda electorala activa in afara perioadei legale de campanie. Mesajul a fost distribuit ca reclama platita, atingand intre 1000-1999 de persoane, demonstrand intentionalitatea si impactul electoral semnificativ.
\end{enumerate}

\vspace{0.5cm}

\subsection{Agendazilei.ro}
Următoarele fapte contravenționale sunt sesizate împotriva acestei entități:

\begin{enumerate}[leftmargin=*, label=\arabic*.)]
    \item difuzarea de materiale de propaganda electorala dupa incheierea perioadei de campanie, concretizata prin publicarea si promovarea platita a unui articol ce vizeaza in mod direct candidatul prezidential Marcel Ciolacu, cu scopul clar de a influenta negativ optiunea de vot a alegatorilor. Materialul, avand ID-ul facebook \href{https://www.facebook.com/ads/library/?id=875282118090212}{875282118090212}, a fost difuzat si dupa ora 18:00 pe 23.11.2024, incalcand astfel prevederile legale privind incetarea campaniei electorale. Impactul este demonstrat prin acoperirea larga (60,000-70,000 impresii) si investitia financiara (300-399 RON) in promovarea acestuia.
    \item difuzarea de materiale de propaganda electorala dupa incheierea campaniei electorale, constand intr-un articol sponsorizat pe Facebook (ID: \href{https://www.facebook.com/ads/library/?id=918152896522002}{918152896522002}) care ataca candidatul ION-MARCEL CIOLACU, prezentand informatii menite sa ii afecteze imaginea si implicit sansele electorale. Materialul a fost promovat dupa ora 18:00 pe 23.11.2024, incalcand astfel prevederile legale privind incheierea campaniei electorale. Caracterul de propaganda este demonstrat de natura platita a postarii, audienta tintita (100,001-500,000 persoane) si continutul explicit negativ la adresa candidatului.
\end{enumerate}

\vspace{0.5cm}

\subsection{Agentia de publicitate Confort Media (Aldo Detail Direct SRL) pentru Partidul National Liberal - Filiala Hunedoara}
Următoarele fapte contravenționale sunt sesizate împotriva acestei entități:

\begin{enumerate}[leftmargin=*, label=\arabic*.)]
    \item publicarea si promovarea unui mesaj de propaganda electorala (ID postare Facebook: \href{https://www.facebook.com/ads/library/?id=861447816196566}{861447816196566}) dupa incheierea perioadei de campanie electorala. Postarea, activa dupa ora 18:00 pe 23.11.2024, promoveaza explicit candidatul Nicolae Ionel Ciuca si indeamna la "vot util", contine numar CMF (11240002), si a avut un impact semnificativ avand intre 45.000-49.999 de afisari. Mesajul reprezinta propaganda electorala conform Art. 36(7) prin referirea directa la candidat, obiectivul electoral explicit si adresarea catre publicul larg prin platforme de social media.
\end{enumerate}

\vspace{0.5cm}

\subsection{AgentiadeInformatii.ro}
Următoarele fapte contravenționale sunt sesizate împotriva acestei entități:

\begin{enumerate}[leftmargin=*, label=\arabic*.)]
    \item difuzarea unui mesaj platit pe Facebook si Instagram (ID postare: \href{https://www.facebook.com/ads/library/?id=1208628637065056}{1208628637065056}) ce constituie propaganda electorala negativa impotriva candidatului Mircea-Dan Geoana, dupa ora 18:00 pe 23.11.2024. Materialul foloseste o intrebare retorica cu conotatii negative ("Romania are nevoie de un presedinte mason?") pentru a influenta decizia de vot a alegatorilor, atingand un public de peste 200.000 de persoane, cu o investitie de aproximativ 950 RON. Mesajul nu are caracter informativ sau jurnalistic, ci urmareste in mod evident sa influenteze optiunea de vot a alegatorilor intr-o perioada in care propaganda electorala este interzisa prin lege.
    \item difuzarea de propaganda electorala dupa incheierea campaniei electorale, constand in publicarea si promovarea unui anunt platit pe Facebook si Instagram (ID postare: \href{https://www.facebook.com/ads/library/?id=1592309648050326}{1592309648050326}) care vizeaza in mod negativ candidatul Mircea-Dan Geoana, folosind intrebarea retorica "Romania are nevoie de un presedinte mason?" pentru a influenta decizia alegatorilor. Postarea a fost difuzata si continua sa fie activa dupa ora 18:00 pe 23.11.2024, incalcand astfel perioada de restrictie electorala. Impactul este demonstrat prin reach-ul substantial (200,000-249,999 impresii) si bugetul semnificativ alocat (900-999 RON).
    \item difuzarea unei reclame platite pe Facebook (ID: \href{https://www.facebook.com/ads/library/?id=3865348743725271}{3865348743725271}) dupa ora 18:00 pe 23.11.2024, care constituie propaganda electorala pentru candidatul George Simion la alegerile prezidentiale. Postarea promoveaza explicit candidatul AUR ca "viitorul presedinte" si prezinta sustinerea unui alt candidat (Calin Georgescu), avand un impact electoral direct. Materialul a fost distribuit ca reclama platita pe Facebook si Instagram, atingand intre 250.000 si 300.000 de afisari, depasind astfel perioada legala de propaganda electorala pentru alegerile prezidentiale.
\end{enumerate}

\vspace{0.5cm}

\subsection{Aladin Georgescu si Partidul Social Democrat}
Următoarele fapte contravenționale sunt sesizate împotriva acestei entități:

\begin{enumerate}[leftmargin=*, label=\arabic*.)]
    \item continuarea propagandei electorale dupa incheierea acesteia, intr-o postare platita pe Facebook (ID: \href{https://www.facebook.com/ads/library/?id=2229317524117504}{2229317524117504}) care indeamna explicit la votarea candidatului ION-MARCEL CIOLACU la alegerile prezidentiale, cu mesajul "La fel de important este votul din data de 24 noiembrie pentru viitorul presedinte al Romaniei, Marcel Ciolacu!". Postarea este activa si promovata dupa ora 18:00 pe 23.11.2024, incalcand perioada legala de campanie electorala. Postarea include numar CMF (11240017), confirmand natura sa de propaganda electorala, si este distribuita catre un public tinta estimat intre 100,001 si 500,000 de persoane.
\end{enumerate}

\vspace{0.5cm}

\subsection{Alianta pentru Unirea Romanilor Valcea}
Următoarele fapte contravenționale sunt sesizate împotriva acestei entități:

\begin{enumerate}[leftmargin=*, label=\arabic*.)]
    \item continuarea propagandei electorale dupa incheierea campaniei electorale, manifestata prin postarea cu ID \href{https://www.facebook.com/ads/library/?id=3847753942163982}{3847753942163982} pe Facebook. Postarea promoveaza explicit candidatul George Simion si programul sau electoral, contine indemnuri directe la vot ("vino la vot"), foloseste hashtag-uri electorale (\#VoteazaAUR) si face promisiuni electorale despre infrastructura, toate acestea dupa ora 18:00 pe 23.11.2024. Postarea este sponsorizata si are un impact semnificativ, atingand intre 10.000 si 14.999 de persoane.
    \item continuarea propagandei electorale dupa incheierea campaniei, materializata prin postarea cu ID-ul \href{https://www.facebook.com/ads/library/?id=914482113515376}{914482113515376} pe Facebook, care promoveaza direct candidatul George Simion si solicita explicit votul cetatenilor ("Vino la vot"). Postarea este platita si distribuita activ dupa ora 18:00 pe 23.11.2024, contine elemente clare de propaganda electorala (CUI MFC 11240014), si vizeaza influentarea directa a votului pentru alegerile prezidentiale.
\end{enumerate}

\vspace{0.5cm}

\subsection{Alianta pentru Unirea Romanilor si SC Total Marketing SRL}
Următoarele fapte contravenționale sunt sesizate împotriva acestei entități:

\begin{enumerate}[leftmargin=*, label=\arabic*.)]
    \item difuzarea de materiale de propaganda electorala dupa incheierea campaniei electorale pentru alegerile prezidentiale, respectiv postarea cu ID-ul \href{https://www.facebook.com/ads/library/?id=562641266468758}{562641266468758} pe platforma Facebook. Materialul promovat contine referinte directe la candidatul prezidential George Simion, fiind difuzat dupa ora 18:00 pe 23.11.2024, intr-un mod care incalca perioada de black-out electoral. Postarea are caracter electoral evident, fiind comandata de partid (CMF 11240014), si a ajuns la un public de 25.000-30.000 persoane prin promovare platita.
\end{enumerate}

\vspace{0.5cm}

\subsection{Alin Calinescu}
Următoarele fapte contravenționale sunt sesizate împotriva acestei entități:

\begin{enumerate}[leftmargin=*, label=\arabic*.)]
    \item publicarea si promovarea platita a unui mesaj de propaganda electorala (ID postare Facebook: \href{https://www.facebook.com/ads/library/?id=1261220475211251}{1261220475211251}) dupa incheierea perioadei de campanie, in data de 23.11.2024. Postarea contine un indemn explicit de vot pentru candidatul Nicolae Ciuca, atacuri la adresa contracandidatului Marcel Ciolacu, si poarta numarul de inregistrare CMF 11240002, demonstrand natura sa de propaganda electorala. Postarea a fost promovata ca reclama platita pe Instagram, vizand un public tinta de pana la 500.000 de persoane, incalcand astfel perioada de liniste electorala dupa ora 18:00 pe 23.11.2024.
    \item continuarea propagandei electorale dupa incheierea perioadei legale de campanie, prin postarea cu ID \href{https://www.facebook.com/ads/library/?id=3743931592523454}{3743931592523454} pe facebook. Postarea, efectuata dupa ora 18:00 pe 23.11.2024, contine mesaje explicite de propaganda electorala, incluzand indemnuri directe de vot pentru candidatul Nicolae Ciuca si atacuri la adresa candidatului Marcel Ciolacu. Postarea este sponsorizata si contine cod CMF 11240002, confirmand natura sa de material de propaganda electorala.
\end{enumerate}

\vspace{0.5cm}

\subsection{Alina Gorghiu (PNL)}
Următoarele fapte contravenționale sunt sesizate împotriva acestei entități:

\begin{enumerate}[leftmargin=*, label=\arabic*.)]
    \item publicarea si promovarea unui mesaj de propaganda electorala (ID Facebook: \href{https://www.facebook.com/ads/library/?id=1285438759162935}{1285438759162935}) dupa incheierea perioadei de campanie electorala pentru alegerile prezidentiale. Postarea, realizata si promovata dupa ora 18:00 pe 23.11.2024, contine referiri directe la candidatii prezidentiali Calin Georgescu si George-Nicolae Simion, include numar CMF (31240003), si urmareste influentarea procesului electoral prin acuzatii privind strategii electorale si manipularea competitiei prezidentiale. Mesajul a fost distribuit catre un public larg prin platformele Facebook si Instagram, cu un buget de promovare intre 100-199 RON si o audienta estimata intre 9.000-9.999 persoane.
\end{enumerate}

\vspace{0.5cm}

\subsection{Anchetatorii.RO}
Următoarele fapte contravenționale sunt sesizate împotriva acestei entități:

\begin{enumerate}[leftmargin=*, label=\arabic*.)]
    \item difuzarea de materiale de propaganda electorala dupa incheierea campaniei electorale pentru alegerile prezidentiale. Postarea promovata (ID: \href{https://www.facebook.com/ads/library/?id=493537339697466}{493537339697466}) contine atacuri directe la adresa candidatei Elena Lasconi, folosind un buget semnificativ de promovare (200-299 RON) pentru a ajunge la un public de peste 1 milion de persoane, cu scopul clar de a influenta negativ opinia publica despre aceasta candidata. Materialul este difuzat dupa ora 18:00 pe 23.11.2024, incalcand explicit perioada de liniste electorala.
    \item difuzarea de propaganda electorala dupa incheierea campaniei electorale, constand intr-o postare sponsorizata pe Facebook si Instagram (ID: \href{https://www.facebook.com/ads/library/?id=544782674990174}{544782674990174}) care vizeaza in mod direct si negativ candidatul Elena Lasconi. Postarea, difuzata dupa ora 18:00 pe 23.11.2024, foloseste simboluri de avertizare si asocieri negative pentru a influenta opinia alegatorilor, atingand un public de peste 50.000 de persoane, cu o investitie intre 300-399 RON. Caracterul de propaganda electorala este evident prin targetarea specifica a candidatului si incercarea de influentare negativa a votului.
\end{enumerate}

\vspace{0.5cm}

\subsection{Andrei Carabelea si Forta Dreptei}
Următoarele fapte contravenționale sunt sesizate împotriva acestei entități:

\begin{enumerate}[leftmargin=*, label=\arabic*.)]
    \item difuzarea unei reclame platite pe Facebook si Instagram (ID postare: \href{https://www.facebook.com/ads/library/?id=1752738545486438}{1752738545486438}) dupa ora 18:00 pe 23.11.2024, cu continut de propaganda electorala pentru candidatul prezidential Ludovic Orban. Postarea contine numar de inregistrare CMF (11240022), promoveaza explicit realizarile si propunerile candidatului si ale partidului sau, in timp ce critica direct contracandidatii, avand un evident scop electoral de influentare a votului. Reclama a fost targetata catre un public larg (100,001-500,000 persoane) si a generat intre 8,000 si 8,999 de afisari.
\end{enumerate}

\vspace{0.5cm}

\subsection{Andreka Adrian-Dan}
Următoarele fapte contravenționale sunt sesizate împotriva acestei entități:

\begin{enumerate}[leftmargin=*, label=\arabic*.)]
    \item promovarea unui mesaj de propaganda electorala pentru candidatul ION-MARCEL CIOLACU dupa incheierea campaniei electorale. Postarea cu ID-ul \href{https://www.facebook.com/ads/library/?id=1950318142131315}{1950318142131315} pe Facebook constituie propaganda electorala activa dupa data de 23.11.2024, ora 18:00, prin faptul ca indeamna explicit la vot ("Duminica alegem"), promoveaza direct un candidat la presedintie si foloseste hashtag-uri si mesaje specifice campaniei electorale. Postarea este promovata ca reclama platita, avand un impact semnificativ, cu o audienta estimata intre 100,001-500,000 de persoane si impresii intre 6,000-6,999.
\end{enumerate}

\vspace{0.5cm}

\subsection{Antonel Tanase}
Următoarele fapte contravenționale sunt sesizate împotriva acestei entități:

\begin{enumerate}[leftmargin=*, label=\arabic*.)]
    \item promovarea unui mesaj de propaganda electorala pentru candidatul Ludovic Orban dupa incheierea perioadei de campanie electorala. Postarea cu ID-ul \href{https://www.facebook.com/ads/library/?id=568335808926391}{568335808926391} pe facebook contine promisiuni electorale explicite ("Ne propunem ca obiectiv ca in decurs de 5 ani venitul mediu al cetatenilor romani sa atinga venitul mediu la nivel european"), promovare directa a candidatului si a platformei sale electorale, fiind difuzata ca reclama platita cu un impact semnificativ (450,000-500,000 impresii) dupa ora 18:00 pe 23.11.2024, incalcand astfel prevederile legale privind incheierea campaniei electorale.
\end{enumerate}

\vspace{0.5cm}

\subsection{Arina Mos}
Următoarele fapte contravenționale sunt sesizate împotriva acestei entități:

\begin{enumerate}[leftmargin=*, label=\arabic*.)]
    \item continuarea propagandei electorale dupa incheierea campaniei electorale, promovand candidatul Nicolae-Ionel Ciuca la functia de presedinte prin intermediul unei postari platite pe Facebook (ID: \href{https://www.facebook.com/ads/library/?id=3410615485741856}{3410615485741856}). Postarea contine numar CMF (11240002), indeamna explicit la vot pentru candidat ("voi vota Nicolae Ionel Ciuca presedinte"), si a fost distribuita dupa ora 18:00 pe 23.11.2024, atingand un public intre 500,001 si 1,000,000 de persoane, constituind astfel propaganda electorala in afara perioadei legale de campanie.
\end{enumerate}

\vspace{0.5cm}

\subsection{Atelierul de Internet SRL}
Următoarele fapte contravenționale sunt sesizate împotriva acestei entități:

\begin{enumerate}[leftmargin=*, label=\arabic*.)]
    \item promovarea unui mesaj de propaganda electorala (ID Facebook: \href{https://www.facebook.com/ads/library/?id=928881002007822}{928881002007822}) dupa ora 18:00 pe 23.11.2024, in perioada campaniei electorale pentru alegerile prezidentiale. Mesajul contine referiri directe la candidatul prezidential Marcel Ciolacu, cu impact negativ asupra acestuia, si include elemente clare de propaganda electorala, fiind marcat cu cod CMF 11240046. Postarea a fost promovata activ pe Facebook, cu un buget intre 300-399 RON, atingand intre 30.000-35.000 de utilizatori, demonstrand astfel intentia clara de a influenta procesul electoral prezidential dupa incheierea perioadei legale de campanie.
    \item difuzarea unui mesaj electoral platit (ID Facebook: \href{https://www.facebook.com/ads/library/?id=930212335660605}{930212335660605}) dupa incheierea campaniei electorale prezidentiale. Postarea contine propaganda electorala negativa la adresa candidatului prezidential Marcel Ciolacu, fiind difuzata dupa ora 18:00 pe 23.11.2024. Mesajul are caracter electoral explicit, fiind insotit de cod CMF (11240046), contine indemnuri directe la vot si este distribuit ca reclama platita pe Facebook, cu un impact estimat intre 15.000 si 19.999 de afisari.
\end{enumerate}

\vspace{0.5cm}

\subsection{Atelierul de internet SRL}
Următoarele fapte contravenționale sunt sesizate împotriva acestei entități:

\begin{enumerate}[leftmargin=*, label=\arabic*.)]
    \item difuzarea de materiale de propaganda electorala dupa incheierea campaniei electorale pentru alegerile prezidentiale. Postarea cu ID-ul \href{https://www.facebook.com/ads/library/?id=1354983342574061}{1354983342574061} de pe Facebook contine sustinere explicita pentru candidatul Mircea Geoana, indemnuri directe la vot si numar CMF (112400400), fiind difuzata dupa ora 18:00 pe 23.11.2024. Materialul a fost promovat ca reclama platita pe Facebook si Instagram, atingand intre 45.000 si 50.000 de afisari, demonstrand intentia clara de influentare a votului in perioada in care propaganda electorala este interzisa prin lege.
    \item promovarea unui material de propaganda electorala (ID postare Facebook: \href{https://www.facebook.com/ads/library/?id=541831591809413}{541831591809413}) dupa incheierea perioadei de campanie electorala pentru alegerile prezidentiale. Materialul contine promovare directa a candidatului Mircea Geoana, indemn explicit la vot, numar CMF (112400400), si a fost distribuit ca reclama platita pe Facebook dupa ora 18:00 pe 23.11.2024, incalcand astfel prevederile legale privind incheierea campaniei electorale. Impactul a fost semnificativ, atingand intre 80.000 si 89.999 de afisari.
    \item difuzarea unui mesaj electoral platit (ID Facebook: \href{https://www.facebook.com/ads/library/?id=548982818002083}{548982818002083}) care ataca direct candidatul prezidential George Simion dupa ora 18:00 pe 23.11.2024. Materialul, care include numar CMF 11240015, constituie propaganda electorala conform Art. 36(7) prin: referirea directa la candidat, obiectivul electoral clar de influentare negativa a votului, si adresabilitatea catre public larg prin promovare platita pe Facebook si Instagram (15,000-19,999 impresii). Mesajul continua sa fie activ si promovat in perioada de interdictie a propagandei electorale pentru alegerile prezidentiale.
\end{enumerate}

\vspace{0.5cm}

\subsection{Bogdan Rodeanu si USR}
Următoarele fapte contravenționale sunt sesizate împotriva acestei entități:

\begin{enumerate}[leftmargin=*, label=\arabic*.)]
    \item continuarea propagandei electorale dupa incheierea acesteia, manifestata prin postarea cu ID \href{https://www.facebook.com/ads/library/?id=1719350121971281}{1719350121971281} pe Facebook. Postarea contine indemnuri directe de vot ("VOTAM USR  pozitia 1"), critici la adresa contracandidatilor, si este distribuita ca reclama platita dupa ora 18:00 pe 23.11.2024. Materialul include numar CMF (11240015), confirmand natura sa de propaganda electorala, si vizeaza in mod direct alegerile prezidentiale pentru care perioada de campanie s-a incheiat.
\end{enumerate}

\vspace{0.5cm}

\subsection{Ciprian Paraschiv}
Următoarele fapte contravenționale sunt sesizate împotriva acestei entități:

\begin{enumerate}[leftmargin=*, label=\arabic*.)]
    \item difuzarea de materiale de propaganda electorala pentru candidatul GEORGE-NICOLAE SIMION dupa incheierea perioadei de campanie electorala. Materialul promotional, identificat prin ID-ul facebook \href{https://www.facebook.com/ads/library/?id=429882539993488}{429882539993488}, contine indemnuri directe de vot ("Votez George Simion!") si promovarea explicita a candidatului ca "Presedintele tuturor romanilor", fiind distribuit ca reclama platita pe Facebook si Instagram, cu un impact estimat de peste 1 milion de persoane, dupa ora 18:00 pe 23.11.2024, incalcand astfel prevederile legale privind incetarea campaniei electorale.
    \item continuarea propagandei electorale dupa incheierea campaniei electorale, promovand in mod explicit candidatul George Simion si atacand contracandidatii (Ciolacu, Ciuca, Lasconi) intr-o postare sponsorizata pe Facebook/Instagram (ID: \href{https://www.facebook.com/ads/library/?id=495679523503238}{495679523503238}) dupa ora 18:00 pe 23.11.2024. Postarea contine indemnuri directe de vot ("votezi George Simion presedinte"), numar CMF (11240014), si targeteaza un public larg estimat la peste 1 milion de persoane, constituind astfel propaganda electorala conform Art. 36(7) din Legea 334/2006.
\end{enumerate}

\vspace{0.5cm}

\subsection{Comentatorii.RO}
Următoarele fapte contravenționale sunt sesizate împotriva acestei entități:

\begin{enumerate}[leftmargin=*, label=\arabic*.)]
    \item difuzarea unui material de propaganda electorala (ID Facebook: \href{https://www.facebook.com/ads/library/?id=8744995008916796}{8744995008916796}) dupa incheierea perioadei de campanie electorala. Materialul promovat, cu o audienta de peste 300.000 de persoane si un buget de aproximativ 650 RON, prezinta in mod explicit sustinerea unui candidat (George Simion) si discreditarea altui candidat (Elena Lasconi), reprezentand propaganda electorala activa dupa ora 18:00 pe 23.11.2024. Impactul este amplificat prin distributia pe multiple platforme (Facebook si Instagram) si targetarea nationala, constituind o incalcare clara a prevederilor legale privind perioada de campanie electorala.
\end{enumerate}

\vspace{0.5cm}

\subsection{Computer Fun SRL}
Următoarele fapte contravenționale sunt sesizate împotriva acestei entități:

\begin{enumerate}[leftmargin=*, label=\arabic*.)]
    \item difuzarea de continut propagandistic electoral dupa incheierea perioadei de campanie electorala, prin intermediul unei reclame platite pe Facebook cu ID-ul \href{https://www.facebook.com/ads/library/?id=3671195386460590}{3671195386460590}. Reclama a fost publicata de PSD Dambovita si promoveaza mesajul "CALEA SIGURA PENTRU \#DAMBOVITA" cu CMF 11240017, avand un impact pozitiv pentru candidatul PSD la prezidentiale. Materialul a continuat sa fie difuzat dupa ora 18:00 pe 23.11.2024, incalcand astfel prevederile legale privind incetarea propagandei electorale.
    \item difuzarea de materiale de propaganda electorala dupa incheierea campaniei electorale pentru alegerile prezidentiale. Materialul promovat pe Facebook si Instagram (ID postare: \href{https://www.facebook.com/ads/library/?id=932359448775477}{932359448775477}) contine elementele specifice materialelor de propaganda electorala (CMF: 11240017), promoveaza un candidat la presedintie si continua sa fie difuzat dupa ora 18:00 pe 23.11.2024, incalcand astfel prevederile legale privind incheierea campaniei electorale. Impactul estimat al materialului este intre 1000-1999 de afisari, demonstrand intentia clara de a influenta procesul electoral in perioada restrictionata.
\end{enumerate}

\vspace{0.5cm}

\subsection{Computer Fun SRL si PSD Dambovita}
Următoarele fapte contravenționale sunt sesizate împotriva acestei entități:

\begin{enumerate}[leftmargin=*, label=\arabic*.)]
    \item continuarea propagandei electorale dupa incheierea acesteia, folosind un anunt sponsorizat pe Facebook si Instagram (ID: \href{https://www.facebook.com/ads/library/?id=1752981645241567}{1752981645241567}) cu mesajul "CALEA SIGURA PENTRU \#GURAFOII" si numar CMF 11240017. Anuntul a fost activ si dupa ora 18:00 pe 23.11.2024, incalcand astfel perioada legala de propaganda electorala pentru alegerile prezidentiale. Materialul prezinta caracteristici clare de propaganda electorala conform Art. 36(7), fiind sponsorizat, avand numar CMF si promovand mesaje cu caracter electoral.
\end{enumerate}

\vspace{0.5cm}

\subsection{Comunitatea Liberala}
Următoarele fapte contravenționale sunt sesizate împotriva acestei entități:

\begin{enumerate}[leftmargin=*, label=\arabic*.)]
    \item publicarea si promovarea unei reclame platite pe Facebook (ID: \href{https://www.facebook.com/ads/library/?id=1598757470719860}{1598757470719860}) dupa ora 18:00 pe 23.11.2024, in care se face in mod explicit propaganda electorala negativa impotriva candidatului NICOLAE-IONEL CIUCA, indemnand direct cetatenii "sa nu-l voteze". Postarea este promovata ca reclama platita, cu un buget intre 100-199 RON si un reach estimat de peste 1 milion de persoane, reprezentand astfel o actiune clara de propaganda electorala in afara perioadei legale permise.
    \item promovarea unui material de propaganda electorala pentru candidata Elena-Valerica Lasconi dupa incheierea perioadei de campanie electorala. Materialul, identificat prin ID-ul postarii Facebook \href{https://www.facebook.com/ads/library/?id=506403022549368}{506403022549368}, reprezinta o incercare directa de a influenta votul prin promovarea platita a unui indemn explicit de vot ("De ce votez cu Lasconi") intr-o perioada interzisa de lege, respectiv dupa ora 18:00 pe 23.11.2024. Postarea are caracter electoral evident, vizeaza un public larg (peste 1 milion de persoane estimate) si depaseste cadrul unei simple opinii personale prin natura sa publicitara si propagandistica.
\end{enumerate}

\vspace{0.5cm}

\subsection{Corneliu Muresan}
Următoarele fapte contravenționale sunt sesizate împotriva acestei entități:

\begin{enumerate}[leftmargin=*, label=\arabic*.)]
    \item publicarea si promovarea unei reclame platite pe Facebook (ID: \href{https://www.facebook.com/ads/library/?id=443944335141611}{443944335141611}) dupa ora 18:00 pe 23.11.2024, continuand propaganda electorala dupa incheierea campaniei. Postarea promoveaza explicit candidatul prezidential ION-MARCEL CIOLACU si realizarile sale ca premier, avand scop electoral evident prin laudarea capacitatilor sale de lider si negociator, utilizand numarul CMF 11240017, specific materialelor de campanie electorala. Materialul a fost distribuit pe Facebook si Instagram catre un public tinta de 100,001-500,000 persoane, reprezentand o incalcare clara a prevederilor legale privind incheierea campaniei electorale.
\end{enumerate}

\vspace{0.5cm}

\subsection{Dan Cosma si AUR}
Următoarele fapte contravenționale sunt sesizate împotriva acestei entități:

\begin{enumerate}[leftmargin=*, label=\arabic*.)]
    \item promovarea unui material electoral platit (ID Facebook: \href{https://www.facebook.com/ads/library/?id=1262654651439486}{1262654651439486}) ce contine indemnuri directe de vot pentru candidatul la presedintie George Simion ("Voteaza George Simion - pozitia 2 la prezidentiale"), dupa incheierea perioadei de campanie electorala prezidentiala. Materialul este difuzat dupa ora 18:00 pe 23.11.2024, include numar CMF (11240014), si reprezinta propaganda electorala conform Art. 36(7) din Legea 334/2006, avand obiectiv electoral explicit si adresandu-se publicului larg prin platformele Facebook si Instagram.
\end{enumerate}

\vspace{0.5cm}

\subsection{Dan Cosma si Alianta pentru Unirea Romanilor (AUR)}
Următoarele fapte contravenționale sunt sesizate împotriva acestei entități:

\begin{enumerate}[leftmargin=*, label=\arabic*.)]
    \item publicarea si promovarea unui anunt sponsorizat pe Facebook (ID: \href{https://www.facebook.com/ads/library/?id=1490494721608433}{1490494721608433}) care continua propaganda electorala pentru candidatul prezidential George Simion dupa incheierea perioadei de campanie. Postarea contine indemnuri directe de vot ("Voteaza George Simion - pozitia 2 la prezidentiale"), fiind difuzata dupa ora 18:00 pe 23.11.2024, intr-un moment in care propaganda electorala pentru alegerile prezidentiale este interzisa prin lege. Materialul este identificat ca propaganda electorala prin prezenta numarului CMF 11240014 si natura sa de continut sponsorizat pe platformele Meta.
\end{enumerate}

\vspace{0.5cm}

\subsection{Dancus Ioan Doru}
Următoarele fapte contravenționale sunt sesizate împotriva acestei entități:

\begin{enumerate}[leftmargin=*, label=\arabic*.)]
    \item promovarea unui mesaj de propaganda electorala platit (900-999 RON) pe platformele Facebook si Instagram (ID postare: \href{https://www.facebook.com/ads/library/?id=1107137107522452}{1107137107522452}), cu impact estimat intre 30.000-35.000 de afisari, in care promoveaza explicit candidatul ION-MARCEL CIOLACU la functia de presedinte, dupa ora 18:00 pe 23.11.2024. Mesajul contine elemente clare de propaganda electorala, inclusiv laude ale capacitatilor diplomatice ale candidatului, promisiuni electorale si comparatii cu alti candidati, cu scopul clar de a influenta alegatorii.
    \item promovarea unui mesaj electoral platit (ID post: \href{https://www.facebook.com/ads/library/?id=8333543073439536}{8333543073439536}) pe platformele Facebook si Instagram, dupa incheierea perioadei de campanie electorala. Postarea promoveaza explicit candidatul ION-MARCEL CIOLACU, folosind sloganul de campanie "\#CaleaSigura" si indeamna direct la vot ("Pe 24 noiembrie, votam \#CaleaSigura pentru Romania!"). Mesajul a fost promovat ca reclama platita, atingand intre 20.000 si 25.000 de afisari, dupa ora 18:00 pe 23.11.2024, incalcand astfel restrictiile privind propaganda electorala dupa incheierea campaniei.
\end{enumerate}

\vspace{0.5cm}

\subsection{DigiPRES.RO}
Următoarele fapte contravenționale sunt sesizate împotriva acestei entități:

\begin{enumerate}[leftmargin=*, label=\arabic*.)]
    \item difuzarea unei reclame platite pe Facebook si Instagram (ID: \href{https://www.facebook.com/ads/library/?id=1565190404109410}{1565190404109410}) ce promoveaza mesaje electorale dupa incheierea campaniei. Materialul prezinta explicit pozitionari electorale privind candidatii la presedintie, cu impact negativ asupra candidatei Elena Lasconi si pozitiv pentru George Simion, atingand un public de peste 300.000 de persoane, dupa ora 18:00 pe 23.11.2024. Continutul depaseste sfera jurnalistica prin natura sa propagandistica si distributia platita la scara larga.
\end{enumerate}

\vspace{0.5cm}

\subsection{DigiPres.ro}
Următoarele fapte contravenționale sunt sesizate împotriva acestei entități:

\begin{enumerate}[leftmargin=*, label=\arabic*.)]
    \item promovarea unei reclame platite pe Facebook (ID: \href{https://www.facebook.com/ads/library/?id=844194280975041}{844194280975041}) dupa ora 18:00 pe 23.11.2024, continand propaganda electorala negativa la adresa candidatului ION-MARCEL CIOLACU. Materialul foloseste limbaj denigrator ("sluga", "pozitie submisiva"), are scop electoral evident prin incercarea de a influenta perceptia alegatorilor asupra candidatului, si este distribuit ca reclama platita catre un public larg (peste 1 milion potentiali destinatari). Aceasta activitate constituie continuarea propagandei electorale dupa incheierea perioadei legale de campanie.
\end{enumerate}

\vspace{0.5cm}

\subsection{Digidev Innotech}
Următoarele fapte contravenționale sunt sesizate împotriva acestei entități:

\begin{enumerate}[leftmargin=*, label=\arabic*.)]
    \item postarea de propaganda electorala pentru alegerile prezidentiale (ID postare Facebook: \href{https://www.facebook.com/ads/library/?id=1155401792993940}{1155401792993940}) dupa incheierea perioadei de campanie. Postarea contine indemnuri directe de vot pentru candidata Elena Lasconi si impotriva candidatilor Marcel Ciolacu si George Simion, fiind distribuita ca reclama platita pe Facebook dupa ora 18:00 pe 23.11.2024, cu un impact estimat intre 10.000 si 14.999 de persoane. Postarea include numar CMF (11240022), confirmand natura sa de propaganda electorala, si face apeluri explicite la influentarea votului pentru alegerile prezidentiale din 24 noiembrie 2024.
    \item continuarea propagandei electorale dupa incheierea campaniei electorale pentru alegerile prezidentiale, prin publicarea si mentinerea activa a unei reclame platite pe Facebook (ID: \href{https://www.facebook.com/ads/library/?id=2631054730430413}{2631054730430413}) dupa ora 18:00 pe 23.11.2024. Reclama contine indemnuri explicite de vot pentru candidata Elena Lasconi la functia de presedinte ("Votati Elena Lasconi pentru presedinte"), precum si referinte negative la adresa altor candidati prezidentiali (Ciolacu si Simion). Materialul prezinta toate elementele constitutive ale propagandei electorale conform Art. 36(7): identifica clar candidatii, are obiectiv electoral explicit, se adreseaza publicului larg prin intermediul unei reclame platite, si depaseste limitele activitatii jurnalistice.
    \item difuzarea unei reclame platite pe Facebook (ID: \href{https://www.facebook.com/ads/library/?id=491308380618000}{491308380618000}) dupa ora 18:00 pe 23.11.2024, care continua propaganda electorala pentru alegerile prezidentiale, facand referire directa la candidatii Marcel Ciolacu si George Simion intr-un mod negativ ("sa spargem blatul dintre Ciolacu si Simion"). Materialul are caracter electoral evident, fiind insotit de CMF 11240022, si a avut un impact semnificativ, atingand intre 25.000 si 29.999 de afisari. Mesajul indeamna in mod explicit la vot si incearca sa influenteze comportamentul electoral al alegatorilor in contextul alegerilor prezidentiale, dupa incheierea perioadei legale de campanie.
    \item difuzarea de materiale de propaganda electorala dupa incheierea campaniei electorale pentru alegerile prezidentiale, prin intermediul unei reclame platite pe Facebook (ID: \href{https://www.facebook.com/ads/library/?id=568591152321296}{568591152321296}) care promoveaza candidatul prezidential Ludovic Orban si partidul Forta Dreptei, prezentand platforma politica si incercand sa influenteze comportamentul electoral. Reclama a fost difuzata dupa ora 18:00 pe 23.11.2024, incalcand astfel perioada legala de campanie electorala. Impactul este demonstrat prin audienta estimata intre 100,001 si 500,000 de persoane si numarul de afisari intre 25,000 si 29,999.
\end{enumerate}

\vspace{0.5cm}

\subsection{Domnul Marian Ciofica}
Următoarele fapte contravenționale sunt sesizate împotriva acestei entități:

\begin{enumerate}[leftmargin=*, label=\arabic*.)]
    \item publicarea si promovarea unei reclame platite pe Facebook (ID: \href{https://www.facebook.com/ads/library/?id=8823369257728083}{8823369257728083}) dupa incheierea perioadei de campanie electorala pentru alegerile prezidentiale. Postarea contine indemnuri directe de vot pentru candidatul Calin Georgescu, foloseste un numar CMF (11240022), si include mesaje explicite de mobilizare electorala ("Alegeti Calin Georgescu!", "Iesiti la vot"). Materialul a fost promovat dupa ora 18:00 pe 23.11.2024, incalcand astfel prevederile legale privind incheierea campaniei electorale pentru alegerile prezidentiale.
\end{enumerate}

\vspace{0.5cm}

\subsection{Domnul Vladut Purja}
Următoarele fapte contravenționale sunt sesizate împotriva acestei entități:

\begin{enumerate}[leftmargin=*, label=\arabic*.)]
    \item promovarea unui mesaj de propaganda electorala dupa incheierea perioadei legale de campanie, constand intr-o postare platita pe Facebook (ID: \href{https://www.facebook.com/ads/library/?id=1266112257969679}{1266112257969679}) care indeamna explicit cetatenii sa voteze cu candidatul ION-MARCEL CIOLACU in data de 24.11.2024. Postarea contine un indemn direct la vot ("Haideti sa fim alaturi de el, duminica, 24.11.2024"), fiind difuzata dupa ora 18:00 pe 23.11.2024, incalcand astfel prevederile legale privind incheierea campaniei electorale.
\end{enumerate}

\vspace{0.5cm}

\subsection{Dumitru Rujan}
Următoarele fapte contravenționale sunt sesizate împotriva acestei entități:

\begin{enumerate}[leftmargin=*, label=\arabic*.)]
    \item publicarea si promovarea unui mesaj de propaganda electorala (ID postare Facebook: \href{https://www.facebook.com/ads/library/?id=4411198482440103}{4411198482440103}) dupa incheierea perioadei de campanie electorala pentru alegerile prezidentiale. Postarea contine indemnuri explicite la vot pentru candidatul Nicolae Ciuca ("eu voi vota Nicolae Ciuca!"), critici directe la adresa contracandidatului Marcel Ciolacu, si promovare activa de mesaje electorale, fiind difuzata dupa ora 18:00 pe 23.11.2024, incalcand astfel prevederile legale privind incetarea campaniei electorale.
\end{enumerate}

\vspace{0.5cm}

\subsection{Eugen Tomac}
Următoarele fapte contravenționale sunt sesizate împotriva acestei entități:

\begin{enumerate}[leftmargin=*, label=\arabic*.)]
    \item publicarea si promovarea unui mesaj de propaganda electorala (ID Facebook: \href{https://www.facebook.com/ads/library/?id=942804337906337}{942804337906337}) dupa ora 18:00 pe 23.11.2024, in perioada de restrictie electorala. Postarea contine indemnuri explicite de vot pentru candidata Elena Lasconi si impotriva candidatilor Marcel Ciolacu, Nicolae Ciuca si George Simion, fiind promovata ca reclama platita pe Facebook si Instagram, cu un impact intre 40.000 si 45.000 de afisari. Mesajul include numar CMF (11240022), confirmand natura sa de propaganda electorala, si contine indemnuri directe la vot ("Iesiti la vot!"), incalcand astfel prevederile legale privind perioada de silentiu electoral.
\end{enumerate}

\vspace{0.5cm}

\subsection{FORTA DREPTEI ILFOV}
Următoarele fapte contravenționale sunt sesizate împotriva acestei entități:

\begin{enumerate}[leftmargin=*, label=\arabic*.)]
    \item continuarea propagandei electorale pentru candidatul prezidential Elena Lasconi dupa incheierea perioadei de campanie, prin postarea cu ID \href{https://www.facebook.com/ads/library/?id=2593648940822239}{2593648940822239} pe platforma Facebook. Postarea contine indemnuri directe de vot ("Pe 24 noiembrie, votati Elena Lasconi Presedinte!") si a fost promovata dupa ora 18:00 pe 23.11.2024, incalcand astfel prevederile legale privind incheierea campaniei electorale. Postarea este una platita, cu un reach estimat intre 500,001 si 1,000,000 de persoane, reprezentand astfel o incercare clara de influentare a votului in perioada de interdictie.
\end{enumerate}

\vspace{0.5cm}

\subsection{FORTA DREPTEI MURES}
Următoarele fapte contravenționale sunt sesizate împotriva acestei entități:

\begin{enumerate}[leftmargin=*, label=\arabic*.)]
    \item continuarea propagandei electorale dupa incheierea perioadei legale de campanie, prin publicarea si promovarea platita a unui mesaj electoral (ID Facebook: \href{https://www.facebook.com/ads/library/?id=1684741228923253}{1684741228923253}) ce contine numar CMF (11240022), care vizeaza in mod direct alegerile prezidentiale, promovand mesaje negative despre candidatul Marcel Ciolacu si PSD, incercand sa influenteze comportamentul electoral al votantilor dupa ora 18:00 pe 23.11.2024. Postarea a avut un impact semnificativ, atingand intre 15.000 si 19.999 de persoane, fiind promovata cu un buget intre 300-399 RON.
    \item difuzarea unui material de propaganda electorala (ID Facebook: \href{https://www.facebook.com/ads/library/?id=932946168775192}{932946168775192}) dupa ora 18:00 pe 23.11.2024, referitor la alegerile prezidentiale. Materialul promoveaza explicit candidatul Elena Lasconi si denigreaza alti candidati (Mircea Geoana, George Simion), continand elemente clare de propaganda electorala, inclusiv numar CMF (11240022), fiind distribuit ca reclama platita pe Facebook cu un impact intre 15.000-19.999 afisari. Materialul incalca prevederile legale privind perioada de campanie electorala pentru alegerile prezidentiale.
\end{enumerate}

\vspace{0.5cm}

\subsection{Forta Dreptei}
Următoarele fapte contravenționale sunt sesizate împotriva acestei entități:

\begin{enumerate}[leftmargin=*, label=\arabic*.)]
    \item difuzarea unui mesaj de propaganda electorala dupa incheierea perioadei de campanie, constand intr-un anunt platit pe Facebook si Instagram (ID: \href{https://www.facebook.com/ads/library/?id=1232129874723325}{1232129874723325}) care promoveaza explicit candidatul ELENA-VALERICA LASCONI prin intermediul unei declaratii de sustinere din partea lui Traian Basescu. Anuntul a fost activ si dupa ora 18:00 pe 23.11.2024, incalcand astfel perioada legala de campanie electorala. Impactul postarii este semnificativ, atingand intre 70.000 si 79.999 de afisari, cu o investitie intre 400 si 499 RON.
    \item difuzarea unei reclame platite pe Facebook si Instagram (ID: \href{https://www.facebook.com/ads/library/?id=515069744866032}{515069744866032}) dupa ora 18:00 pe 23.11.2024, care promoveaza explicit transferul de voturi catre candidatul Elena Lasconi, constituind astfel propaganda electorala in afara perioadei permise. Reclama a avut un impact semnificativ, atingand intre 100.000 si 124.999 de afisari, cu o investitie intre 400-499 RON, reprezentand o incercare clara de influentare a intentiei de vot in perioada in care propaganda electorala este interzisa prin lege.
    \item difuzarea de materiale de propaganda electorala dupa incheierea campaniei electorale pentru alegerile prezidentiale. Postarea cu ID-ul \href{https://www.facebook.com/ads/library/?id=937805918197892}{937805918197892} contine un indemn explicit la vot ("Votati verde, votati Forta Dreptei!"), fiind promovata ca reclama platita pe Facebook dupa ora 18:00 pe 23.11.2024. Materialul foloseste metafore politice si critici la adresa guvernarii actuale, avand un evident caracter de propaganda electorala, cu un buget semnificativ de promovare (2000-2500 RON) si o audienta estimata de peste 200,000 de persoane.
\end{enumerate}

\vspace{0.5cm}

\subsection{Forta Dreptei - Calarasi, prin intermediul Buyer Brain SRL}
Următoarele fapte contravenționale sunt sesizate împotriva acestei entități:

\begin{enumerate}[leftmargin=*, label=\arabic*.)]
    \item difuzarea de materiale de propaganda electorala dupa incheierea campaniei electorale pentru alegerile prezidentiale. Postarea cu ID-ul \href{https://www.facebook.com/ads/library/?id=894985495946854}{894985495946854} promoveaza explicit candidatura Elenei Lasconi si denigreaza candidatura lui Marcel Ciolacu, avand un clear obiectiv electoral si adresandu-se publicului larg prin intermediul unei reclame platite pe Facebook si Instagram, dupa ora 18:00 pe 23.11.2024. Postarea a avut un impact semnificativ, atingand intre 2000 si 2999 de persoane.
\end{enumerate}

\vspace{0.5cm}

\subsection{Forta Dreptei, prin reprezentantul Eugen Tomac}
Următoarele fapte contravenționale sunt sesizate împotriva acestei entități:

\begin{enumerate}[leftmargin=*, label=\arabic*.)]
    \item continuarea propagandei electorale dupa incheierea campaniei electorale pentru alegerile prezidentiale. Postarea cu ID \href{https://www.facebook.com/ads/library/?id=531701209853921}{531701209853921} contine indemnuri directe de vot pentru candidata Elena Lasconi ("Pe 24 noiembrie, votam Elena Lasconi"), fiind o postare platita pe Facebook cu un impact intre 15.000-19.999 de afisari, difuzata dupa ora 18:00 pe 23.11.2024. Postarea include cod mandatar (11240022) si reprezinta propaganda electorala conform Art. 36(7), avand obiectiv electoral explicit si adresandu-se publicului larg.
\end{enumerate}

\vspace{0.5cm}

\subsection{Gabriel Valer Zetea si PSD}
Următoarele fapte contravenționale sunt sesizate împotriva acestei entități:

\begin{enumerate}[leftmargin=*, label=\arabic*.)]
    \item difuzarea unui mesaj de propaganda electorala dupa incheierea perioadei legale de campanie, in data de 23.11.2024, dupa ora 18:00. Postarea cu ID-ul \href{https://www.facebook.com/ads/library/?id=1478717420197889}{1478717420197889} promoveaza in mod explicit candidatul ION-MARCEL CIOLACU, folosind realizarea aderarii la Schengen pentru a sublinia calitatile sale de lider si capacitatea sa de a conduce tara, utilizand hashtag-uri de campanie (\#CaleaSigura) si limbaj laudativ specific propagandei electorale. Postarea este sponsorizata si targetata catre un public larg (100,001-500,000 persoane), demonstrand intentia clara de influentare a votului.
\end{enumerate}

\vspace{0.5cm}

\subsection{Gazetarii.RO}
Următoarele fapte contravenționale sunt sesizate împotriva acestei entități:

\begin{enumerate}[leftmargin=*, label=\arabic*.)]
    \item difuzarea de continut de propaganda electorala dupa incheierea perioadei de campanie, concretizat intr-o postare sponsorizata pe Facebook (ID: \href{https://www.facebook.com/ads/library/?id=1349724882848299}{1349724882848299}) care il vizeaza direct pe candidatul ION-MARCEL CIOLACU, prin atacuri la adresa credibilitatii sale, cu scopul clar de a influenta negativ opinia alegatorilor. Postarea a fost promovata dupa ora 18:00 pe 23.11.2024, in perioada de interdictie legala a propagandei electorale, atingand un public estimat de peste 1 milion de persoane prin intermediul platformelor Facebook si Instagram.
\end{enumerate}

\vspace{0.5cm}

\subsection{Gheorghe Valentin Rotar - Primar}
Următoarele fapte contravenționale sunt sesizate împotriva acestei entități:

\begin{enumerate}[leftmargin=*, label=\arabic*.)]
    \item promovarea unui material de propaganda electorala pentru candidatul prezidential Nicolae Ciuca dupa incheierea perioadei de campanie. Postarea cu ID-ul \href{https://www.facebook.com/ads/library/?id=1309174787164909}{1309174787164909} pe facebook reprezinta promovare electorala prin distribuirea unui mesaj pozitiv al candidatului, fiind sponsorizata pentru a ajunge la un public larg (7000-7999 impresii). Postarea a fost facuta si promovata dupa ora 18:00 pe 23.11.2024, incalcand astfel prevederile legale privind incheierea campaniei electorale.
\end{enumerate}

\vspace{0.5cm}

\subsection{Ghergu Nicolae-Marius}
Următoarele fapte contravenționale sunt sesizate împotriva acestei entități:

\begin{enumerate}[leftmargin=*, label=\arabic*.)]
    \item continuarea propagandei electorale dupa incheierea campaniei, manifestata prin postarea cu ID \href{https://www.facebook.com/ads/library/?id=1266932291022514}{1266932291022514} pe facebook. Postarea, efectuata dupa ora 18:00 pe 23.11.2024, contine elementele specifice materialelor de propaganda electorala (CMF 11240060), un mesaj electoral explicit ("Schimba-i!"), si critici la adresa conducerii actuale, fiind promovata prin publicitate platita cu un impact estimat intre 90.000 si 99.999 de afisari. Materialul are scop electoral evident si vizeaza influentarea votului in cadrul alegerilor prezidentiale.
\end{enumerate}

\vspace{0.5cm}

\subsection{Heus Media Service}
Următoarele fapte contravenționale sunt sesizate împotriva acestei entități:

\begin{enumerate}[leftmargin=*, label=\arabic*.)]
    \item promovarea platita a unui mesaj cu caracter electoral care promoveaza indirect candidatul UDMR la presedintie, Kelemen Hunor, prin intermediul declaratiilor premierului ungar Viktor Orban, dupa ora 18:00 pe 23.11.2024. Postarea cu ID-ul \href{https://www.facebook.com/ads/library/?id=1631249137600414}{1631249137600414} a fost distribuita pe Facebook si Instagram, cu un buget intre 300-399 RON, atingand 25,000-30,000 de persoane, reprezentand propaganda electorala continua dupa incheierea perioadei legale de campanie.
    \item difuzarea unui mesaj de propaganda electorala platit (ID Facebook: \href{https://www.facebook.com/ads/library/?id=559832876901964}{559832876901964}) dupa incheierea perioadei de campanie electorala. Materialul promoveaza realizarile UDMR si indirect pe candidatul HUNOR KELEMEN, prin intermediul declaratiilor lui Viktor Orban despre succesul UDMR in negocierile Schengen, cu un indemn explicit de distribuire ("Oszd meg!"). Postarea a fost promovata dupa ora 18:00 pe 23.11.2024, avand un impact semnificativ cu 45.000-50.000 de afisari si un buget de aproximativ 650 RON.
\end{enumerate}

\vspace{0.5cm}

\subsection{Horia Constantinescu}
Următoarele fapte contravenționale sunt sesizate împotriva acestei entități:

\begin{enumerate}[leftmargin=*, label=\arabic*.)]
    \item promovarea unui mesaj electoral platit pe Facebook (ID: \href{https://www.facebook.com/ads/library/?id=1289994562135510}{1289994562135510}) dupa incheierea perioadei de campanie electorala, respectiv dupa ora 18:00 pe 23.11.2024. Mesajul contine o sustinere explicita a candidatului ION-MARCEL CIOLACU si indeamna in mod direct alegatorii sa il sustina, avand un impact semnificativ prin targetarea unei audiente estimate de peste 1 milion de persoane, cu 10.000-14.999 afisari efective. Mesajul reprezinta propaganda electorala conform Art. 36(7) prin referirea directa la candidat, obiectivul electoral explicit si adresabilitatea catre publicul larg prin intermediul platformelor de social media.
\end{enumerate}

\vspace{0.5cm}

\subsection{I AM ONLINE SRL}
Următoarele fapte contravenționale sunt sesizate împotriva acestei entități:

\begin{enumerate}[leftmargin=*, label=\arabic*.)]
    \item difuzarea de materiale de propaganda electorala dupa incheierea campaniei electorale pentru alegerile prezidentiale. Postarea cu ID-ul \href{https://www.facebook.com/ads/library/?id=1324022541930513}{1324022541930513} contine un indemn explicit de a vota candidatul George Simion la alegerile prezidentiale, fiind difuzata ca reclama platita pe Facebook si Instagram, cu un impact intre 30.000 si 35.000 de afisari, dupa ora 18:00 pe 23.11.2024. Postarea include cod mandatar financiar (11240014) si reprezinta propaganda electorala conform Art. 36(7) prin prezenta unui mesaj electoral explicit, identificarea clara a candidatului, si adresabilitatea catre publicul larg prin natura sa de continut sponsorizat.
    \item promovarea candidatului George Simion la presedintie prin intermediul unei reclame platite pe Facebook (ID: \href{https://www.facebook.com/ads/library/?id=1693913901167510}{1693913901167510}) dupa ora 18:00 pe 23.11.2024. Postarea contine promisiuni electorale explicite ("Cu AUR la guvernare si George Simion presedinte, agricultorii vor fi sprijiniti"), identificator CMF 11240014, si indeamna in mod direct la sustinerea candidatului in perioada in care propaganda electorala este interzisa. Impactul postarii este semnificativ, avand intre 25.000 si 29.999 de afisari, reprezentand o incalcare clara a prevederilor legale privind incheierea campaniei electorale.
\end{enumerate}

\vspace{0.5cm}

\subsection{Ilie Suciu si Partidul National Liberal}
Următoarele fapte contravenționale sunt sesizate împotriva acestei entități:

\begin{enumerate}[leftmargin=*, label=\arabic*.)]
    \item promovarea unui mesaj de propaganda electorala pentru candidatul NICOLAE-IONEL CIUCA la functia de presedinte dupa incheierea perioadei de campanie. Postarea cu ID-ul \href{https://www.facebook.com/ads/library/?id=2321738621330179}{2321738621330179} pe Facebook contine indemnuri directe la vot ("votati la presedintie candidatul"), argumente electorale si instructiuni specifice de vot, fiind difuzata dupa ora 18:00 pe 23.11.2024. Mesajul a fost promovat ca reclama platita, atingand intre 2000-2999 de persoane, constituind astfel un act deliberat de propaganda electorala in afara perioadei legale de campanie.
\end{enumerate}

\vspace{0.5cm}

\subsection{Ilie-Alin Colesa}
Următoarele fapte contravenționale sunt sesizate împotriva acestei entități:

\begin{enumerate}[leftmargin=*, label=\arabic*.)]
    \item promovarea unui mesaj electoral platit pe platformele Facebook si Instagram (ID postare: \href{https://www.facebook.com/ads/library/?id=8799652853454129}{8799652853454129}) dupa ora 18:00 pe 23.11.2024, mesaj care vizeaza in mod direct candidatul Calin Georgescu, punand sub semnul intrebarii credibilitatea sa religioasa si morala. Postarea a avut un impact semnificativ, atingand intre 5.000-5.999 de persoane, fiind distribuita in majoritatea judetelor tarii, reprezentand astfel o forma clara de propaganda electorala in afara perioadei legale permise.
\end{enumerate}

\vspace{0.5cm}

\subsection{InfoBraila}
Următoarele fapte contravenționale sunt sesizate împotriva acestei entități:

\begin{enumerate}[leftmargin=*, label=\arabic*.)]
    \item difuzarea de materiale de propaganda electorala dupa incheierea campaniei electorale pentru alegerile prezidentiale. Postarea cu ID-ul \href{https://www.facebook.com/ads/library/?id=1058273652450836}{1058273652450836} constituie propaganda electorala prin mentionarea directa a candidatului George Simion, folosind un mesaj care poate influenta comportamentul electoral al votantilor, fiind difuzata ca reclama platita pe Facebook si Instagram dupa ora 18:00 pe 23.11.2024. Impactul este demonstrat prin reach-ul de peste 30.000 de afisari si bugetul alocat de peste 200 RON.
\end{enumerate}

\vspace{0.5cm}

\subsection{Infomed Pro SRL}
Următoarele fapte contravenționale sunt sesizate împotriva acestei entități:

\begin{enumerate}[leftmargin=*, label=\arabic*.)]
    \item difuzarea de materiale de propaganda electorala (ID postare Facebook: \href{https://www.facebook.com/ads/library/?id=1028956809243157}{1028956809243157}) dupa incheierea perioadei de campanie electorala pentru alegerile prezidentiale. Postarea contine indemnuri directe de vot pentru candidatul ION-MARCEL CIOLACU, folosind un CMF valid (11240017), intr-o perioada in care propaganda electorala pentru alegerile prezidentiale este interzisa (dupa ora 18:00 pe 23.11.2024). Impactul postarii este semnificativ, atingand intre 8.000 si 8.999 de persoane, reprezentand o incalcare clara a prevederilor legale privind incheierea campaniei electorale.
    \item difuzarea de materiale de propaganda electorala (post Facebook ID: \href{https://www.facebook.com/ads/library/?id=1072866431252300}{1072866431252300}) dupa incheierea perioadei de campanie electorala pentru alegerile prezidentiale. Materialul promoveaza direct candidatul ION-MARCEL CIOLACU, folosind CMF 11240017, hashtag-uri de partid (\#PSD) si prezentand realizari administrative in scop electoral. Postarea este promovata ca reclama platita pe Facebook si Instagram dupa ora 18:00 pe 23.11.2024, incalcand astfel prevederile legale privind interdictia propagandei electorale dupa incheierea campaniei.
    \item difuzarea de materiale de propaganda electorala dupa incheierea campaniei electorale, concretizata prin promovarea platita a realizarilor guvernamentale ale candidatului ION-MARCEL CIOLACU pe Facebook si Instagram (ID postare: \href{https://www.facebook.com/ads/library/?id=1072969724289605}{1072969724289605}). Postarea, activa dupa ora 18:00 pe 23.11.2024, promoveaza explicit realizarile guvernamentale ale candidatului la presedintie, utilizand mesaje care il asociaza direct cu rezultate pozitive in domeniul sanatatii, avand un evident caracter de propaganda electorala prin natura sa promotionala si contextul electoral.
    \item promovarea unui anunt platit pe Facebook (ID: \href{https://www.facebook.com/ads/library/?id=1083249396324223}{1083249396324223}) care continua propaganda electorala dupa incheierea acesteia. Anuntul contine un indemn explicit de a vota candidatul ION-MARCEL CIOLACU la alegerile prezidentiale ("Duminica aceasta, votati Marcel Ciolacu presedinte"), fiind difuzat dupa ora 18:00 pe 23.11.2024, cand perioada de campanie electorala pentru alegerile prezidentiale s-a incheiat. Materialul include numar CMF (11240017) si constituie propaganda electorala conform Art. 36(7), avand obiectiv electoral explicit si adresandu-se publicului larg prin intermediul platformelor de social media.
    \item promovarea unui material de propaganda electorala (ID Facebook: \href{https://www.facebook.com/ads/library/?id=1087041495976896}{1087041495976896}) dupa incheierea perioadei de campanie electorala prezidentiala. Materialul promovat face referire directa la candidatul prezidential Marcel Ciolacu si programul sau de tara, fiind difuzat dupa ora 18:00 pe 23.11.2024. Postarea contine numar CMF (11240017), are caracter electoral explicit si a fost promovata platit pe platformele Facebook si Instagram, cu un impact estimat intre 100,001 si 500,000 de persoane.
    \item promovarea unui mesaj de propaganda electorala (ID postare Facebook: \href{https://www.facebook.com/ads/library/?id=1093387848852522}{1093387848852522}) dupa ora 18:00 pe 23.11.2024, in favoarea candidatului ION-MARCEL CIOLACU. Postarea contine numar CMF (11240017), promoveaza realizarile candidatului in calitate de prim-ministru, cu referire directa la majorarile salariale implementate sub conducerea sa, avand un obiectiv electoral clar de influentare a votului. Mesajul a fost distribuit ca reclama platita pe Facebook si Instagram, cu un impact estimat intre 4.000 si 4.999 de afisari.
    \item promovarea unui mesaj electoral platit pe Facebook si Instagram (ID postare: \href{https://www.facebook.com/ads/library/?id=1094718548873755}{1094718548873755}) dupa incheierea perioadei de campanie electorala pentru alegerile prezidentiale. Mesajul contine indemnuri directe de vot pentru candidatul Marcel Ciolacu la alegerile prezidentiale ("votati Marcel Ciolacu presedinte pe 24 noiembrie"), fiind difuzat dupa ora 18:00 pe 23.11.2024, incalcand astfel prevederile legale privind perioada de campanie electorala. Mesajul este in mod clar unul de propaganda electorala, continand numarul CMF 11240017, si este distribuit catre un public larg estimat intre 100,001 si 500,000 de persoane.
    \item continuarea propagandei electorale dupa incheierea campaniei electorale prezidentiale, prin publicarea si sponsorizarea unui mesaj pe Facebook (ID post: \href{https://www.facebook.com/ads/library/?id=1109886473308382}{1109886473308382}) care indeamna explicit la votarea candidatului ION-MARCEL CIOLACU la alegerile prezidentiale. Postarea, activa dupa ora 18:00 pe 23.11.2024, contine numar CMF (CMF11240017), utilizeaza mesaje de campanie si indeamna direct la vot pentru un candidat prezidential specific, incalcand astfel perioada de blackout electoral. Mesajul a fost distribuit catre un public tinta estimat intre 100,001 si 500,000 de persoane, amplificand astfel impactul incalcarii.
    \item promovarea unui material de propaganda electorala (ID postare Facebook: \href{https://www.facebook.com/ads/library/?id=1111404077659539}{1111404077659539}) dupa incheierea perioadei de campanie. Materialul promoveaza explicit candidatul prezidential Marcel Ciolacu si PSD, folosind numar CMF (11240017), fiind o reclama platita pe Facebook si Instagram cu mesaje electorale clare ("lupta alaturi de presedintele Marcel Ciolacu pentru romani"). Postarea a fost activa si dupa ora 18:00 pe 23.11.2024, incalcand astfel perioada legala de campanie electorala.
    \item promovarea unui material de propaganda electorala dupa incheierea perioadei de campanie, respectiv dupa ora 18:00 pe 23.11.2024. Materialul, cu ID-ul \href{https://www.facebook.com/ads/library/?id=1113785710353254}{1113785710353254} pe Facebook, promoveaza candidatul prezidential Marcel Ciolacu prin evidentierea realizarilor sale guvernamentale si multumiri explicite, depasind sfera comunicarii administrative obiective. Mesajul a fost promovat activ prin publicitate platita pe Facebook si Instagram, cu un buget intre 100-199 RON si o audienta estimata intre 100,001-500,000 persoane, demonstrand intentia clara de influentare a votului.
    \item continuarea propagandei electorale pentru candidatul prezidential Marcel Ciolacu dupa incheierea perioadei legale de campanie. Postarea cu ID-ul \href{https://www.facebook.com/ads/library/?id=1120710852905228}{1120710852905228} contine mesaje directe de propaganda electorala, folosind CMF 11240017, promovand explicit programul candidatului si indemnand la vot pentru PSD in alegerile prezidentiale. Postarea este promovata dupa ora 18:00 pe 23.11.2024, avand un impact semnificativ cu peste 10,000 de afisari si un buget de promovare intre 200-299 RON.
    \item difuzarea de materiale de propaganda electorala dupa incheierea perioadei de campanie, constand intr-o postare sponsorizata pe Facebook/Instagram (ID: \href{https://www.facebook.com/ads/library/?id=1120850192933191}{1120850192933191}) care indeamna explicit la votarea candidatului Marcel Ciolacu ("Voteaza Marcel Ciolacu, calea sigura pentru Romania!"), asociaza realizari guvernamentale cu candidatul si foloseste mesaje de campanie pentru influentarea votului. Postarea a fost activa si dupa ora 18:00 pe 23.11.2024, incalcand astfel prevederile legale privind incheierea campaniei electorale.
    \item continuarea propagandei electorale dupa incheierea campaniei prezidentiale, promovand candidatul ION-MARCEL CIOLACU pentru functia de presedinte prin intermediul unei reclame platite pe Facebook (ID: \href{https://www.facebook.com/ads/library/?id=1126375062244385}{1126375062244385}) cu mesajul explicit "Votati Marcel Ciolacu, presedinte!". Materialul promotional, marcat cu CMF 11240017, a avut un impact semnificativ, atingand intre 15.000 si 19.999 de persoane dupa ora 18:00 pe 23.11.2024, incalcand astfel perioada legala de campanie electorala pentru alegerile prezidentiale.
    \item difuzarea unei reclame platite pe Facebook (ID: \href{https://www.facebook.com/ads/library/?id=1306321060785441}{1306321060785441}) ce continua propaganda electorala dupa incheierea campaniei pentru alegerile prezidentiale. Postarea contine un indemn explicit de a vota candidatul Marcel Ciolacu la alegerile prezidentiale ("voteaza Marcel Ciolacu la alegerile prezidentiale"), fiind difuzata dupa ora 18:00 pe 23.11.2024. Materialul este unul de propaganda electorala conform Art. 36(7), continand numar CMF (11240017), fiind platit pentru a ajunge la un public larg (3000-3999 impresii), si avand obiectiv electoral explicit.
    \item difuzarea de materiale de propaganda electorala dupa incheierea campaniei electorale, respectiv postarea cu ID-ul \href{https://www.facebook.com/ads/library/?id=1317717352743487}{1317717352743487} pe facebook. Postarea promoveaza explicit candidatul ION-MARCEL CIOLACU pentru functia de presedinte al Romaniei, prezentand motive pentru care alegatorii ar trebui sa il voteze, avand un caracter vadit de propaganda electorala. Materialul este difuzat sub forma de reclama platita pe Facebook si Instagram, cu un public tinta estimat intre 100,001 si 500,000 de persoane, dupa ora 18:00 pe 23.11.2024, incalcand astfel prevederile legale privind perioada de campanie electorala.
    \item difuzarea de materiale de propaganda electorala dupa incheierea campaniei electorale pentru alegerile prezidentiale, prin postarea cu ID \href{https://www.facebook.com/ads/library/?id=1354493685519333}{1354493685519333} pe Facebook. Postarea contine indemnul explicit de a vota candidatul Marcel Ciolacu la alegerile prezidentiale ("Pe 24 noiembrie, voteaza Marcel Ciolacu"), fiind difuzata si dupa ora 18:00 pe 23.11.2024. Materialul include numar CMF (11240017), reprezinta propaganda electorala conform Art. 36(7), fiind o postare platita ce se adreseaza unui public larg (100,001-500,000 persoane) cu scop electoral explicit.
    \item promovarea unui mesaj de propaganda electorala dupa incheierea perioadei legale de campanie, referitor la candidatul prezidential Marcel Ciolacu. Postarea ID \href{https://www.facebook.com/ads/library/?id=1481881965839933}{1481881965839933} continua sa fie activa si promovata dupa ora 18:00 pe 23.11.2024, contine numar CMF (11240017), promoveaza explicit programul de guvernare si promisiunile electorale ale candidatului, fiind targetata catre populatia judetului Bihor, cu un buget de promovare intre 100-199 RON si o audienta estimata intre 100,001-500,000 persoane.
    \item promovarea unui mesaj de propaganda electorala pentru candidatul prezidential Marcel Ciolacu dupa incheierea perioadei de campanie. Postarea cu ID \href{https://www.facebook.com/ads/library/?id=1501110307184723}{1501110307184723} contine indemnuri directe de vot ("va indemn sa ne sustineti, votand Marcel Ciolacu presedinte"), are caracter electoral explicit, fiind difuzata ca reclama platita pe Facebook si Instagram dupa ora 18:00 pe 23.11.2024. Postarea include numar CMF (11240017), confirmand natura sa de propaganda electorala, si vizeaza un public larg estimat intre 100,001-500,000 persoane.
    \item promovarea unui mesaj de propaganda electorala pentru candidatul prezidential Marcel Ciolacu dupa incheierea perioadei de campanie. Postarea cu ID-ul \href{https://www.facebook.com/ads/library/?id=1520096789382440}{1520096789382440} contine indemnuri directe la vot ("votam cu domnul presedinte Marcel Ciolacu") si a fost difuzata dupa ora 18:00 pe 23.11.2024, incalcand astfel prevederile legale privind incheierea campaniei electorale. Mesajul a avut un impact semnificativ, atingand intre 50.000 si 59.999 de persoane pe platformele Facebook si Instagram, reprezintand o incalcare clara a prevederilor legale privind incetarea propagandei electorale.
    \item promovarea unui material de propaganda electorala (ID Facebook: \href{https://www.facebook.com/ads/library/?id=1529351454396166}{1529351454396166}) dupa incheierea perioadei de campanie electorala. Materialul promoveaza explicit candidatul ION-MARCEL CIOLACU si programul sau electoral, contine numar CMF (11240017), si este distribuit ca reclama platita pe Facebook si Instagram dupa ora 18:00 pe 23.11.2024. Materialul prezinta promisiuni electorale si puncte din programul de guvernare, avand un evident caracter de propaganda electorala in favoarea candidatului PSD la presedintie.
    \item difuzarea de materiale de propaganda electorala (post-ul Facebook ID \href{https://www.facebook.com/ads/library/?id=1559839654674397}{1559839654674397}) dupa incheierea perioadei de campanie electorala. Materialul promoveaza explicit candidatul ION-MARCEL CIOLACU la functia de presedinte, continand indemnuri directe la vot ("votati Marcel Ciolacu presedinte in aceasta duminica"), fiind difuzat ca reclama platita pe Facebook dupa ora 18:00 pe 23.11.2024. Postarea include numar CMF (11240017), confirmand natura sa de material electoral, si are un impact semnificativ, atingand intre 10,000 si 14,999 de persoane.
    \item difuzarea de materiale de propaganda electorala dupa incheierea perioadei de campanie, prin promovarea unui mesaj platit pe Facebook (ID: \href{https://www.facebook.com/ads/library/?id=1610575576479237}{1610575576479237}) care prezinta realizarile si lauda activitatea candidatului prezidential Marcel Ciolacu, actual prim-ministru. Postarea, activa dupa ora 18:00 pe 23.11.2024, reprezinta propaganda electorala prin promovarea explicita a realizarilor guvernamentale ale candidatului, cu intentia clara de a influenta preferintele electorale ale votantilor.
    \item continuarea propagandei electorale dupa incheierea campaniei, prin publicarea si mentinerea activa a unei reclame pe Facebook (ID: \href{https://www.facebook.com/ads/library/?id=1610696299520643}{1610696299520643}) care promoveaza explicit candidatul PSD Ion Marcel Ciolacu si critica candidatul PNL Nicolae Ciuca, continand mesaje electorale clare si numar CMF (11240017). Reclama a fost difuzata si dupa ora 18:00 pe 23.11.2024, incalcand astfel perioada legala de campanie electorala. Efectul electoral este evident prin contrastul facut intre realizarile PSD si promisiunile PNL, precum si prin mesajul final care indeamna la sustinerea PSD la guvernare.
    \item promovarea unui mesaj de propaganda electorala pentru candidatul prezidential Marcel Ciolacu dupa incheierea perioadei de campanie. Postarea cu ID-ul \href{https://www.facebook.com/ads/library/?id=1696543720906867}{1696543720906867} contine numar CMF (CMF11240017), promoveaza explicit programul electoral si candidatul la prezidentiale Marcel Ciolacu, foloseste hashtag-uri de campanie si este promovata ca reclama platita pe Facebook dupa ora 18:00 pe 23.11.2024. Mesajul are caracter electoral explicit si vizeaza influentarea votului pentru alegerile prezidentiale.
    \item promovarea unui mesaj de propaganda electorala (ID Facebook: \href{https://www.facebook.com/ads/library/?id=1697891587608035}{1697891587608035}) dupa ora 18:00 pe 23.11.2024. Postarea promoveaza explicit realizarile guvernamentale ale candidatului prezidential Marcel Ciolacu, folosind un CMF (11240017), si este distribuita ca reclama platita pe Facebook si Instagram. Mesajul are caracter electoral explicit, promovand realizarile guvernarii conduse de candidatul prezidential Marcel Ciolacu in perioada in care propaganda electorala pentru alegerile prezidentiale este interzisa.
    \item difuzarea de materiale de propaganda electorala pentru candidatul prezidential Marcel Ciolacu dupa incheierea perioadei de campanie. Postarea cu ID-ul \href{https://www.facebook.com/ads/library/?id=1702946183611560}{1702946183611560} contine promisiuni electorale explicite, targetare platita catre public larg (100,001-500,000 persoane), cod unic CMF 11240017, si promoveaza activ candidatul si partidul sau dupa ora 18:00 pe 23.11.2024, incalcand astfel prevederile legale privind incetarea propagandei electorale.
    \item continuarea propagandei electorale dupa incheierea campaniei electorale pentru alegerile prezidentiale, prin promovarea unui mesaj electoral explicit pentru candidatul Marcel Ciolacu ("votati Marcel Ciolacu pentru un viitor sigur!") intr-o postare sponsorizata pe Facebook si Instagram (ID: \href{https://www.facebook.com/ads/library/?id=1742822496636862}{1742822496636862}) dupa ora 18:00 pe 23.11.2024. Postarea contine elemente clare de propaganda electorala, inclusiv indemnuri directe la vot si numar CMF (11240017), incalcand astfel prevederile legale privind incheierea campaniei electorale.
    \item promovarea unui mesaj de propaganda electorala (ID Facebook: \href{https://www.facebook.com/ads/library/?id=1839933430165436}{1839933430165436}) dupa ora 18:00 pe 23.11.2024, in favoarea candidatului ION-MARCEL CIOLACU. Materialul include numar CMF (11240017), promoveaza explicit candidatul si partidul sau prin referinte directe, hashtag-uri oficiale ale partidului (\#caleasigurapentruromania), si prezinta un mesaj electoral explicit care influenteaza alegatorii. Reclama este platita si activa, cu un buget intre 200-299 RON si o audienta estimata intre 100,001-500,000 persoane.
    \item promovarea unui material de propaganda electorala (ID postare Facebook: \href{https://www.facebook.com/ads/library/?id=1938529673225729}{1938529673225729}) dupa incheierea perioadei de campanie prezidentiala. Materialul, publicat si promovat dupa ora 18:00 pe 23.11.2024, contine referiri directe la candidatul prezidential Marcel Ciolacu si alegerile prezidentiale, foloseste numar CMF (11240017), si are un obiectiv electoral clar exprimat prin sloganul "PSD, Calea Sigura pentru Romania!". Postarea a fost promovata activ, atingand intre 15.000 si 19.999 de persoane, reprezentand astfel o continuare deliberata a propagandei electorale dupa incheierea perioadei legale de campanie.
    \item continuarea difuzarii de materiale de propaganda electorala dupa incheierea campaniei electorale prezidentiale. Materialul promotional cu ID-ul \href{https://www.facebook.com/ads/library/?id=2094189494329421}{2094189494329421} pe facebook promoveaza in mod direct candidatul prezidential Marcel Ciolacu, prin prezentarea masurilor si propunerilor acestuia, contine numar oficial de material electoral CMF11240017, si continua sa fie difuzat ca reclama platita dupa ora 18:00 pe 23.11.2024. Materialul are obiectiv electoral explicit si se adreseaza publicului larg prin intermediul platformelor Facebook si Instagram.
    \item promovarea unui mesaj de propaganda electorala pentru candidatul ION-MARCEL CIOLACU dupa incheierea perioadei de campanie, prin intermediul unei reclame platite pe Facebook (ID: \href{https://www.facebook.com/ads/library/?id=2312255482440591}{2312255482440591}). Postarea, publicata dupa ora 18:00 pe 23.11.2024, promoveaza realizarile guvernamentale ale candidatului, foloseste hashtag-ul de campanie "\#caleasigurapentruromania", si are numarul de inregistrare CMF11240017, confirmand natura sa de propaganda electorala. Impactul estimat este intre 15.000 si 19.999 de afisari, cu un buget intre 200-299 RON.
    \item promovarea unui mesaj de propaganda electorala dupa incheierea perioadei de campanie, prin intermediul unei reclame platite pe Facebook (ID: 27581136358200824) care promoveaza candidatul ION-MARCEL CIOLACU intr-o lumina pozitiva, asociindu-l cu realizari in domeniul sanatatii. Materialul, distribuit dupa ora 18:00 pe 23.11.2024, reprezinta propaganda electorala prin asocierea directa a candidatului cu realizari guvernamentale si promovarea explicita a imaginii sale, depasind scopul informativ si intrand in sfera propagandei electorale.
    \item promovarea unui material de propaganda electorala (ID Facebook: \href{https://www.facebook.com/ads/library/?id=376470698883242}{376470698883242}) dupa incheierea perioadei de campanie. Materialul promoveaza explicit candidatul ION-MARCEL CIOLACU prin utilizarea hashtag-ului "\#CiolacuPresedinte" si continut electoral specific (CMF 11240017), fiind difuzat ca reclama platita pe Facebook si Instagram dupa ora 18:00 pe 23.11.2024. Impactul este semnificativ, atingand intre 150.000 si 175.000 de afisari, reprezentand o incalcare clara a prevederilor legale privind incetarea propagandei electorale.
    \item continuarea propagandei electorale dupa incheierea campaniei, promovand realizarile guvernamentale ale candidatului prezidential Marcel Ciolacu. Postarea cu ID \href{https://www.facebook.com/ads/library/?id=3790873984507071}{3790873984507071} pe Facebook contine numar CMF (11240017), hashtag de campanie si mesaje electorale explicite, fiind o reclama platita cu impact intre 10.000-14.999 impresii, difuzata dupa ora 18:00 pe 23.11.2024. Materialul reprezinta in mod clar propaganda electorala conform Art. 36(7), avand obiectiv electoral si adresandu-se publicului larg prin intermediul platformelor de social media.
    \item difuzarea de materiale de propaganda electorala pentru candidatul ION-MARCEL CIOLACU dupa incheierea campaniei electorale. Postarea cu ID-ul \href{https://www.facebook.com/ads/library/?id=386986527742462}{386986527742462} contine indemnuri directe la vot ("Sustine-ne duminica aceasta, votand Marcel Ciolacu presedinte!"), fiind o reclama platita pe Facebook si Instagram, cu un buget intre 100-199 RON si o audienta estimata intre 100,001-500,000 persoane. Postarea include numarul oficial de campanie CMF 11240017, confirmand natura sa de propaganda electorala. Aceasta activitate continua dupa ora 18:00 pe 23.11.2024, incalcand astfel prevederile legale privind perioada de campanie electorala.
    \item promovarea unui mesaj electoral platit pe Facebook (ID postare: \href{https://www.facebook.com/ads/library/?id=3986809324924481}{3986809324924481}) care face propaganda electorala pentru candidatul prezidential Marcel Ciolacu dupa incheierea perioadei de campanie. Postarea contine indemnuri directe la vot ("votati Marcel Ciolacu presedinte"), promisiuni electorale specifice (cresterea pensiilor) si are un caracter vadit de propaganda electorala, fiind difuzata dupa ora 18:00 pe 23.11.2024, cand perioada de campanie electorala pentru alegerile prezidentiale s-a incheiat. Postarea este platita si targetata catre un public larg (100,001-500,000 persoane), folosind platformele Facebook si Instagram.
    \item difuzarea de materiale de propaganda electorala (ID postare Facebook: \href{https://www.facebook.com/ads/library/?id=419482617925440}{419482617925440}) dupa incheierea perioadei de campanie electorala. Materialul promovat reprezinta propaganda electorala explicita pentru candidatul ION-MARCEL CIOLACU, avand scop electoral clar definit, adresandu-se publicului larg prin intermediul unei reclame platite pe Facebook si Instagram, cu mesaje precum "Marcel Ciolacu, calea sigura pentru Romania" si indemnuri clare de sustinere a candidatului la functia de presedinte. Postarea a fost promovata dupa ora 18:00 pe 23.11.2024, incalcand astfel prevederile legale privind incheierea campaniei electorale.
    \item difuzarea de materiale de propaganda electorala dupa incheierea campaniei electorale pentru alegerile prezidentiale, intr-o postare sponsorizata pe Facebook (ID: \href{https://www.facebook.com/ads/library/?id=448675854942347}{448675854942347}) care promoveaza candidatul ION-MARCEL CIOLACU si programul sau electoral. Postarea contine numar de inregistrare CMF 11240017, foloseste mesaje specifice campaniei electorale, si a fost difuzata dupa ora 18:00 pe 23.11.2024, incalcand astfel prevederile legale privind incheierea campaniei electorale.
    \item promovarea unui mesaj electoral platit (ID Facebook: \href{https://www.facebook.com/ads/library/?id=539951575522072}{539951575522072}) ce indeamna explicit la votarea candidatului prezidential Marcel Ciolacu, dupa incheierea perioadei legale de campanie electorala. Mesajul contine indemnuri directe de vot ("va asteptam sa-l votati pe presedintele Marcel Ciolacu"), fiind difuzat dupa ora 18:00 pe 23.11.2024, incalcand astfel prevederile legale privind incheierea campaniei electorale. Materialul este identificabil ca propaganda electorala prin prezenta codului CMF 11240017 si natura sa de continut sponsorizat pe platformele Facebook si Instagram.
    \item continuarea propagandei electorale dupa incheierea campaniei pentru alegerile prezidentiale, promovand candidatul ION-MARCEL CIOLACU si PSD prin postarea cu ID \href{https://www.facebook.com/ads/library/?id=539970938841239}{539970938841239} pe Facebook. Postarea contine promisiuni electorale specifice, indemnuri directe la vot si foloseste numarul CMF11240017, fiind difuzata dupa ora 18:00 pe 23.11.2024. Postarea a avut un impact semnificativ, atingand intre 15.000 si 19.999 de persoane, reprezentand o incalcare clara a prevederilor legale privind incheierea campaniei electorale.
    \item promovarea unui material de propaganda electorala pentru candidatul ION-MARCEL CIOLACU dupa incheierea perioadei de campanie electorala. Postarea cu ID \href{https://www.facebook.com/ads/library/?id=540099652150122}{540099652150122} promoveaza programul de guvernare si propunerile candidatului, fiind distribuit ca reclama platita pe Facebook si Instagram, cu un impact estimat intre 100,001 si 500,000 de persoane, dupa ora 18:00 pe 23.11.2024. Materialul contine numarul de inregistrare CMF 11240017, confirmand natura sa de propaganda electorala.
    \item difuzarea de materiale de propaganda electorala (ID postare Facebook: \href{https://www.facebook.com/ads/library/?id=576695268147731}{576695268147731}) dupa incheierea perioadei de campanie electorala. Postarea promoveaza explicit candidatul PSD Marcel Ciolacu si denigreaza candidatul PNL Nicolae Ciuca, contine numar CMF (11240017), si este difuzata ca reclama platita pe Facebook si Instagram dupa ora 18:00 pe 23.11.2024. Materialul include promisiuni electorale specifice si mesaje de campanie, fiind in mod clar propaganda electorala conform Art. 36(7) din Legea 334/2006.
    \item promovarea unui mesaj de propaganda electorala pentru candidatul prezidential Marcel Ciolacu dupa incheierea perioadei de campanie electorala. Postarea cu ID-ul \href{https://www.facebook.com/ads/library/?id=579459671127548}{579459671127548} contine un indemn explicit de vot ("votati Marcel Ciolacu presedinte"), fiind o reclama platita pe Facebook cu impact intre 100,001 si 500,000 de persoane, difuzata dupa ora 18:00 pe 23.11.2024. Mesajul include numar CMF (11240017) si reprezinta propaganda electorala conform Art. 36(7), avand obiectiv electoral explicit si adresandu-se publicului larg.
    \item difuzarea de materiale de propaganda electorala pentru alegerile prezidentiale dupa incheierea perioadei legale de campanie. Postarea cu ID-ul \href{https://www.facebook.com/ads/library/?id=595398042978959}{595398042978959} promoveaza explicit victoria in alegerile prezidentiale ("castigam primul tur al alegerilor prezidentiale"), contine numar CMF (11240017), si promoveaza masurile PSD in contextul alegerilor prezidentiale. Postarea este sponsorizata si activa dupa ora 18:00 pe 23.11.2024, atingand intre 4000-4999 de persoane. Materialul reprezinta propaganda electorala conform Art. 36(7) din Legea 334/2006, indeplinind toate conditiile: referire directa la partid, obiectiv electoral explicit, adresare catre public larg prin sponsorizare, si depaseste limitele activitatii jurnalistice.
    \item promovarea unui mesaj de propaganda electorala pentru candidatul la presedintie Marcel Ciolacu dupa incheierea perioadei de campanie electorala. Postarea cu ID-ul \href{https://www.facebook.com/ads/library/?id=599690146053145}{599690146053145} contine un indemn explicit la vot ("votati Marcel Ciolacu - presedinte!"), fiind publicata si promovata ca reclama platita pe Facebook si Instagram dupa ora 18:00 pe 23.11.2024. Mesajul are caracter electoral evident, fiind insotit de cod CMF 11240017, si urmareste influentarea directa a intentiei de vot a alegatorilor pentru scrutinul prezidential.
    \item difuzarea de materiale de propaganda electorala dupa incheierea perioadei de campanie, prin postarea cu ID \href{https://www.facebook.com/ads/library/?id=792635229631100}{792635229631100} pe Facebook. Postarea promoveaza explicit candidatul ION-MARCEL CIOLACU, contine numar CMF (11240017), si a fost difuzata ca reclama platita dupa ora 18:00 pe 23.11.2024. Materialul face referire directa la realizarile guvernamentale ale candidatului, foloseste hashtag-uri de campanie, si are scop electoral explicit, fiind distribuit contra cost catre un public larg (25,000-29,999 impresii).
    \item promovarea unui material electoral platit (ID Facebook: \href{https://www.facebook.com/ads/library/?id=829926469106440}{829926469106440}) ce face referire directa la realizarile candidatului prezidential Marcel Ciolacu, dupa ora 18:00 pe 23.11.2024. Materialul, care include CMF11240017, are un impact electoral semnificativ, atingand intre 15.000 si 19.999 de persoane, promovand imaginea candidatului prezidential intr-un mod pozitiv prin evidentierea realizarilor sale guvernamentale, in perioada in care propaganda electorala pentru alegerile prezidentiale este interzisa.
    \item promovarea unui mesaj electoral explicit pentru candidatul prezidential Marcel Ciolacu dupa incheierea perioadei de campanie electorala. Postarea cu ID-ul \href{https://www.facebook.com/ads/library/?id=862107419116899}{862107419116899} contine indemnuri directe de vot ("votati Marcel Ciolacu pentru un viitor sigur") si a fost difuzata dupa ora 18:00 pe 23.11.2024, incalcand astfel prevederile legale privind incetarea propagandei electorale. Mesajul are caracter electoral evident, fiind marcat cu CMF 11240017, si vizeaza influentarea directa a votului pentru alegerile prezidentiale.
    \item publicarea si promovarea unui mesaj de propaganda electorala (ID Facebook: \href{https://www.facebook.com/ads/library/?id=891258106323887}{891258106323887}) dupa incheierea perioadei de campanie electorala pentru alegerile prezidentiale. Postarea, publicata si promovata dupa ora 18:00 pe 23.11.2024, contine in mod explicit indemnuri de a vota candidatul prezidential Marcel Ciolacu in data de 24 noiembrie, folosind si numar oficial de inregistrare CMF 11240017. Mesajul este distribuit prin platforme de social media (Facebook si Instagram) ca reclama platita, avand potential de expunere intre 100.001 si 500.000 de persoane, demonstrand astfel intentia clara de influentare a votului in perioada in care propaganda electorala este interzisa prin lege.
    \item distribuirea de continut promotional electoral pentru candidatul prezidential PSD dupa incheierea campaniei electorale. Postarea cu ID-ul \href{https://www.facebook.com/ads/library/?id=925257992403519}{925257992403519} promoveaza explicit castigarea alegerilor prezidentiale de catre PSD, foloseste numar CMF 11240017, si contine mesaje de propaganda electorala directa. Postarea este inca activa si promovata cu bani dupa ora 18:00 pe 23.11.2024, incalcand astfel perioada legala de campanie electorala pentru alegerile prezidentiale.
    \item difuzarea de materiale de propaganda electorala dupa incheierea campaniei electorale pentru alegerile prezidentiale, constand intr-o postare sponsorizata pe Facebook si Instagram (ID: \href{https://www.facebook.com/ads/library/?id=925418612429053}{925418612429053}) care promoveaza candidatul prezidential Marcel Ciolacu si programul sau electoral, utilizand numarul de inregistrare CMF11240017. Postarea a fost difuzata dupa ora 18:00 pe 23.11.2024, incalcand astfel perioada legala de campanie electorala. Materialul are caracter electoral evident, fiind directionat catre un public larg estimat intre 100.001 si 500.000 de persoane, si promoveaza in mod direct candidatul la presedintie Marcel Ciolacu si programul sau electoral.
    \item promovarea unui mesaj electoral platit (ID Facebook: \href{https://www.facebook.com/ads/library/?id=937024215153673}{937024215153673}) care continua propaganda electorala pentru candidatul prezidential Marcel Ciolacu dupa incheierea perioadei legale de campanie. Postarea, activa dupa ora 18:00 pe 23.11.2024, contine indemnuri directe la vot ("va asteptam sa-l votati pe presedintele Marcel Ciolacu"), fiind marcata cu numar CMF 11240017, demonstrand caracterul sau explicit de propaganda electorala. Mesajul a avut un impact semnificativ, atingand intre 2000-2999 de afisari pe platformele Facebook si Instagram.
\end{enumerate}

\vspace{0.5cm}

\subsection{Infomed Pro SRL si Gabriel Traian Ghilea}
Următoarele fapte contravenționale sunt sesizate împotriva acestei entități:

\begin{enumerate}[leftmargin=*, label=\arabic*.)]
    \item difuzarea de materiale de propaganda electorala dupa incheierea campaniei electorale prezidentiale, constand intr-o postare sponsorizata pe Facebook (ID: \href{https://www.facebook.com/ads/library/?id=1283780659398512}{1283780659398512}) care indeamna explicit la votarea candidatului Marcel Ciolacu pentru functia de presedinte ("votati Marcel Ciolacu presedinte"). Postarea, difuzata dupa ora 18:00 pe 23.11.2024, include numar CMF (11240017), are caracter electoral explicit si targeteaza un public larg (100,001-500,000 persoane) prin intermediul unei reclame platite pe platformele Facebook si Instagram.
    \item promovarea si indemnul direct de a vota candidatul prezidential Marcel Ciolacu dupa incheierea perioadei de campanie electorala. Postarea cu ID-ul \href{https://www.facebook.com/ads/library/?id=1332526384577298}{1332526384577298} contine mesaje explicite de sustinere si indemn la vot ("votati Marcel Ciolacu presedinte pe 24 noiembrie"), fiind promovata dupa ora 18:00 pe 23.11.2024. Postarea este platita si distribuita activ pe platformele Facebook si Instagram, avand un CMF alocat (11240017), ceea ce confirma natura sa de propaganda electorala.
\end{enumerate}

\vspace{0.5cm}

\subsection{Infomed Pro SRL si PSD}
Următoarele fapte contravenționale sunt sesizate împotriva acestei entități:

\begin{enumerate}[leftmargin=*, label=\arabic*.)]
    \item difuzarea de materiale de propaganda electorala dupa incheierea campaniei electorale pentru alegerile prezidentiale, respectiv dupa ora 18:00 pe 23.11.2024. Postarea cu ID-ul \href{https://www.facebook.com/ads/library/?id=1117507146617421}{1117507146617421} contine indemnuri directe de a vota cu candidatul ION-MARCEL CIOLACU, folosind un mesaj explicit ("votam cu domnul presedinte Marcel Ciolacu"), difuzat prin intermediul unei reclame platite pe Facebook si Instagram, cu o audienta estimata intre 45.000 si 50.000 de persoane. Materialul reprezinta propaganda electorala conform Art. 36(7), indeplinind toate conditiile: referire directa la candidat, obiectiv electoral explicit si adresabilitate catre publicul larg.
\end{enumerate}

\vspace{0.5cm}

\subsection{Infomed Pro SRL si PSD Bihor}
Următoarele fapte contravenționale sunt sesizate împotriva acestei entități:

\begin{enumerate}[leftmargin=*, label=\arabic*.)]
    \item difuzarea de materiale de propaganda electorala pentru candidatul prezidential Marcel Ciolacu (postare Facebook ID: \href{https://www.facebook.com/ads/library/?id=1563753904343301}{1563753904343301}) dupa incheierea perioadei de campanie electorala prezidentiala. Postarea contine indemnuri directe de vot ("Votati Marcel Ciolacu, presedinte!"), fiind promovata ca reclama platita pe Facebook si Instagram dupa ora 18:00 pe 23.11.2024, avand un impact potential asupra 100,001-500,000 de persoane. Materialul include numar CMF (11240017), confirmand natura sa de propaganda electorala oficiala.
\end{enumerate}

\vspace{0.5cm}

\subsection{Innen Analitiq}
Următoarele fapte contravenționale sunt sesizate împotriva acestei entități:

\begin{enumerate}[leftmargin=*, label=\arabic*.)]
    \item difuzarea de materiale de propaganda electorala (ID postare Facebook: \href{https://www.facebook.com/ads/library/?id=1091320795959890}{1091320795959890}) dupa incheierea perioadei de campanie electorala. Postarea, care include numar CMF 11240002, promoveaza explicit candidatul NICOLAE-IONEL CIUCA pentru functia de presedinte, folosind mesaje de campanie si fiind difuzata ca reclama platita pe Facebook si Instagram dupa ora 18:00 pe 23.11.2024. Materialul reprezinta propaganda electorala conform Art. 36(7) prin referirea directa la candidat, obiectivul electoral explicit si adresabilitatea catre publicul larg prin intermediul platformelor de social media.
    \item difuzarea de materiale de propaganda electorala dupa incheierea campaniei electorale pentru alegerile prezidentiale. Postarea cu ID-ul \href{https://www.facebook.com/ads/library/?id=3768559450053507}{3768559450053507} promoveaza explicit candidatul Nicolae-Ionel Ciuca pentru functia de presedinte, contine numar CMF (11240002), foloseste mesaje de campanie directe ("Nicolae Ciuca, presedinte!"), si a fost distribuita dupa ora 18:00 pe 23.11.2024 prin intermediul unei reclame platite pe Facebook si Instagram, cu un impact estimat intre 3.000 si 3.999 de afisari. Materialul reprezinta propaganda electorala conform Art. 36(7), indeplinind toate criteriile: identifica clar candidatul, are obiectiv electoral explicit si se adreseaza publicului larg prin intermediul unei reclame platite.
    \item promovarea unui mesaj electoral platit pe Facebook si Instagram (ID: \href{https://www.facebook.com/ads/library/?id=529016233306349}{529016233306349}) pentru candidatul Nicolae Ciuca, dupa incheierea perioadei de campanie electorala. Mesajul contine numar CMF (11240002), promoveaza explicit candidatul pentru functia de presedinte, si a fost distribuit dupa ora 18:00 pe 23.11.2024 catre un public tinta de 500,001-1,000,000 persoane, reprezentand astfel propaganda electorala in afara perioadei legale de campanie.
\end{enumerate}

\vspace{0.5cm}

\subsection{Ion Iordache}
Următoarele fapte contravenționale sunt sesizate împotriva acestei entități:

\begin{enumerate}[leftmargin=*, label=\arabic*.)]
    \item difuzarea unui mesaj de propaganda electorala (ID postare Facebook: \href{https://www.facebook.com/ads/library/?id=582178580862377}{582178580862377}) dupa incheierea perioadei de campanie electorala. Postarea promoveaza explicit candidatul Nicolae-Ionel Ciuca pentru functia de presedinte, contine numar CMF (11240002), si reprezinta propaganda electorala activa dupa ora 18:00 pe 23.11.2024, avand un impact direct asupra procesului electoral prin mesajul "Nicolae Ciuca trebuie sa fie presedintele Romaniei!" si chemarea la actiune "Tu decizi!". Postarea a fost promovata ca reclama platita pe Facebook, cu un reach estimat intre 10.000-14.999 de persoane.
    \item publicarea si promovarea activa a unui mesaj de propaganda electorala (ID Facebook: \href{https://www.facebook.com/ads/library/?id=878684511102919}{878684511102919}) dupa incheierea perioadei legale de campanie. Postarea, care include numarul CMF 11240002, promoveaza explicit candidatul Nicolae Ciuca pentru functia de presedinte, folosind mesaje de campanie si promisiuni electorale ("Pentru o Oltenie mai prospera"), fiind difuzata dupa ora 18:00 pe 23.11.2024. Impactul postarii este semnificativ, avand intre 60.000 si 70.000 de afisari, cu un buget de promovare intre 600-699 RON, reprezentand astfel o incalcare clara si deliberata a prevederilor legale privind perioada de campanie electorala.
\end{enumerate}

\vspace{0.5cm}

\subsection{Ion Iordache si PNL Gorj}
Următoarele fapte contravenționale sunt sesizate împotriva acestei entități:

\begin{enumerate}[leftmargin=*, label=\arabic*.)]
    \item difuzarea de materiale de propaganda electorala (postare platita pe Facebook cu ID \href{https://www.facebook.com/ads/library/?id=1058089099448408}{1058089099448408}) dupa incheierea perioadei de campanie electorala. Postarea promoveaza explicit candidatul Nicolae Ciuca la functia de presedinte, folosind mesaje de sustinere directa ("Nicolae Ciuca este cel mai bun si are sprijinul nostru!"), avand un CMF asociat (11240002) si fiind difuzata dupa ora 18:00 pe 23.11.2024. Postarea a avut un impact semnificativ, atingand intre 25.000 si 29.999 de persoane, cu o investitie de aproximativ 450 RON in promovare, reprezentand o incercare clara de a influenta votul dupa incheierea campaniei electorale.
    \item continuarea propagandei electorale dupa incheierea perioadei legale de campanie, prin postarea cu ID \href{https://www.facebook.com/ads/library/?id=1276702273486390}{1276702273486390} pe Facebook. Postarea contine propaganda electorala explicita in favoarea candidatului Nicolae Ciuca, folosind CMF 11240002, cu mesaje directe de sustinere si indemnuri la vot ("Nicolae Ciuca este cel mai bun si are sprijinul nostru!", "Tu decizi!"). Postarea este promovata cu bani (700-799 RON) si a ajuns la 70,000-80,000 de persoane dupa ora 18:00 pe 23.11.2024, incalcand astfel prevederile legale privind incheierea campaniei electorale.
    \item continuarea propagandei electorale dupa incheierea campaniei, prin intermediul unei postari sponsorizate pe Facebook (ID: \href{https://www.facebook.com/ads/library/?id=548069484753309}{548069484753309}) care promoveaza explicit candidatul Nicolae Ciuca la functia de presedinte. Postarea, activa dupa ora 18:00 pe 23.11.2024, contine elementele clare ale propagandei electorale: numar CMF (11240002), mesaj explicit de sustinere ("Nicolae Ciuca este cel mai bun si are sprijinul nostru!"), si indeamna direct alegatorii sa voteze ("Tu decizi!"). Impactul electoral este amplificat prin bugetul alocat (400-499 RON) si audienta tintita (30,000-35,000 impresii).
\end{enumerate}

\vspace{0.5cm}

\subsection{Ion Marian Ciofica, candidat Camera Deputatilor Forta Dreptei Gorj}
Următoarele fapte contravenționale sunt sesizate împotriva acestei entități:

\begin{enumerate}[leftmargin=*, label=\arabic*.)]
    \item promovarea unui mesaj de propaganda electorala pentru candidatul prezidential Calin Georgescu dupa incheierea perioadei de campanie electorala. Postarea cu ID-ul \href{https://www.facebook.com/ads/library/?id=1334206434372922}{1334206434372922} pe facebook contine indemnuri directe la vot ("Va chem alaturi de mine sa sustinem", "Votul pentru Calin Georgescu este un vot pentru independenta"), fiind promovata dupa ora 18:00 pe 23.11.2024, intr-o postare platita ce a ajuns la 3000-4000 de persoane din mai multe judete. Mesajul incalca prevederile legale prin continuarea propagandei electorale dupa incheierea acesteia.
\end{enumerate}

\vspace{0.5cm}

\subsection{Legacy Marketing SRL}
Următoarele fapte contravenționale sunt sesizate împotriva acestei entități:

\begin{enumerate}[leftmargin=*, label=\arabic*.)]
    \item difuzarea de materiale de propaganda electorala dupa incheierea campaniei electorale pentru alegerile prezidentiale. Postarea cu ID-ul \href{https://www.facebook.com/ads/library/?id=2403627686650111}{2403627686650111} promoveaza explicit candidatul George Simion pentru functia de presedinte, folosind sloganuri de campanie, indemnuri la vot si numar CMF (11240014). Materialul a fost distribuit ca reclama platita pe Facebook si Instagram dupa ora 18:00 pe 23.11.2024, avand un impact semnificativ (80.000-89.999 impresii). Acesta reprezinta o incalcare clara a perioadei legale de campanie electorala, cu intentia vadita de a influenta procesul electoral.
\end{enumerate}

\vspace{0.5cm}

\subsection{Lorand Toth}
Următoarele fapte contravenționale sunt sesizate împotriva acestei entități:

\begin{enumerate}[leftmargin=*, label=\arabic*.)]
    \item promovarea unui mesaj de propaganda electorala pentru candidatul USR la prezidentiale Elena Lasconi, dupa incheierea perioadei de campanie. Postarea cu ID-ul \href{https://www.facebook.com/ads/library/?id=1575859769965869}{1575859769965869} constituie propaganda electorala conform Art. 36(7) prin: prezenta numarului CMF (11240015), promovarea directa a candidatului, obiectiv electoral explicit prin indemnul la vot ("Turul 1 este decisiv"), si adresarea catre publicul larg prin reclama platita pe Facebook si Instagram. Mentionam ca aceasta postare sponsorizata continua sa fie activa dupa ora 18:00 pe 23.11.2024, incalcand astfel prevederile legale privind perioada de campanie electorala.
\end{enumerate}

\vspace{0.5cm}

\subsection{MEGASOFT SYSTEMS SRL}
Următoarele fapte contravenționale sunt sesizate împotriva acestei entități:

\begin{enumerate}[leftmargin=*, label=\arabic*.)]
    \item difuzarea de materiale de propaganda electorala (ID Facebook: \href{https://www.facebook.com/ads/library/?id=2459673824239734}{2459673824239734}) dupa ora 18:00 pe 23.11.2024. Materialul promoveaza explicit candidatul ION-MARCEL CIOLACU, folosind numarul CMF 11240017, si are ca scop influentarea votului prin afirmatii precum "singurul care a dovedit ca se poate acest lucru este Marcel Ciolacu". Postarea este platita si are o audienta estimata intre 500,001 si 1,000,000 de persoane, demonstrand caracterul sau de propaganda electorala in masa.
    \item difuzarea de materiale de propaganda electorala dupa incheierea campaniei electorale pentru alegerile prezidentiale, intr-o postare sponsorizata pe Facebook si Instagram (ID: \href{https://www.facebook.com/ads/library/?id=291752920699971}{291752920699971}). Postarea promoveaza explicit candidatul ION-MARCEL CIOLACU si partidul PSD, folosind sloganuri de campanie si asocieri pozitive cu proiecte de infrastructura, avand un CMF alocat (11240017), ceea ce confirma natura sa de propaganda electorala. Materialul a fost difuzat dupa ora 18:00 pe 23.11.2024, incalcand astfel prevederile legale privind incheierea campaniei electorale.
    \item difuzarea de materiale de propaganda electorala dupa incheierea campaniei electorale pentru alegerile prezidentiale. Postarea cu ID-ul \href{https://www.facebook.com/ads/library/?id=869919955125660}{869919955125660} promoveaza direct candidatul ION-MARCEL CIOLACU si PSD, folosind realizari guvernamentale pentru a influenta votul si contine elementele specifice campaniei electorale (CMF 11240017). Postarea este promovata ca reclama platita pe Facebook si Instagram, cu un impact estimat intre 200,000 si 250,000 de afisari, dupa ora 18:00 pe 23.11.2024, incalcand astfel perioada legala de campanie electorala.
\end{enumerate}

\vspace{0.5cm}

\subsection{MEGASOFT SYSTEMS SRL si PSD}
Următoarele fapte contravenționale sunt sesizate împotriva acestei entități:

\begin{enumerate}[leftmargin=*, label=\arabic*.)]
    \item continuarea propagandei electorale dupa incheierea campaniei prezidentiale, promovand candidatul ION-MARCEL CIOLACU intr-un material publicitar platit pe Facebook (ID: \href{https://www.facebook.com/ads/library/?id=977130791118334}{977130791118334}). Materialul, publicat dupa ora 18:00 pe 23.11.2024, contine numar CMF (11240017), promoveaza explicit candidatul si partidul, si indeamna alegatorii sa urmeze "calea sigura" propusa de candidat, avand un impact semnificativ cu peste 25.000 de afisari. Mesajul combina in mod deliberat elementele campaniei parlamentare cu promovarea candidatului prezidential, incalcand astfel perioada de restrictie pentru campania prezidentiala.
\end{enumerate}

\vspace{0.5cm}

\subsection{MEGASOFT SYSTEMS SRL si Partidul Social Democrat}
Următoarele fapte contravenționale sunt sesizate împotriva acestei entități:

\begin{enumerate}[leftmargin=*, label=\arabic*.)]
    \item continuarea propagandei electorale dupa incheierea campaniei, prin promovarea unui mesaj electoral platit (ID Facebook: \href{https://www.facebook.com/ads/library/?id=575355425040710}{575355425040710}) care il mentioneaza explicit pe candidatul prezidential Marcel Ciolacu si programul PSD, folosind numar CMF (11240017), intr-un mod care vizeaza influentarea votului. Postarea continua sa fie activa si dupa ora 18:00 pe 23.11.2024, incalcand astfel perioada legala de campanie electorala. Mesajul are caracter electoral evident, solicitand explicit sustinerea ("Haideti cu noi pe calea sigura!") si promovand platforma electorala a candidatului.
\end{enumerate}

\vspace{0.5cm}

\subsection{Maramures INFO News}
Următoarele fapte contravenționale sunt sesizate împotriva acestei entități:

\begin{enumerate}[leftmargin=*, label=\arabic*.)]
    \item promovarea platita a unui mesaj de propaganda electorala pentru candidatul prezidential Elena Lasconi (USR) dupa incheierea perioadei de campanie electorala. Postarea cu ID-ul \href{https://www.facebook.com/ads/library/?id=1077556713676876}{1077556713676876} pe facebook promoveaza in mod explicit candidatul, prezentand-o drept "un presedinte care intelege cu adevarat nevoile si dorintele cetatenilor" si "un exemplu de onestitate, curaj si determinare", avand efect electoral direct prin atingerea a peste 4000 de persoane prin promovare platita, dupa ora 18:00 pe 23.11.2024. Mesajul reprezinta propaganda electorala conform Art. 36(7) din Legea 334/2006, indeplinind toate conditiile: referire directa la candidat, obiectiv electoral si adresabilitate catre publicul larg prin promovare platita.
\end{enumerate}

\vspace{0.5cm}

\subsection{Maramures INFO News si Brian Cristian (USR)}
Următoarele fapte contravenționale sunt sesizate împotriva acestei entități:

\begin{enumerate}[leftmargin=*, label=\arabic*.)]
    \item difuzarea de materiale de propaganda electorala (ID postare Facebook: \href{https://www.facebook.com/ads/library/?id=1323418585516701}{1323418585516701}) dupa incheierea campaniei electorale pentru alegerile prezidentiale. Postarea, realizata si promovata dupa ora 18:00 pe 23.11.2024, contine mesaje clare de sustinere electorala si indemn la vot pentru candidatul USR la presedintie, avand un impact estimat intre 100,001 si 500,000 de persoane prin distribuire platita pe Facebook si Instagram. Mesajul include referiri directe la alegerile prezidentiale din 24 noiembrie si caracterizari ale candidatului sustinut, depasind astfel cadrul legal al perioadei de campanie electorala.
\end{enumerate}

\vspace{0.5cm}

\subsection{Maramuresanul.ro}
Următoarele fapte contravenționale sunt sesizate împotriva acestei entități:

\begin{enumerate}[leftmargin=*, label=\arabic*.)]
    \item promovarea candidatului prezidential Marcel Ciolacu intr-o postare platita pe Facebook (ID: \href{https://www.facebook.com/ads/library/?id=1930004284174540}{1930004284174540}) dupa incheierea perioadei de campanie electorala. Postarea prezinta realizarile guvernamentale ale candidatului intr-un mod care depaseste simpla informare jurnalistica, avand un evident scop electoral prin evidentierea pozitiva a activitatii sale guvernamentale. Aceasta promovare continua dupa ora 18:00 pe 23.11.2024, incalcand astfel prevederile legale privind incheierea campaniei electorale.
\end{enumerate}

\vspace{0.5cm}

\subsection{Marian Ciofica}
Următoarele fapte contravenționale sunt sesizate împotriva acestei entități:

\begin{enumerate}[leftmargin=*, label=\arabic*.)]
    \item publicarea si promovarea unei reclame platite pe Facebook (ID: \href{https://www.facebook.com/ads/library/?id=1303880497450662}{1303880497450662}) dupa ora 18:00 pe 23.11.2024, in care face propaganda electorala explicita in favoarea candidatului Calin Georgescu, incercand sa influenteze decizia de vot a alegatorilor prin argumentarea in favoarea acestuia si impotriva altor candidati. Postarea a fost promovata cu bani (intre 0-99 RON) si a ajuns la 2000-2999 de persoane, constituind astfel o actiune deliberata de propaganda electorala in perioada in care aceasta este interzisa prin lege.
    \item distribuirea unui anunt sponsorizat pe Facebook si Instagram (ID post: \href{https://www.facebook.com/ads/library/?id=1622851758269156}{1622851758269156}) dupa ora 18:00 pe 23.11.2024, care contine indemnuri explicite de a vota candidatul prezidential Calin Georgescu in ziua alegerilor ("Eu votez Calin Georgescu, iar daca va regasiti in propunerea mea haideti va rog sa sustinem pe 24 Noiembrie 2024 candidatura dansului"). Postarea constituie propaganda electorala conform Art. 36(7) prin prezenta numarului CMF (11240022), adresarea catre public larg prin promovare platita, si obiectivul electoral explicit de influentare a votului pentru alegerile prezidentiale.
    \item difuzarea unei reclame platite pe Facebook (ID postare: \href{https://www.facebook.com/ads/library/?id=1624012695133033}{1624012695133033}) cu continut de propaganda electorala in favoarea candidatului Calin Georgescu, dupa incheierea perioadei de campanie electorala. Postarea, activa dupa ora 18:00 pe 23.11.2024, contine mesaje directe de sustinere a candidatului ("Sustin Calin Georgescu!") si are un impact semnificativ, atingand intre 9.000 si 9.999 de afisari in mai multe judete, reprezentand o clara tentativa de influentare a alegatorilor in afara perioadei legale de campanie.
\end{enumerate}

\vspace{0.5cm}

\subsection{Marketing on Line and Business Advanced Communication - M.O.B.A.C. si Forta Dreptei - Bacau}
Următoarele fapte contravenționale sunt sesizate împotriva acestei entități:

\begin{enumerate}[leftmargin=*, label=\arabic*.)]
    \item promovarea unui material de propaganda electorala (ID postare Facebook: \href{https://www.facebook.com/ads/library/?id=1566318563996555}{1566318563996555}) dupa incheierea perioadei de campanie pentru alegerile prezidentiale. Postarea promoveaza candidatul la presedintie Ludovic Orban si programul sau de guvernare, fiind distribuita dupa ora 18:00 pe 23.11.2024, incalcand astfel prevederile legale privind incheierea campaniei electorale. Materialul are caracter electoral evident, fiind marcat cu cod CMF (11240022), si a fost distribuit ca reclama platita pe Facebook cu un impact estimat intre 45,000-50,000 de afisari.
\end{enumerate}

\vspace{0.5cm}

\subsection{Maszol.ro}
Următoarele fapte contravenționale sunt sesizate împotriva acestei entități:

\begin{enumerate}[leftmargin=*, label=\arabic*.)]
    \item publicarea si promovarea unui material de propaganda electorala dupa incheierea campaniei electorale (postare ID: \href{https://www.facebook.com/ads/library/?id=1094621292192183}{1094621292192183}). Materialul contine evaluari directe ale candidatilor la presedintie (Kelemen Hunor, Elena Lasconi, George Simion), prezentandu-l intr-o lumina favorabila pe Kelemen Hunor ca "castigator al dezbaterii" si facand caracterizari negative la adresa celorlalti candidati. Postarea este sponsorizata si difuzata dupa ora 18:00 pe 23.11.2024, incalcand astfel prevederile legale privind incheierea campaniei electorale.
\end{enumerate}

\vspace{0.5cm}

\subsection{Matei Stefanescu}
Următoarele fapte contravenționale sunt sesizate împotriva acestei entități:

\begin{enumerate}[leftmargin=*, label=\arabic*.)]
    \item promovarea unui mesaj de propaganda electorala (ID Facebook: \href{https://www.facebook.com/ads/library/?id=1404871247307885}{1404871247307885}) dupa ora 18:00 pe 23.11.2024, continand indemnul explicit de a vota candidatul George Simion la alegerile prezidentiale ("Mergem pana la capat! Duminica avem sansa istorica sa votam primul Presedinte al romanilor, iar nu al sistemului! George Simion Presedinte!"). Mesajul contine numar CMF (11240014), este platit pentru distributie pe Facebook si Instagram, si are un impact potential de peste 1 milion de utilizatori, demonstrand clar intentia de propaganda electorala in afara perioadei legale permise.
\end{enumerate}

\vspace{0.5cm}

\subsection{Media Flux Dambovita SRL}
Următoarele fapte contravenționale sunt sesizate împotriva acestei entități:

\begin{enumerate}[leftmargin=*, label=\arabic*.)]
    \item difuzarea de materiale de propaganda electorala (ID Facebook: \href{https://www.facebook.com/ads/library/?id=551750367733446}{551750367733446}) dupa incheierea perioadei de campanie electorala. Materialul promoveaza explicit candidatul NICOLAE-IONEL CIUCA la functia de presedinte, contine numar CMF (11240002), face apel direct la vot pentru data de 24 noiembrie si este difuzat ca reclama platita pe Facebook si Instagram dupa ora 18:00 pe 23.11.2024, incalcand astfel prevederile legale privind incheierea campaniei electorale.
\end{enumerate}

\vspace{0.5cm}

\subsection{Media New Strategy}
Următoarele fapte contravenționale sunt sesizate împotriva acestei entități:

\begin{enumerate}[leftmargin=*, label=\arabic*.)]
    \item promovarea unui mesaj electoral platit pe Facebook (ID: \href{https://www.facebook.com/ads/library/?id=470040132292081}{470040132292081}) care indeamna explicit la votarea candidatului Elena Lasconi la alegerile prezidentiale ("Votati cu candidatul sustinut de Forta Dreptei la alegerile prezidentiale, Elena Lasconi"). Materialul este difuzat dupa ora 18:00 pe 23.11.2024, cand campania electorala pentru alegerile prezidentiale s-a incheiat. Mesajul este unul platit, cu numar CMF (11240022), care vizeaza un public larg (10,001-50,000 persoane) si are caracter explicit de propaganda electorala pentru alegerile prezidentiale.
\end{enumerate}

\vspace{0.5cm}

\subsection{Media Smart SRL}
Următoarele fapte contravenționale sunt sesizate împotriva acestei entități:

\begin{enumerate}[leftmargin=*, label=\arabic*.)]
    \item promovarea unui material de propaganda electorala (ID Facebook: \href{https://www.facebook.com/ads/library/?id=1546185322672477}{1546185322672477}) dupa ora 18:00 pe 23.11.2024, continand promovarea directa a candidatului prezidential Marcel Ciolacu si a PSD. Materialul contine numar CMF (11240017), prezinta realizari si promisiuni electorale, fiind distribuit ca reclama platita pe Facebook si Instagram, cu un reach estimat intre 100,001-500,000 persoane. Mesajul promoveaza explicit candidatul la presedintie Marcel Ciolacu, subliniind capacitatile sale de lider si realizarile in contextul aderarii la Schengen, avand un evident caracter de propaganda electorala in perioada in care aceasta este interzisa prin lege.
\end{enumerate}

\vspace{0.5cm}

\subsection{Moldova Invest}
Următoarele fapte contravenționale sunt sesizate împotriva acestei entități:

\begin{enumerate}[leftmargin=*, label=\arabic*.)]
    \item promovarea unui material electoral platit (ID: \href{https://www.facebook.com/ads/library/?id=590483866782246}{590483866782246}) care face referire directa la candidatul prezidential George Nicolae Simion dupa incheierea perioadei de campanie electorala. Materialul, publicat dupa ora 18:00 pe 23.11.2024, face conexiuni directe intre succesul prezidential al candidatului si sansele electorale parlamentare, reprezentand astfel continuarea propagandei electorale pentru alegerile prezidentiale dupa incheierea acesteia. Impactul este amplificat prin utilizarea platformei Facebook cu targetare spre un public de 100,001-500,000 persoane si promovare platita.
\end{enumerate}

\vspace{0.5cm}

\subsection{Moza Costel -Consilier Local Oradea}
Următoarele fapte contravenționale sunt sesizate împotriva acestei entități:

\begin{enumerate}[leftmargin=*, label=\arabic*.)]
    \item promovarea unui mesaj de propaganda electorala platit (ID Facebook: \href{https://www.facebook.com/ads/library/?id=492771710468418}{492771710468418}) care indeamna in mod direct la votarea candidatului George Simion in ziua alegerilor ("Pe 24 noiembrie, voteaza George Simion Presedinte!"). Materialul prezinta element de identificare specific campaniei electorale (CMF 11240014), are un impact semnificativ (35,000-39,999 impresii) si continua propaganda electorala dupa ora 18:00 pe 23.11.2024, incalcand astfel perioada de restrictie prevazuta de lege. Mesajul este distribuit ca reclama platita pe Facebook, cu un buget intre 300-399 RON.
\end{enumerate}

\vspace{0.5cm}

\subsection{Neata Eugen}
Următoarele fapte contravenționale sunt sesizate împotriva acestei entități:

\begin{enumerate}[leftmargin=*, label=\arabic*.)]
    \item promovarea unui mesaj de propaganda electorala (ID postare Facebook: \href{https://www.facebook.com/ads/library/?id=587640377105732}{587640377105732}) dupa incheierea perioadei de campanie electorala. Postarea, care include numarul CMF 11240017, promoveaza direct candidatul ION-MARCEL CIOLACU pentru functia de presedinte, contine indemnuri explicite la vot ("Duminica mergem cu totii la vot") si sugereaza direct optiunea de vot ("Nu ma indoiesc ca aceasta decizie este Marcel Ciolacu, presedinte al romanilor"). Mesajul este distribuit ca reclama platita pe Facebook si Instagram dupa ora 18:00 pe 23.11.2024, incalcand astfel prevederile legale privind perioada de campanie electorala.
    \item promovarea unui mesaj de propaganda electorala (ID Facebook: \href{https://www.facebook.com/ads/library/?id=667140289053461}{667140289053461}) dupa incheierea perioadei de campanie electorala. Postarea, efectuata dupa ora 18:00 pe 23.11.2024, contine elementele definitorii ale propagandei electorale conform Art. 36(7): identificarea clara a candidatului (Marcel Ciolacu), obiectiv electoral explicit ("Ne pregatim cu totii pentru a-l alege pe Marcel Ciolacu in functia de presedinte al tarii"), adresabilitate catre publicul larg (post sponsorizat pe Facebook), si prezenta numarului CMF 11240017. Mesajul reprezinta o continuare a propagandei electorale dupa incheierea campaniei, fiind o incalcare directa a prevederilor legale.
\end{enumerate}

\vspace{0.5cm}

\subsection{Oeconomus}
Următoarele fapte contravenționale sunt sesizate împotriva acestei entități:

\begin{enumerate}[leftmargin=*, label=\arabic*.)]
    \item difuzarea unui mesaj de propaganda electorala platit (ID postare Facebook: \href{https://www.facebook.com/ads/library/?id=1118160340017424}{1118160340017424}) dupa incheierea perioadei de campanie electorala pentru alegerile prezidentiale. Postarea promoveaza explicit candidatul UDMR Kelemen Hunor si partidul UDMR, fiind difuzata dupa ora 18:00 pe 23.11.2024, continand mesaje care influenteaza in mod direct procesul electoral prezidential. Mesajul este difuzat contra cost, targetand populatia vorbitoare de limba maghiara, cu un impact estimat intre 100,001 si 500,000 de persoane.
\end{enumerate}

\vspace{0.5cm}

\subsection{Organizatia Declic}
Următoarele fapte contravenționale sunt sesizate împotriva acestei entități:

\begin{enumerate}[leftmargin=*, label=\arabic*.)]
    \item distribuirea unui material de propaganda electorala dupa incheierea perioadei de campanie, constand intr-o postare sponsorizata pe Instagram (ID: \href{https://www.facebook.com/ads/library/?id=398079753384999}{398079753384999}) ce vizeaza in mod direct si negativ candidatul prezidential Marcel Ciolacu. Postarea, difuzata dupa ora 18:00 pe 23.11.2024, foloseste limbaj critic si emotiv pentru a influenta decizia alegatorilor, depasind limitele informarii obiective si constituind propaganda electorala prin prezentarea negativa a candidatului si a actiunilor sale politice.
    \item promovarea unui mesaj platit pe Facebook (ID: \href{https://www.facebook.com/ads/library/?id=595688596237600}{595688596237600}) care aduce critici directe candidatului prezidential Marcel Ciolacu, dupa ora 18:00 pe 23.11.2024. Mesajul, desi prezinta aparent o critica pe tema Codului Silvic, este propaganda electorala negativa in perioada interzisa, fiind un continut platit care il vizeaza direct pe candidatul la presedintie si partidul sau, cu potential impact asupra deciziei de vot a cetatenilor.
\end{enumerate}

\vspace{0.5cm}

\subsection{Organizatia PNL Santana}
Următoarele fapte contravenționale sunt sesizate împotriva acestei entități:

\begin{enumerate}[leftmargin=*, label=\arabic*.)]
    \item publicarea si promovarea unui mesaj de propaganda electorala (ID postare Facebook: \href{https://www.facebook.com/ads/library/?id=599969435814755}{599969435814755}) dupa incheierea perioadei de campanie electorala pentru alegerile prezidentiale. Postarea contine indemnuri explicite de vot pentru candidatul Nicolae Ciuca la alegerile prezidentiale ("In 24 noiembrie 2024 pozitia 4"), fiind promovata dupa ora 18:00 pe 23.11.2024. Mesajul include cod mandatar financiar (11240002), demonstrand caracterul sau oficial de propaganda electorala, si foloseste realizari administrative pentru a influenta votul in alegerile prezidentiale.
\end{enumerate}

\vspace{0.5cm}

\subsection{PNL Arges}
Următoarele fapte contravenționale sunt sesizate împotriva acestei entități:

\begin{enumerate}[leftmargin=*, label=\arabic*.)]
    \item promovarea unui mesaj electoral platit (ID postare Facebook: \href{https://www.facebook.com/ads/library/?id=1092751075297851}{1092751075297851}) dupa incheierea perioadei de campanie electorala. Postarea, publicata si promovata dupa ora 18:00 pe 23.11.2024, contine promovare directa a candidatului Nicolae Ciuca, avand numar CMF 11240002, cu argumente explicite pentru sustinerea acestuia in alegerile prezidentiale. Mesajul a fost difuzat contra cost pe platformele Facebook si Instagram, cu un buget intre 200-299 RON si o audienta estimata intre 100,001-500,000 persoane, reprezentand astfel o incalcare clara a prevederilor legale privind incheierea campaniei electorale.
    \item publicarea si promovarea unei reclame electorale (ID Facebook: \href{https://www.facebook.com/ads/library/?id=597485686033312}{597485686033312}) dupa incheierea perioadei de campanie electorala. Postarea contine propaganda electorala negativa la adresa candidatului ION-MARCEL CIOLACU, avand un indemn explicit de a nu-l vota ("Trimite acasa Ciolacul mincinos!"), fiind promovata ca reclama platita pe Facebook si Instagram, cu un impact semnificativ (peste 30.000 de afisari). Postarea include numar CMF (11240002), confirmand natura sa de material electoral. Aceasta activitate a continuat dupa ora 18:00 pe 23.11.2024, incalcand astfel prevederile legale privind incheierea campaniei electorale.
    \item difuzarea unui mesaj de propaganda electorala (ID postare Facebook: \href{https://www.facebook.com/ads/library/?id=607248631642642}{607248631642642}) dupa incheierea perioadei de campanie electorala. Materialul promovat vizeaza in mod direct candidatul ION-MARCEL CIOLACU, contine numar CMF (11240002), are obiectiv electoral explicit prin indemnul "Trimite acasa Ciolacul mincinos!" si este difuzat ca reclama platita pe Instagram dupa ora 18:00 pe 23.11.2024, incalcand astfel prevederile legale privind incetarea propagandei electorale.
\end{enumerate}

\vspace{0.5cm}

\subsection{PNL Calarasi si Datablitz SRL}
Următoarele fapte contravenționale sunt sesizate împotriva acestei entități:

\begin{enumerate}[leftmargin=*, label=\arabic*.)]
    \item continuarea propagandei electorale pentru candidatul prezidential Nicolae Ciuca dupa incheierea perioadei legale de campanie. Postarea cu ID-ul \href{https://www.facebook.com/ads/library/?id=1259843452016145}{1259843452016145} promoveaza explicit pozitia candidatului pe buletinul de vot pentru alegerile prezidentiale ("Nicolae Ciuca, pozitia 4 pe buletinul de vot pe 24 noiembrie"), constituind propaganda electorala dupa ora 18:00 pe 23.11.2024. Postarea este sponsorizata, avand un CMF valid (11240002), si a ajuns la un numar semnificativ de persoane (35.000-40.000 impresii).
\end{enumerate}

\vspace{0.5cm}

\subsection{PNL DOLJ}
Următoarele fapte contravenționale sunt sesizate împotriva acestei entități:

\begin{enumerate}[leftmargin=*, label=\arabic*.)]
    \item promovarea candidatului la presedintie Nicolae Ciuca intr-o postare platita pe Facebook (ID: \href{https://www.facebook.com/ads/library/?id=2356166724718218}{2356166724718218}) dupa incheierea perioadei de campanie electorala prezidentiala. Postarea, publicata dupa ora 18:00 pe 23.11.2024, il prezinta pe Nicolae Ciuca intr-o lumina pozitiva ca "model nou de om politic" si "model nou de om de stat", constituind astfel propaganda electorala in afara perioadei legale permise. Efectul electoral este evident prin asocierea pozitiva si prezentarea candidatului ca model de urmat, intr-un context electoral activ.
\end{enumerate}

\vspace{0.5cm}

\subsection{PNL DOLJ si Oltenia STAR}
Următoarele fapte contravenționale sunt sesizate împotriva acestei entități:

\begin{enumerate}[leftmargin=*, label=\arabic*.)]
    \item difuzarea de materiale de propaganda electorala (ID Facebook: \href{https://www.facebook.com/ads/library/?id=903805185049021}{903805185049021}) dupa incheierea perioadei de campanie electorala pentru alegerile prezidentiale. Postarea contine un indemn explicit de a vota candidatul Nicolae Ciuca la presedintie, fiind promovata ca reclama platita pe Facebook si Instagram dupa ora 18:00 pe 23.11.2024. Materialul include cod AEP (11240002), confirmand natura sa de propaganda electorala, si are un impact estimat intre 8000-8999 de afisari, demonstrand intentia clara de influentare a votului in perioada interzisa de lege.
\end{enumerate}

\vspace{0.5cm}

\subsection{PNL Galati}
Următoarele fapte contravenționale sunt sesizate împotriva acestei entități:

\begin{enumerate}[leftmargin=*, label=\arabic*.)]
    \item publicarea si promovarea unei reclame pe Facebook (ID: \href{https://www.facebook.com/ads/library/?id=539157538894856}{539157538894856}) dupa incheierea perioadei de campanie electorala. Postarea contine mesaje electorale explicite, face referire directa la candidati prezidentiali (Ciolacu, Simion), include indemnuri la vot ("iesi la vot"), si este marcata cu cod CMF (11240002), confirmand natura sa de propaganda electorala. Reclama este programata sa ruleze si dupa ora 18:00 pe 23.11.2024, incalcand astfel explicit prevederile legale privind incheierea campaniei electorale.
    \item promovarea unui mesaj de propaganda electorala pentru candidatul Nicolae-Ionel Ciuca dupa incheierea perioadei de campanie electorala, prin intermediul unei reclame platite pe Facebook (ID: \href{https://www.facebook.com/ads/library/?id=542353765276315}{542353765276315}). Mesajul contine indemnuri directe la vot ("Pe \#24Noiembrie votati"), promoveaza explicit candidatul ("viitorul presedinte al Romaniei") si foloseste numar CMF (11240002), fiind difuzat dupa ora 18:00 pe 23.11.2024, incalcand astfel prevederile legale privind incheierea campaniei electorale.
\end{enumerate}

\vspace{0.5cm}

\subsection{PNL IASI}
Următoarele fapte contravenționale sunt sesizate împotriva acestei entități:

\begin{enumerate}[leftmargin=*, label=\arabic*.)]
    \item difuzarea de materiale de propaganda electorala dupa incheierea campaniei electorale pentru alegerile prezidentiale, respectiv dupa ora 18:00 pe 23.11.2024. Postarea cu ID-ul \href{https://www.facebook.com/ads/library/?id=591241616682140}{591241616682140} contine un indemn explicit la vot pentru candidatul Nicolae Ciuca ("VOTAM Nicolae Ciuca Presedinte!"), prezinta mesaje de campanie electorala si este promovata ca reclama platita pe platformele Facebook si Instagram, avand un impact estimat intre 2000-2999 de persoane. Materialul reprezinta in mod clar propaganda electorala conform Art. 36(7) din Legea 334/2006, indeplinind toate conditiile: referire directa la candidat, obiectiv electoral explicit si adresabilitate catre publicul larg.
\end{enumerate}

\vspace{0.5cm}

\subsection{PNL IASI, prin reprezentantii sai legali}
Următoarele fapte contravenționale sunt sesizate împotriva acestei entități:

\begin{enumerate}[leftmargin=*, label=\arabic*.)]
    \item difuzarea unui mesaj de propaganda electorala dupa incheierea campaniei electorale pentru alegerile prezidentiale, intr-o postare platita pe Facebook (ID: \href{https://www.facebook.com/ads/library/?id=586462180413319}{586462180413319}). Postarea contine un indemn explicit de a vota candidatul Nicolae Ionel Ciuca la alegerile prezidentiale ("pe 24 noiembrie sa-l votam pe Nicolae Ionel Ciuca presedinte"), fiind difuzata dupa ora 18:00 pe 23.11.2024, cu un impact semnificativ demonstrat prin numarul de afisari (60.000-70.000). Materialul reprezinta propaganda electorala conform Art. 36(7), indeplinind toate conditiile: referire directa la candidat, obiectiv electoral explicit si adresabilitate catre publicul larg prin natura sa de continut sponsorizat.
\end{enumerate}

\vspace{0.5cm}

\subsection{PNL Maramures}
Următoarele fapte contravenționale sunt sesizate împotriva acestei entități:

\begin{enumerate}[leftmargin=*, label=\arabic*.)]
    \item promovarea unui mesaj de propaganda electorala pe Facebook (ID postare: \href{https://www.facebook.com/ads/library/?id=1080939136916586}{1080939136916586}) dupa incheierea perioadei de campanie. Postarea contine un indemn explicit de vot pentru candidatul Nicolae-Ionel Ciuca, foloseste numar CMF (11240002), si are caracter de propaganda electorala fiind distribuita dupa ora 18:00 pe 23.11.2024. Mesajul a fost promovat ca reclama platita, cu un impact estimat intre 10.000 si 14.999 de afisari, reprezentand o incercare clara de influentare a votului in perioada de interdictie.
    \item continuarea propagandei electorale dupa incheierea campaniei electorale pentru alegerile prezidentiale, prin postarea cu ID \href{https://www.facebook.com/ads/library/?id=3326577030806855}{3326577030806855} pe facebook. Postarea, sponsorizata si distribuita dupa ora 18:00 pe 23.11.2024, promoveaza explicit candidatul Nicolae-Ionel Ciuca la functia de presedinte, contine numar CMF (11240002), si incearca sa influenteze alegatorii prin mesaje despre "experienta si viziunea sa vor aduce stabilitate si prosperitate Romaniei". Postarea a avut un impact semnificativ, atingand intre 20.000 si 24.999 de afisari.
    \item difuzarea unui mesaj de propaganda electorala dupa incheierea campaniei electorale prezidentiale, folosind o postare platita pe Facebook (ID: \href{https://www.facebook.com/ads/library/?id=458881507235742}{458881507235742}) care promoveaza explicit candidatura lui Nicolae Ionel Ciuca la presedintia Romaniei. Postarea, difuzata dupa ora 18:00 pe 23.11.2024, contine elemente clare de propaganda electorala (CMF:11240002), are caracter electoral explicit si a fost distribuita catre un public larg (15.000-19.999 impresii), incalcand astfel prevederile legale privind incheierea campaniei electorale.
\end{enumerate}

\vspace{0.5cm}

\subsection{PNL Maramures, prin reprezentantii sai}
Următoarele fapte contravenționale sunt sesizate împotriva acestei entități:

\begin{enumerate}[leftmargin=*, label=\arabic*.)]
    \item continuarea propagandei electorale dupa incheierea campaniei electorale pentru alegerile prezidentiale, manifestata prin postarea platita pe Facebook cu ID \href{https://www.facebook.com/ads/library/?id=1628682881410360}{1628682881410360}, care indeamna explicit la votarea candidatului Nicolae Ciuca ("Pe 24 noiembrie, votam impreuna Nicolae Ciuca!"). Postarea este activa si distribuita dupa ora 18:00 pe 23.11.2024, incalcand astfel perioada legala de campanie electorala. Materialul are caracter explicit de propaganda electorala, fiind marcat cu cod CMF 11240002, tintind un public intre 50,001-100,000 de persoane, cu un buget de promovare intre 100-199 RON.
\end{enumerate}

\vspace{0.5cm}

\subsection{PNL Mures}
Următoarele fapte contravenționale sunt sesizate împotriva acestei entități:

\begin{enumerate}[leftmargin=*, label=\arabic*.)]
    \item distribuirea unui material de propaganda electorala (ID postare Facebook: \href{https://www.facebook.com/ads/library/?id=489373544122065}{489373544122065}) dupa incheierea perioadei de campanie electorala prezidentiala. Postarea promoveaza explicit candidatul la presedintie Nicolae Ciuca, declarandu-l "alegerea rationala pentru viitorul Romaniei", avand un impact direct asupra alegerilor prezidentiale. Materialul este distribuit ca reclama platita pe Facebook si Instagram, cu o audienta estimata intre 100,001-500,000 persoane, dupa ora 18:00 pe 23.11.2024, incalcand astfel prevederile legale privind incheierea campaniei electorale.
\end{enumerate}

\vspace{0.5cm}

\subsection{PNL Olt}
Următoarele fapte contravenționale sunt sesizate împotriva acestei entități:

\begin{enumerate}[leftmargin=*, label=\arabic*.)]
    \item difuzarea unui mesaj de propaganda electorala pentru candidatul prezidential Nicolae Ciuca dupa incheierea perioadei de campanie electorala. Postarea cu ID-ul \href{https://www.facebook.com/ads/library/?id=2483117645229321}{2483117645229321} contine un indemn direct de vot ("Votati Nicolae Ciuca, Presedinte al Romaniei!"), are numar CMF (11240002), si a fost promovata ca reclama platita pe Facebook si Instagram, cu un impact estimat intre 125.000 si 150.000 de afisari, dupa ora 18:00 pe 23.11.2024. Postarea reprezinta o continuare a propagandei electorale dupa incheierea acesteia, incalcand astfel prevederile legale.
    \item promovarea unui mesaj electoral explicit ("Votati Nicolae Ciuca, Presedinte al Romaniei!") dupa incheierea perioadei de campanie electorala pentru alegerile prezidentiale. Postarea cu ID-ul \href{https://www.facebook.com/ads/library/?id=479625595132799}{479625595132799} pe facebook reprezinta propaganda electorala conform Art. 36(7), avand CMF (11240002), fiind distribuita ca reclama platita cu impact intre 100.000-124.999 impresii, dupa ora 18:00 pe 23.11.2024. Mesajul reprezinta in mod clar o incercare de influentare a votului pentru candidatul prezidential Nicolae Ciuca, intr-o perioada in care propaganda electorala pentru alegerile prezidentiale este interzisa prin lege.
\end{enumerate}

\vspace{0.5cm}

\subsection{PNL Pascani}
Următoarele fapte contravenționale sunt sesizate împotriva acestei entități:

\begin{enumerate}[leftmargin=*, label=\arabic*.)]
    \item difuzarea unui mesaj de propaganda electorala dupa incheierea campaniei electorale pentru alegerile prezidentiale, intr-o postare sponsorizata pe Facebook (ID: \href{https://www.facebook.com/ads/library/?id=1292348225285327}{1292348225285327}) care promoveaza explicit candidatul Nicolae-Ionel Ciuca la presedintie si indeamna direct la vot pentru acesta ("sa votam pe 24 noiembrie pe Nicolae Nicolae Ionel Ciuca presedinte"). Postarea este activa si dupa ora 18:00 pe 23.11.2024, contine numar CMF (11240002), si a avut un impact semnificativ, atingand intre 40.000 si 45.000 de afisari.
\end{enumerate}

\vspace{0.5cm}

\subsection{PNL SALAJ}
Următoarele fapte contravenționale sunt sesizate împotriva acestei entități:

\begin{enumerate}[leftmargin=*, label=\arabic*.)]
    \item promovarea candidatului la presedintie Nicolae-Ionel Ciuca intr-o postare platita pe Facebook si Instagram (ID postare: \href{https://www.facebook.com/ads/library/?id=1287775152348244}{1287775152348244}) dupa incheierea perioadei de campanie electorala prezidentiala. Postarea, publicata si promovata dupa ora 18:00 pe 23.11.2024, contine in mod explicit indemnul de sustinere a candidatului la presedintie si reprezinta propaganda electorala conform Art. 36(7), avand cod AEP 11240002, obiectiv electoral explicit, si fiind distribuita catre un public larg estimat intre 100,001 si 500,000 de persoane.
    \item difuzarea de materiale de propaganda electorala pentru candidatul la presedintie Nicolae-Ionel Ciuca dupa incheierea perioadei de campanie electorala. Materialul, avand ID-ul postarii pe facebook \href{https://www.facebook.com/ads/library/?id=510922691951131}{510922691951131}, constituie propaganda electorala conform Art. 36(7) prin: (a) referirea directa la candidatul Nicolae-Ionel Ciuca, (b) utilizarea in afara perioadei permise, (c) obiectivul electoral explicit de influentare a votului prin indemnuri directe de a-l vota pe candidat, si (d) depasirea limitelor activitatii jurnalistice. Postarea a fost promovata dupa ora 18:00 pe 23.11.2024, avand un impact semnificativ prin atingerea a 35,000-39,999 de persoane.
\end{enumerate}

\vspace{0.5cm}

\subsection{PNL Tecuci}
Următoarele fapte contravenționale sunt sesizate împotriva acestei entități:

\begin{enumerate}[leftmargin=*, label=\arabic*.)]
    \item publicarea si promovarea unei reclame electorale (ID Facebook: \href{https://www.facebook.com/ads/library/?id=972415984698063}{972415984698063}) dupa incheierea perioadei de campanie electorala. Postarea contine propaganda electorala explicita, avand numar CMF 11240002, facand referiri directe la candidatii prezidentiali Simion si Ciolacu intr-un mod negativ, si indeamna explicit la vot intr-o anumita directie. Reclama este inca activa si a fost promovata dupa ora 18:00 pe 23.11.2024, incalcand astfel prevederile legale privind incheierea campaniei electorale.
\end{enumerate}

\vspace{0.5cm}

\subsection{PNL Valcea}
Următoarele fapte contravenționale sunt sesizate împotriva acestei entități:

\begin{enumerate}[leftmargin=*, label=\arabic*.)]
    \item promovarea unui mesaj de propaganda electorala (ID Facebook: \href{https://www.facebook.com/ads/library/?id=486201574473068}{486201574473068}) dupa incheierea perioadei de campanie. Postarea, publicata ca reclama platita pe Facebook si Instagram, promoveaza explicit candidatul Nicolae Ciuca pentru functia de presedinte, solicitand in mod direct votul cetatenilor pentru data de 24 noiembrie, continand si cod AEP (11240002). Mesajul a fost difuzat dupa ora 18:00 pe 23.11.2024, incalcand astfel prevederile legale privind incetarea campaniei electorale.
\end{enumerate}

\vspace{0.5cm}

\subsection{POD TV}
Următoarele fapte contravenționale sunt sesizate împotriva acestei entități:

\begin{enumerate}[leftmargin=*, label=\arabic*.)]
    \item publicarea si promovarea unui material de propaganda electorala (ID Facebook: \href{https://www.facebook.com/ads/library/?id=854887666859658}{854887666859658}) dupa ora 18:00 pe 23.11.2024. Materialul contine cod CMF11240027, vizeaza in mod direct candidatul George Nicolae Simion, foloseste limbaj emotional si acuzatii pentru a influenta decizia de vot, si a fost distribuit ca reclama platita pe Facebook cu un buget semnificativ (600-699 RON) si un impact mare (60,000-70,000 impresii). Materialul depaseste limitele activitatii jurnalistice obiective si reprezinta propaganda electorala activa dupa incheierea perioadei legale de campanie.
\end{enumerate}

\vspace{0.5cm}

\subsection{PS News}
Următoarele fapte contravenționale sunt sesizate împotriva acestei entități:

\begin{enumerate}[leftmargin=*, label=\arabic*.)]
    \item promovarea unei reclame platite pe Facebook (ID: \href{https://www.facebook.com/ads/library/?id=1089620366052762}{1089620366052762}) care indeamna in mod explicit la votarea candidatului Nicolae Ciuca, dupa incheierea perioadei de campanie electorala. Postarea, activa dupa ora 18:00 pe 23.11.2024, constituie propaganda electorala prin faptul ca: (1) promoveaza direct un candidat la presedintie, (2) include un indemn explicit la vot pentru respectivul candidat, (3) este distribuita ca continut platit catre un public larg estimat la peste 1 milion de persoane, incalcand astfel perioada de liniste electorala prevazuta de lege.
    \item promovarea unui articol cu continut electoral dupa incheierea campaniei electorale, respectiv dupa ora 18:00 pe 23.11.2024. Postarea platita cu ID-ul \href{https://www.facebook.com/ads/library/?id=1730750311038529}{1730750311038529} promoveaza activ un sondaj electoral ce prezinta sansele candidatilor Nicolae Ciuca si George Simion pentru turul doi al alegerilor prezidentiale, avand potentialul de a influenta comportamentul electoral al alegatorilor in ziua votului. Audienta estimata depaseste 1 milion de persoane, demonstrand impactul semnificativ al acestei incalcari.
    \item promovarea platita a unui material ce prezinta in mod pozitiv realizarile candidatului prezidential Ion Marcel Ciolacu dupa incheierea campaniei electorale. Materialul, cu ID-ul postarii pe facebook \href{https://www.facebook.com/ads/library/?id=1786970205398033}{1786970205398033}, a fost promovat dupa ora 18:00 pe 23.11.2024, prezentand realizarile candidatului intr-un mod care poate influenta preferintele electorale, acest lucru constituind propaganda electorala in perioada restrictionata. Efectul electoral este evident prin asocierea pozitiva a candidatului cu succese in domeniul reindustrializarii, depasind sfera simplei informari jurnalistice prin promovarea platita si targetata.
    \item promovarea unui material de propaganda electorala pentru candidatul NICOLAE-IONEL CIUCA dupa incheierea perioadei de campanie. Postarea cu ID-ul \href{https://www.facebook.com/ads/library/?id=1882700792138874}{1882700792138874} pe facebook reprezinta propaganda electorala prin promovarea sustinerii internationale a candidatului, fiind sponsorizata si distribuita dupa ora 18:00 pe 23.11.2024, cu potential de a influenta decizia alegatorilor. Materialul depaseste limitele activitatii jurnalistice obiective, avand caracter promotional evident in favoarea candidatului.
    \item promovarea unui mesaj electoral platit care il vizeaza direct pe candidatul GEORGE-NICOLAE SIMION, comparandu-l negativ cu Alexandr Stoianoglo, cu scopul clar de a influenta negativ optiunea alegatorilor. Postarea cu ID-ul \href{https://www.facebook.com/ads/library/?id=444559491737765}{444559491737765} reprezinta propaganda electorala activa dupa data de 23.11.2024, ora 18:00, fiind o reclama platita cu un reach estimat de peste 1 milion de persoane, care continua sa ruleze in perioada de restrictie electorala.
    \item promovarea unui material publicitar platit pe Facebook (ID: \href{https://www.facebook.com/ads/library/?id=916375703921699}{916375703921699}) cu continut electoral in favoarea candidatului ION-MARCEL CIOLACU, dupa incheierea perioadei de campanie electorala. Materialul, difuzat dupa ora 18:00 pe 23.11.2024, prezinta o analiza favorabila a campaniei candidatului ("campanie structurata, dialog extern consistent"), constituind astfel propaganda electorala in perioada interzisa. Caracterul platit al materialului si audienta tinta estimata de peste 1 milion de persoane demonstreaza intentia clara de influentare a procesului electoral.
\end{enumerate}

\vspace{0.5cm}

\subsection{PS News si Partidul National Liberal}
Următoarele fapte contravenționale sunt sesizate împotriva acestei entități:

\begin{enumerate}[leftmargin=*, label=\arabic*.)]
    \item difuzarea de materiale de propaganda electorala dupa incheierea perioadei de campanie, constand in promovarea candidatului Nicolae Ciuca la functia de presedinte prin intermediul unei reclame platite pe Facebook (ID: \href{https://www.facebook.com/ads/library/?id=532419989618931}{532419989618931}). Materialul prezinta in mod explicit sustinerea candidatului pentru functia de presedinte al Romaniei, depasind caracterul informativ-jurnalistic si avand scop electoral evident. Publicarea si promovarea acestui material dupa ora 18:00 pe 23.11.2024 reprezinta o incalcare clara a prevederilor legale privind incheierea campaniei electorale.
\end{enumerate}

\vspace{0.5cm}

\subsection{PSD Arges}
Următoarele fapte contravenționale sunt sesizate împotriva acestei entități:

\begin{enumerate}[leftmargin=*, label=\arabic*.)]
    \item continuarea propagandei electorale dupa incheierea campaniei pentru alegerile prezidentiale, prin promovarea unui mesaj electoral platit (ID postare Facebook: \href{https://www.facebook.com/ads/library/?id=1129170215457617}{1129170215457617}) ce il prezinta intr-o lumina pozitiva pe candidatul prezidential Marcel Ciolacu, asociindu-l cu cresterile de pensii si reformele guvernamentale. Postarea este activa si dupa ora 18:00 pe 23.11.2024, incalcand astfel perioada de restrictie a propagandei electorale pentru alegerile prezidentiale.
    \item continuarea propagandei electorale dupa incheierea perioadei legale de campanie, prin promovarea si evidentierea realizarilor candidatului prezidential ION-MARCEL CIOLACU intr-o postare platita pe Facebook (ID: \href{https://www.facebook.com/ads/library/?id=1583680075585052}{1583680075585052}). Postarea contine numar CMF (11240017), vizeaza un public larg (100,001-500,000 persoane) si continua sa fie activa si dupa ora 18:00 pe 23.11.2024, incalcand astfel prevederile legale privind incheierea campaniei electorale.
\end{enumerate}

\vspace{0.5cm}

\subsection{PSD BN}
Următoarele fapte contravenționale sunt sesizate împotriva acestei entități:

\begin{enumerate}[leftmargin=*, label=\arabic*.)]
    \item continuarea propagandei electorale dupa incheierea campaniei electorale pentru alegerile prezidentiale, manifestata prin publicarea si mentinerea activa a unei reclame platite pe Facebook (ID: \href{https://www.facebook.com/ads/library/?id=2310944462575110}{2310944462575110}) care indeamna direct la votarea candidatului ION-MARCEL CIOLACU si a partidului PSD. Postarea include numere oficiale CMF (31240010 si 11240017), demonstrand natura sa de propaganda electorala, si a continuat sa fie difuzata dupa ora 18:00 pe 23.11.2024, incalcand astfel perioada legala de campanie electorala. Impactul acestei incalcari este amplificat de audienta estimata intre 100,001 si 500,000 de persoane si numarul de afisari intre 10,000 si 14,999.
\end{enumerate}

\vspace{0.5cm}

\subsection{PSD Brasov}
Următoarele fapte contravenționale sunt sesizate împotriva acestei entități:

\begin{enumerate}[leftmargin=*, label=\arabic*.)]
    \item promovarea electorala a candidatului prezidential Ion-Marcel Ciolacu dupa incheierea perioadei de campanie electorala, prin intermediul unei reclame platite pe Facebook (ID: \href{https://www.facebook.com/ads/library/?id=1507022259999923}{1507022259999923}) care promoveaza direct candidatul si viziunea sa, avand un impact semnificativ (125.000-150.000 impresii). Postarea, activa dupa ora 18:00 pe 23.11.2024, reprezinta o continuare a propagandei electorale dupa incheierea acesteia, cu un efect electoral direct in favoarea candidatului. Materialul contine numar CMF (11240017) si este finantat de organizatia politica, demonstrand natura sa de propaganda electorala.
    \item difuzarea unei reclame platite pe Facebook si Instagram (ID: \href{https://www.facebook.com/ads/library/?id=1744268173063704}{1744268173063704}) dupa ora 18:00 pe 23.11.2024, care promoveaza candidatul prezidential Marcel Ciolacu. Materialul contine numar CMF (11240017), prezinta explicit candidatul si promisiunile sale pentru regiunea Brasov, are caracter electoral evident si a fost distribuit catre un public larg (80,000-90,000 impresii), fiind o continuare clara a propagandei electorale dupa incheierea perioadei legale de campanie.
    \item difuzarea unui material de propaganda electorala (ID postare Facebook: \href{https://www.facebook.com/ads/library/?id=2065759973878657}{2065759973878657}) dupa incheierea perioadei de campanie electorala. Materialul promovat prezinta candidatul ION-MARCEL CIOLACU intr-o lumina favorabila, evidentiind realizarile sale ca premier si viziunea pentru dezvoltarea Brasovului, avand un evident caracter de propaganda electorala demonstrat prin prezenta numarului CMF 11240017, atingerea unui public larg prin promovare platita (90.000-99.999 impresii), si depasirea limitelor activitatii jurnalistice de informare. Aceasta actiune a continuat si dupa ora 18:00 pe 23.11.2024, incalcand astfel prevederile legale privind incheierea campaniei electorale.
    \item publicarea si mentinerea activa a unei reclame electorale (ID: \href{https://www.facebook.com/ads/library/?id=492335780485449}{492335780485449}) dupa incheierea perioadei de campanie electorala. Postarea contine referiri directe la candidatul prezidential Marcel Ciolacu, foloseste numar CMF (11240017), are obiectiv electoral explicit prin indemnul la vot pentru data de 24 noiembrie, si a fost promovata ca reclama platita pe Facebook si Instagram, cu un impact estimat intre 100.000 si 124.999 de afisari, dupa ora 18:00 pe 23.11.2024. Materialul reprezinta propaganda electorala conform Art. 36(7) din Legea 334/2006, indeplinind toate criteriile prevazute de lege.
    \item continuarea propagandei electorale dupa incheierea perioadei legale de campanie, prin postarea cu ID \href{https://www.facebook.com/ads/library/?id=550408210951641}{550408210951641} pe Facebook. Postarea, realizata ca material publicitar platit (800-899 RON), promoveaza candidatul ION-MARCEL CIOLACU si solicita in mod explicit votul pentru acesta in alegerile prezidentiale, atingand un public intre 100.000-124.999 persoane dupa ora 18:00 pe 23.11.2024. Materialul contine numar CMF (11240017), confirmand natura sa de propaganda electorala.
    \item difuzarea de materiale de propaganda electorala (IDpostare Facebook: \href{https://www.facebook.com/ads/library/?id=551155220867298}{551155220867298}) dupa incheierea perioadei de campanie electorala. Materialul promovat prezinta in mod direct candidatul ION-MARCEL CIOLACU, contine mesaje electorale explicite si promisiuni de campanie, fiind distribuit ca reclama platita pe Facebook si Instagram dupa ora 18:00 pe 23.11.2024. Materialul include numar CMF (11240017), confirmand natura sa de propaganda electorala, si promoveaza explicit candidatul la presedintie si programul sau electoral, incalcand astfel prevederile legale privind incheierea campaniei electorale.
    \item continuarea propagandei electorale dupa incheierea acesteia, promovand in mod activ candidatul ION-MARCEL CIOLACU prin intermediul unei reclame platite pe Facebook (ID: \href{https://www.facebook.com/ads/library/?id=567368295837608}{567368295837608}) care prezinta viziunea si promisiunile electorale ale candidatului pentru functia de presedinte. Materialul, care include numarul de marketing electoral CMF 11240017, promoveaza in mod direct platforma prezidentiala a candidatului dupa ora 18:00 pe 23.11.2024, constituind astfel o incalcare clara a prevederilor legale privind incheierea campaniei electorale.
    \item continuarea propagandei electorale dupa incheierea campaniei electorale, manifestata prin promovarea platita a unui material electoral (ID Facebook: \href{https://www.facebook.com/ads/library/?id=992563739298912}{992563739298912}) ce il prezinta in mod favorabil pe candidatul prezidential Marcel Ciolacu. Materialul, activ si dupa ora 18:00 pe 23.11.2024, foloseste numar CMF (11240017), are caracter electoral evident si este distribuit catre un public larg (500,001-1,000,000 persoane) prin platformele Facebook si Instagram, cu un buget semnificativ (600-699 RON). Continutul promoveaza explicit imaginea si realizarile candidatului in perioada interzisa legal pentru propaganda electorala.
\end{enumerate}

\vspace{0.5cm}

\subsection{PSD Buzau}
Următoarele fapte contravenționale sunt sesizate împotriva acestei entități:

\begin{enumerate}[leftmargin=*, label=\arabic*.)]
    \item promovarea continua a candidatului prezidential Marcel Ciolacu prin intermediul unei reclame platite pe Facebook (ID: \href{https://www.facebook.com/ads/library/?id=541801878737165}{541801878737165}) dupa incheierea perioadei de campanie electorala. Postarea, activa dupa ora 18:00 pe 23.11.2024, include elemente clare de propaganda electorala (CMF 11240017), promoveaza explicit candidatul la presedintie si foloseste sloganuri de campanie, atingand un public tinta de peste 40,000 de persoane prin investitii publicitare semnificative (1000-1499 RON).
\end{enumerate}

\vspace{0.5cm}

\subsection{PSD Dolj}
Următoarele fapte contravenționale sunt sesizate împotriva acestei entități:

\begin{enumerate}[leftmargin=*, label=\arabic*.)]
    \item difuzarea unui material de propaganda electorala (ID postare Facebook: \href{https://www.facebook.com/ads/library/?id=1068912664984786}{1068912664984786}) dupa incheierea perioadei de campanie electorala. Materialul promoveaza in mod direct candidatul prezidential Ion Marcel Ciolacu, evidentiind realizarile acestuia ca prim-ministru in contextul aderarii Romaniei la spatiul Schengen. Postarea include numar CMF (11240017), este sponsorizata pentru a ajunge la un public larg (100,001-500,000 persoane) si are un evident caracter electoral, fiind difuzata dupa ora 18:00 pe 23.11.2024, incalcand astfel prevederile legale privind incetarea campaniei electorale.
    \item promovarea unui material de propaganda electorala pentru candidatul ION-MARCEL CIOLACU dupa incheierea perioadei de campanie electorala. Materialul, cu ID-ul postarii pe facebook \href{https://www.facebook.com/ads/library/?id=1076877734172969}{1076877734172969}, a fost promovat ca reclama platita pe Facebook si Instagram, cu un buget intre 100-199 RON si o audienta estimata intre 100,001-500,000 persoane, reprezentand propaganda electorala activa dupa ora 18:00 pe 23.11.2024. Postarea contine mesajul "De ce Marcel Ciolacu?" si hashtag-urile electorale \#MarcelCiolacu \#Echilibru \#Stabilitate, fiind in mod clar material de propaganda electorala conform Art. 36(7).
    \item publicarea si promovarea unei reclame pe Facebook (ID: \href{https://www.facebook.com/ads/library/?id=1086980712697748}{1086980712697748}) dupa incheierea perioadei de propaganda electorala. Postarea, difuzata dupa ora 18:00 pe 23.11.2024, reprezinta propaganda electorala negativa indreptata impotriva candidatului Nicolae-Ionel Ciuca, urmarind sa ii afecteze imaginea si sansele electorale prin referiri la presupuse esecuri anterioare. Materialul a fost distribuit ca reclama platita, cu un buget intre 100-199 RON si o audienta estimata intre 100,001-500,000 persoane, demonstrand intentia clara de a influenta procesul electoral dupa incheierea perioadei legale de propaganda.
    \item promovarea unui material de propaganda electorala (ID postare Facebook: \href{https://www.facebook.com/ads/library/?id=1104185277756075}{1104185277756075}) dupa incheierea perioadei de campanie electorala. Materialul, publicat si promovat dupa ora 18:00 pe 23.11.2024, prezinta realizarile candidatului ION-MARCEL CIOLACU, fiind marcat cu cod CMF 11240017, constituind astfel propaganda electorala explicita. Postarea este sponsorizata si distribuita pe Facebook si Instagram, avand un impact estimat intre 3.000 si 3.999 de afisari, reprezentand o incercare clara de influentare a votului in perioada in care propaganda electorala este interzisa prin lege.
    \item promovarea unui material de propaganda electorala (ID postare Facebook: \href{https://www.facebook.com/ads/library/?id=1108531926873914}{1108531926873914}) dupa incheierea perioadei de campanie electorala pentru alegerile prezidentiale. Materialul promovat dupa ora 18:00 pe 23.11.2024 contine referinte directe la candidatul prezidential Marcel Ciolacu, foloseste sloganuri de campanie ("Calea sigura pentru Romania"), include numar CMF (11240017) si este distribuit ca material sponsorizat pe Facebook si Instagram, avand ca scop influentarea votului in favoarea candidatului PSD la presedintie.
    \item continuarea propagandei electorale dupa incheierea campaniei electorale, prin promovarea unui mesaj electoral platit (ID Facebook: \href{https://www.facebook.com/ads/library/?id=1247733936473803}{1247733936473803}) care il sustine explicit pe candidatul prezidential Marcel Ciolacu. Postarea, activa dupa ora 18:00 pe 23.11.2024, contine numar oficial CMF (11240017), foloseste sloganuri de campanie (\#CaleaSiguraPentruRomania) si promoveaza explicit candidatul in contextul alegerilor prezidentiale. Impactul estimat al postarii este intre 45.000 si 50.000 de afisari, demonstrand intentia clara de a influenta electoratul dupa incheierea perioadei legale de campanie.
    \item difuzarea de materiale de propaganda electorala (ID Facebook: \href{https://www.facebook.com/ads/library/?id=1272562420436699}{1272562420436699}) dupa ora 18:00 pe 23.11.2024. Materialul promovat prezinta realizarile guvernului condus de candidatul prezidential Marcel Ciolacu, foloseste hashtag-uri de campanie (\#CaleaSigura), contine numar CMF (11240017) si face referire directa la alegerile din 24 noiembrie. Postarea este platita si targetata catre un public larg (15.000-19.999 impresii), promovand explicit realizarile si promisiunile PSD in contextul alegerilor prezidentiale.
    \item promovarea unui material de propaganda electorala (ID Facebook: \href{https://www.facebook.com/ads/library/?id=1292682541619455}{1292682541619455}) dupa ora 18:00 pe 23.11.2024, care promoveaza candidatul ION-MARCEL CIOLACU intr-un mod pozitiv, prezentand realizarile sale ca prim-ministru si negocierile pentru aderarea la Schengen. Materialul contine numar CMF (11240017), este platit si distribuit pe Facebook si Instagram, si are un obiectiv electoral clar de a influenta alegatorii in perioada in care propaganda electorala este interzisa prin lege.
    \item promovarea unui mesaj electoral platit pe Facebook (ID: \href{https://www.facebook.com/ads/library/?id=1334442494200293}{1334442494200293}) dupa incheierea perioadei de campanie electorala. Materialul promoveaza explicit candidatul ION-MARCEL CIOLACU si indeamna direct la vot ("duminica, 24 noiembrie, votam Marcel Ciolacu!"), continand elemente specifice de propaganda electorala (CMF 11240017, hashtag-uri de campanie, mesaje electorale). Postarea este activa si promovata dupa ora 18:00 pe 23.11.2024, incalcand astfel prevederile legale privind incetarea campaniei electorale.
    \item continuarea propagandei electorale dupa incheierea campaniei electorale pentru alegerile prezidentiale, prin promovarea unui material electoral platit pe Facebook (ID: \href{https://www.facebook.com/ads/library/?id=1355346872396754}{1355346872396754}) care il promoveaza direct pe candidatul ION-MARCEL CIOLACU, folosind numarul CMF 11240017, hashtag-uri de campanie si mesaje electorale. Materialul a fost activ si distribuit dupa ora 18:00 pe 23.11.2024, incalcand astfel perioada legala de campanie electorala pentru alegerile prezidentiale.
    \item promovarea unui mesaj de propaganda electorala (ID Facebook: \href{https://www.facebook.com/ads/library/?id=1644691303149522}{1644691303149522}) dupa incheierea perioadei de campanie, respectiv dupa ora 18:00 pe 23.11.2024. Postarea promoveaza realizarile candidatului prezidential Marcel Ciolacu, folosind hashtag-uri electorale (\#MarcelCiolacu, \#PSD), continand numar CMF (11240017) si fiind promovata platit pe platformele Facebook si Instagram catre un public tinta de 100.001-500.000 persoane, cu scopul evident de influentare a votului.
    \item promovarea unui mesaj electoral platit pe Facebook si Instagram (ID postare: \href{https://www.facebook.com/ads/library/?id=1736664547156289}{1736664547156289}) dupa ora 18:00 pe 23.11.2024. Mesajul promoveaza candidatul ION-MARCEL CIOLACU intr-o maniera pozitiva, folosind metafore sportive si termeni laudativi ("Lider, strateg, antrenor"), cu un buget de 100-199 RON si o audienta estimata intre 100,001-500,000 persoane. Caracterul electoral al mesajului este evident prin promovarea directa a candidatului si folosirea unor tehnici de marketing politic pentru influentarea votantilor.
    \item publicarea si promovarea unei reclame platite pe Facebook (ID: \href{https://www.facebook.com/ads/library/?id=1778143106258366}{1778143106258366}) dupa ora 18:00 pe 23.11.2024, continand propaganda electorala in favoarea candidatului ION-MARCEL CIOLACU. Postarea promoveaza in mod explicit realizarile si calitatile candidatului ("Cu diplomatie si ambitie, Marcel Ciolacu indeplineste un mare vis al romanilor"), folosind mesaje cu caracter electoral explicit in perioada de restrictie, cu intentie clara de influentare a alegatorilor. Materialul indeplineste toate conditiile art. 36(7) pentru a fi considerat propaganda electorala, fiind promovat activ prin plata pentru distributie pe platformele Meta (Facebook si Instagram).
    \item continuarea propagandei electorale dupa incheierea campaniei, prin promovarea unui material electoral platit (ID Facebook: \href{https://www.facebook.com/ads/library/?id=3780141848966060}{3780141848966060}) ce il prezinta in mod pozitiv pe candidatul prezidential Marcel Ciolacu, cu numar CMF 11240017, distribuit dupa ora 18:00 pe 23.11.2024. Materialul promoveaza realizarile candidatului si foloseste hashtag-uri de campanie (\#PSD \#CaleaSiguraPentruRomania), avand un evident caracter de propaganda electorala, fiind targetat catre un public larg (5000-6000 impresii estimate).
    \item difuzarea de materiale de propaganda electorala (ID postare Facebook: \href{https://www.facebook.com/ads/library/?id=379816521808234}{379816521808234}) dupa incheierea perioadei de campanie electorala. Materialul promovat dupa ora 18:00 pe 23.11.2024 contine in mod explicit promovarea candidatului ION-MARCEL CIOLACU la presedintie, utilizand sloganul "\#MarcelCiolacuPresedinte", fiind insotit de numar CMF (11240017) si constituind propaganda electorala conform Art. 36(7) prin referirea directa la candidat, obiectivul electoral explicit si adresabilitatea catre publicul larg prin intermediul unei reclame platite pe Facebook si Instagram.
    \item promovarea unui material publicitar platit pe Facebook (ID: \href{https://www.facebook.com/ads/library/?id=419365714570211}{419365714570211}) care face referire directa la candidatul prezidential Marcel Ciolacu, dupa incheierea perioadei de campanie electorala. Materialul, distribuit dupa ora 18:00 pe 23.11.2024, are un buget de promovare intre 100-199 RON si a ajuns la un public estimat intre 100,001-500,000 persoane, reprezentand o continuare clara a propagandei electorale dupa incheierea acesteia. Postarea face parte dintr-o strategie de umanizare a candidatului, avand hashtag-uri specifice campaniei (\#MarcelCiolacu) si fiind sponsorizata de organizatia judeteana a partidului.
    \item promovarea unui mesaj de propaganda electorala dupa incheierea perioadei de campanie, constand intr-o postare platita pe Facebook (ID: \href{https://www.facebook.com/ads/library/?id=4246401465586841}{4246401465586841}) care indeamna explicit cetatenii sa voteze candidatul PSD Marcel Ciolacu la alegerile prezidentiale. Postarea, facuta dupa ora 18:00 pe 23.11.2024, reprezinta o continuare ilegala a propagandei electorale, avand un impact estimat intre 2000-2999 de afisari si promovand explicit un candidat la presedintie dupa incheierea perioadei legale de campanie.
    \item difuzarea de materiale de propaganda electorala (ID postare Facebook: \href{https://www.facebook.com/ads/library/?id=465635329442590}{465635329442590}) dupa incheierea perioadei de campanie electorala. Materialul promovat prezinta realizarile candidatului ION-MARCEL CIOLACU intr-o lumina favorabila, folosind evenimente oficiale pentru promovare electorala, dupa ora 18:00 pe 23.11.2024. Postarea include numar CMF (11240017), confirmand natura sa de material electoral, si este distribuita ca reclama platita pe Facebook si Instagram, vizand un public tinta de peste 100.000 de persoane.
    \item promovarea unui material de propaganda electorala (ID Facebook: \href{https://www.facebook.com/ads/library/?id=467492126361933}{467492126361933}) dupa incheierea perioadei de campanie electorala. Materialul promoveaza candidatul ION-MARCEL CIOLACU intr-o lumina pozitiva, folosind hashtag-uri de campanie (\#CaleaSiguraPentruRomania), numar CMF (11240017) si mesaje care ii evidentiaza realizarile ca premier in contextul campaniei prezidentiale. Postarea este platita si targetata catre un public larg (100,001-500,000 persoane), fiind activa dupa ora 18:00 pe 23.11.2024, incalcand astfel prevederile legale privind incetarea propagandei electorale.
    \item publicarea si promovarea unui mesaj de propaganda electorala (ID postare Facebook: \href{https://www.facebook.com/ads/library/?id=482922284799882}{482922284799882}) dupa incheierea perioadei de campanie electorala pentru alegerile prezidentiale. Postarea, publicata dupa ora 18:00 pe 23.11.2024, promoveaza in mod direct candidatul ION-MARCEL CIOLACU, folosind numar CMF (11240017), hashtag-uri de campanie si mesaje care ii evidentiaza realizarile in calitate de prim-ministru, avand ca scop influentarea alegatorilor. Postarea este sponsorizata si distribuita pe platformele Facebook si Instagram, avand un impact estimat intre 3000-3999 de afisari.
    \item continuarea propagandei electorale dupa incheierea acesteia, manifestata prin publicarea si promovarea unei reclame platite pe Facebook (ID: \href{https://www.facebook.com/ads/library/?id=536199915962104}{536199915962104}) care vizeaza in mod direct si negativ candidatul la presedintie Nicolae-Ionel Ciuca. Postarea, avand un buget de 300-399 RON si o audienta estimata intre 60.000-70.000 de persoane, a continuat sa fie difuzata si dupa ora 18:00 pe 23.11.2024, incalcand astfel prevederile legale privind incheierea campaniei electorale. Mesajul are un caracter evident de propaganda electorala, urmarind sa influenteze negativ opinia alegatorilor despre candidat prin ridiculizarea rezultatelor sale electorale anterioare.
    \item promovarea unui mesaj de propaganda electorala dupa incheierea perioadei de campanie, respectiv dupa ora 18:00 pe 23.11.2024. Postarea cu ID-ul \href{https://www.facebook.com/ads/library/?id=550787227746775}{550787227746775} promoveaza in mod direct candidatul ION-MARCEL CIOLACU, prezentandu-l intr-o lumina pozitiva prin evidentierea realizarilor sale politice si folosind hashtag-uri specific electorale (\#caleasigurapentruromani). Mesajul este platit si targetat catre un public larg (100,001-500,000 persoane), fiind distribuit pe Facebook si Instagram, reprezentand astfel o continuare clara a propagandei electorale dupa incheierea perioadei legale de campanie.
    \item distribuirea de materiale de propaganda electorala platita pe platformele Facebook si Instagram (ID postare: \href{https://www.facebook.com/ads/library/?id=554122070651046}{554122070651046}) dupa ora 18:00 pe 23.11.2024. Postarea contine critici directe si acuzatii la adresa candidatului Nicolae-Ionel Ciuca, are scop electoral explicit, contine numar CMF (1124001), si este distribuit ca material publicitar platit catre un public larg (8000-9000 impresii). Mesajul urmareste in mod vadit influentarea votului prin discreditarea candidatului si continua sa fie activ in perioada de interdictie a propagandei electorale pentru alegerile prezidentiale.
    \item promovarea unui mesaj electoral platit pe Facebook (ID: \href{https://www.facebook.com/ads/library/?id=596284366168045}{596284366168045}) ce continua propaganda electorala pentru candidatul prezidential Marcel Ciolacu dupa incheierea perioadei de campanie. Postarea contine indemnuri directe de vot ("Votati Marcel Ciolacu, presedintele Romaniei!") si este promovata activ dupa ora 18:00 pe 23.11.2024, incalcand astfel prevederile legale privind incetarea campaniei electorale. Mesajul are caracter electoral explicit si este distribuit contra cost pe platformele Facebook si Instagram, cu un impact estimat intre 1000-1999 de utilizatori.
    \item continuarea propagandei electorale dupa incheierea campaniei pentru alegerile prezidentiale, prin postarea cu ID \href{https://www.facebook.com/ads/library/?id=873431361659277}{873431361659277} pe Facebook. Postarea, facuta dupa ora 18:00 pe 23.11.2024, contine materiale de propaganda electorala cu referire directa la alegerile prezidentiale, folosind sloganuri de campanie ("Calea Sigura Pentru Romania"), indemnuri directe la vot pentru data de 24 noiembrie, si promovarea indirecta a candidatului PSD la presedintie. Postarea include numar CMF (11240017), confirmand natura sa de propaganda electorala, si a fost promovata ca reclama platita pe Facebook si Instagram, atingand intre 10.000-14.999 de afisari.
    \item promovarea unui mesaj electoral platit (ID Facebook: \href{https://www.facebook.com/ads/library/?id=878774651078667}{878774651078667}) dupa incheierea perioadei de campanie electorala. Postarea, efectuata dupa ora 18:00 pe 23.11.2024, promoveaza realizarile lui Marcel Ciolacu, candidat la presedintie, folosind numar CMF (11240017), hashtag-uri de campanie si mesaje specifice propagandei electorale. Postarea are caracter electoral evident, fiind promovata platit catre un public larg (100,001-500,000 persoane), cu un buget intre 100-199 RON, pe platformele Facebook si Instagram.
    \item promovarea unui material de propaganda electorala (ID Facebook: \href{https://www.facebook.com/ads/library/?id=917142056664792}{917142056664792}) dupa incheierea perioadei de campanie electorala. Materialul promoveaza candidatul ION-MARCEL CIOLACU, folosind realizarile sale ca prim-ministru pentru a influenta alegatorii, continand elemente clare de propaganda electorala (CMF 11240017, hashtag-uri de campanie, mesaje electorale). Postarea a fost promovata activ dupa ora 18:00 pe 23.11.2024, incalcand astfel prevederile legale privind perioada de campanie electorala.
    \item difuzarea unui mesaj de propaganda electorala dupa incheierea perioadei de campanie, respectiv dupa ora 18:00 pe 23.11.2024. Postarea cu ID-ul \href{https://www.facebook.com/ads/library/?id=925092865738975}{925092865738975} contine un indemn direct la vot pentru candidatul ION-MARCEL CIOLACU, utilizeaza numar CMF (11240017), si este promovata ca reclama platita pe platformele Facebook si Instagram, avand un impact estimat intre 4.000 si 4.999 de afisari. Mesajul include in mod explicit indemnul "Duminica, 24 noiembrie, votam Marcel Ciolacu" si foloseste sloganuri de campanie, constituind astfel propaganda electorala activa in ziua votului.
    \item continuarea propagandei electorale dupa incheierea campaniei, prin promovarea unui material electoral platit pe Facebook (ID: \href{https://www.facebook.com/ads/library/?id=936257145070434}{936257145070434}) care promoveaza candidatul ION-MARCEL CIOLACU, folosind hashtag-ul \#MarcelCiolacuPresedinte si prezentand realizarile acestuia in calitate de prim-ministru ca argument electoral. Materialul a fost difuzat dupa ora 18:00 pe 23.11.2024, incalcand astfel prevederile legale privind incheierea campaniei electorale. Materialul contine cod CMF (11240017), confirmand natura sa de propaganda electorala.
    \item difuzarea de continut propagandistic electoral (ID postare Facebook: \href{https://www.facebook.com/ads/library/?id=957521712887209}{957521712887209}) dupa incheierea perioadei de campanie electorala. Postarea promoveaza explicit candidatul ION-MARCEL CIOLACU si partidul PSD, folosind slogan de campanie si numar CMF (11240017), constituind astfel propaganda electorala activa dupa ora 18:00 pe 23.11.2024. Mesajul a fost distribuit ca reclama platita pe Facebook si Instagram, cu un impact estimat intre 15.000-19.999 de afisari, demonstrand intentia clara de influentare a votului prin continuarea propagandei electorale in afara perioadei legale.
\end{enumerate}

\vspace{0.5cm}

\subsection{PSD Giurgiu}
Următoarele fapte contravenționale sunt sesizate împotriva acestei entități:

\begin{enumerate}[leftmargin=*, label=\arabic*.)]
    \item continuarea propagandei electorale dupa incheierea campaniei electorale pentru alegerile prezidentiale. Postarea cu ID-ul \href{https://www.facebook.com/ads/library/?id=1241527783630357}{1241527783630357} contine mesaje directe de sustinere a candidatului prezidential Marcel Ciolacu ("E timpul pentru Marcel Ciolacu - presedinte"), fiind promovata ca reclama platita pe Facebook dupa ora 18:00 pe 23.11.2024. Postarea include cod CMF (11240017), are caracter electoral explicit si se adreseaza unui public larg (125,000-150,000 impresii), reprezentand astfel propaganda electorala conform Art. 36(7).
    \item difuzarea unei reclame platite pe Facebook si Instagram (ID: \href{https://www.facebook.com/ads/library/?id=1411711393139703}{1411711393139703}) dupa ora 18:00 pe 23.11.2024, care promoveaza candidatul ION-MARCEL CIOLACU la presedintie. Postarea contine promisiuni electorale specifice, numar CMF (11240017), si a avut un impact semnificativ, atingand intre 70.000 si 79.999 de persoane. Mesajul reprezinta propaganda electorala clara, cu obiectiv electoral explicit, adresandu-se publicului larg si promovand direct un candidat la presedintie dupa incheierea perioadei legale de campanie.
    \item promovarea unui mesaj electoral platit pe Facebook (ID: \href{https://www.facebook.com/ads/library/?id=2394020644263136}{2394020644263136}) dupa incheierea perioadei de campanie electorala pentru alegerile prezidentiale. Postarea promoveaza explicit candidatul ION-MARCEL CIOLACU ca "viitorul presedinte" si face promisiuni electorale in numele acestuia, avand un impact direct asupra alegerilor prezidentiale. Mesajul a fost promovat dupa ora 18:00 pe 23.11.2024, incalcand astfel perioada legala de campanie electorala. Postarea a avut o audienta semnificativa, atingand intre 250.000 si 300.000 de afisari, cu o investitie intre 2.500 si 3.000 RON.
    \item promovarea unui mesaj de propaganda electorala (ID postare Facebook: \href{https://www.facebook.com/ads/library/?id=547070228109693}{547070228109693}) dupa incheierea perioadei de campanie electorala pentru alegerile prezidentiale. Postarea, efectuata dupa ora 18:00 pe 23.11.2024, contine numar CMF (11240017), ataca direct un contracandidat (PNL/Ciuca), foloseste retorica electorala si incearca sa influenteze opinia publica in perioada de interdictie. Mesajul este sponsorizat si distribuit catre un public tinta de 100,001-500,000 persoane, amplificand astfel impactul incalcarii.
\end{enumerate}

\vspace{0.5cm}

\subsection{PSD Gorj}
Următoarele fapte contravenționale sunt sesizate împotriva acestei entități:

\begin{enumerate}[leftmargin=*, label=\arabic*.)]
    \item promovarea unui mesaj electoral platit pe Facebook (ID: \href{https://www.facebook.com/ads/library/?id=1068055818117576}{1068055818117576}) care sustine explicit candidatura lui Marcel Ciolacu la presedintie ("Marcel Ciolacu pentru Presedintia Romaniei"), dupa ora 18:00 pe 23.11.2024, perioada in care campania electorala pentru alegerile prezidentiale este inchisa. Mesajul include un indemn explicit la vot ("Votul tau conteaza") si promisiuni electorale, fiind distribuit ca reclama platita pe Facebook si Instagram, cu un impact estimat intre 10,000 si 15,000 de afisari.
    \item promovarea candidatului ION-MARCEL CIOLACU la functia de presedinte prin intermediul unei reclame platite pe Facebook (ID: \href{https://www.facebook.com/ads/library/?id=3845124615805481}{3845124615805481}) dupa incheierea perioadei de campanie electorala. Postarea contine elemente clare de propaganda electorala, incluzand hashtag-uri de campanie (\#MarcelCiolacu\_PRESEDINTE), simboluri de partid si indemnuri directe la vot, fiind difuzata dupa ora 18:00 pe 23.11.2024, incalcand astfel prevederile legale privind incheierea campaniei electorale.
    \item difuzarea unui mesaj de propaganda electorala pentru candidatul prezidential Marcel Ciolacu dupa incheierea perioadei de campanie electorala. Materialul publicitar, cu ID-ul \href{https://www.facebook.com/ads/library/?id=888133213528318}{888133213528318} pe facebook, promovat ca reclama platita, prezinta un indemn explicit de sustinere a candidatului la functia de presedinte, fiind distribuit dupa ora 18:00 pe 23.11.2024. Mesajul a avut un impact semnificativ, atingand intre 10.000 si 14.999 de persoane, fiind distribuit atat pe Facebook cat si pe Instagram, reprezentand astfel o incalcare clara a prevederilor legale privind perioada de campanie electorala.
\end{enumerate}

\vspace{0.5cm}

\subsection{PSD Iasi}
Următoarele fapte contravenționale sunt sesizate împotriva acestei entități:

\begin{enumerate}[leftmargin=*, label=\arabic*.)]
    \item continuarea propagandei electorale dupa incheierea campaniei electorale prezidentiale, prin promovarea candidatului ION-MARCEL CIOLACU intr-o postare platita pe Facebook si Instagram (ID: \href{https://www.facebook.com/ads/library/?id=1795323061006990}{1795323061006990}). Postarea, difuzata dupa ora 18:00 pe 23.11.2024, promoveaza in mod direct realizarile si imaginea candidatului la presedintie Marcel Ciolacu, avand un impact estimat intre 100.000 si 124.999 de afisari, constituind astfel propaganda electorala conform Art. 36(7) din Legea 334/2006. Mesajul are caracter electoral explicit, prezentand realizarile candidatului si criticand opozitia politica, depasind sfera comunicarii administrative.
    \item promovarea unui material electoral (ID Facebook: \href{https://www.facebook.com/ads/library/?id=543626805255602}{543626805255602}) ce face referire la "planul de tara al presedintelui Marcel Ciolacu" dupa ora 18:00 pe 23.11.2024, in perioada de restrictie electorala pentru alegerile prezidentiale. Desi postarea se concentreaza aparent pe alegerile parlamentare, includerea referintei la planul prezidential al candidatului Marcel Ciolacu reprezinta o continuare a propagandei electorale prezidentiale dupa incheierea acesteia, cu un impact estimat intre 15.000 si 19.999 de afisari.
    \item promovarea unui mesaj de propaganda electorala dupa incheierea perioadei de campanie pentru alegerile prezidentiale, in data de 23.11.2024, dupa ora 18:00. Postarea cu ID-ul \href{https://www.facebook.com/ads/library/?id=556304500453927}{556304500453927} promoveaza realizarile si imaginea candidatului prezidential Marcel Ciolacu, folosind resurse financiare semnificative (buget 2000-2499 RON) pentru a atinge un public de 800.000-900.000 de persoane, cu un mesaj clar electoral care prezinta PSD si pe Marcel Ciolacu ca "calea sigura pentru dezvoltare", incercand astfel sa influenteze decizia de vot a alegatorilor.
    \item continuarea propagandei electorale dupa incheierea campaniei electorale pentru alegerile prezidentiale, prin promovarea unui material publicitar platit (ID Facebook: \href{https://www.facebook.com/ads/library/?id=586428570613663}{586428570613663}) care il prezinta in mod favorabil pe candidatul prezidential Marcel Ciolacu, dupa ora 18:00 pe 23.11.2024. Materialul contine numar CMF (11240017), face promisiuni electorale si critica oponentii politici, avand un evident caracter de propaganda electorala pentru candidatul prezidential al PSD.
    \item continuarea propagandei electorale pentru candidatul prezidential Marcel Ciolacu dupa incheierea perioadei legale de campanie. Postarea cu ID-ul \href{https://www.facebook.com/ads/library/?id=8590905931028294}{8590905931028294} reprezinta propaganda electorala activa dupa ora 18:00 pe 23.11.2024, continand mesaje de sustinere explicita a candidatului la presedintie si fiind distribuita ca reclama platita pe platformele Facebook si Instagram, cu o audienta estimata de peste 1 milion de persoane si intre 60.000-70.000 de afisari. Postarea include numar CMF (11240017) si reprezinta o incalcare clara a prevederilor legale privind incheierea campaniei electorale.
\end{enumerate}

\vspace{0.5cm}

\subsection{PSD Maramures}
Următoarele fapte contravenționale sunt sesizate împotriva acestei entități:

\begin{enumerate}[leftmargin=*, label=\arabic*.)]
    \item promovarea unui mesaj electoral platit (ID Facebook: \href{https://www.facebook.com/ads/library/?id=1307053713792365}{1307053713792365}) ce contine propaganda electorala pentru candidatul prezidential Marcel Ciolacu dupa incheierea perioadei de campanie. Postarea, ce include numar CMF 11240017, a fost difuzata pe Facebook si Instagram cu un buget intre 1000-1499 RON si a ajuns la 125.000-150.000 de persoane, reprezentand astfel o incalcare clara a prevederilor legale privind incetarea propagandei electorale dupa ora 18:00 pe 23.11.2024. Mesajul contine referiri directe la candidatura prezidentiala si solicita explicit sprijin electoral.
    \item publicarea si promovarea unei postari cu caracter electoral (ID Facebook: \href{https://www.facebook.com/ads/library/?id=1667935190435016}{1667935190435016}) dupa incheierea perioadei de campanie electorala prezidentiala. Postarea promoveaza explicit candidatul ION-MARCEL CIOLACU, folosind resurse financiare pentru distribuire (400-499 RON) si atingand un public de 20,000-25,000 persoane, dupa ora 18:00 pe 23.11.2024. Materialul contine numar CMF (11240017), demonstrand caracterul sau electoral explicit, si prezinta realizarile candidatului intr-o lumina favorabila, cu scopul clar de a influenta preferintele electorale.
    \item promovarea unui material de propaganda electorala (ID postare Facebook: \href{https://www.facebook.com/ads/library/?id=3855023434716826}{3855023434716826}) dupa ora 18:00 pe 23.11.2024. Materialul promoveaza explicit candidatul ION-MARCEL CIOLACU pentru functia de presedinte, contine numar CMF (11240017), foloseste hashtag-uri de campanie (\#MarcelCiolacupresedinte), face promisiuni electorale si este distribuit ca reclama platita pe Facebook si Instagram cu un reach estimat intre 35.000-39.999 impresii. Materialul reprezinta propaganda electorala conform Art. 36(7) prin referirea directa la candidat, obiectivul electoral explicit si adresarea catre publicul larg.
    \item promovarea unui mesaj de propaganda electorala pentru alegerile prezidentiale (mentionand explicit "campanie prezidentiala") dupa incheierea perioadei legale de campanie, intr-o postare sponsorizata pe Facebook (ID: \href{https://www.facebook.com/ads/library/?id=445600828269259}{445600828269259}) cu impact intre 40.000 si 45.000 de afisari. Postarea, realizata dupa ora 18:00 pe 23.11.2024, contine elemente clare de propaganda electorala, inclusiv numar CMF (11240017), slogan de campanie si mesaje care promoveaza candidatul PSD la presedintie, incalcand astfel prevederile legale privind incheierea campaniei electorale.
    \item promovarea unui mesaj electoral platit (ID: \href{https://www.facebook.com/ads/library/?id=620276456991285}{620276456991285}) ce face referire directa la candidatul prezidential Marcel Ciolacu, dupa incheierea perioadei de campanie electorala. Postarea, avand CMF 11240017, a fost distribuita pe Facebook si Instagram cu un buget intre 300-399 RON, atingand intre 15.000-19.999 de persoane, promovand explicit mesaje electorale si planul de masuri al candidatului, dupa ora 18:00 pe 23.11.2024, incalcand astfel prevederile legale privind incheierea campaniei electorale.
\end{enumerate}

\vspace{0.5cm}

\subsection{PSD Maramures, prin reprezentantul Gabriel Valer Zetea}
Următoarele fapte contravenționale sunt sesizate împotriva acestei entități:

\begin{enumerate}[leftmargin=*, label=\arabic*.)]
    \item continuarea propagandei electorale pentru candidatul prezidential Marcel Ciolacu dupa incheierea perioadei de campanie. Postarea cu ID-ul \href{https://www.facebook.com/ads/library/?id=1120036146358587}{1120036146358587} promoveaza explicit "proiectele de tara propuse de Marcel Ciolacu" intr-o postare platita, cu impact semnificativ (45,000-50,000 impresii), difuzata dupa ora 18:00 pe 23.11.2024. Postarea include elementele specifice propagandei electorale: CMF 11240017, sloganuri de campanie, si promovare directa a candidatului prezidential intr-o perioada interzisa de lege.
\end{enumerate}

\vspace{0.5cm}

\subsection{PSD Neamt, prin intermediul lui Ciprian Serban}
Următoarele fapte contravenționale sunt sesizate împotriva acestei entități:

\begin{enumerate}[leftmargin=*, label=\arabic*.)]
    \item distribuirea unui mesaj de propaganda electorala (ID postare Facebook: \href{https://www.facebook.com/ads/library/?id=2031677793938855}{2031677793938855}) dupa incheierea campaniei electorale. Postarea promoveaza explicit candidatul ION-MARCEL CIOLACU si realizarile PSD, folosind numar CMF (CMF-11240017), fiind difuzata dupa ora 18:00 pe 23.11.2024. Mesajul are caracter electoral evident, promovand realizarile candidatului si partidului, contrastand cu opozitia politica si folosind sloganuri de campanie ("CALEA SIGURA spre Europa!").
\end{enumerate}

\vspace{0.5cm}

\subsection{PSD Neamt, prin reprezentantul Adrian Paduraru}
Următoarele fapte contravenționale sunt sesizate împotriva acestei entități:

\begin{enumerate}[leftmargin=*, label=\arabic*.)]
    \item promovarea candidatului la presedintie Marcel Ciolacu intr-o reclama platita pe Facebook (ID: \href{https://www.facebook.com/ads/library/?id=1223021858771631}{1223021858771631}) dupa incheierea perioadei de campanie electorala prezidentiala. Postarea, difuzata dupa ora 18:00 pe 23.11.2024, prezinta realizarile guvernamentale ale lui Marcel Ciolacu in contextul aderarii la Schengen, asociind direct acest succes cu candidatul la presedintie si solicitand explicit sprijinul alegatorilor. Materialul include numar CMF (11240017), confirmand natura sa de propaganda electorala, si a fost distribuit catre un public tinta de 100.001-500.000 persoane.
\end{enumerate}

\vspace{0.5cm}

\subsection{PSD Neamt, prin reprezentantul Tifui Dumitru Bogdan}
Următoarele fapte contravenționale sunt sesizate împotriva acestei entități:

\begin{enumerate}[leftmargin=*, label=\arabic*.)]
    \item continuarea propagandei electorale dupa incheierea campaniei electorale pentru alegerile prezidentiale. Postarea cu ID-ul \href{https://www.facebook.com/ads/library/?id=1274716527004296}{1274716527004296} contine un mesaj explicit de indemn la vot pentru candidatul Marcel Ciolacu, include promisiuni electorale si este promovata ca reclama platita pe Facebook dupa ora 18:00 pe 23.11.2024. Postarea include numar CMF (CMF-11240017), confirmand natura sa de propaganda electorala, si are un reach estimat intre 100,001 si 500,000 de persoane, demonstrand impactul semnificativ al acestei incalcari.
\end{enumerate}

\vspace{0.5cm}

\subsection{PSD OLT}
Următoarele fapte contravenționale sunt sesizate împotriva acestei entități:

\begin{enumerate}[leftmargin=*, label=\arabic*.)]
    \item difuzarea de materiale de propaganda electorala dupa incheierea campaniei electorale pentru alegerile prezidentiale. Postarea cu ID-ul \href{https://www.facebook.com/ads/library/?id=928493602047350}{928493602047350} promoveaza explicit candidatul ION-MARCEL CIOLACU pentru functia de presedinte, continand indemnuri directe la vot ("duminica, 24 noiembrie ALEGEM...MARCEL CIOLACU, PRESEDINTE"). Aceasta este o reclama platita, cu un impact estimat intre 100,001-500,000 de persoane, difuzata dupa ora 18:00 pe 23.11.2024, incalcand astfel perioada de restrictie electorala. Materialul intruneste toate criteriile propagandei electorale conform Art. 36(7): identifica clar candidatul, are obiectiv electoral explicit si se adreseaza publicului larg prin intermediul platformelor de social media.
\end{enumerate}

\vspace{0.5cm}

\subsection{PSD Pascani}
Următoarele fapte contravenționale sunt sesizate împotriva acestei entități:

\begin{enumerate}[leftmargin=*, label=\arabic*.)]
    \item promovarea unui mesaj electoral in favoarea candidatului prezidential ION-MARCEL CIOLACU dupa incheierea campaniei electorale, prin postarea cu ID \href{https://www.facebook.com/ads/library/?id=581864764347888}{581864764347888} pe facebook. Postarea, efectuata dupa ora 18:00 pe 23.11.2024, contine mesaje clare de propaganda electorala cum ar fi "Cu o guvernare social-democrata si Marcel Ciolacu in fruntea tarii, Romania are sansa unui progres pe termen lung", are numar CMF 11240017, si este o reclama platita care a ajuns la intre 10.000 si 15.000 de persoane, constituind astfel o clara actiune de propaganda electorala in afara perioadei permise de lege.
    \item continuarea propagandei electorale dupa incheierea campaniei electorale, manifestata prin publicarea si promovarea unei reclame platite pe Facebook (ID: \href{https://www.facebook.com/ads/library/?id=920139349642495}{920139349642495}) dupa ora 18:00 pe 23.11.2024. Postarea contine in mod explicit indemnuri la vot pentru candidatul ION-MARCEL CIOLACU, promisiuni electorale si mesaje de campanie, fiind marcata cu cod CMF 11240017, ceea ce confirma natura sa de material electoral. Materialul a fost distribuit activ si dupa incheierea perioadei legale de campanie, avand impact direct asupra procesului electoral prin atingerea a peste 1000 de utilizatori.
\end{enumerate}

\vspace{0.5cm}

\subsection{PSD Salaj}
Următoarele fapte contravenționale sunt sesizate împotriva acestei entități:

\begin{enumerate}[leftmargin=*, label=\arabic*.)]
    \item promovarea unui material de propaganda electorala (ID postare Facebook: \href{https://www.facebook.com/ads/library/?id=1933217477167387}{1933217477167387}) dupa incheierea perioadei de campanie electorala. Materialul promovat dupa ora 18:00 pe 23.11.2024 contine referiri directe la candidatul prezidential Marcel Ciolacu, ii atribuie realizari specifice (aderarea la Schengen), include numar CMF (11240017), si este distribuit ca reclama platita catre un public larg (10,000-14,999 impresii), avand ca scop influentarea optiunilor de vot prin asocierea candidatului cu succese politice majore.
\end{enumerate}

\vspace{0.5cm}

\subsection{PSD Teleorman, prin pagina "Ioana Panagoret"}
Următoarele fapte contravenționale sunt sesizate împotriva acestei entități:

\begin{enumerate}[leftmargin=*, label=\arabic*.)]
    \item difuzarea de materiale de propaganda electorala (ID postare Facebook: \href{https://www.facebook.com/ads/library/?id=3754486404881677}{3754486404881677}) dupa incheierea campaniei electorale pentru alegerile prezidentiale. Postarea promoveaza realizarile candidatului prezidential Marcel Ciolacu, folosind un CMF (11240017), intr-o postare platita pe Facebook si Instagram, cu impact intre 6000-6999 de afisari, dupa ora 18:00 pe 23.11.2024. Materialul foloseste hashtag-uri electorale si prezinta realizari ale candidatului in calitate de premier, avand scop electoral evident.
\end{enumerate}

\vspace{0.5cm}

\subsection{PSD Tulcea}
Următoarele fapte contravenționale sunt sesizate împotriva acestei entități:

\begin{enumerate}[leftmargin=*, label=\arabic*.)]
    \item difuzarea unui mesaj de propaganda electorala (ID Facebook: \href{https://www.facebook.com/ads/library/?id=1500539287325035}{1500539287325035}) dupa ora 18:00 pe 23.11.2024. Mesajul promoveaza in mod direct candidatul ION-MARCEL CIOLACU, actualul premier si candidat la presedintie, prezentand realizarile acestuia si ale partidului sau, cu intentia clara de a influenta preferintele electorale. Postarea include numar CMF (11240017), confirma natura sa de propaganda electorala si este platita pentru a ajunge la un public larg (100,001-500,000 persoane).
    \item difuzarea de materiale de propaganda electorala (ID postare Facebook: \href{https://www.facebook.com/ads/library/?id=599290196096649}{599290196096649}) dupa incheierea perioadei de campanie electorala pentru alegerile prezidentiale. Postarea promoveaza candidatul ION-MARCEL CIOLACU intr-un mod pozitiv, prin asocierea cu realizari administrative si folosind simbolurile partidului, avand si numar CMF (11240017). Materialul a fost promovat ca reclama platita pe Facebook si Instagram dupa ora 18:00 pe 23.11.2024, avand un impact estimat intre 15.000 si 19.999 de afisari, incalcand astfel prevederile legale privind incheierea campaniei electorale.
    \item promovarea unui material de propaganda electorala (ID Facebook: \href{https://www.facebook.com/ads/library/?id=873014781330956}{873014781330956}) dupa incheierea perioadei de campanie electorala pentru alegerile prezidentiale. Materialul promoveaza realizarile candidatului ION-MARCEL CIOLACU si ale PSD, folosind numar CMF (11240017), fiind difuzat dupa ora 18:00 pe 23.11.2024. Postarea platita vizeaza un public larg (intre 100.001 si 500.000 persoane) si prezinta explicit realizarile candidatului in calitate de prim-ministru, cu scopul de a influenta votul.
\end{enumerate}

\vspace{0.5cm}

\subsection{PSDMM}
Următoarele fapte contravenționale sunt sesizate împotriva acestei entități:

\begin{enumerate}[leftmargin=*, label=\arabic*.)]
    \item continuarea propagandei electorale dupa incheierea campaniei electorale pentru alegerile prezidentiale, prin promovarea activa si platita a candidatului Marcel Ciolacu la functia de presedinte (postare ID: \href{https://www.facebook.com/ads/library/?id=1650154845845951}{1650154845845951}). Postarea contine numar CMF (11240017), foloseste explicit sloganuri de campanie si hashtag-ul "\#MarcelCiolacupresedinte", fiind o forma clara de propaganda electorala difuzata dupa ora 18:00 pe 23.11.2024. Impactul postarii este semnificativ, atingand intre 35.000 si 39.999 de persoane, cu un buget intre 1.000 si 1.499 RON.
\end{enumerate}

\vspace{0.5cm}

\subsection{PSDMM, prin reprezentantul Vasile Roman}
Următoarele fapte contravenționale sunt sesizate împotriva acestei entități:

\begin{enumerate}[leftmargin=*, label=\arabic*.)]
    \item continuarea propagandei electorale dupa incheierea campaniei electorale pentru alegerile prezidentiale, prin promovarea candidatului Marcel Ciolacu la functia de presedinte intr-o postare platita pe Facebook si Instagram (ID postare: \href{https://www.facebook.com/ads/library/?id=1986893058390371}{1986893058390371}). Postarea contine elementele specifice propagandei electorale, inclusiv numar CMF (11240017), sloganuri de campanie si indemnuri explicite de sustinere a candidatului la presedintie, fiind difuzata dupa ora 18:00 pe 23.11.2024, incalcand astfel prevederile legale privind incheierea campaniei electorale.
\end{enumerate}

\vspace{0.5cm}

\subsection{Pagina "5news.RO"}
Următoarele fapte contravenționale sunt sesizate împotriva acestei entități:

\begin{enumerate}[leftmargin=*, label=\arabic*.)]
    \item difuzarea unui mesaj de propaganda electorala dupa incheierea perioadei legale de campanie, in data de 23.11.2024, dupa ora 18:00. Postarea cu ID-ul \href{https://www.facebook.com/ads/library/?id=1249273286195639}{1249273286195639} contine un mesaj negativ la adresa candidatului ELENA-VALERICA LASCONI, cu intentia vadita de a influenta optiunea de vot a alegatorilor, fiind promovata prin intermediul unei reclame platite pe Facebook si Instagram, cu o audienta estimata de peste 1 milion de persoane si un numar de impresii intre 20.000 si 24.999. Mesajul "LASCONI II FACE TRADATORI PE ROMANI" reprezinta in mod clar propaganda electorala negativa, fiind difuzat dupa incheierea perioadei legale de campanie.
    \item promovarea unui mesaj electoral negativ ("Lasconi da dovada de aroganta") impotriva candidatei ELENA-VALERICA LASCONI, dupa incheierea perioadei de campanie electorala. Postarea cu ID-ul \href{https://www.facebook.com/ads/library/?id=1316851392647324}{1316851392647324} a fost promovata ca reclama platita pe Facebook si Instagram dupa ora 18:00 pe 23.11.2024, atingand intre 6000-6999 de persoane, reprezentand o continuare ilegala a propagandei electorale dupa incheierea acesteia.
\end{enumerate}

\vspace{0.5cm}

\subsection{Pagina "Alegerea de AUR"}
Următoarele fapte contravenționale sunt sesizate împotriva acestei entități:

\begin{enumerate}[leftmargin=*, label=\arabic*.)]
    \item difuzarea de materiale de propaganda electorala pentru candidatul prezidential George Nicolae Simion (AUR) dupa incheierea perioadei de campanie electorala. Postarea cu ID-ul \href{https://www.facebook.com/ads/library/?id=1123473639412728}{1123473639412728} promoveaza in mod direct candidatul si planul sau electoral pentru Botosani, folosind explicit indemnul la vot (\#VoteazaAUR) si este o reclama platita pe Facebook si Instagram, activa dupa ora 18:00 pe 23.11.2024, ceea ce constituie propaganda electorala ilegala in perioada post-campanie.
\end{enumerate}

\vspace{0.5cm}

\subsection{Pagina "Constanta MEA"}
Următoarele fapte contravenționale sunt sesizate împotriva acestei entități:

\begin{enumerate}[leftmargin=*, label=\arabic*.)]
    \item difuzarea de propaganda electorala dupa incheierea campaniei electorale, constand in promovarea platita a unei postari pe Facebook (ID: \href{https://www.facebook.com/ads/library/?id=3863248660623792}{3863248660623792}) care vizeaza in mod direct candidatii la presedintie Nicolae-Ionel Ciuca (negativ) si Mircea-Dan Geoana (pozitiv), avand ca scop influentarea comportamentului electoral. Postarea a fost publicata si promovata dupa ora 18:00 pe 23.11.2024, incalcand astfel perioada legala de campanie electorala. Mesajul foloseste hashtag-uri electorale si indeamna in mod explicit la orientarea votului catre un anumit candidat.
\end{enumerate}

\vspace{0.5cm}

\subsection{Pagina "Fluxdestiri"}
Următoarele fapte contravenționale sunt sesizate împotriva acestei entități:

\begin{enumerate}[leftmargin=*, label=\arabic*.)]
    \item promovarea unei reclame platite pe Facebook si Instagram (ID: \href{https://www.facebook.com/ads/library/?id=1797273990811118}{1797273990811118}) in data de 23.11.2024, dupa ora 18:00, continand propaganda electorala negativa la adresa candidatului Nicolae-Ionel Ciuca. Postarea, cu un buget de peste 1000 RON si un impact de peste 80.000 de afisari, contine acuzatii directe de manipulare a sondajelor de opinie, fiind conceputa si distribuita cu scopul clar de a influenta negativ optiunea de vot a alegatorilor in ziua scrutinului prezidential. Mesajul depaseste limitele unei simple opinii personale, fiind o actiune organizata si platita de propaganda electorala.
\end{enumerate}

\vspace{0.5cm}

\subsection{Pagina "Fruncea"}
Următoarele fapte contravenționale sunt sesizate împotriva acestei entități:

\begin{enumerate}[leftmargin=*, label=\arabic*.)]
    \item publicarea si mentinerea activa a unei reclame cu ID-ul \href{https://www.facebook.com/ads/library/?id=1087378399405279}{1087378399405279} ce reprezinta propaganda electorala negativa impotriva candidatilor Mircea-Dan Geoana si Cristian Diaconescu, dupa ora 18:00 pe 23.11.2024. Reclama platita, cu un buget de 400-499 RON si cu o audienta estimata intre 100,001-500,000 persoane, urmareste in mod direct influentarea votului prin asocieri negative si acuzatii directe la adresa candidatilor, depasind cadrul unei simple opinii personale sau al unei informari jurnalistice.
    \item promovarea unui anunt platit (ID: \href{https://www.facebook.com/ads/library/?id=1122389519319036}{1122389519319036}) ce constituie propaganda electorala negativa impotriva candidatului Mircea-Dan Geoana, dupa ora 18:00 pe 23.11.2024. Postarea, care a avut un impact semnificativ (80,000-89,999 impresii) si un buget substantial (400-499 RON), are ca scop influentarea opiniei publice prin generarea de controverse despre familia candidatului. Continutul nu are caracter informativ sau jurnalistic, ci reprezinta propaganda electorala mascata sub forma unei intrebari provocatoare.
\end{enumerate}

\vspace{0.5cm}

\subsection{Pagina "Fruncea" (ID: \href{https://www.facebook.com/ads/library/?id=104898731336618}{104898731336618})}
Următoarele fapte contravenționale sunt sesizate împotriva acestei entități:

\begin{enumerate}[leftmargin=*, label=\arabic*.)]
    \item promovarea unui mesaj de propaganda electorala dupa incheierea perioadei legale de campanie, referitor la candidatii GEORGE-NICOLAE SIMION si ION-MARCEL CIOLACU. Postarea (ID: \href{https://www.facebook.com/ads/library/?id=386674257771104}{386674257771104}) constituie propaganda electorala prin natura sa de mesaj platit, cu impact intre 15.000-19.999 de afisari, vizand sa influenteze opinia alegatorilor despre candidati dupa ora 18:00 pe 23.11.2024. Mesajul foloseste limbaj sensationalist si face afirmatii despre strategii electorale, depasind simpla exprimare a unei opinii personale.
\end{enumerate}

\vspace{0.5cm}

\subsection{Pagina "Lasa-i acasa"}
Următoarele fapte contravenționale sunt sesizate împotriva acestei entități:

\begin{enumerate}[leftmargin=*, label=\arabic*.)]
    \item difuzarea de materiale de propaganda electorala dupa incheierea perioadei de campanie, constand in publicarea unei reclame platite pe Facebook si Instagram (ID: \href{https://www.facebook.com/ads/library/?id=927543008740969}{927543008740969}) care il vizeaza direct pe candidatul George Nicolae Simion, cu un impact negativ evident, atingand intre 400.000 si 450.000 de afisari, cu o cheltuiala intre 2.500 si 3.000 RON, dupa ora 18:00 pe 23.11.2024. Materialul are caracter explicit electoral, vizand sa influenteze negativ perceptia alegatorilor fata de candidat prin comparatii denigratoare si limbaj depreciativ.
\end{enumerate}

\vspace{0.5cm}

\subsection{Pagina "Lasa-i acasa" (ID: \href{https://www.facebook.com/ads/library/?id=248488498687026}{248488498687026})}
Următoarele fapte contravenționale sunt sesizate împotriva acestei entități:

\begin{enumerate}[leftmargin=*, label=\arabic*.)]
    \item difuzarea de materiale de propaganda electorala (ID postare: \href{https://www.facebook.com/ads/library/?id=889814966684713}{889814966684713}) dupa incheierea perioadei de campanie. Postarea vizeaza in mod direct candidatul GEORGE-NICOLAE SIMION, intr-o maniera negativa, facand referire la presupuse fapte de evaziune fiscala. Materialul este promovat ca reclama platita pe Facebook si Instagram, cu un impact estimat intre 100,001 si 500,000 de persoane, fiind difuzat dupa ora 18:00 pe 23.11.2024, in perioada de prohibitie electorala. Continutul are caracter explicit electoral si urmareste influentarea comportamentului de vot al alegatorilor.
\end{enumerate}

\vspace{0.5cm}

\subsection{Pagina "Opinii Independente" (ID: \href{https://www.facebook.com/ads/library/?id=614966562181447}{614966562181447})}
Următoarele fapte contravenționale sunt sesizate împotriva acestei entități:

\begin{enumerate}[leftmargin=*, label=\arabic*.)]
    \item promovarea unui material de propaganda electorala (ID postare: \href{https://www.facebook.com/ads/library/?id=824813186321358}{824813186321358}) pentru candidatul independent Mircea-Dan Geoana, dupa incheierea perioadei de campanie electorala. Materialul, publicat pe data de 22.11.2024, promoveaza o alianta strategica cu impact electoral direct, fiind distribuit ca reclama platita pe Facebook si Instagram, cu un reach estimat intre 100,001 si 500,000 de persoane, dupa ora 18:00 pe 23.11.2024. Actiunea constituie continuarea propagandei electorale dupa incheierea acesteia, fiind in mod clar destinata influentarii votantilor in perioada de interdictie.
\end{enumerate}

\vspace{0.5cm}

\subsection{Pagina "Vocea Botosani"}
Următoarele fapte contravenționale sunt sesizate împotriva acestei entități:

\begin{enumerate}[leftmargin=*, label=\arabic*.)]
    \item difuzarea de materiale de propaganda electorala platite pe platformele Facebook si Instagram (ID postare: \href{https://www.facebook.com/ads/library/?id=1273462180683780}{1273462180683780}) dupa ora 18:00 pe 23.11.2024. Materialul vizeaza direct candidatul la presedintie George Simion, prezentand informatii menite sa influenteze negativ opinia publica prin difuzarea unei convorbiri private si comentarii tendentioase. Postarea este promovata cu bani (intre 0-99 RON) si a ajuns la 3000-3999 de persoane, demonstrand caracterul intentionat de propaganda electorala si influentare a votului in perioada in care acest lucru este interzis prin lege.
\end{enumerate}

\vspace{0.5cm}

\subsection{Pagina de Facebook "AUR - Salasu de Sus" (ID: \href{https://www.facebook.com/ads/library/?id=296678066869537}{296678066869537})}
Următoarele fapte contravenționale sunt sesizate împotriva acestei entități:

\begin{enumerate}[leftmargin=*, label=\arabic*.)]
    \item continuarea propagandei electorale dupa incheierea campaniei, promovand candidatul George Nicolae Simion la functia de presedinte prin intermediul unei reclame platite pe Facebook (ID postare: \href{https://www.facebook.com/ads/library/?id=2006851919778923}{2006851919778923}). Reclama, care a generat intre 25.000 si 29.999 de afisari, promoveaza explicit candidatura la presedintie si indeamna alegatorii sa voteze pentru schimbare, continuand sa fie difuzata dupa ora 18:00 pe 23.11.2024, incalcand astfel perioada de restrictie electorala.
\end{enumerate}

\vspace{0.5cm}

\subsection{Partidul Alianta pentru Unirea Romanilor (AUR)}
Următoarele fapte contravenționale sunt sesizate împotriva acestei entități:

\begin{enumerate}[leftmargin=*, label=\arabic*.)]
    \item difuzarea de materiale de propaganda electorala (postare Facebook ID: \href{https://www.facebook.com/ads/library/?id=533765789578462}{533765789578462}) dupa incheierea perioadei de campanie electorala. Postarea, publicata si promovata ca reclama platita dupa ora 18:00 pe 23.11.2024, promoveaza explicit candidatul George Simion pentru functia de presedinte, continand elementele specifice materialelor de propaganda electorala (CMF 11240014), si fiind distribuita catre un public larg (30,000-35,000 impresii). Mesajul "George Simion presedinte!" reprezinta in mod clar un indemn electoral, care incalca prevederile legale privind perioada de campanie electorala.
\end{enumerate}

\vspace{0.5cm}

\subsection{Partidul DREPT, prin reprezentant Felicia Ovanesian}
Următoarele fapte contravenționale sunt sesizate împotriva acestei entități:

\begin{enumerate}[leftmargin=*, label=\arabic*.)]
    \item distribuirea unui mesaj sponsorizat pe Facebook (ID postare: \href{https://www.facebook.com/ads/library/?id=1016870357150374}{1016870357150374}) dupa ora 18:00 pe 23.11.2024, continand propaganda electorala referitoare la candidatii la presedintie Elena Lasconi, George Simion si Marcel Ciolacu, facand predictii si comentarii care pot influenta comportamentul electoral al votantilor in perioada in care campania electorala prezidentiala este incheiata. Mesajul a avut un impact semnificativ, atingand intre 40.000 si 45.000 de persoane, fiind marcat cu cod CMF 11240046, confirmand natura sa de comunicare electorala.
\end{enumerate}

\vspace{0.5cm}

\subsection{Partidul National Liberal}
Următoarele fapte contravenționale sunt sesizate împotriva acestei entități:

\begin{enumerate}[leftmargin=*, label=\arabic*.)]
    \item continuarea propagandei electorale pentru candidatul prezidential Nicolae Ciuca dupa incheierea perioadei de campanie, in postarea cu ID \href{https://www.facebook.com/ads/library/?id=594402326490969}{594402326490969}. Postarea, fiind publicata si activa dupa ora 18:00 pe 23.11.2024, promoveaza explicit candidatura la presedintie a lui Nicolae Ciuca ("Cu Nicolae Ciuca Presedinte"), constituind astfel propaganda electorala in afara perioadei legale. Postarea este marcata oficial ca propaganda electorala prin numarul CMF 11240002, este platita si targetata catre un public larg (1001-5000 persoane), avand un clar obiectiv electoral de influentare a votului pentru alegerile prezidentiale.
\end{enumerate}

\vspace{0.5cm}

\subsection{Partidul National Liberal (PNL) si Ucu Dima}
Următoarele fapte contravenționale sunt sesizate împotriva acestei entități:

\begin{enumerate}[leftmargin=*, label=\arabic*.)]
    \item promovarea unui material de propaganda electorala pentru candidatul prezidential Nicolae Ciuca (postare Facebook ID: \href{https://www.facebook.com/ads/library/?id=1652235849023143}{1652235849023143}) dupa incheierea perioadei de campanie electorala. Materialul contine in mod explicit indemnuri de vot si promovare electorala pentru candidatul la presedintie, fiind difuzat dupa ora 18:00 pe 23.11.2024, incalcand astfel prevederile legale privind incheierea campaniei electorale. Materialul include numar CMF (11240002), confirmand natura sa de propaganda electorala, si este promovat ca reclama platita pe platformele Facebook si Instagram.
\end{enumerate}

\vspace{0.5cm}

\subsection{Partidul National Liberal (PNL), prin intermediul paginii "Ucu Dima"}
Următoarele fapte contravenționale sunt sesizate împotriva acestei entități:

\begin{enumerate}[leftmargin=*, label=\arabic*.)]
    \item promovarea unui material de propaganda electorala (ID postare Facebook: \href{https://www.facebook.com/ads/library/?id=1306460553865972}{1306460553865972}) dupa incheierea perioadei de campanie pentru alegerile prezidentiale, respectiv dupa ora 18:00 pe 23.11.2024. Materialul il promoveaza explicit pe candidatul Nicolae Ciuca in contextul presedintiei, mentionand "Cu Nicolae Ciuca Presedinte", reprezentand astfel o continuare a propagandei electorale pentru alegerile prezidentiale dupa incheierea acesteia.
    \item continuarea propagandei electorale pentru candidatul prezidential Nicolae Ciuca dupa incheierea perioadei legale de campanie. Postarea cu ID-ul \href{https://www.facebook.com/ads/library/?id=1857303024677817}{1857303024677817} contine referiri directe la candidatura prezidentiala ("Cu Nicolae Ciuca Presedinte"), fiind promovata ca reclama platita pe Facebook si Instagram dupa ora 18:00 pe 23.11.2024. Materialul include numar CMF (11240002) si reprezinta propaganda electorala conform Art. 36(7) din Legea 334/2006, avand obiectiv electoral explicit si adresandu-se publicului larg.
\end{enumerate}

\vspace{0.5cm}

\subsection{Partidul National Liberal - filiala Botosani}
Următoarele fapte contravenționale sunt sesizate împotriva acestei entități:

\begin{enumerate}[leftmargin=*, label=\arabic*.)]
    \item difuzarea unui material de propaganda electorala (ID postare Facebook: \href{https://www.facebook.com/ads/library/?id=995712202316464}{995712202316464}) dupa incheierea perioadei de campanie electorala pentru alegerile prezidentiale. Postarea contine referinte directe la candidatul prezidential Nicolae-Ionel Ciuca, foloseste numar CMF (11240002), si are scop electoral explicit, fiind difuzata dupa ora 18:00 pe 23.11.2024. Materialul a avut un impact semnificativ, atingand intre 35.000 si 40.000 de impresii, fiind promovat ca reclama platita pe platformele Facebook si Instagram.
\end{enumerate}

\vspace{0.5cm}

\subsection{Partidul National Liberal Braila}
Următoarele fapte contravenționale sunt sesizate împotriva acestei entități:

\begin{enumerate}[leftmargin=*, label=\arabic*.)]
    \item promovarea unui material de propaganda electorala (ID postare Facebook: \href{https://www.facebook.com/ads/library/?id=397977850062132}{397977850062132}) dupa incheierea perioadei de campanie electorala. Postarea, efectuata dupa ora 18:00 pe 23.11.2024, contine indemnuri directe de vot pentru candidatul Nicolae Ciuca ("Votati un presedinte corect- pozitia 4, Nicolae Ciuca"), atacuri la adresa contracandidatului Marcel Ciolacu, si foloseste numarul oficial de material electoral CMF 11240002. Materialul a fost distribuit ca reclama platita pe Facebook si Instagram, cu un impact estimat intre 45,000 si 50,000 de afisari, reprezentand o incalcare clara a prevederilor legale privind incetarea propagandei electorale.
\end{enumerate}

\vspace{0.5cm}

\subsection{Partidul National Liberal Teleorman}
Următoarele fapte contravenționale sunt sesizate împotriva acestei entități:

\begin{enumerate}[leftmargin=*, label=\arabic*.)]
    \item continuarea propagandei electorale dupa incheierea campaniei, prin postarea cu ID \href{https://www.facebook.com/ads/library/?id=1294693224868664}{1294693224868664} pe Facebook, care indeamna explicit la votarea candidatului Nicolae Ciuca in data de 24 noiembrie ("Pe 24 noiembrie votam Nicolae Ciuca presedinte!"). Postarea este una platita, cu numar CMF 31240003, distribuita dupa ora 18:00 pe 23.11.2024, avand un impact estimat intre 100,001 si 500,000 de persoane, reprezentand astfel o incalcare clara a prevederilor legale privind incheierea campaniei electorale.
\end{enumerate}

\vspace{0.5cm}

\subsection{Partidul National Liberal si Ucu Dima}
Următoarele fapte contravenționale sunt sesizate împotriva acestei entități:

\begin{enumerate}[leftmargin=*, label=\arabic*.)]
    \item promovarea cu plata pe platformele Facebook si Instagram a unui mesaj de propaganda electorala pentru candidatul prezidential Nicolae Ciuca (postare cu ID \href{https://www.facebook.com/ads/library/?id=482379781056772}{482379781056772}) dupa incheierea perioadei de campanie electorala. Postarea contine un indemn explicit de sustinere a candidatului la presedintie, foloseste numar CMF (11240002), si este promovata dupa ora 18:00 pe 23.11.2024, incalcand astfel prevederile legale privind incheierea campaniei electorale. Materialul are caracter evident de propaganda electorala, fiind utilizat pentru influentarea votului si promovarea directa a candidatului.
    \item continuarea propagandei electorale pentru candidatul prezidential Nicolae Ciuca dupa incheierea campaniei electorale, prin postarea cu ID \href{https://www.facebook.com/ads/library/?id=487741760321294}{487741760321294} pe Facebook. Postarea contine materiale de propaganda electorala (conform Art. 36(7) din Legea 334/2006) fiind o reclama platita, cu CMF 11240002, care promoveaza explicit candidatul la presedintie si indeamna la vot, difuzata dupa ora 18:00 pe 23.11.2024. Impactul electoral este evident prin mesajul direct de sustinere si indemnul la vot pentru candidatul prezidential.
    \item promovarea unui mesaj electoral (ID Facebook: \href{https://www.facebook.com/ads/library/?id=516328028047409}{516328028047409}) ce contine referinte directe la candidatul prezidential Nicolae Ciuca, dupa incheierea perioadei de campanie electorala. Postarea, promovata dupa ora 18:00 pe 23.11.2024, include un CMF (11240002), contine indemnuri directe la vot si promoveaza candidatul la presedintie intr-o perioada in care propaganda electorala pentru alegerile prezidentiale este interzisa. Efectul electoral este evident prin prezentarea explicita a beneficiilor votarii PNL si a candidatului sau la presedintie, depasind cadrul legal al campaniei parlamentare in desfasurare.
    \item continuarea propagandei electorale dupa incheierea perioadei de campanie pentru alegerile prezidentiale, promovand candidatul Nicolae Ciuca intr-o postare platita pe Facebook (ID: \href{https://www.facebook.com/ads/library/?id=589233960303601}{589233960303601}) dupa ora 18:00 pe 23.11.2024. Postarea contine referinte directe la candidatul prezidential Nicolae Ciuca, il promoveaza in calitate de viitor presedinte, foloseste numar CMF (11240002), si reprezinta propaganda electorala conform Art. 36(7) prin referirea directa la candidat, obiectivul electoral explicit si adresarea catre publicul larg prin intermediul unei reclame platite pe platformele sociale.
\end{enumerate}

\vspace{0.5cm}

\subsection{Partidul National Liberal, prin intermediul paginii "Raluca Turcan"}
Următoarele fapte contravenționale sunt sesizate împotriva acestei entități:

\begin{enumerate}[leftmargin=*, label=\arabic*.)]
    \item difuzarea unei reclame platite pe Facebook (ID: \href{https://www.facebook.com/ads/library/?id=930222319052589}{930222319052589}) care continua propaganda electorala pentru candidatul prezidential Nicolae Ciuca dupa incheierea perioadei legale de campanie. Postarea, activa dupa ora 18:00 pe 23.11.2024, include elemente clare de propaganda electorala precum hashtag-ul \#NicolaeCiucaPresedinte, are numar CMF (11240002), si reprezinta un indemn explicit la vot ("Votul liberal este votul nostru natural!"). Postarea a fost promovata ca reclama platita, atingand intre 4000-4999 de utilizatori, reprezentand astfel o incercare deliberata de a influenta alegatorii in afara perioadei legale de campanie.
\end{enumerate}

\vspace{0.5cm}

\subsection{Partidul National Liberal, prin intermediul paginii "Ucu Dima"}
Următoarele fapte contravenționale sunt sesizate împotriva acestei entități:

\begin{enumerate}[leftmargin=*, label=\arabic*.)]
    \item difuzarea unui material de propaganda electorala (ID postare Facebook: \href{https://www.facebook.com/ads/library/?id=1069524538193933}{1069524538193933}) dupa incheierea perioadei de campanie electorala prezidentiala. Postarea promoveaza explicit candidatul Nicolae Ciuca pentru functia de presedinte, folosind expresii precum "il sustinem cu incredere pe Nicolae Ciuca" si "alegerea ideala pentru functia de Presedinte al Romaniei", intr-o postare platita, cu CMF 11240002, difuzata dupa ora 18:00 pe 23.11.2024. Materialul are caracter explicit de propaganda electorala, fiind destinat influentarii votului pentru alegerile prezidentiale, intr-o perioada in care aceasta activitate este interzisa prin lege.
    \item publicarea si promovarea platita a unui mesaj de propaganda electorala (ID postare Facebook: \href{https://www.facebook.com/ads/library/?id=1812516469285879}{1812516469285879}) dupa incheierea perioadei de campanie electorala prezidentiala. Postarea, marcata cu cod CMF 11240002, promoveaza explicit candidatul Nicolae Ciuca pentru functia de presedinte, folosind expresii precum "il sustinem cu incredere pe Nicolae Ciuca" si "alegerea ideala pentru functia de Presedinte al Romaniei", reprezentand astfel continuarea propagandei electorale dupa ora 18:00 pe 23.11.2024. Mesajul a fost promovat activ prin publicitate platita pe platformele Facebook si Instagram, avand un impact potential asupra a 1001-5000 de persoane.
    \item continuarea propagandei electorale pentru candidatul prezidential Nicolae Ciuca dupa incheierea perioadei de campanie. Postarea cu ID-ul \href{https://www.facebook.com/ads/library/?id=445998178606699}{445998178606699} reprezinta material de propaganda electorala platit, distribuit pe Facebook si Instagram dupa ora 18:00 pe 23.11.2024, care promoveaza explicit candidatul la presedintie Nicolae Ciuca, mentionandu-l ca "Nicolae Ciuca Presedinte". Postarea include numar CMF (11240002) si reprezinta in mod clar material de propaganda electorala conform Art. 36(7), avand obiectiv electoral explicit si adresandu-se publicului larg prin intermediul platformelor de social media.
\end{enumerate}

\vspace{0.5cm}

\subsection{Partidul National Liberal, prin reprezentantul Alin Calinescu}
Următoarele fapte contravenționale sunt sesizate împotriva acestei entități:

\begin{enumerate}[leftmargin=*, label=\arabic*.)]
    \item continuarea propagandei electorale dupa incheierea campaniei electorale pentru alegerile prezidentiale, prin postarea cu ID \href{https://www.facebook.com/ads/library/?id=1827516994652644}{1827516994652644} pe facebook. Postarea contine indemnuri directe la vot pentru candidatul Nicolae Ciuca ("Haideti la vot duminica, haideti sa trimitem in turul doi adevaratul candidat de dreapta: Nicolae Ciuca"), are caracter de propaganda electorala confirmat prin prezenta numarului CMF 11240002, si a fost difuzata dupa ora 18:00 pe 23.11.2024, incalcand astfel prevederile legale privind incetarea campaniei electorale.
\end{enumerate}

\vspace{0.5cm}

\subsection{Partidul Politic ALIANTA PENTRU UNIREA ROMANILOR}
Următoarele fapte contravenționale sunt sesizate împotriva acestei entități:

\begin{enumerate}[leftmargin=*, label=\arabic*.)]
    \item continuarea propagandei electorale pentru candidatul la presedintie George Simion dupa incheierea perioadei de campanie electorala, folosind hashtag-ul \#GeorgeSimionPresedinte si promovand explicit candidatura sa la presedintie in postarea cu ID-ul \href{https://www.facebook.com/ads/library/?id=1261056901921116}{1261056901921116}. Postarea este inca activa si promovata dupa ora 18:00 pe 23.11.2024, incalcand astfel prevederile legale privind incheierea campaniei electorale prezidentiale. Materialul constituie propaganda electorala conform Art. 36(7), avand CMF 11240014, referindu-se direct la candidat, avand obiectiv electoral explicit si adresandu-se publicului larg prin intermediul unei reclame platite pe Facebook.
    \item publicarea si promovarea unei reclame platite pe Facebook si Instagram (ID: \href{https://www.facebook.com/ads/library/?id=466425329793976}{466425329793976}) dupa ora 18:00 pe 23.11.2024, continand mesaje de propaganda electorala ("Mergeti la vot!" si acuzatii de frauda electorala). Postarea a avut un impact semnificativ, atingand intre 250.000 si 300.000 de persoane, cu o cheltuiala intre 1000-1499 RON, reprezentand o incercare clara de influentare a comportamentului electoral in ziua votului.
\end{enumerate}

\vspace{0.5cm}

\subsection{Partidul Politic ALIANTA PENTRU UNIREA ROMANILOR, prin Sergiu Mihalcea}
Următoarele fapte contravenționale sunt sesizate împotriva acestei entități:

\begin{enumerate}[leftmargin=*, label=\arabic*.)]
    \item publicarea si promovarea unei reclame pe Facebook (ID: \href{https://www.facebook.com/ads/library/?id=1714538599122156}{1714538599122156}) dupa incheierea campaniei electorale pentru alegerile prezidentiale. Postarea, publicata dupa ora 18:00 pe 23.11.2024, contine propaganda electorala explicita pentru candidatul prezidential George Simion, include indemnuri directe la vot ("votez George Simion!", "Votati AUR"), si foloseste numar CMF (11240014), demonstrand natura sa de propaganda electorala. Impactul a fost semnificativ, atingand intre 10.000 si 14.999 de persoane, fiind o incalcare clara a perioadei de restrictie electorala.
\end{enumerate}

\vspace{0.5cm}

\subsection{Partidul Social Democrat}
Următoarele fapte contravenționale sunt sesizate împotriva acestei entități:

\begin{enumerate}[leftmargin=*, label=\arabic*.)]
    \item publicarea si promovarea platita a unei postari cu ID-ul \href{https://www.facebook.com/ads/library/?id=1461437151191808}{1461437151191808} pe Facebook dupa ora 18:00 pe 23.11.2024, care promoveaza candidatul prezidential Marcel Ciolacu, folosind hashtag-ul campaniei si prezentand realizarile guvernamentale intr-un mod care influenteaza alegatorii. Postarea a avut un impact semnificativ, ajungand la peste 125.000 de persoane, si a continuat sa fie difuzata dupa incheierea perioadei de campanie electorala, incalcand astfel prevederile legale privind incheierea propagandei electorale.
    \item continuarea propagandei electorale dupa incheierea acesteia, promovand candidatul ION-MARCEL CIOLACU prin intermediul unei reclame platite pe Facebook (ID: \href{https://www.facebook.com/ads/library/?id=2863568223850289}{2863568223850289}). Postarea promoveaza realizarile si viziunea candidatului, folosind hashtag-uri de campanie si mesaje electorale, atingand un public tinta de peste 100.000 de persoane, dupa ora 18:00 pe 23.11.2024. Materialul indeplineste toate conditiile unui material de propaganda electorala conform Art. 36(7), referindu-se direct la candidat, avand obiectiv electoral si adresandu-se publicului larg.
\end{enumerate}

\vspace{0.5cm}

\subsection{Partidul Social Democrat - Organizatia Judeteana Bistrita-Nasaud}
Următoarele fapte contravenționale sunt sesizate împotriva acestei entități:

\begin{enumerate}[leftmargin=*, label=\arabic*.)]
    \item continuarea propagandei electorale pentru candidatul prezidential Marcel Ciolacu dupa incheierea perioadei de campanie, folosind o postare platita pe Facebook (ID: \href{https://www.facebook.com/ads/library/?id=595862286340426}{595862286340426}) ce indeamna explicit la vot pentru candidatul prezidential in datele de 24 noiembrie si 8 decembrie 2024. Postarea, activa dupa ora 18:00 pe 23.11.2024, are un caracter evident de propaganda electorala, confirmata prin prezenta codului CMF 11240017, si a ajuns la un public tinta estimat intre 100,001 si 500,000 de persoane, constituind astfel o incalcare clara a prevederilor legale privind incheierea campaniei electorale.
    \item continuarea propagandei electorale pentru candidatul prezidential Marcel Ciolacu dupa incheierea perioadei legale de campanie. Postarea cu ID-ul \href{https://www.facebook.com/ads/library/?id=603296632079168}{603296632079168} pe facebook contine indemnuri directe de vot ("Pe 24 noiembrie si 8 decembrie votam  Marcel Ciolacu, Presedinte al Romaniei!"), fiind o reclama platita care a continuat sa ruleze dupa ora 18:00 pe 23.11.2024, incalcand astfel prevederile legale privind incheierea campaniei electorale. Postarea include numar CMF (11240017), confirmand natura sa de propaganda electorala, si a avut un impact semnificativ, ajungand la un public tinta de peste 100.000 de persoane.
\end{enumerate}

\vspace{0.5cm}

\subsection{Partidul Social Democrat - Organizatia Judeteana Hunedoara}
Următoarele fapte contravenționale sunt sesizate împotriva acestei entități:

\begin{enumerate}[leftmargin=*, label=\arabic*.)]
    \item difuzarea unei reclame platite pe Facebook si Instagram (ID: \href{https://www.facebook.com/ads/library/?id=1540641106581455}{1540641106581455}) dupa ora 18:00 pe 23.11.2024, care indeamna in mod explicit la votarea candidatului ION-MARCEL CIOLACU la alegerile prezidentiale, utilizand mesajul "Pe 24 noiembrie si 8 decembrie \#votam Marcel Ciolacu - presedinte al Romaniei!". Reclama a fost distribuita catre un public tinta estimat intre 100,001 si 500,000 de persoane, reprezentand propaganda electorala activa in afara perioadei legale de campanie.
\end{enumerate}

\vspace{0.5cm}

\subsection{Partidul Social Democrat Bistrita-Nasaud}
Următoarele fapte contravenționale sunt sesizate împotriva acestei entități:

\begin{enumerate}[leftmargin=*, label=\arabic*.)]
    \item promovarea unui mesaj de propaganda electorala pentru candidatul la presedintie Marcel Ciolacu dupa incheierea perioadei de campanie electorala. Postarea cu ID-ul \href{https://www.facebook.com/ads/library/?id=1626528474611407}{1626528474611407} contine in mod explicit indemnul "Pe 24 noiembrie si 8 decembrie votam  Marcel Ciolacu, Presedinte al Romaniei!" si a fost distribuita ca reclama platita pe Facebook si Instagram dupa ora 18:00 pe 23.11.2024, incalcand astfel prevederile legale privind perioada de campanie electorala.
    \item publicarea si promovarea unei reclame platite pe Facebook (ID: \href{https://www.facebook.com/ads/library/?id=447243754691508}{447243754691508}) care contine propaganda electorala pentru candidatul prezidential Marcel Ciolacu dupa ora 18:00 pe 23.11.2024. Postarea contine indemnuri directe la vot ("Pe 24 noiembrie si 8 decembrie votam  Marcel Ciolacu, Presedinte al Romaniei!"), fiind difuzata ca reclama platita pe platformele Facebook si Instagram, cu un impact estimat intre 100,001 si 500,000 de persoane. Materialul este marcat cu cod CMF 11240017, confirmand natura sa de propaganda electorala.
    \item promovarea unui material electoral (ID Facebook: \href{https://www.facebook.com/ads/library/?id=480709598330522}{480709598330522}) ce contine indemnuri explicite de vot pentru candidatul prezidential Marcel Ciolacu ("Pe 24 noiembrie si 8 decembrie votam  Marcel Ciolacu, Presedinte al Romaniei!") dupa incheierea perioadei de campanie electorala. Materialul, distribuit ca reclama platita pe Facebook si Instagram, reprezinta propaganda electorala conform Art. 36(7), avand CMF 11240017, si continua sa fie activ si dupa ora 18:00 pe 23.11.2024, incalcand astfel prevederile legale privind incheierea campaniei electorale prezidentiale.
\end{enumerate}

\vspace{0.5cm}

\subsection{Partidul Social Democrat Bucuresti}
Următoarele fapte contravenționale sunt sesizate împotriva acestei entități:

\begin{enumerate}[leftmargin=*, label=\arabic*.)]
    \item promovarea unui mesaj electoral platit (ID postare Facebook: \href{https://www.facebook.com/ads/library/?id=1116163450109715}{1116163450109715}) dupa ora 18:00 pe 23.11.2024, care promoveaza indirect candidatul la presedintie Marcel Ciolacu si partidul PSD. Postarea contine numar CMF (11240017), foloseste hashtag-uri electorale (\#AlegeriPrezidentiale), si face referire directa la alegerile prezidentiale, continuand astfel propaganda electorala dupa incheierea perioadei legale de campanie. Mesajul a fost distribuit pe Facebook si Instagram, cu o audienta estimata intre 500,001 si 1,000,000 de persoane.
    \item publicarea si promovarea unei reclame pe Facebook (ID: \href{https://www.facebook.com/ads/library/?id=1263403118138230}{1263403118138230}) dupa ora 18:00 pe 23.11.2024, in care se face propaganda electorala pentru candidatul prezidential Marcel Ciolacu. Materialul contine numar CMF (11240017), prezinta viziunea si realizarile candidatului, fiind difuzat in perioada de interdictie a campaniei electorale pentru alegerile prezidentiale. Postarea este una platita, cu audienta estimata de peste 1 milion de persoane, avand clar scopul de a influenta alegatorii in favoarea candidatului PSD.
    \item promovarea unui mesaj de propaganda electorala pentru candidatul la presedintie Marcel Ciolacu dupa incheierea perioadei de campanie electorala. Postarea cu ID-ul \href{https://www.facebook.com/ads/library/?id=887823363487514}{887823363487514}, publicata si promovata dupa ora 18:00 pe 23.11.2024, contine mesaje clare de sustinere electorala precum "Echipa social-democrata, prin Marcel Ciolacu la presedintia Romaniei [...] sunt calea sigura pentru Bucuresti si Romania", constituind astfel propaganda electorala interzisa in aceasta perioada. Postarea este sponsorizata si targetata catre un public larg (peste 1 milion de persoane potentiale), avand un caracter explicit de propaganda electorala, demonstrat si prin prezenta numarului CMF 11240017.
\end{enumerate}

\vspace{0.5cm}

\subsection{Partidul Social Democrat Buzau}
Următoarele fapte contravenționale sunt sesizate împotriva acestei entități:

\begin{enumerate}[leftmargin=*, label=\arabic*.)]
    \item promovarea candidatului Ion-Marcel Ciolacu intr-o postare sponsorizata pe Facebook (ID: \href{https://www.facebook.com/ads/library/?id=1037871088116062}{1037871088116062}) dupa ora 18:00 pe 23.11.2024. Postarea constituie propaganda electorala conform Art. 36(7) prin: referirea directa la candidat, utilizarea in perioada campaniei, obiectivul electoral clar de influentare a votantilor, si depasirea limitelor activitatii jurnalistice. Materialul include CMF 11240017, confirmand natura sa de propaganda electorala, si a avut un impact semnificativ, atingand intre 40.000-45.000 de impresii.
    \item distribuirea de materiale de propaganda electorala dupa incheierea perioadei de campanie, prin postarea cu ID \href{https://www.facebook.com/ads/library/?id=2274374652938316}{2274374652938316} pe Facebook. Materialul promoveaza candidatul prezidential Marcel Ciolacu, folosind simboluri de partid si sloganuri electorale, fiind distribuit ca reclama platita dupa ora 18:00 pe 23.11.2024, cu un impact estimat intre 10,000 si 14,999 de persoane. Materialul contine cod unic CMF 11240017, confirmat ca material de propaganda electorala.
\end{enumerate}

\vspace{0.5cm}

\subsection{Partidul Social Democrat Buzau prin MRC SHOW LED SRL}
Următoarele fapte contravenționale sunt sesizate împotriva acestei entități:

\begin{enumerate}[leftmargin=*, label=\arabic*.)]
    \item difuzarea de materiale de propaganda electorala (ID postare Facebook: \href{https://www.facebook.com/ads/library/?id=580083777794224}{580083777794224}) dupa incheierea perioadei de campanie electorala. Materialul, publicat si promovat dupa ora 18:00 pe 23.11.2024, contine elementele specifice propagandei electorale: CMF 11240017, vizeaza in mod direct un candidat la presedintie (Mircea Geoana), are obiectiv electoral explicit de influentare a votului ("Cine vrea inca un presedinte..."), si este distribuit contra cost catre un public larg (100,001-500,000 persoane) pe platformele Facebook si Instagram.
\end{enumerate}

\vspace{0.5cm}

\subsection{Partidul Social Democrat Org. Jud. Botosani}
Următoarele fapte contravenționale sunt sesizate împotriva acestei entități:

\begin{enumerate}[leftmargin=*, label=\arabic*.)]
    \item continuarea propagandei electorale dupa incheierea acesteia, prin publicarea si mentinerea activa a unei reclame pe Facebook (ID: \href{https://www.facebook.com/ads/library/?id=1101884608608594}{1101884608608594}) care face propaganda electorala negativa impotriva candidatului Nicolae-Ionel Ciuca, dupa ora 18:00 pe 23.11.2024. Postarea contine numar CMF (11240017), are scop electoral explicit prin indemnul direct legat de vot ("Fiecare vot pentru Ciuca si PNL este un vot pentru Guvernul Iohannis!"), si a fost difuzata ca reclama platita pe platformele Facebook si Instagram, cu un impact estimat intre 15.000 si 19.999 de afisari.
    \item difuzarea de materiale de propaganda electorala dupa incheierea campaniei electorale pentru alegerile prezidentiale. Postarea cu ID-ul \href{https://www.facebook.com/ads/library/?id=1651224968820842}{1651224968820842} reprezinta propaganda electorala negativa la adresa candidatului Nicolae-Ionel Ciuca, fiind difuzata dupa ora 18:00 pe 23.11.2024, contine numar CMF (11240017), este platita si distribuita pe Facebook si Instagram, avand un impact estimat intre 15.000 si 19.999 de afisari. Mesajul are caracter electoral explicit, vizand sa influenteze negativ opinia alegatorilor fata de candidatul PNL la presedintie.
    \item publicarea si promovarea unei reclame platite pe Facebook (ID: \href{https://www.facebook.com/ads/library/?id=2011678159255070}{2011678159255070}) dupa ora 18:00 pe 23.11.2024, continand propaganda electorala care promoveaza in mod explicit realizarile guvernului condus de candidatul prezidential Marcel Ciolacu si partidul PSD. Postarea contine numar CMF (11240017), are caracter electoral explicit, a fost promovata cu sume semnificative (800-899 RON) si a avut un impact larg (100,000-124,999 impresii), reprezentand astfel o continuare clara a propagandei electorale dupa incheierea perioadei legale.
    \item promovarea unui material de propaganda electorala (ID Facebook: \href{https://www.facebook.com/ads/library/?id=578469851326101}{578469851326101}) dupa ora 18:00 pe 23.11.2024, material care il promoveaza pe candidatul prezidential Marcel Ciolacu prin evidentierea realizarilor sale ca premier si promisiuni electorale. Materialul contine numar CMF (11240017), are caracter electoral explicit, si a fost distribuit ca reclama platita pe Facebook si Instagram, cu un impact intre 30.000 si 35.000 de afisari, demonstrand intentia clara de a influenta alegatorii in perioada in care propaganda electorala este interzisa prin lege.
\end{enumerate}

\vspace{0.5cm}

\subsection{Partidul Uniunea Salvati Romania - Buzau}
Următoarele fapte contravenționale sunt sesizate împotriva acestei entități:

\begin{enumerate}[leftmargin=*, label=\arabic*.)]
    \item promovarea pe Facebook a unui material de propaganda electorala (ID postare: \href{https://www.facebook.com/ads/library/?id=2283471848692348}{2283471848692348}) dupa incheierea perioadei de campanie electorala. Materialul promoveaza explicit candidatul ELENA-VALERICA LASCONI la functia de presedinte, prezinta programul electoral si indeamna direct la vot pentru data de 24 noiembrie. Postarea a fost promovata ca reclama platita, avand un impact semnificativ (175,000-199,999 impresii), dupa ora 18:00 pe 23.11.2024, incalcand astfel prevederile legale privind incheierea campaniei electorale.
\end{enumerate}

\vspace{0.5cm}

\subsection{Patridul National Liberal Teleorman}
Următoarele fapte contravenționale sunt sesizate împotriva acestei entități:

\begin{enumerate}[leftmargin=*, label=\arabic*.)]
    \item publicarea si promovarea unui mesaj de propaganda electorala (ID postare Facebook: \href{https://www.facebook.com/ads/library/?id=605488808500280}{605488808500280}) dupa incheierea perioadei de campanie, respectiv dupa ora 18:00 pe 23.11.2024. Postarea contine numar CMF (11240002), reprezinta material electoral platit cu impact intre 15.000-19.999 impresii, si indeamna in mod direct la participarea la vot prin mesajul "Viitorul Romaniei nu se decide in sondaje, ci la vot!", constituind astfel propaganda electorala activa in ziua votului.
\end{enumerate}

\vspace{0.5cm}

\subsection{Peter Costea}
Următoarele fapte contravenționale sunt sesizate împotriva acestei entități:

\begin{enumerate}[leftmargin=*, label=\arabic*.)]
    \item difuzarea de materiale de propaganda electorala dupa incheierea campaniei electorale, constand intr-o reclama platita pe Facebook (ID: \href{https://www.facebook.com/ads/library/?id=3859478547647884}{3859478547647884}) care indeamna explicit la votarea candidatului George Simion ("De ce si eu voi vota pentru GEORGE SIMION") si discrediteaza candidatul Cristian Terhes. Materialul propagandistic este difuzat dupa ora 18:00 pe 23.11.2024, incalcand explicit prevederile legale privind incheierea campaniei electorale. Reclama este configurata sa ajunga la un public tinta de 100,001-500,000 de persoane, demonstrand intentia clara de influentare a votului.
\end{enumerate}

\vspace{0.5cm}

\subsection{Platforma 5news.RO}
Următoarele fapte contravenționale sunt sesizate împotriva acestei entități:

\begin{enumerate}[leftmargin=*, label=\arabic*.)]
    \item difuzarea unui mesaj de propaganda electorala negativa dupa incheierea perioadei de campanie, vizand direct candidatii prezidentiali Nicolae-Ionel Ciuca si Ion-Marcel Ciolacu. Postarea platita, cu ID-ul \href{https://www.facebook.com/ads/library/?id=485554463896332}{485554463896332}, a fost difuzata pe Facebook si Instagram dupa ora 18:00 pe 23.11.2024, continand mesajul "GUVERNELE CIUCA SI CIOLACU INGROAPA AGRICULTURA ROMANEASCA", reprezentand propaganda electorala negativa ce poate influenta decizia de vot a cetatenilor. Mesajul a avut un impact semnificativ, atingand intre 10.000 si 14.999 de afisari.
\end{enumerate}

\vspace{0.5cm}

\subsection{PresaLibera.ro}
Următoarele fapte contravenționale sunt sesizate împotriva acestei entități:

\begin{enumerate}[leftmargin=*, label=\arabic*.)]
    \item difuzarea de materiale de propaganda electorala dupa incheierea perioadei de campanie, constand in promovarea platita pe platformele Facebook si Instagram a unui mesaj ce vizeaza sansele candidatului George Simion in alegerile prezidentiale. Postarea cu ID-ul \href{https://www.facebook.com/ads/library/?id=1374414093939771}{1374414093939771} a fost difuzata dupa ora 18:00 pe 23.11.2024, incalcand explicit prevederile legale privind incheierea campaniei electorale. Mesajul are caracter electoral explicit, fiind distribuit catre un public tinta estimat la peste 1 milion de persoane, cu un buget intre 300-399 RON si generand intre 35.000-40.000 de afisari.
    \item publicarea si promovarea platita a unui mesaj de propaganda electorala dupa incheierea perioadei de campanie, referitor la candidatul prezidential Elena-Valerica Lasconi. Postarea cu ID-ul \href{https://www.facebook.com/ads/library/?id=1635942760687821}{1635942760687821} a fost promovata dupa ora 18:00 pe 23.11.2024, atingand intre 50.000 si 59.999 de persoane, cu o investitie intre 400 si 499 RON. Mesajul are caracter electoral explicit, incercand sa influenteze comportamentul de vot al alegatorilor prin asocieri negative si afirmatii tendentioase despre candidat si partidele politice implicate.
    \item promovarea unui material de propaganda electorala dupa incheierea campaniei electorale, constand intr-o postare platita pe Facebook (ID: \href{https://www.facebook.com/ads/library/?id=2062388527550424}{2062388527550424}) care promoveaza sansele electorale ale candidatului George Simion. Materialul, difuzat dupa ora 18:00 pe 23.11.2024, foloseste markeri emotionali si prezinta explicit sansele electorale ale candidatului, avand un impact semnificativ cu peste 80.000 de afisari si o audienta estimata de peste 1 milion de persoane. Cheltuielile de promovare intre 200-299 RON demonstreaza intentia clara de propaganda electorala in afara perioadei legale.
\end{enumerate}

\vspace{0.5cm}

\subsection{Rafael Nichita}
Următoarele fapte contravenționale sunt sesizate împotriva acestei entități:

\begin{enumerate}[leftmargin=*, label=\arabic*.)]
    \item continuarea propagandei electorale dupa incheierea campaniei electorale pentru alegerile prezidentiale, concretizata prin postarea cu ID \href{https://www.facebook.com/ads/library/?id=445644034932892}{445644034932892} pe facebook. Postarea contine indemnuri directe de a vota candidatul Nicolae Ciuca, foloseste numar CMF (11240002), este promovata ca reclama platita si vizeaza in mod explicit influentarea votului pentru alegerile prezidentiale dupa ora 18:00 pe 23.11.2024. Postarea este activa si distribuite in continuare prin platformele Facebook si Instagram, cu un buget de promovare intre 300-399 RON si un impact estimat intre 10,000-14,999 de afisari.
\end{enumerate}

\vspace{0.5cm}

\subsection{Raluca Giorgiana Dumitrescu}
Următoarele fapte contravenționale sunt sesizate împotriva acestei entități:

\begin{enumerate}[leftmargin=*, label=\arabic*.)]
    \item continuarea propagandei electorale dupa incheierea campaniei electorale, promovand in mod activ candidatul ION-MARCEL CIOLACU la functia de presedinte prin intermediul unei reclame platite pe Facebook (ID: \href{https://www.facebook.com/ads/library/?id=562233786388115}{562233786388115}) care indeamna explicit la alegerea acestuia ca presedinte ("Alegem Marcel Ciolacu - Presedintele Romaniei!"). Postarea, fiind activa si promovata dupa ora 18:00 pe 23.11.2024, incalca in mod direct prevederile legale privind incetarea campaniei electorale. Materialul contine elemente clare de propaganda electorala, inclusiv numar de mandatar financiar (CUI 11240017), si a avut un impact semnificativ, atingand intre 10.000 si 14.999 de afisari.
\end{enumerate}

\vspace{0.5cm}

\subsection{Razvan Biro}
Următoarele fapte contravenționale sunt sesizate împotriva acestei entități:

\begin{enumerate}[leftmargin=*, label=\arabic*.)]
    \item promovarea unui mesaj electoral explicit pe Facebook si Instagram (ID post: \href{https://www.facebook.com/ads/library/?id=921810482745028}{921810482745028}) dupa incheierea perioadei de campanie electorala. Postarea contine un indemn direct de a vota candidatul George Simion la alegerile prezidentiale ("Duminica votam George Simion presedinte!"), fiind realizata si promovata dupa ora 18:00 pe 23.11.2024, cu un impact semnificativ (20,000-24,999 impresii). Mesajul include numar CMF (11240014), confirmand natura sa de propaganda electorala, si reprezinta o incalcare clara a interdictiei de propaganda electorala dupa incheierea campaniei.
\end{enumerate}

\vspace{0.5cm}

\subsection{Razvan Biro si AUR}
Următoarele fapte contravenționale sunt sesizate împotriva acestei entități:

\begin{enumerate}[leftmargin=*, label=\arabic*.)]
    \item continuarea propagandei electorale dupa incheierea campaniei electorale pentru alegerile prezidentiale, prin promovarea unui material electoral platit pe Facebook (ID: \href{https://www.facebook.com/ads/library/?id=1289695989147162}{1289695989147162}) care face referire directa la candidatul George Simion si programul sau electoral ("Planul Simion"), contine numar CMF (11240014), prezinta promisiuni electorale specifice privind reducerea taxelor si indeamna explicit la vot pentru data de 24 noiembrie. Materialul a fost promovat dupa ora 18:00 pe 23.11.2024, incalcand astfel perioada legala de campanie electorala.
\end{enumerate}

\vspace{0.5cm}

\subsection{RedNews si Oancea Silviu}
Următoarele fapte contravenționale sunt sesizate împotriva acestei entități:

\begin{enumerate}[leftmargin=*, label=\arabic*.)]
    \item distribuirea unei reclame platite pe Facebook (ID: \href{https://www.facebook.com/ads/library/?id=1251511855972536}{1251511855972536}) ce promoveaza candidatul George Nicolae Simion si partidul AUR, continand mesaje de propaganda electorala ("Voteaza \#SIMION") si elemente din programul electoral, dupa incheierea perioadei legale de campanie electorala (dupa ora 18:00 pe 23.11.2024). Reclama a avut un impact semnificativ, atingand intre 5000-5999 de utilizatori, reprezentand o incalcare clara a prevederilor legale privind continuarea propagandei electorale dupa incheierea campaniei.
\end{enumerate}

\vspace{0.5cm}

\subsection{Ropres.ro}
Următoarele fapte contravenționale sunt sesizate împotriva acestei entități:

\begin{enumerate}[leftmargin=*, label=\arabic*.)]
    \item distribuirea de continut platit pe platformele Facebook si Instagram (ID postare: \href{https://www.facebook.com/ads/library/?id=1003904621501720}{1003904621501720}) dupa ora 18:00 pe 23.11.2024, continuand propaganda electorala dupa incheierea campaniei. Postarea vizeaza direct candidatii ELENA-VALERICA LASCONI si NICOLAE-IONEL CIUCA, prezentand in mod negativ sansele lor electorale, cu intentia clara de a influenta comportamentul electoral al votantilor. Continutul platit a avut o audienta estimata de peste 1 milion de persoane si a generat intre 8.000 si 8.999 de afisari, demonstrand impactul semnificativ al acestei incalcari.
    \item difuzarea de propaganda electorala dupa incheierea campaniei electorale, constand intr-o postare sponsorizata pe Facebook si Instagram (ID: \href{https://www.facebook.com/ads/library/?id=539308318987599}{539308318987599}) care il vizeaza direct si in mod negativ pe candidatul prezidential Marcel Ciolacu. Postarea, difuzata dupa ora 18:00 pe 23.11.2024, foloseste un limbaj tendentios si acuzatii care au scopul clar de a influenta negativ opinia alegatorilor fata de candidat, atingand un public estimat de peste 1 milion de persoane. Materialul depaseste limitele activitatii jurnalistice obiective si reprezinta propaganda electorala explicita intr-o perioada in care aceasta este interzisa prin lege.
\end{enumerate}

\vspace{0.5cm}

\subsection{Rus Marinel}
Următoarele fapte contravenționale sunt sesizate împotriva acestei entități:

\begin{enumerate}[leftmargin=*, label=\arabic*.)]
    \item promovarea unui mesaj de propaganda electorala (ID Facebook: 28450776881176277) dupa ora 18:00 pe 23.11.2024, continand indemnuri explicite de vot pentru candidatul USR Elena Lasconi la alegerile prezidentiale. Materialul contine numar CMF (11240015), mesaje directe de sustinere electorala, strategii de vot ("Turul 1 = Turul decisiv!") si indemnuri clare pentru mobilizarea la vot, fiind distribuit ca reclama platita pe Facebook si Instagram, cu un impact estimat intre 1000-1999 de afisari.
\end{enumerate}

\vspace{0.5cm}

\subsection{SALAMINA PRINT SRL}
Următoarele fapte contravenționale sunt sesizate împotriva acestei entități:

\begin{enumerate}[leftmargin=*, label=\arabic*.)]
    \item publicarea si promovarea pe Facebook a unui mesaj de propaganda electorala (ID postare: \href{https://www.facebook.com/ads/library/?id=952853273557827}{952853273557827}) dupa incheierea perioadei de campanie electorala pentru alegerile prezidentiale. Postarea contine referiri directe la candidatul prezidential Ludovic Orban, indemnuri explicite la vot ("Pe 1 decembrie votati Forta Dreptei!"), si este marcata cu numar CMF (11240022), confirmand natura sa de propaganda electorala. Postarea este activa si promovata dupa ora 18:00 pe 23.11.2024, incalcand astfel prevederile legale privind incheierea campaniei electorale.
\end{enumerate}

\vspace{0.5cm}

\subsection{SC EURO MEDIA BIS SRL}
Următoarele fapte contravenționale sunt sesizate împotriva acestei entități:

\begin{enumerate}[leftmargin=*, label=\arabic*.)]
    \item difuzarea de materiale de propaganda electorala (postare Facebook ID: \href{https://www.facebook.com/ads/library/?id=1080818846619500}{1080818846619500}) dupa incheierea campaniei electorale. Materialul promovat contine indemnuri directe la vot ("Duminica vom face primul pas pe acest drum dand un VOT\_UTIL pentru Romania!"), foloseste numar CMF (11240002), si promoveaza candidatul PNL la presedintie. Postarea a fost promovata si dupa ora 18:00 pe 23.11.2024, incalcand astfel prevederile legale privind incheierea campaniei electorale. Impactul postarii este semnificativ, atingand intre 15.000 si 19.999 de persoane, cu o investitie intre 300-399 RON.
    \item continuarea propagandei electorale dupa incheierea campaniei electorale pentru alegerile prezidentiale, prin postarea cu ID \href{https://www.facebook.com/ads/library/?id=1806660900143980}{1806660900143980} pe platforma Facebook. Postarea, publicata si promovata dupa ora 18:00 pe 23.11.2024, contine mesaje explicite de propaganda electorala, criticand candidatii Marcel Ciolacu, Elena Lasconi si George Simion, in timp ce promoveaza candidatul Nicolae Ciuca, folosind numarul oficial de campanie CMF 11240002. Postarea include indemnuri directe la vot si caracterizari ale candidatilor, incalcand astfel prevederile legale privind incheierea campaniei electorale.
    \item difuzarea unui mesaj de propaganda electorala (ID Facebook: \href{https://www.facebook.com/ads/library/?id=3997343703873698}{3997343703873698}) dupa ora 18:00 pe 23.11.2024, referitor la alegerile prezidentiale. Mesajul face referire directa la candidatul prezidential George Nicolae Simion si posibilitatea intrarii acestuia in turul doi, constituind astfel propaganda electorala in afara perioadei permise. Materialul este platit si distribuit pe platformele Facebook si Instagram, cu un impact semnificativ (30.000-35.000 impresii), continand si numar CMF (11240022), demonstrand natura sa de propaganda electorala.
    \item difuzarea de materiale de propaganda electorala (ID postare Facebook: \href{https://www.facebook.com/ads/library/?id=752286993767980}{752286993767980}) dupa incheierea perioadei de campanie electorala. Materialul promovat reprezinta propaganda electorala explicita in favoarea candidatului Nicolae Ciuca (PNL) la presedintie, continand mesaje care indeamna direct la sustinerea acestuia si critici la adresa contracandidatilor, fiind difuzat dupa ora 18:00 pe 23.11.2024. Postarea include numar CMF (11240002), confirmand natura sa de material de propaganda electorala, si a fost distribuita ca reclama platita pe Facebook si Instagram, cu un impact semnificativ (15,000-20,000 impresii).
\end{enumerate}

\vspace{0.5cm}

\subsection{SC NESTAL ALLNEGREA SRL}
Următoarele fapte contravenționale sunt sesizate împotriva acestei entități:

\begin{enumerate}[leftmargin=*, label=\arabic*.)]
    \item publicarea si promovarea unui mesaj electoral referitor la candidatul prezidential Ludovic Orban dupa incheierea perioadei de campanie electorala. Postarea cu ID-ul \href{https://www.facebook.com/ads/library/?id=924450862487265}{924450862487265} constituie propaganda electorala conform Art. 36(7) prin referirea directa la candidatul prezidential, avand obiectiv electoral explicit si adresandu-se publicului larg prin intermediul unei reclame platite pe Facebook. Postarea a fost activa dupa ora 18:00 pe 23.11.2024, incalcand astfel prevederile legale privind incheierea campaniei electorale prezidentiale.
\end{enumerate}

\vspace{0.5cm}

\subsection{SC VIDEO DANCRYS-STUDIO S.R.L}
Următoarele fapte contravenționale sunt sesizate împotriva acestei entități:

\begin{enumerate}[leftmargin=*, label=\arabic*.)]
    \item promovarea unui mesaj de propaganda electorala dupa incheierea perioadei legale de campanie, constand intr-o postare platita pe Facebook (ID: \href{https://www.facebook.com/ads/library/?id=1328124568153532}{1328124568153532}) care indeamna explicit la votarea candidatului George Simion la alegerile prezidentiale ("La alegerile prezidentiale, votam George Simion"). Postarea a fost activa dupa ora 18:00 pe 23.11.2024, avand un impact semnificativ demonstrat prin numarul de afisari (4000-4999 impresii) si audienta tintita (500,001-1,000,000 persoane). Mesajul contine elemente clare de propaganda electorala, incluzand indemnuri directe la vot si promovarea candidatului.
    \item promovarea unui mesaj de propaganda electorala pentru candidatul George Nicolae Simion (AUR) dupa incheierea perioadei de campanie electorala. Postarea cu ID \href{https://www.facebook.com/ads/library/?id=3822009734713369}{3822009734713369} reprezinta un anunt platit pe Facebook si Instagram, care promoveaza explicit candidatul la presedintie George Simion prin hashtag-uri precum \#GeorgeSimionPresedinte si continut care il prezinta intr-o lumina favorabila in contextul alegerilor prezidentiale. Postarea a fost activa si distribuita dupa ora 18:00 pe 23.11.2024, incalcand astfel prevederile legale privind incetarea campaniei electorale.
    \item difuzarea de materiale de propaganda electorala dupa incheierea campaniei electorale. Postarea cu ID-ul \href{https://www.facebook.com/ads/library/?id=451750557962158}{451750557962158} pe Facebook promoveaza explicit candidatul George Simion la functia de presedinte ("George Simion Presedinte!"), contine numar CMF (31240006), este material publicitar platit cu impact intre 10.000-14.999 afisari, difuzat pe Facebook si Instagram dupa ora 18:00 pe 23.11.2024, cand campania electorala era incheiata legal. Materialul are obiectiv electoral clar si vizeaza influentarea votului pentru candidatul AUR la presedintie.
    \item promovarea unui mars electoral in sprijinul candidatului George Simion prin intermediul unei reclame platite pe Facebook si Instagram (ID: \href{https://www.facebook.com/ads/library/?id=546433941508702}{546433941508702}) dupa ora 18:00 pe 23.11.2024. Reclama are caracter electoral explicit, promovand un mars de sustinere pentru candidatul la presedintie si utilizand hashtag-uri electorale (\#georgesimionpresedinte). Continutul a fost distribuit catre un public larg estimat intre 500.001 si 1.000.000 de persoane, avand un impact direct asupra procesului electoral in perioada in care propaganda electorala este interzisa prin lege.
\end{enumerate}

\vspace{0.5cm}

\subsection{SC VIDEO DANCRYS-STUDIO S.R.L.}
Următoarele fapte contravenționale sunt sesizate împotriva acestei entități:

\begin{enumerate}[leftmargin=*, label=\arabic*.)]
    \item continuarea propagandei electorale dupa incheierea campaniei, manifestata prin publicarea si promovarea platita a unui mesaj electoral explicit in favoarea candidatului GEORGE-NICOLAE SIMION (postarea Facebook ID: \href{https://www.facebook.com/ads/library/?id=528827646645998}{528827646645998}). Materialul promovat contine mesaje directe de sustinere electorala ("va fi presedintele care va reconstrui Romania"), hashtag-uri de campanie (\#GeorgeSimionPresedinte) si este difuzat dupa ora 18:00 pe 23.11.2024, intr-un moment in care propaganda electorala este interzisa prin lege. Impactul este amplificat prin natura platita a postarii, care a generat intre 35.000 si 40.000 de afisari.
\end{enumerate}

\vspace{0.5cm}

\subsection{SC VIDEO DANCRYS-STUDIO SRL}
Următoarele fapte contravenționale sunt sesizate împotriva acestei entități:

\begin{enumerate}[leftmargin=*, label=\arabic*.)]
    \item promovarea unui mesaj de propaganda electorala pentru candidatul GEORGE-NICOLAE SIMION (AUR) dupa incheierea perioadei de campanie electorala. Postarea cu ID-ul \href{https://www.facebook.com/ads/library/?id=1136207171352012}{1136207171352012} este un anunt platit pe Facebook si Instagram, cu un buget de 100-199 RON, care a ajuns la 10.000-14.999 de persoane, continand mesajul explicit "George Simion Presedinte" insotit de CMF 31240006, hashtag-uri electorale si simboluri nationale, constituind astfel propaganda electorala activa dupa ora 18:00 pe 23.11.2024.
    \item difuzarea de materiale de propaganda electorala dupa incheierea perioadei de campanie, constand intr-o postare platita pe Facebook si Instagram (ID: \href{https://www.facebook.com/ads/library/?id=1146734783545268}{1146734783545268}) care promoveaza direct candidatul GEORGE-NICOLAE SIMION prin organizarea unui mars de sustinere si utilizarea hashtag-urilor de campanie (\#georgesimionpresedinte). Postarea a fost activa dupa ora 18:00 pe 23.11.2024, atingand intre 5.000 si 5.999 de afisari, avand un impact electoral direct prin mobilizarea populatiei intr-o actiune politica explicita de sustinere a unui candidat la presedintie.
    \item difuzarea de materiale de propaganda electorala pentru candidatul GEORGE-NICOLAE SIMION (AUR) dupa incheierea perioadei de campanie electorala. Postarea cu ID-ul \href{https://www.facebook.com/ads/library/?id=3801991510054994}{3801991510054994} pe facebook contine mesajul "George Simion Presedinte!" si foloseste CMF 31240006, fiind o reclama platita cu impact intre 10.000-14.999 de afisari, difuzata dupa ora 18:00 pe 23.11.2024. Materialul reprezinta propaganda electorala conform Art. 36(7) fiind direct referitor la un candidat, avand obiectiv electoral explicit si adresandu-se publicului larg prin intermediul platformelor de social media.
    \item promovarea unui mars de sustinere pentru candidatul prezidential George Simion dupa incheierea perioadei de campanie electorala. Postarea cu ID-ul \href{https://www.facebook.com/ads/library/?id=568354982553353}{568354982553353} reprezinta propaganda electorala explicita prin organizarea si promovarea unui eveniment de sustinere a candidatului, folosind hashtag-uri precum \#georgesimionpresedinte si promovand un mars electoral. Reclama a fost activa si dupa ora 18:00 pe 23.11.2024, incalcand astfel prevederile legale privind incheierea campaniei electorale. Impactul a fost semnificativ, atingand intre 500.001 si 1.000.000 de persoane pe platformele Facebook si Instagram.
    \item difuzarea de propaganda electorala dupa incheierea perioadei de campanie, respectiv dupa ora 18:00 pe 23.11.2024. Postarea cu ID-ul \href{https://www.facebook.com/ads/library/?id=579583984614765}{579583984614765} contine indemnuri explicite de vot pentru candidatul George Simion la alegerile prezidentiale ("La alegerile prezidentiale, votam George Simion"), fiind promovata activ prin reclama platita pe platformele Facebook si Instagram, cu un reach estimat intre 500.001 si 1.000.000 de utilizatori. Mesajul are un evident caracter de propaganda electorala, folosind hashtag-uri electorale si indemnuri directe la vot pentru un candidat specific.
    \item promovarea unui material de propaganda electorala pentru candidatul GEORGE-NICOLAE SIMION dupa incheierea perioadei de campanie electorala. Materialul, cu ID-ul \href{https://www.facebook.com/ads/library/?id=584647100592642}{584647100592642} pe Facebook, promoveaza un mars de sustinere pentru candidat, foloseste hashtag-uri electorale (\#georgesimionpresedinte) si are un obiectiv electoral clar, fiind difuzat dupa ora 18:00 pe 23.11.2024. Postarea este sponsorizata si are un impact estimat intre 500,001 si 1,000,000 de persoane, demonstrand intentia clara de a influenta alegatorii in perioada in care propaganda electorala este interzisa prin lege.
    \item difuzarea de continut electoral pentru candidatul GEORGE-NICOLAE SIMION (AUR) dupa incheierea perioadei de campanie electorala. Materialul publicitar (ID Facebook: \href{https://www.facebook.com/ads/library/?id=908952204223735}{908952204223735}) contine mesaje directe de sustinere ("George Simion Presedinte!"), are numar CMF (31240006), si a fost distribuit ca reclama platita pe Facebook si Instagram, cu un buget intre 100-199 RON si o audienta estimata intre 500,001-1,000,000 persoane, dupa ora 18:00 pe 23.11.2024. Materialul reprezinta in mod clar propaganda electorala conform Art. 36(7) din Legea 334/2006, avand obiectiv electoral explicit si adresandu-se publicului larg.
    \item difuzarea de materiale de propaganda electorala (ID postare Facebook: \href{https://www.facebook.com/ads/library/?id=943728441010897}{943728441010897}) dupa incheierea perioadei de campanie electorala. Materialul, care include numarul CMF 11240014, promoveaza explicit candidatul George Simion pentru alegerile prezidentiale, fiind difuzat ca reclama platita pe Facebook si Instagram, cu impact potential intre 500,001 si 1,000,000 de persoane, dupa ora 18:00 pe 23.11.2024. Continutul postarii indica clar intentia de propaganda electorala prin mesajul "il sustin pe George Simion pentru alegerile prezidentiale din 2024" si folosirea hashtag-urilor relevante pentru campania electorala.
\end{enumerate}

\vspace{0.5cm}

\subsection{Sabin Sarmas}
Următoarele fapte contravenționale sunt sesizate împotriva acestei entități:

\begin{enumerate}[leftmargin=*, label=\arabic*.)]
    \item difuzarea unui mesaj electoral platit (ID Facebook: \href{https://www.facebook.com/ads/library/?id=1789161281900055}{1789161281900055}) care face referire directa la candidatul prezidential Elena Lasconi si strategia sa electorala, dupa incheierea perioadei de campanie electorala (dupa ora 18:00 pe 23.11.2024). Postarea, marcata cu cod CMF 14240010, constituie propaganda electorala conform Art. 36(7) prin referirea directa la un candidat prezidential si obiectivul sau electoral explicit, fiind difuzata catre un public larg estimat intre 100.001-500.000 de persoane prin platformele Facebook si Instagram.
\end{enumerate}

\vspace{0.5cm}

\subsection{Screen Native}
Următoarele fapte contravenționale sunt sesizate împotriva acestei entități:

\begin{enumerate}[leftmargin=*, label=\arabic*.)]
    \item promovarea unui mesaj de propaganda electorala pentru candidatul NICOLAE-IONEL CIUCA (PNL) la alegerile prezidentiale, dupa incheierea perioadei de campanie. Postarea cu ID \href{https://www.facebook.com/ads/library/?id=552165887544658}{552165887544658} contine indemnuri directe la vot ("Mediasul voteaza presedintele liberal Nicolae Ciuca!"), este marcata cu CMF 11240002, si a fost promovata ca reclama platita pe Instagram, cu un buget intre 400-499 RON si o audienta estimata intre 100,001-500,000 persoane, dupa ora 18:00 pe 23.11.2024.
\end{enumerate}

\vspace{0.5cm}

\subsection{Senatorul Artene Singeorzan si USR}
Următoarele fapte contravenționale sunt sesizate împotriva acestei entități:

\begin{enumerate}[leftmargin=*, label=\arabic*.)]
    \item promovarea unui mesaj electoral platit (ID Facebook: \href{https://www.facebook.com/ads/library/?id=1534691147214423}{1534691147214423}) ce face propaganda electorala pentru candidatul prezidential Elena Lasconi dupa incheierea perioadei de campanie. Postarea, activa si dupa ora 18:00 pe 23.11.2024, contine numar CMF (11240015), indeamna explicit la vot ("Turul 1 = Turul decisiv!"), promoveaza direct candidatul si solicita activ sustinerea electoratului, fiind difuzata ca reclama platita pe Facebook si Instagram cu un reach estimat intre 100,001-500,000 persoane.
\end{enumerate}

\vspace{0.5cm}

\subsection{Senatorul Sorin Cristian Alic}
Următoarele fapte contravenționale sunt sesizate împotriva acestei entități:

\begin{enumerate}[leftmargin=*, label=\arabic*.)]
    \item publicarea si mentinerea activa a unei reclame electorale (ID Facebook: \href{https://www.facebook.com/ads/library/?id=497335456648147}{497335456648147}) dupa incheierea perioadei de campanie electorala. Postarea contine un indemn explicit de a vota candidatul George Simion, fiind distribuita ca reclama platita pe Facebook si Instagram, cu un impact estimat intre 8.000 si 8.999 de afisari. Materialul include numar CMF ( - ), confirmand natura sa de propaganda electorala. Reclama a ramas activa si dupa ora 18:00 pe 23.11.2024, incalcand astfel prevederile legale privind incheierea campaniei electorale.
\end{enumerate}

\vspace{0.5cm}

\subsection{Serban Todorica-Constantin}
Următoarele fapte contravenționale sunt sesizate împotriva acestei entități:

\begin{enumerate}[leftmargin=*, label=\arabic*.)]
    \item difuzarea de materiale de propaganda electorala dupa incheierea campaniei electorale, constand intr-o postare sponsorizata pe Facebook si Instagram (ID: \href{https://www.facebook.com/ads/library/?id=1745223982905054}{1745223982905054}) care indeamna explicit la votarea candidatului Nicolae-Ionel Ciuca pentru functia de presedinte. Postarea a fost difuzata si mentinuta activ dupa ora 18:00 pe 23.11.2024, avand un impact semnificativ cu peste 15.000 de afisari si un buget de promovare de peste 200 RON. Mesajul contine elemente clare de propaganda electorala, incluzand indemnul direct la vot ("Votez") si utilizarea simbolisticii partidului.
\end{enumerate}

\vspace{0.5cm}

\subsection{Sorin Nacuta si Partidul National Liberal}
Următoarele fapte contravenționale sunt sesizate împotriva acestei entități:

\begin{enumerate}[leftmargin=*, label=\arabic*.)]
    \item difuzarea unui material de propaganda electorala (ID Facebook: \href{https://www.facebook.com/ads/library/?id=501390856272523}{501390856272523}) ce promoveaza candidatul Nicolae-Ionel Ciuca la alegerile prezidentiale, dupa incheierea perioadei de campanie electorala. Postarea, care este platita si are un impact estimat intre 900.000 si 1.000.000 de afisari, continua sa fie activa si dupa ora 18:00 pe 23.11.2024, incalcand astfel prevederile legale privind incheierea campaniei electorale. Materialul contine indemnuri directe de vot si promoveaza explicit candidatul la presedintie, depasind cadrul legal permis pentru perioada post-campanie.
\end{enumerate}

\vspace{0.5cm}

\subsection{TOHANEANU NICOLAE-VLADUT PERSOANA FIZICA AUTORIZATA}
Următoarele fapte contravenționale sunt sesizate împotriva acestei entități:

\begin{enumerate}[leftmargin=*, label=\arabic*.)]
    \item difuzarea de materiale de propaganda electorala dupa incheierea campaniei electorale pentru alegerile prezidentiale, prin intermediul unei postari sponsorizate pe Facebook (ID: \href{https://www.facebook.com/ads/library/?id=1255773598984272}{1255773598984272}). Materialul promoveaza candidatul PNCR Cristian Terhes si ataca alti candidati, folosind hashtag-uri electorale si numere CMF (11240003, 31240004), fiind difuzat dupa ora 18:00 pe 23.11.2024, incalcand astfel perioada legala de campanie electorala. Postarea are caracter explicit electoral, vizand influentarea votului prin mesaje de sustinere si critica a candidatilor.
    \item promovarea unui material de propaganda electorala pentru alegerile prezidentiale (ID postare Facebook: \href{https://www.facebook.com/ads/library/?id=3928901367359145}{3928901367359145}) dupa incheierea perioadei de campanie. Materialul contine numere CMF (11240003, 31240004), promoveaza direct candidatul Cristian Terhes folosind hashtag-ul \#TerhesPresedinte, critica alti candidati prezidentiali, si este distribuit ca reclama platita pe Facebook si Instagram dupa ora 18:00 pe 23.11.2024, incalcand astfel prevederile legale privind incheierea campaniei electorale.
\end{enumerate}

\vspace{0.5cm}

\subsection{TOP DUO CONSULTING S.R.L.}
Următoarele fapte contravenționale sunt sesizate împotriva acestei entități:

\begin{enumerate}[leftmargin=*, label=\arabic*.)]
    \item difuzarea de materiale de propaganda electorala dupa incheierea campaniei electorale pentru alegerile prezidentiale, constand intr-o postare platita pe Facebook (ID: \href{https://www.facebook.com/ads/library/?id=1195555961926959}{1195555961926959}) care promoveaza candidatul George Simion si contine mesaje electorale explicit formulate. Postarea include numar CMF (11240014), promisiuni electorale si este difuzata dupa ora 18:00 pe 23.11.2024, incalcand astfel prevederea legala privind incetarea campaniei electorale. Impactul postarii este semnificativ, avand o audienta estimata intre 500,001 si 1,000,000 de persoane.
    \item difuzarea unei reclame platite pe Facebook si Instagram (ID: \href{https://www.facebook.com/ads/library/?id=1573390819943317}{1573390819943317}) care promoveaza candidatul prezidential George Simion dupa incheierea perioadei de campanie electorala. Materialul, publicat dupa ora 18:00 pe 23.11.2024, contine referiri directe la candidatul prezidential ("Cu AUR si George Simion"), foloseste un numar CMF (11240014), si are un impact semnificativ, fiind distribuit catre un public larg (25,000-29,999 impresii). Mesajul are caracter electoral explicit si promoveaza candidatul in mod direct, incalcand astfel prevederile legale privind perioada de propaganda electorala.
    \item difuzarea de materiale de propaganda electorala dupa incheierea perioadei de campanie pentru alegerile prezidentiale. Postarea cu ID-ul \href{https://www.facebook.com/ads/library/?id=492205546486907}{492205546486907} promoveaza in mod explicit candidatul la presedintie George Simion, fiind o reclama platita cu impact semnificativ (15.000-19.999 impresii), difuzata dupa ora 18:00 pe 23.11.2024. Materialul contine numar CMF (11240014), hashtag-uri electorale si mesaje clare de sustinere politica, constituind astfel propaganda electorala in afara perioadei legale permise.
    \item difuzarea de materiale de propaganda electorala (ID postare Facebook: \href{https://www.facebook.com/ads/library/?id=538309045777464}{538309045777464}) dupa incheierea campaniei electorale pentru alegerile prezidentiale. Postarea contine mesaje directe de sustinere a candidatului George Simion, foloseste numar CMF (11240014), include indemnuri explicite la vot ("Sustine-ne la alegeri"), si a fost difuzata dupa ora 18:00 pe 23.11.2024. Impactul postarii este semnificativ, avand intre 25.000 si 29.999 de afisari, fiind astfel o incalcare clara a prevederilor legale privind incheierea campaniei electorale.
    \item promovarea unui mesaj de propaganda electorala (ID Facebook: \href{https://www.facebook.com/ads/library/?id=914473200271350}{914473200271350}) ce face referire directa si pozitiva la candidatul prezidential George Simion, dupa ora 18:00 pe 23.11.2024. Mesajul, care include numarul CMF 11240014, reprezinta o continuare a propagandei electorale dupa incheierea perioadei legale pentru alegerile prezidentiale, fiind distribuit ca reclama platita pe Facebook cu un impact potential de peste 1 milion de persoane. Continutul promoveaza explicit candidatul la presedintie intr-un context electoral, incalcand astfel prevederile legale privind perioada de campanie.
\end{enumerate}

\vspace{0.5cm}

\subsection{Total Impact}
Următoarele fapte contravenționale sunt sesizate împotriva acestei entități:

\begin{enumerate}[leftmargin=*, label=\arabic*.)]
    \item difuzarea de materiale de propaganda electorala dupa incheierea perioadei de campanie, constand intr-o postare platita pe Facebook (ID: \href{https://www.facebook.com/ads/library/?id=972207998081030}{972207998081030}) care promoveaza candidatul Nicolae Ciuca la presedintie. Postarea contine mesaje clare de sustinere electorala, foloseste simbolistica partidului si are ca scop influentarea votului, fiind difuzata dupa ora 18:00 pe 23.11.2024. Impactul estimat al postarii este intre 100,001 si 500,000 de persoane, demonstrand intentia clara de a influenta procesul electoral in afara perioadei legale de campanie.
\end{enumerate}

\vspace{0.5cm}

\subsection{Total Impact si AUR}
Următoarele fapte contravenționale sunt sesizate împotriva acestei entități:

\begin{enumerate}[leftmargin=*, label=\arabic*.)]
    \item difuzarea unui mesaj de propaganda electorala pentru alegerile prezidentiale (ID postare Facebook: \href{https://www.facebook.com/ads/library/?id=606526731946679}{606526731946679}) dupa data de 23.11.2024, ora 18:00. Postarea promovata face referire directa la candidatul prezidential George Simion, fiind distribuita pe Facebook si Instagram ca reclama platita, cu un impact estimat intre 100,001 si 500,000 de persoane, constituind astfel un act de propaganda electorala in afara perioadei legale permise.
\end{enumerate}

\vspace{0.5cm}

\subsection{UNIWORLD - MEDIA SRL}
Următoarele fapte contravenționale sunt sesizate împotriva acestei entități:

\begin{enumerate}[leftmargin=*, label=\arabic*.)]
    \item difuzarea unei reclame platite pe Facebook (ID: \href{https://www.facebook.com/ads/library/?id=1074079691186801}{1074079691186801}) care face propaganda electorala explicita pentru candidatul prezidential George Simion dupa incheierea perioadei de campanie. Postarea contine un indemn direct la vot ("Votati [...] pe George Simion presedinte"), fiind difuzata dupa ora 18:00 pe 23.11.2024, atingand intre 4000-4999 de persoane. Materialul promite in mod explicit constructia unei baze olimpice de canotaj in schimbul votului, ceea ce confirma caracterul sau de propaganda electorala conform Art. 36(7) din Legea 334/2006.
    \item difuzarea unui material de propaganda electorala (ID Facebook: \href{https://www.facebook.com/ads/library/?id=1309103433586021}{1309103433586021}) dupa incheierea perioadei de campanie electorala. Materialul promoveaza explicit candidatul George Simion pentru alegerile prezidentiale, contine indemnuri directe la vot ("va invit sa votati George Simion si A.U.R") si face referire explicita la data de 24 noiembrie, dupa ora 18:00 pe 23.11.2024. Materialul a fost distribuit ca reclama platita pe Facebook, cu un buget semnificativ (1500-1999 RON) si o audienta estimata intre 100,001 si 500,000 de persoane, demonstrand intentia clara de influentare a votului pentru alegerile prezidentiale in afara perioadei legale de campanie.
    \item difuzarea de materiale de propaganda electorala dupa incheierea campaniei electorale prezidentiale, constand intr-o postare sponsorizata pe Facebook (ID: \href{https://www.facebook.com/ads/library/?id=1652036598729276}{1652036598729276}) care promoveaza candidatul George Simion si partidul AUR, facand apel explicit la vot ("votati AUR") si promovand platforma electorala a candidatului ("Planul pentru o Romanie dreapta"). Postarea a fost difuzata si mentinuta activ dupa ora 18:00 pe 23.11.2024, incalcand astfel prevederile legale privind incheierea campaniei electorale prezidentiale.
    \item promovarea unui mesaj de propaganda electorala pentru candidatul George Simion la functia de presedinte (ID postare Facebook: \href{https://www.facebook.com/ads/library/?id=2963929023764464}{2963929023764464}), dupa ora 18:00 pe 23.11.2024. Postarea contine elemente clare de propaganda electorala: numar unic de inregistrare CMF11240014, hashtag-uri electorale (\#GeorgeSimionPresedinte), promovare explicita a candidatului si reprezinta o reclama platita cu impact semnificativ (60.000-70.000 impresii). Mesajul combina o poveste personala cu propaganda electorala directa, avand ca scop influentarea votului in favoarea candidatului AUR la prezidentiale.
    \item difuzarea unei reclame platite pe Facebook (ID: \href{https://www.facebook.com/ads/library/?id=3623273844640710}{3623273844640710}) ce continua propaganda electorala pentru candidatul prezidential George Simion dupa incheierea perioadei legale de campanie. Postarea, activa dupa ora 18:00 pe 23.11.2024, promoveaza explicit candidatura prezidentiala prin utilizarea hashtag-ului "\#GeorgeSimionPresedinte" si mesaje de sustinere directa, fiind o reclama platita cu impact semnificativ (125,000-150,000 afisari). Materialul include cod CMF 11240014, confirmand natura sa de propaganda electorala.
    \item promovarea unui material de propaganda electorala (ID Facebook: \href{https://www.facebook.com/ads/library/?id=3925497371059759}{3925497371059759}) dupa incheierea perioadei de campanie electorala. Materialul promoveaza explicit candidatul GEORGE-NICOLAE SIMION pentru functia de presedinte, folosind hashtag-uri specifice campaniei (\#GeorgeSimionPresedinte), continut electoral explicit si numar unic de inregistrare CMF11240014. Postarea a fost promovata ca reclama platita pe Facebook, cu un impact intre 20.000 si 24.999 de afisari, dupa ora 18:00 pe 23.11.2024, incalcand astfel prevederile legale privind incetarea propagandei electorale.
    \item difuzarea de materiale de propaganda electorala dupa incheierea perioadei legale de campanie, promovand candidatul George Nicolae Simion si partidul AUR prin intermediul unei reclame platite pe Facebook (ID: \href{https://www.facebook.com/ads/library/?id=457836440239634}{457836440239634}). Materialul prezinta realizarile candidatului in domeniul sanatatii si este distribuit dupa ora 18:00 pe 23.11.2024, cu un impact semnificativ (20,000-24,999 impresii), reprezentand o incercare clara de a influenta alegatorii in perioada interzisa prin lege.
    \item promovarea unui mesaj de propaganda electorala pentru candidatul prezidential George Simion (prin referirea la "Planul Simion") dupa incheierea perioadei de campanie electorala. Postarea cu ID-ul \href{https://www.facebook.com/ads/library/?id=588232327103229}{588232327103229} pe facebook, publicata si promovata dupa ora 18:00 pe 23.11.2024, reprezinta o continuare a propagandei electorale pentru alegerile prezidentiale prin promovarea explicita a planului candidatului AUR la presedintie, incalcand astfel prevederile legale privind incheierea campaniei electorale. Efectul electoral este evident prin indemnul direct la vot si asocierea cu candidatul prezidential.
    \item distribuirea de materiale de propaganda electorala dupa incheierea campaniei electorale pentru alegerile prezidentiale, constand intr-o postare sponsorizata pe Facebook (ID: \href{https://www.facebook.com/ads/library/?id=612675074424929}{612675074424929}) care indeamna in mod explicit la votarea candidatului George Simion. Postarea contine indemnuri directe de vot ("Eu il votez pe George Simion si va indemn sa faceti la fel"), utilizeaza simbolistica electorala si este difuzata dupa ora 18:00 pe 23.11.2024, incalcand astfel prevederile legale privind incheierea campaniei electorale.
    \item difuzarea de materiale de propaganda electorala dupa incheierea campaniei electorale pentru alegerile prezidentiale. Postarea cu ID-ul \href{https://www.facebook.com/ads/library/?id=959848279534435}{959848279534435} contine un indemn explicit de a vota candidatul George Simion la presedintie si partidul AUR, fiind difuzata dupa ora 18:00 pe 23.11.2024. Materialul are caracter electoral evident, fiind distribuit catre un public larg (40,000-45,000 impresii), cu un buget semnificativ (200-299 RON), reprezentand o incalcare clara a prevederilor legale privind incetarea propagandei electorale.
\end{enumerate}

\vspace{0.5cm}

\subsection{UNIWORLD - MEDIA SRL si AUR}
Următoarele fapte contravenționale sunt sesizate împotriva acestei entități:

\begin{enumerate}[leftmargin=*, label=\arabic*.)]
    \item continuarea propagandei electorale pentru candidatul prezidential George Nicolae Simion dupa incheierea perioadei de campanie. Postarea cu ID-ul \href{https://www.facebook.com/ads/library/?id=1354010492236303}{1354010492236303} contine indemnuri explicite de vot ("sa-l sustina pe George Simion") si promovare electorala activa dupa ora 18:00 pe 23.11.2024, intr-o perioada in care propaganda electorala pentru alegerile prezidentiale este interzisa. Postarea a fost promovata ca reclama platita pe Facebook, cu un impact estimat intre 10.000-14.999 de afisari, demonstrand intentia clara de a influenta votul pentru alegerile prezidentiale intr-o perioada nepermisa de lege.
    \item difuzarea de materiale de propaganda electorala dupa incheierea perioadei legale de campanie. Postarea cu ID-ul \href{https://www.facebook.com/ads/library/?id=1611606109739658}{1611606109739658} contine un indemn explicit de a vota candidatul George Simion la presedintie, folosind formularea "Votati AUR si George Simion, presedintele Romaniei!", constituind astfel propaganda electorala activa. Postarea a fost promovata dupa ora 18:00 pe 23.11.2024, atingand intre 20.000 si 24.999 de afisari, cu un buget de 200-299 RON, demonstrand intentia clara de a influenta votul prin continuarea propagandei electorale dupa incheierea perioadei legale.
    \item difuzarea de materiale de propaganda electorala dupa incheierea campaniei electorale pentru alegerile prezidentiale, constand intr-o postare sponsorizata pe Facebook (ID: \href{https://www.facebook.com/ads/library/?id=871483411851862}{871483411851862}) ce indeamna explicit la votarea candidatului George Simion la functia de presedinte ("Votati AUR si pe George Simion, presedintele Romaniei!"). Postarea a fost difuzata dupa ora 18:00 pe 23.11.2024, atingand intre 25.000 si 29.999 de afisari, cu un buget intre 200-299 RON, reprezentand o incalcare clara a perioadei legale de campanie electorala.
\end{enumerate}

\vspace{0.5cm}

\subsection{UNIWORLD - MEDIA SRL si Cornel Ciobanu}
Următoarele fapte contravenționale sunt sesizate împotriva acestei entități:

\begin{enumerate}[leftmargin=*, label=\arabic*.)]
    \item difuzarea de materiale de propaganda electorala dupa incheierea campaniei electorale pentru alegerile prezidentiale. Materialul promovat (ID Facebook: 27593234510292262) contine indemnuri explicite de vot pentru candidatul AUR la presedintie, George Simion, fiind difuzat si dupa ora 18:00 pe 23.11.2024. Postarea reprezinta propaganda electorala conform Art. 36(7), avand obiectiv electoral explicit, adresandu-se publicului larg prin intermediul unei campanii platite pe Facebook, cu un reach estimat intre 150.000-175.000 de persoane si un buget de peste 2.000 RON.
\end{enumerate}

\vspace{0.5cm}

\subsection{UNIWORLD - MEDIA SRL si Doina Mihailescu}
Următoarele fapte contravenționale sunt sesizate împotriva acestei entități:

\begin{enumerate}[leftmargin=*, label=\arabic*.)]
    \item difuzarea unui material de propaganda electorala (ID postare Facebook: \href{https://www.facebook.com/ads/library/?id=1139425044849387}{1139425044849387}) care promoveaza candidatul prezidential George Simion dupa incheierea perioadei de campanie electorala. Materialul, difuzat dupa ora 18:00 pe 23.11.2024, face referire directa la "planul lui George Simion" si exprima sustinerea pentru acesta, constituind astfel propaganda electorala continuata dupa incheierea campaniei. Impactul a fost semnificativ, postarea avand intre 70.000-79.999 de afisari, fiind sponsorizata cu suma de 600-699 RON.
\end{enumerate}

\vspace{0.5cm}

\subsection{UNIWORLD - MEDIA SRL si Ilie-Vasile Sirbu}
Următoarele fapte contravenționale sunt sesizate împotriva acestei entități:

\begin{enumerate}[leftmargin=*, label=\arabic*.)]
    \item publicarea si promovarea unei reclame pe Facebook (ID: \href{https://www.facebook.com/ads/library/?id=2795941963927464}{2795941963927464}) ce contine propaganda electorala explicita pentru candidatul prezidential George Simion, dupa incheierea perioadei de campanie electorala. Postarea include indemnul direct "Va indemn sa-l sustineti pe George Simion pe 24 noiembrie" si a fost difuzata dupa ora 18:00 pe 23.11.2024, incalcand astfel prevederile legale privind incheierea campaniei electorale. Impactul postarii este semnificativ, avand intre 100.000 si 124.999 de afisari, reprezentand astfel o incercare clara de influentare a votului in perioada de interdictie.
\end{enumerate}

\vspace{0.5cm}

\subsection{UNIWORLD - MEDIA SRL si Partidul AUR}
Următoarele fapte contravenționale sunt sesizate împotriva acestei entități:

\begin{enumerate}[leftmargin=*, label=\arabic*.)]
    \item difuzarea de materiale de propaganda electorala dupa incheierea campaniei electorale. Postarea cu ID-ul \href{https://www.facebook.com/ads/library/?id=3554420868191864}{3554420868191864} promoveaza in mod direct candidatul George Simion, folosind hashtag-ul \#VoteazaAUR si distribuind afise electorale, avand un impact direct asupra alegerilor prezidentiale. Actiunea a continuat si dupa ora 18:00 pe 23.11.2024, incalcand astfel prevederile legale privind incheierea campaniei electorale. Impactul este demonstrat prin reach-ul de peste 25.000 de persoane si bugetul de promovare de peste 200 RON.
\end{enumerate}

\vspace{0.5cm}

\subsection{USR}
Următoarele fapte contravenționale sunt sesizate împotriva acestei entități:

\begin{enumerate}[leftmargin=*, label=\arabic*.)]
    \item promovarea unui mesaj electoral platit pe Facebook (ID: \href{https://www.facebook.com/ads/library/?id=569180649096733}{569180649096733}) care continua propaganda electorala pentru candidatul prezidential Elena Lasconi dupa incheierea perioadei legale de campanie. Postarea, activa dupa ora 18:00 pe 23.11.2024, contine un mesaj explicit de sustinere electorala ("Elena Lasconi la Presedintie"), are numar CMF (11240015), si a ajuns la un public estimat intre 200,000 si 250,000 de persoane, reprezentand o clara continuare a propagandei electorale in afara perioadei permise de lege.
\end{enumerate}

\vspace{0.5cm}

\subsection{USR Arad, prin reprezentantii sai legali}
Următoarele fapte contravenționale sunt sesizate împotriva acestei entități:

\begin{enumerate}[leftmargin=*, label=\arabic*.)]
    \item difuzarea de materiale de propaganda electorala dupa incheierea campaniei electorale pentru alegerile prezidentiale. Postarea cu ID-ul \href{https://www.facebook.com/ads/library/?id=3739588549685648}{3739588549685648} pe Facebook, publicata si promovata dupa ora 18:00 pe 23.11.2024, face referire directa la candidatii prezidentiali Ciolacu, Simion si Lasconi, in context electoral, cu mesaj ce vizeaza influentarea votului. Postarea a fost promovata cu sume intre 200-299 RON, atingand intre 10,000-14,999 persoane, demonstrand intentia clara de propaganda electorala in afara perioadei legale permise.
\end{enumerate}

\vspace{0.5cm}

\subsection{USR Arges}
Următoarele fapte contravenționale sunt sesizate împotriva acestei entități:

\begin{enumerate}[leftmargin=*, label=\arabic*.)]
    \item publicarea si promovarea unei reclame pe Facebook (ID: \href{https://www.facebook.com/ads/library/?id=3525561504410923}{3525561504410923}) dupa incheierea perioadei de campanie electorala pentru alegerile prezidentiale. Postarea contine propaganda electorala explicit pentru candidata Elena Lasconi si impotriva candidatilor Marcel Ciolacu si George Simion, fiind difuzata dupa ora 18:00 pe 23.11.2024. Postarea include numar CMF (11240015), apeluri directe la vot si mesaje care influenteaza preferintele electorale, atingand un public intre 35.000-40.000 de persoane prin promovare platita.
\end{enumerate}

\vspace{0.5cm}

\subsection{USR Bistrita-Nasaud}
Următoarele fapte contravenționale sunt sesizate împotriva acestei entități:

\begin{enumerate}[leftmargin=*, label=\arabic*.)]
    \item promovarea unui mesaj de propaganda electorala (ID Facebook: \href{https://www.facebook.com/ads/library/?id=895329446034288}{895329446034288}) dupa incheierea perioadei de campanie electorala. Postarea, publicata ca reclama platita pe Facebook si Instagram, promoveaza direct candidatul Elena Lasconi pentru functia de presedinte, contine numar CMF (11240015), si face apel explicit la vot ("Turul 1 = Turul decisiv"). Mesajul a fost difuzat dupa ora 18:00 pe 23.11.2024, incalcand astfel prevederile legale privind incetarea campaniei electorale.
\end{enumerate}

\vspace{0.5cm}

\subsection{USR Braila}
Următoarele fapte contravenționale sunt sesizate împotriva acestei entități:

\begin{enumerate}[leftmargin=*, label=\arabic*.)]
    \item promovarea unui mesaj electoral platit (ID Facebook: \href{https://www.facebook.com/ads/library/?id=572976648757324}{572976648757324}) dupa incheierea perioadei de campanie electorala prezidentiala, respectiv dupa ora 18:00 pe 23.11.2024. Postarea contine elementele specifice propagandei electorale: numar CMF (11240015), critica directa a candidatului prezidential Marcel Ciolacu, indemn explicit de a nu vota PSD (\#NuVotatiPSD), si promovare politica cu obiectiv electoral clar. Mesajul a fost distribuit ca reclama platita pe Facebook si Instagram, cu un impact estimat intre 20.000 si 24.999 de afisari.
\end{enumerate}

\vspace{0.5cm}

\subsection{USR Bucuresti}
Următoarele fapte contravenționale sunt sesizate împotriva acestei entități:

\begin{enumerate}[leftmargin=*, label=\arabic*.)]
    \item publicarea si mentinerea activa a unei reclame electorale (ID: \href{https://www.facebook.com/ads/library/?id=1078508163752941}{1078508163752941}) dupa incheierea perioadei de campanie electorala. Reclama contine un indemn explicit de vot pentru candidata Elena Lasconi ("Votam Elena Lasconi duminica aceasta"), fiind difuzata pe Instagram cu un impact semnificativ (45,000-50,000 impresii). Materialul este marcat cu codurile CMF 11240015 si CMF:31240009, confirmand natura sa de propaganda electorala. Aceasta activitate continua dupa ora 18:00 pe 23.11.2024, incalcand astfel prevederile legale privind incheierea campaniei electorale.
\end{enumerate}

\vspace{0.5cm}

\subsection{USR Caras-Severin}
Următoarele fapte contravenționale sunt sesizate împotriva acestei entități:

\begin{enumerate}[leftmargin=*, label=\arabic*.)]
    \item continuarea propagandei electorale dupa incheierea campaniei, promovand programul electoral al candidatei Elena Lasconi prin intermediul unei postari platite pe Facebook (ID: \href{https://www.facebook.com/ads/library/?id=1247985159866179}{1247985159866179}). Postarea contine promisiuni electorale specifice privind reducerea taxelor pentru antreprenori si PFA-uri, fiind marcata cu cod CMF 11240015, si a fost difuzata dupa ora 18:00 pe 23.11.2024, incalcand astfel prevederile legale privind incetarea campaniei electorale. Impactul postarii este semnificativ, atingand intre 8.000 si 8.999 de persoane, cu un buget de promovare intre 200-299 RON.
\end{enumerate}

\vspace{0.5cm}

\subsection{USR Constanta}
Următoarele fapte contravenționale sunt sesizate împotriva acestei entități:

\begin{enumerate}[leftmargin=*, label=\arabic*.)]
    \item publicarea si promovarea unei reclame pe Facebook (ID: 27606559912324043) ce contine propaganda electorala dupa incheierea perioadei legale de campanie, criticand direct candidatii prezidentiali Marcel Ciolacu si Nicolae Ciuca, cu scopul de a influenta comportamentul electoral al votantilor. Postarea, efectuata dupa ora 18:00 pe 23.11.2024, contine numar CMF (11240015), demonstrand caracterul sau electoral explicit, si a fost promovata catre un public tinta de 100,001-500,000 persoane, reprezentand o incalcare clara a prevederilor legale privind incheierea campaniei electorale.
    \item promovarea unui mesaj electoral platit (ID: \href{https://www.facebook.com/ads/library/?id=439252415885874}{439252415885874}) pe Facebook dupa incheierea campaniei electorale pentru alegerile prezidentiale. Postarea, publicata dupa ora 18:00 pe 23.11.2024, contine indemnuri directe de vot pentru candidatul prezidential Elena Lasconi ("\#ElenaLasconiPresedinte"), atacuri la adresa contracandidatilor Marcel Ciolacu si George Simion, precum si elemente specifice campaniei electorale (numar CMF: 11240015). Mesajul a avut un impact semnificativ, ajungand la 10.000-14.999 de utilizatori prin distributie platita.
    \item difuzarea unui mesaj de propaganda electorala dupa incheierea campaniei electorale pentru alegerile prezidentiale, prin postarea cu ID \href{https://www.facebook.com/ads/library/?id=8481490908566311}{8481490908566311} pe Facebook. Postarea, publicata si promovata dupa ora 18:00 pe 23.11.2024, contine mesaje clare de propaganda electorala, avand CMF 11240015, atacand direct candidatii Marcel Ciolacu, George Simion si Nicolae Ciuca, in timp ce promoveaza candidatul Elena Lasconi. Postarea a fost distribuita catre un public larg (20.000-25.000 de persoane) prin publicitate platita pe Facebook si Instagram, avand un obiectiv electoral explicit.
\end{enumerate}

\vspace{0.5cm}

\subsection{USR Dambovita}
Următoarele fapte contravenționale sunt sesizate împotriva acestei entități:

\begin{enumerate}[leftmargin=*, label=\arabic*.)]
    \item promovarea candidatului ELENA-VALERICA LASCONI intr-o postare platita pe Facebook (ID: \href{https://www.facebook.com/ads/library/?id=1194999424934329}{1194999424934329}) dupa incheierea perioadei de campanie electorala. Postarea prezinta realizarile candidatului si foloseste mesaje emotionale pentru a influenta alegatorii, avand un CMF alocat (11240015) si fiind difuzata dupa ora 18:00 pe 23.11.2024. Impactul postarii este semnificativ, atingand intre 90.000 si 99.999 de persoane, cu o investitie intre 2.000 si 2.499 RON.
\end{enumerate}

\vspace{0.5cm}

\subsection{USR Galati}
Următoarele fapte contravenționale sunt sesizate împotriva acestei entități:

\begin{enumerate}[leftmargin=*, label=\arabic*.)]
    \item difuzarea unui mesaj de propaganda electorala (ID Facebook: \href{https://www.facebook.com/ads/library/?id=1102673304560545}{1102673304560545}) dupa ora 18:00 pe 23.11.2024. Postarea contine materiale de propaganda electorala clara, cu referire directa la candidatul prezidential Marcel Ciolacu si indemnuri explicite de vot pentru USR, fiind distribuita ca reclama platita pe Facebook si Instagram. Materialul include CMF 11240015, confirmand natura sa de propaganda electorala, si continua sa fie activ dupa incheierea perioadei legale de campanie pentru alegerile prezidentiale.
\end{enumerate}

\vspace{0.5cm}

\subsection{USR Ilfov}
Următoarele fapte contravenționale sunt sesizate împotriva acestei entități:

\begin{enumerate}[leftmargin=*, label=\arabic*.)]
    \item continuarea propagandei electorale dupa incheierea campaniei electorale prezidentiale, manifestata prin postarea cu ID \href{https://www.facebook.com/ads/library/?id=449144147850872}{449144147850872} pe platforma Facebook. Postarea, activa si promovata dupa ora 18:00 pe 23.11.2024, contine material de propaganda electorala care il vizeaza direct pe candidatul prezidential Nicolae Ciuca, facand asocieri negative si indemnuri explicite la vot impotriva acestuia. Postarea include numar CMF (11240015), confirmand natura sa de propaganda electorala, si a fost distribuita catre un public larg (90,000-99,999 impresii) prin publicitate platita pe Facebook si Instagram.
    \item difuzarea unui mesaj de propaganda electorala (ID Facebook: \href{https://www.facebook.com/ads/library/?id=571435878604456}{571435878604456}) dupa incheierea perioadei de campanie electorala. Materialul promovat, continand CMF 11240015, il vizeaza direct pe candidatul ION-MARCEL CIOLACU intr-un mod negativ si face apel explicit la vot ("Pe 1 decembrie avem ocazia sa alegem"), fiind difuzat ca reclama platita pe Facebook si Instagram dupa ora 18:00 pe 23.11.2024. Postarea a avut un impact semnificativ, atingand intre 100.000 si 124.999 de afisari, cu o investitie de aproximativ 2.250 RON.
\end{enumerate}

\vspace{0.5cm}

\subsection{USR Maramures}
Următoarele fapte contravenționale sunt sesizate împotriva acestei entități:

\begin{enumerate}[leftmargin=*, label=\arabic*.)]
    \item promovarea unui material de propaganda electorala pentru candidatul prezidential Elena Lasconi (USR) dupa incheierea perioadei de campanie electorala. Materialul, cu ID-ul postarii pe facebook \href{https://www.facebook.com/ads/library/?id=828153992620006}{828153992620006}, continand codurile CMF 11240015 si CMF 31240009, promoveaza in mod direct candidatul si programul sau electoral, fiind difuzat ca reclama platita dupa ora 18:00 pe 23.11.2024. Materialul are caracter electoral evident, prezentand promisiuni si angajamente electorale specifice, fiind distribuit ca reclama platita catre un public larg.
\end{enumerate}

\vspace{0.5cm}

\subsection{Ucu Dima si PNL}
Următoarele fapte contravenționale sunt sesizate împotriva acestei entități:

\begin{enumerate}[leftmargin=*, label=\arabic*.)]
    \item promovarea unui material de propaganda electorala (ID post Facebook: \href{https://www.facebook.com/ads/library/?id=546432674845737}{546432674845737}) dupa incheierea perioadei de campanie electorala pentru alegerile prezidentiale. Materialul face referire directa la candidatul prezidential Nicolae Ciuca, promovandu-l intr-un mod pozitiv si solicitand explicit votul pentru PNL si candidatul sau, dupa ora 18:00 pe 23.11.2024. Materialul contine numar CMF (11240002), demonstrand natura sa de propaganda electorala, si este distribuit ca reclama platita pe platformele Facebook si Instagram.
\end{enumerate}

\vspace{0.5cm}

\subsection{Ucu Dima si PNL Arad}
Următoarele fapte contravenționale sunt sesizate împotriva acestei entități:

\begin{enumerate}[leftmargin=*, label=\arabic*.)]
    \item promovarea unui material de propaganda electorala (ID postare Facebook: \href{https://www.facebook.com/ads/library/?id=1256654532239498}{1256654532239498}) dupa incheierea perioadei de campanie electorala pentru alegerile prezidentiale. Materialul face referire directa la candidatul Nicolae Ciuca, promovandu-l explicit pentru functia de Presedinte al Romaniei, folosind mesaje de sustinere si indemnuri la vot. Postarea contine numar CMF (11240002), confirmand natura sa de propaganda electorala, si a fost difuzata dupa ora 18:00 pe 23.11.2024, incalcand astfel restrictiile legale privind perioada de campanie electorala pentru alegerile prezidentiale.
    \item continuarea propagandei electorale dupa incheierea campaniei electorale prezidentiale, prin promovarea activa si platita a candidatului Nicolae Ciuca la functia de presedinte ("il sustinem cu incredere pe Nicolae Ciuca - un lider de exceptie"). Postarea cu ID-ul \href{https://www.facebook.com/ads/library/?id=554639790609638}{554639790609638} pe facebook este o reclama platita, cu CMF 11240002, care incalca explicit prevederile legale privind incheierea campaniei electorale, fiind difuzata dupa ora 18:00 pe 23.11.2024. Mesajul promovat are caracter electoral explicit si vizeaza influentarea votului pentru alegerile prezidentiale.
    \item difuzarea de materiale de propaganda electorala (ID postare Facebook: \href{https://www.facebook.com/ads/library/?id=836994191768944}{836994191768944}) dupa incheierea perioadei de campanie electorala pentru alegerile prezidentiale. Postarea, publicata si promovata dupa ora 18:00 pe 23.11.2024, contine mesaje explicite de sustinere a candidatului prezidential Nicolae Ciuca, folosind numarul CMF 11240002, si face apel direct la vot. Materialul depaseste sfera comunicarii despre alegerile parlamentare, intrand in sfera propagandei electorale pentru alegerile prezidentiale, unde perioada de campanie s-a incheiat.
\end{enumerate}

\vspace{0.5cm}

\subsection{Ucu Dima si Partidul National Liberal}
Următoarele fapte contravenționale sunt sesizate împotriva acestei entități:

\begin{enumerate}[leftmargin=*, label=\arabic*.)]
    \item promovarea unui mesaj electoral platit (ID Facebook: \href{https://www.facebook.com/ads/library/?id=1242890290351940}{1242890290351940}) dupa incheierea perioadei de campanie electorala pentru alegerile prezidentiale. Postarea promoveaza explicit candidatul Nicolae Ciuca la presedintie, prezentandu-l ca viitor presedinte, intr-un context electoral explicit, cu numar CMF 11240002, dupa ora 18:00 pe 23.11.2024. Mesajul combina propaganda electorala pentru alegerile parlamentare cu cea pentru alegerile prezidentiale, incalcand astfel prevederile legale privind incheierea campaniei electorale prezidentiale.
    \item promovarea unui mesaj electoral platit pe Facebook (ID: \href{https://www.facebook.com/ads/library/?id=1290694521930956}{1290694521930956}) dupa incheierea perioadei de campanie electorala pentru alegerile prezidentiale. Postarea, difuzata dupa ora 18:00 pe 23.11.2024, contine propaganda electorala explicita pentru candidatul prezidential Nicolae Ciuca, descriindu-l ca "lider de exceptie" si "alegerea ideala pentru functia de Presedinte al Romaniei". Mesajul este confirmat ca fiind propaganda electorala prin prezenta codului CMF 11240002, reprezinta o incalcare clara a prevederilor legale privind incetarea campaniei electorale.
    \item promovarea candidatului la presedintie Nicolae Ciuca intr-o reclama platita pe Facebook (ID: \href{https://www.facebook.com/ads/library/?id=581946374325171}{581946374325171}) dupa incheierea perioadei de campanie electorala. Postarea contine referinte directe la candidatul prezidential si indemnuri la vot, fiind difuzata dupa ora 18:00 pe 23.11.2024. Materialul include numar CMF (11240002), confirmand natura sa de propaganda electorala, si foloseste platformele Facebook si Instagram pentru a influenta alegatorii in favoarea candidatului mentionat.
\end{enumerate}

\vspace{0.5cm}

\subsection{Uniunea Salvati Romania}
Următoarele fapte contravenționale sunt sesizate împotriva acestei entități:

\begin{enumerate}[leftmargin=*, label=\arabic*.)]
    \item publicarea si promovarea unui mesaj de propaganda electorala (ID postare Facebook: \href{https://www.facebook.com/ads/library/?id=1099811228424088}{1099811228424088}) dupa incheierea perioadei de campanie electorala pentru alegerile prezidentiale. Postarea, marcata cu CMF 11240015, contine critici directe la adresa candidatilor prezidentiali Marcel Ciolacu si Nicolae Ciuca, avand scop electoral explicit si fiind difuzata dupa ora 18:00 pe 23.11.2024, incalcand astfel prevederile legale privind incheierea campaniei electorale.
    \item continuarea propagandei electorale dupa incheierea campaniei electorale, prin promovarea unui material electoral platit pe platformele Facebook si Instagram (ID postare: \href{https://www.facebook.com/ads/library/?id=884481050533794}{884481050533794}) care promoveaza candidatul Elena Lasconi si denigreaza candidatii Ciolacu, Simion, Ciuca si Geoana. Materialul este activ si dupa ora 18:00 pe 23.11.2024, contine numar CMF (11240015), are caracter electoral explicit si este distribuit catre un public larg (10,000-14,999 impresii). Postarea reprezinta propaganda electorala conform Art. 36(7) din Legea 334/2006, indeplinind toate criteriile legale pentru aceasta incadrare.
    \item difuzarea unei reclame platite pe Facebook si Instagram (ID: \href{https://www.facebook.com/ads/library/?id=931967841627574}{931967841627574}) care promoveaza candidatul Elena Lasconi dupa ora 18:00 pe 23.11.2024. Mesajul "Elena Lasconi nu este raul cel mai mic, ci singura care poate invinge raul cel mare" reprezinta in mod clar propaganda electorala, avand scop electoral explicit, adresandu-se unui public larg (100,001-500,000 persoane tintite), cu un buget de 100-199 RON si 8,000-8,999 afisari. Mesajul incalca prevederile legale privind incetarea campaniei electorale si continua sa influenteze decizia de vot in perioada interzisa.
\end{enumerate}

\vspace{0.5cm}

\subsection{Uniunea Salvati Romania (USR)}
Următoarele fapte contravenționale sunt sesizate împotriva acestei entități:

\begin{enumerate}[leftmargin=*, label=\arabic*.)]
    \item continuarea propagandei electorale pentru candidatul prezidential Elena Lasconi dupa incheierea perioadei legale de campanie, prin intermediul unei postari sponsorizate pe Facebook (ID: \href{https://www.facebook.com/ads/library/?id=1075626137371290}{1075626137371290}). Postarea, difuzata dupa ora 18:00 pe 23.11.2024, contine referinte directe la candidatul prezidential si indemnuri explicite la vot, avand un impact semnificativ demonstrat prin reach-ul de peste 40,000 de impresii. Materialul este in mod clar propaganda electorala, avand numar CMF (11240015) si fiind platit de partidul USR.
    \item continuarea propagandei electorale dupa incheierea perioadei legale, prin publicarea si mentinerea activa a unei reclame pe Facebook (ID: \href{https://www.facebook.com/ads/library/?id=571074518803045}{571074518803045}) dupa ora 18:00 pe 23.11.2024. Postarea contine propaganda electorala explicita, avand CMF 11240015, promovand candidatul USR la presedintie si atacand direct un contracandidat, folosind hashtag-ul \#ElenaLasconiPresedinte si indemnuri directe la vot. Postarea a avut un impact semnificativ, atingand intre 25.000 si 29.999 de persoane, fiind difuzata pe multiple platforme (Facebook si Instagram).
\end{enumerate}

\vspace{0.5cm}

\subsection{Uniunea Salvati Romania - USR}
Următoarele fapte contravenționale sunt sesizate împotriva acestei entități:

\begin{enumerate}[leftmargin=*, label=\arabic*.)]
    \item difuzarea unui material de propaganda electorala (ID postare Facebook: \href{https://www.facebook.com/ads/library/?id=1612369773046758}{1612369773046758}) dupa incheierea perioadei de campanie electorala pentru alegerile prezidentiale. Materialul face referiri directe si negative la candidatii prezidentiali Marcel Ciolacu si Nicolae Ciuca, incercand sa influenteze opinia publica impotriva acestora dupa ora 18:00 pe 23.11.2024. Postarea este sponsorizata, are un grad mare de expunere (peste 125.000 de impresii) si contine elementele specifice materialelor de propaganda electorala, inclusiv numar CMF (11240015).
    \item continuarea propagandei electorale dupa incheierea campaniei electorale pentru alegerile prezidentiale, prin publicarea si mentinerea activa a unei reclame platite pe Facebook (ID: \href{https://www.facebook.com/ads/library/?id=463364796334558}{463364796334558}) dupa ora 18:00 pe 23.11.2024. Postarea contine mesaje explicite de sustinere a candidatului Elena Lasconi si critici la adresa candidatului Marcel Ciolacu, avand scop electoral evident prin indemnul direct "Veniti la vot. \#ElenaLasconiPresedinte". Materialul este marcat oficial ca propaganda electorala prin prezenta numarului CMF 11240015.
\end{enumerate}

\vspace{0.5cm}

\subsection{Uniunea Salvati Romania filiala Constanta}
Următoarele fapte contravenționale sunt sesizate împotriva acestei entități:

\begin{enumerate}[leftmargin=*, label=\arabic*.)]
    \item promovarea unui mesaj electoral platit pe Facebook (ID: \href{https://www.facebook.com/ads/library/?id=914310827466619}{914310827466619}) dupa incheierea perioadei de campanie electorala pentru alegerile prezidentiale. Postarea contine atacuri directe la adresa candidatilor prezidentiali Marcel Ciolacu si Nicolae Ciuca, cu intentia clara de a influenta comportamentul electoral al votantilor, fiind distribuita dupa ora 18:00 pe 23.11.2024. Mesajul include numar CMF (11240015), confirmand natura sa de material de propaganda electorala, si reprezinta o incalcare clara a prevederilor legale privind incetarea campaniei electorale.
\end{enumerate}

\vspace{0.5cm}

\subsection{VESTEA MEDIA}
Următoarele fapte contravenționale sunt sesizate împotriva acestei entități:

\begin{enumerate}[leftmargin=*, label=\arabic*.)]
    \item promovarea unui material de propaganda electorala negativa impotriva candidatului Ion-Marcel Ciolacu dupa incheierea perioadei de campanie electorala. Materialul, cu ID-ul postarii \href{https://www.facebook.com/ads/library/?id=1516528592335743}{1516528592335743}, este o reclama platita pe Facebook si Instagram care vizeaza direct candidatul, folosind un ton denigrator si acuzator, cu scopul clar de a influenta negativ opinia publica si intentiile de vot. Acesta a fost promovat dupa ora 18:00 pe 23.11.2024, incalcand astfel prevederile legale privind incetarea propagandei electorale.
\end{enumerate}

\vspace{0.5cm}

\subsection{VIVINET Agency (SC NANOPRO ARL)}
Următoarele fapte contravenționale sunt sesizate împotriva acestei entități:

\begin{enumerate}[leftmargin=*, label=\arabic*.)]
    \item difuzarea de materiale de propaganda electorala dupa incheierea campaniei electorale pentru alegerile prezidentiale, constand intr-o postare platita pe Facebook (ID: \href{https://www.facebook.com/ads/library/?id=833832162042609}{833832162042609}) care promoveaza explicit candidatul NICOLAE-IONEL CIUCA la presedintie prin utilizarea hashtag-ului "\#ciucapresedinte" si mesaje de sustinere politica. Postarea, difuzata dupa ora 18:00 pe 23.11.2024, a avut un impact semnificativ, atingand intre 30.000 si 35.000 de afisari, reprezentand o incalcare clara a perioadei de restrictie electorala.
\end{enumerate}

\vspace{0.5cm}

\subsection{VOCEA TV}
Următoarele fapte contravenționale sunt sesizate împotriva acestei entități:

\begin{enumerate}[leftmargin=*, label=\arabic*.)]
    \item difuzarea de propaganda electorala dupa incheierea perioadei legale de campanie, concretizata prin publicarea unui anunt platit pe Facebook (ID: \href{https://www.facebook.com/ads/library/?id=1615145879409512}{1615145879409512}) care dezinformeaza cu privire la retragerea candidatului Ludovic Orban si redirectionarea voturilor catre Elena Lasconi. Aceasta actiune constituie propaganda electorala explicita, fiind realizata dupa ora 18:00 pe 23.11.2024, cu impact direct asupra intentiei de vot a cetatenilor, atingand intre 4000-4999 de persoane prin reclama platita.
\end{enumerate}

\vspace{0.5cm}

\subsection{Vocea Botosani}
Următoarele fapte contravenționale sunt sesizate împotriva acestei entități:

\begin{enumerate}[leftmargin=*, label=\arabic*.)]
    \item difuzarea de propaganda electorala pentru candidatul NICOLAE-IONEL CIUCA dupa incheierea perioadei de campanie, prin intermediul unei reclame platite pe Facebook (ID: \href{https://www.facebook.com/ads/library/?id=1631655967707245}{1631655967707245}) care continua sa ruleze dupa ora 18:00 pe 23.11.2024. Materialul promovat contine numar CMF31240003, promoveaza explicit candidatul si realizarile sale, foloseste endorsement-ul unui fost ambasador american, si are scop electoral explicit, fiind difuzat catre un public larg estimat intre 100,001-500,000 persoane pe Facebook si Instagram.
\end{enumerate}

\vspace{0.5cm}

\subsection{Voga Design \& Architecture}
Următoarele fapte contravenționale sunt sesizate împotriva acestei entități:

\begin{enumerate}[leftmargin=*, label=\arabic*.)]
    \item difuzarea unui material de propaganda electorala (ID postare Facebook: \href{https://www.facebook.com/ads/library/?id=1922465028238746}{1922465028238746}) dupa ora 18:00 pe 23.11.2024. Materialul prezinta caracteristici clare de propaganda electorala: numar unic de inregistrare CMF:11200014, indeamna explicit la vot pentru candidatul George Simion, contine hashtag-uri de campanie (\#GeorgeSimion \#Presedinte), si este distribuit ca reclama platita pe Facebook cu impact intre 20.000-24.999 impresii. Mesajul denigreaza alte partide si candidati, promovand in mod direct un candidat la presedintie, incalcand astfel prevederile legale privind perioada de campanie electorala.
    \item difuzarea de materiale de propaganda electorala (ID postare Facebook: \href{https://www.facebook.com/ads/library/?id=4091020691135777}{4091020691135777}) dupa incheierea perioadei de campanie electorala. Postarea, care include numar CMF 11240014, promoveaza direct candidatul GEORGE-NICOLAE SIMION, folosind mesaje cu caracter electoral explicit ("Alegeti curajul, demnitatea, omul care lupta pentru limba noastra") si este difuzata ca reclama platita pe Facebook dupa ora 18:00 pe 23.11.2024, cu potential de expunere intre 500,001 si 1,000,000 de utilizatori.
\end{enumerate}

\vspace{0.5cm}

\subsection{WEBASSIST CONSULTING SRL}
Următoarele fapte contravenționale sunt sesizate împotriva acestei entități:

\begin{enumerate}[leftmargin=*, label=\arabic*.)]
    \item difuzarea de materiale de propaganda electorala dupa incheierea perioadei de campanie, concretizata prin postarea cu ID \href{https://www.facebook.com/ads/library/?id=923407932626906}{923407932626906} pe platforma Facebook. Postarea, publicata si promovata dupa ora 18:00 pe 23.11.2024, contine mesaje explicite de propaganda electorala, face referire directa la mai multi candidati la presedintie (Nicolae Ciuca, Elena Lasconi, Marcel Ciolacu, George Simion), incearca sa influenteze comportamentul electoral prin solicitarea explicita de retragere a unui candidat in favoarea altuia si promoveaza explicit un candidat in detrimentul celorlalti. Mesajul a fost distribuit ca reclama platita, cu un buget substantial si o audienta estimata intre 125.000 si 150.000 de persoane, demonstrand caracterul sau de propaganda electorala sistematica si intentionata.
\end{enumerate}

\vspace{0.5cm}

\subsection{WEBSITESDESIGN SRL}
Următoarele fapte contravenționale sunt sesizate împotriva acestei entități:

\begin{enumerate}[leftmargin=*, label=\arabic*.)]
    \item difuzarea de materiale de propaganda electorala (ID postare Facebook: \href{https://www.facebook.com/ads/library/?id=559852190367969}{559852190367969}) dupa incheierea perioadei de campanie electorala. Materialul promovat contine indemnuri directe la vot ("Pe 1 decembrie votam DREPT"), identificare clara a candidatului (Pozitia 13), numar unic de inregistrare CMF 11240046, si promisiuni electorale explicite. Postarea a fost promovata ca reclama platita pe Facebook incepand cu data de 23.11.2024, dupa ora 18:00, cand perioada de campanie electorala era incheiata legal.
\end{enumerate}

\vspace{0.5cm}

\subsection{WEDEV NOVAT SRL}
Următoarele fapte contravenționale sunt sesizate împotriva acestei entități:

\begin{enumerate}[leftmargin=*, label=\arabic*.)]
    \item difuzarea de materiale de propaganda electorala pentru candidatul prezidential Marcel Ciolacu (ID postare Facebook: \href{https://www.facebook.com/ads/library/?id=1102376751503672}{1102376751503672}) dupa ora 18:00 pe 23.11.2024. Materialul promovat prezinta realizarile guvernamentale ale candidatului, contine indemnuri explicite la vot pentru data de 24 noiembrie si foloseste numar CMF (11240017), confirmand natura sa de propaganda electorala. Postarea este promovata ca reclama platita pe Facebook, cu un impact estimat intre 25.000 si 29.999 de afisari, reprezentand o incalcare clara a perioadei de propaganda electorala permisa de lege.
    \item promovarea unui material de propaganda electorala (ID Facebook: \href{https://www.facebook.com/ads/library/?id=2533014050362871}{2533014050362871}) dupa incheierea perioadei de campanie electorala. Materialul promoveaza explicit candidatul ION-MARCEL CIOLACU in context electoral, criticand in acelasi timp contra-candidatul NICOLAE-IONEL CIUCA, utilizand resurse financiare pentru promovare si targetand un public larg (500,001-1,000,000 persoane). Postarea este activa si promovata dupa ora 18:00 pe 23.11.2024, incalcand astfel prevederile legale privind incetarea propagandei electorale.
\end{enumerate}

\vspace{0.5cm}

\subsection{WEDEV NOVAT SRL si PSD Cluj}
Următoarele fapte contravenționale sunt sesizate împotriva acestei entități:

\begin{enumerate}[leftmargin=*, label=\arabic*.)]
    \item difuzarea unui material de propaganda electorala dupa incheierea perioadei de campanie, constand intr-o postare sponsorizata pe Facebook si Instagram (ID: \href{https://www.facebook.com/ads/library/?id=1260244428457755}{1260244428457755}) care promoveaza candidatul prezidential Ion-Marcel Ciolacu si partidul PSD, folosind realizari guvernamentale pentru a influenta optiunile electorale ale pensionarilor din judetul Cluj. Postarea a fost difuzata dupa ora 18:00 pe 23.11.2024, incalcand explicit prevederile legale privind incheierea campaniei electorale. Impactul a fost semnificativ, atingand intre 6000-7000 de afisari.
\end{enumerate}

\vspace{0.5cm}

\subsection{WEDEV NOVAT SRL si PSD Cluj-Organizatia Judeteana}
Următoarele fapte contravenționale sunt sesizate împotriva acestei entități:

\begin{enumerate}[leftmargin=*, label=\arabic*.)]
    \item difuzarea de materiale de propaganda electorala (ID Facebook: \href{https://www.facebook.com/ads/library/?id=1104015198016822}{1104015198016822}) dupa ora 18:00 pe 23.11.2024. Materialul promovat prezinta explicit candidatul ION-MARCEL CIOLACU intr-o lumina pozitiva, contine numar CMF (11240017), este platit si distribuit ca reclama pe Facebook, are obiectiv electoral explicit prin promovarea realizarilor si promisiunilor candidatului, si continua sa fie activ in perioada de restrictie electorala, influentand astfel procesul electoral in favoarea candidatului PSD.
    \item difuzarea unei reclame platite pe Facebook (ID: \href{https://www.facebook.com/ads/library/?id=1655170358449919}{1655170358449919}) dupa ora 18:00 pe 23.11.2024, care promoveaza candidatul prezidential Marcel Ciolacu. Postarea contine numar CMF (11240017), are caracter electoral explicit, promoveaza direct candidatul la presedintie si viziunea sa economica, si a avut un impact semnificativ, ajungand la peste 8000 de persoane. Materialul constituie propaganda electorala conform Art. 36(7) din Legea 334/2006, intrucat se refera direct la candidat, are obiectiv electoral si se adreseaza publicului larg prin intermediul unei platforme de social media.
\end{enumerate}

\vspace{0.5cm}

\subsection{Zarnescu Samuel si Alianta pentru Unirea Romanilor (AUR)}
Următoarele fapte contravenționale sunt sesizate împotriva acestei entități:

\begin{enumerate}[leftmargin=*, label=\arabic*.)]
    \item difuzarea unui mesaj de propaganda electorala dupa incheierea perioadei de campanie, promovand explicit candidatul George Simion la presedintie prin intermediul unei reclame platite pe Facebook (ID: \href{https://www.facebook.com/ads/library/?id=1535094047370846}{1535094047370846}). Mesajul contine indemnuri directe de vot ("Votam pentru George Simion presedintele Romaniei"), fiind distribuit dupa ora 18:00 pe 23.11.2024, cu un impact estimat intre 10.000-14.999 de afisari, reprezentand o incalcare clara a prevederilor legale privind incheierea campaniei electorale.
\end{enumerate}

\vspace{0.5cm}

\subsection{ZiarulEconomic}
Următoarele fapte contravenționale sunt sesizate împotriva acestei entități:

\begin{enumerate}[leftmargin=*, label=\arabic*.)]
    \item difuzarea de materiale de propaganda electorala dupa incheierea campaniei electorale pentru alegerile prezidentiale, intr-o postare platita pe Facebook (ID: \href{https://www.facebook.com/ads/library/?id=2008908872902760}{2008908872902760}) difuzata dupa ora 18:00 pe 23.11.2024. Postarea contine mesaje explicite de sustinere a candidatului George Simion si atacuri la adresa candidatei Elena Lasconi, utilizand hashtaguri electorale (\#GeorgeSimionPresedinte) si indemnuri clare de vot, atingand un public de peste 40.000 de persoane, cu un buget de promovare de aproximativ 350 RON. Materialul intruneste toate elementele constitutive ale propagandei electorale conform Art. 36(7), avand obiectiv electoral explicit si depasind limitele activitatii jurnalistice.
\end{enumerate}

\vspace{0.5cm}

\section{Solicitări}

Față de cele de mai sus, solicit:

\begin{enumerate}[leftmargin=*, label=\arabic*.]
    \item Constatarea contravențiilor săvârșite;
    \item Identificarea persoanelor vinovate;
    \item Aplicarea sancțiunilor prevăzute de lege.
\end{enumerate}

\section{Anexe}

Anexez prezentei plângeri următoarele dovezi:

\begin{enumerate}[leftmargin=*, label=\arabic*.]
    \item Capturi de ecran ale postărilor care fac obiectul sesizării;
    \item Dovada calității de observator electoral.
\end{enumerate}

\vspace{1cm}
\noindent Data: \today

\vspace{1.5cm}
\noindent Observator electoral,\\[0.3cm]
Deleanu Ștefan-Lucian

\vspace{1cm}
\noindent Semnătura: [SEMNAT ELECTRONIC]

\end{document}
