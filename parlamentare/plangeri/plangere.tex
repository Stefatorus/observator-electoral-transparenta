\documentclass[a4paper,12pt]{article}
\usepackage[romanian]{babel}
\usepackage[utf8]{inputenc}
\usepackage[T1]{fontenc}
\usepackage{geometry}
\usepackage{enumitem}
\usepackage{titlesec}
\usepackage{parskip}
\usepackage{microtype}
\usepackage[none]{hyphenat}
\usepackage{ragged2e}
\usepackage{url}
\usepackage{xurl}
\usepackage{hyperref}
\usepackage{fancyhdr}
\usepackage{listings}
\usepackage{tocloft}
\usepackage{lastpage}

\geometry{
    a4paper,
    margin=2.5cm,
    includehead
}

\pagestyle{fancy}
\fancyhf{}
\lhead{Plângere Contravențională}
\rhead{pagina \thepage/\pageref{LastPage}}
\renewcommand{\headrulewidth}{0.4pt}

\renewcommand{\contentsname}{Cuprins}
\setcounter{tocdepth}{2}
\renewcommand{\cftsecfont}{\normalsize\bfseries}
\renewcommand{\cftsubsecfont}{\normalsize}
\renewcommand{\cftsecpagefont}{\normalsize}
\renewcommand{\cftsubsecpagefont}{\normalsize}
\setlength{\cftbeforesecskip}{5pt}
\setlength{\cftbeforesubsecskip}{2pt}

\titleformat{\section}
  {\normalfont\Large\bfseries}{\thesection.}{0.5em}{}
\titleformat{\subsection}
  {\normalfont\large\bfseries}{\thesubsection.}{0.5em}{}

\begin{document}

\begin{flushleft}
    \normalsize
    Către: IPJ Cluj\\
    cabinet@cj.politiaromana.ro\\
\end{flushleft}

\begin{flushleft}
    \normalsize
    Către: Inspectoratul General al Jandarmeriei Romane\\
    jandarmerie@mai.gov.ro\\
\end{flushleft}

\begin{flushleft}
    \normalsize
    Către: BIROUL ELECTORAL CENTRAL\\
    secretariat@bec.ro\\
\end{flushleft}

\begin{flushleft}
    \normalsize
    Către: Biroul electoral județean nr. 13 CLUJ\\
    bej.cluj@bec.ro\\
\end{flushleft}

\vspace{1cm}

Subsemnatul Deleanu Ștefan-Lucian, identificat prin act de identitate electronic nr. CJ10026, domiciliat în Jud. Cluj, Cluj-Napoca, Str. Aurel Vlaicu, nr. 2, bloc 5A, Sc. I, etaj 7, ap. 28, în virtutea calității de observator electoral acreditat de Funky Citizens, asociație legal acreditată de Autoritatea Electorală Permanentă prin ACREDITAREA nr. 30688/13.10.2024, în conformitate cu și în temeiul dispozițiilor imperative ale Legii 208/2015 privind alegerea Senatului și a Camerei Deputaților, cu modificările și completările ulterioare, formulez și înaintez prezenta:

\vspace{0.5cm}
\begin{center}
\textbf{\Large PLÂNGERE CONTRAVENȚIONALĂ}
\end{center}
\vspace{0.5cm}

prin intermediul căreia sesizez și aduc la cunoștință săvârșirea contravențiilor prevăzute și definite de art. 98 lit. t), art. 99 alin (1), alin. (2) lit. a) din cuprinsul Legii 208/2015 privind alegerea Senatului și a Camerei Deputaților.

În procesul de evaluare a caracterului de propagandă electorală al aspectelor semnalate, am avut în considerare definiția statuată de art. 36 pct. 7 din Legea 334/2006, republicată, cu modificările și completările ulterioare.

Dat fiind natura generală și amploarea conduitei contravenționale, ce ia forma a peste 100 de fapte contravenționale distincte, cu un caracter efemer, se impune cu necesitate constatarea cu celeritate a acestora, motiv pentru care am procedat la sesizarea, în mod concomitent, a tuturor organelor abilitate în acest sens, respectiv: ofițerii, agenții și subofițerii din cadrul Poliției Române, Poliției de Frontieră Române și Jandarmeriei Române, precum și polițiștii locali.

Mai mult, fiind vorba de aspecte ce influenteaza scrutinul parlamentar, am procedat la notificarea Biroului Electoral Central și a Biroului Electoral Județean nr. 13 CLUJ, în vederea luării măsurilor legale ce se impun.

Intrucat caracterul analizei este unul cu un caracter subiectiv, este notabil ca anumite aspecte sesizate pot fi interpretate diferit de către organele abilitate, motiv pentru care se impune o analiză detaliată a postărilor cu potențial caracter electoral propagandistic, în conformitate cu prevederile legale în vigoare.

Spre exemplu, multe partide au incercat substituirea campaniei intr-o campanie pentru "alegeri locale", insa scopul fiind vadit pentru atragerea unui numar cat mai mare de locuri la alegerile parlamentare. Alte partide folosesc PBN-uri (private blog networks) pentru a crea ideea ca postarile sunt neutre, jurnalistice, insa in realitate sunt postari cu caracter electoral.

Desi Facebook obliga furnizarea de informatii reale despre titularul reclamei si cel care plateste pentru ea, aceste informatii sunt deseori false, iar Facebook nu verifica aceste informatii. Este notabil faptul ca declararea in fals a acestor informatii este de natura sa se incadreze la Art. 325 Cod Penal - Falsul informatic, astfel incat daca organul de constatare a contraventiilor are si competenta in constatarea faptelor penale, ii rugam sa se sesizeze si cu privire la aceste aspecte.

\noindent\rule{\textwidth}{1pt}

Organul competent cu constatarea contravențiilor poate accesa link-urile în formă originală pentru o analiză detaliată a postărilor cu potențial caracter electoral propagandistic, prin \textbf{dublu click pe ID-urile postărilor respective}.

\textbf{Metodologia aplicată în cadrul studiului ce a fundamentat prezenta plângere, cu scopul asigurării unui caracter echidistant și obiectiv în analiza naturii postărilor, poate fi consultată accesând următorul link:}

\href{https://github.com/Stefatorus/observator-electoral-transparenta}{https://github.com/Stefatorus/observator-electoral-transparenta}

Analizele au fost efectuate in perioada 30.11.2024, incepand cu ora 18:00, pana in data de 30.11.2024, la ora 19:30, folosind analiza automata cu AI, si a vizat EXCLUSIV platforma "META" (Facebook, Instagram, WhatsApp).

\tableofcontents
\newpage

\section{Împotriva numiților}

\subsection{ALVERNA CONNECT SRL}
Următoarele fapte contravenționale sunt sesizate împotriva acestei entități:

\begin{enumerate}[leftmargin=*, label=\arabic*.)]
    \item publicarea unei reclame platite pe Facebook dupa ora 18:00 pe 30.11.2024, cu ID-ul postarii \href{https://www.facebook.com/ads/library/?id=1085453249594392}{1085453249594392}, care, desi nu mentioneaza explicit un candidat, contine un numar CMF (11240027), este directionata catre un public specific si are ca scop influentarea votului in favoarea lui Sorin Nas prin asocierea acestuia cu valori traditionale.  Mesajul vag ("Credinta in Dumnezeu si in valorile noastre este forta care ne-a tinut uniti de-a lungul istoriei") este utilizat intr-un context electoral, amplificat de link-ul catre pagina de Facebook a lui Sorin Nas, avand un efect electoral pozitiv.  Reclama, cu o cheltuiala de sub 100 RON, a atins peste 1 milion de oameni, conform datelor din Ad Library, contactul fiind office@neximo.ro si +40745223355.
    \item publicarea, dupa ora 18:00 pe 30.11.2024, a unei postari pe Facebook (ID: \href{https://www.facebook.com/ads/library/?id=563009356458380}{563009356458380}) cu scopul de a influenta votul alegatorilor in favoarea Partidului S.O.S. Romania la alegerile parlamentare si a lui Calin Georgescu la alegerile prezidentiale,  prin afirmatii precum "Cum votam pe 1 si 8 Decembrie? - 1 Decembrie votam Partidul S.O.S. Romania la Parlamentare - 8 Decembrie votam Calin Georgescu in turul II la Prezidentiale!",  avand un efect electoral clar si vizibil, demonstrat de bugetul alocat reclamei (sub 100 RON) si de o estimare a audientei de peste 1 milion de persoane. Prezenta codului CMF: 11240027 confirma natura electorala a postarii.  Postarea, prin continutul sau explicit si indemnul direct la vot, constituie propaganda electorala dupa incheierea perioadei legale de campanie, incalcand astfel prevederile legale.  Datele de contact ale advertiserului sunt: office@neximo.ro si +40745223355.
    \item publicarea unei postari platite pe Facebook (ID: \href{https://www.facebook.com/ads/library/?id=594468426393957}{594468426393957}), dupa ora 18:00 pe 30.11.2024, care promoveaza Partidul S.O.S. Romania pentru alegerile parlamentare, mentionand pozitia lor pe buletinul de vot (\#9) si indemnand la vot ("Votam Partidul S.O.S. Romania la Parlamentare").  Postarea contine un numar CMF (11240027), indicand clar natura sa de propaganda electorala.  Efectul electoral este evident, avand ca obiectiv influentarea voturilor in favoarea partidului mentionat.  Chemarea la actiune directa, combinata cu bugetul alocat reclamei (sub 100 RON, conform datelor din Ad Library) si o estimare a audientei de peste 1 milion de persoane, demonstreaza intentia clara de a influenta rezultatul alegerilor.
\end{enumerate}

\vspace{0.5cm}

\subsection{AUR}
Următoarele fapte contravenționale sunt sesizate împotriva acestei entități:

\begin{enumerate}[leftmargin=*, label=\arabic*.)]
    \item publicarea unei postari pe Facebook (ID: \href{https://www.facebook.com/ads/library/?id=1551664468825639}{1551664468825639}) dupa ora 18:00 pe 30.11.2024, care constituie propaganda electorala interzisa. Postarea, care a beneficiat de o cheltuiala de RON 100-RON 199 si a atins o audienta de peste 1 milion de persoane, promoveaza explicit candidatul Dan Cosma (AUR), solicitand voturi pentru acesta si pentru partid, folosind fraze precum Un vot pentru viitorul nostru, Haideti la urne!, si La parlamentare, votati AUR.  Prezenta numarului CMF:11240014 confirma natura electorala a postarii.  Aceste actiuni au un efect electoral clar, vizand influentarea voturilor in favoarea lui Dan Cosma si AUR, incalcand astfel prevederile legale privind interzicerea propagandei electorale dupa incheierea perioadei legale.  Informatiile de contact ale advertiserului sunt: cosdan63@gmail.com si +407666663210.
    \item publicarea unei reclame platite pe Facebook (ID postare: \href{https://www.facebook.com/ads/library/?id=1734603870433438}{1734603870433438}) dupa ora 18:00 pe 30.11.2024, care promoveaza candidatura sa pentru Senat si Camera Deputatilor din partea AUR, cu mentionarea explicita a pozitiei sale pe buletinul de vot ("pozitia 1 pe buletinul de vot") si link-uri catre platforma sa electorala (planulsimion.ro si programaur.ro), avand un efect electoral clar de influentare a voturilor in favoarea sa si a partidului AUR. Reclama, cu o cheltuiala estimata intre 100 si 199 RON, a atins o audienta estimata intre 100.000 si 500.000 de persoane in judetul Bistrita-Nasaud, conform datelor din Facebook Ad Library. Prezenta numarului CMF 11240014 confirma natura electorala a reclamei.  Aceasta actiune constituie propaganda electorala dupa incheierea perioadei legale, incalcand prevederile legale.  Datele de contact ale advertiserului sunt: cosdan63@gmail.com si +40766663210.
    \item publicarea unei postari pe Facebook (ID: \href{https://www.facebook.com/ads/library/?id=472663805845598}{472663805845598}) dupa ora 18:00 pe 30.11.2024, care promoveaza explicit candidatura sa la Senat din partea AUR, folosind fraze precum "Nu pot ei fura cat putem noi vota!" si "La parlamentare, votati AUR  pozitia 1 pe buletinul de vot!", avand un efect electoral clar de influentare a votului in favoarea sa si a partidului AUR. Postarea, care a beneficiat de o investitie financiara intre 100 si 199 RON, a generat intre 10.000 si 15.000 de impresii, cu o estimare a acoperirii de peste 1 milion de persoane, contine numarul CMF: 11240014 si este asociata cu adresa de email cosdan63@gmail.com si numarul de telefon +407666663210.  Aceste elemente, coroborate cu lipsa oricarui element jurnalistic obiectiv, demonstreaza clar intentia de a influenta procesul electoral prin propaganda electorala interzisa dupa incheierea perioadei legale de campanie.
    \item publicarea unei reclame platite pe Facebook dupa ora 18:00 pe 30.11.2024, cu ID-ul \href{https://www.facebook.com/ads/library/?id=499403779084639}{499403779084639}, care contine un mesaj electoral persuasiv, utilizand limbaj patriotic puternic ("unim toata suflarea romaneasca," "rescriem istoria," "redobandirii demnitatii si suveranitatii nationale") si un apel direct la actiune in ziua alegerilor ("duminica, prin votul nostru").  Prezenta numarului CMF 11240014 indica clar natura electorala a reclamei, iar datele din sectiunea `aaa\_info` confirma ca platitorul este Alianta pentru Unirea Romanilor, beneficiarul fiind Razvan Biro.  Efectul electoral este evident, avand in vedere limbajul utilizat si scopul clar de a influenta votul in favoarea AUR si/sau a lui Razvan Biro. Reclama a avut o estimare a audientei de 100.000-500.000 de persoane, iar cheltuielile au fost sub 100 RON, conform datelor din Ad Library.  Datele de contact ale advertiserului sunt contact@razvanbiro.ro si +40751303330.
    \item publicarea unei postari platite pe Facebook (ID \href{https://www.facebook.com/ads/library/?id=555150543924818}{555150543924818}) dupa ora 18:00 pe 30.11.2024, care promoveaza indirect candidatul George Simion la presedintie, asociindu-l cu AUR si indemnand la vot pentru acesta, avand un efect electoral clar asupra alegerilor parlamentare. Postarea, cu un buget de 600-699 RON, a atins 100.000-125.000 de impresii si a vizat in principal judetul Hunedoara, utilizand un limbaj persuasiv ("moment istoric," "schimbare adevarata") si continand numarul CMF 11240014.  Aceasta postare, desi nu mentioneaza explicit alegerile parlamentare, are ca obiectiv influentarea votului in favoarea AUR, reprezentand astfel propaganda electorala interzisa dupa incheierea perioadei legale de campanie.  Contact: iuliangros8@gmail.com, +40728044104, https://www.facebook.com/aursalasudesus/.
    \item publicarea, dupa ora 18:00 pe 30.11.2024, a unei postari pe Facebook (ID: \href{https://www.facebook.com/ads/library/?id=578490801208263}{578490801208263}) cu continut electoral, care promoveaza candidatura lui Marian Dumitru pentru Senat din partea AUR, cu mesajul explicit Votati AUR  viitorul apartine celor curajosi!, avand ca efect electoral influentarea voturilor in favoarea candidatului si partidului mentionat. Postarea, realizata la comanda Partidului Alianta pentru Unirea Romanilor si identificata prin CMF 11240014, se adreseaza publicului larg din Calarasi, depasind limitele activitatii jurnalistice si constituind material de propaganda electorala dupa incheierea perioadei legale de campanie.  Postarea a generat intre 10.000 si 15.000 de impresii, cu o estimare a acoperirii de 100.000-500.000 de persoane,  costul fiind intre 200 si 299 RON.  Informatiile de contact ale advertiserului sunt: redactie@calarasipress.ro si +40734642449.
    \item publicarea unei postari pe Facebook (ID: \href{https://www.facebook.com/ads/library/?id=593615980022376}{593615980022376}) dupa ora 18:00 pe 30.11.2024, care contine propaganda electorala pentru candidatul AUR Dan Cosma,  cu un buget de 200-299 RON,  ajungand la peste 1 milion de oameni,  prin utilizarea CMF-ului 11240014,  mentionarea explicita a pozitiei sale pe buletinul de vot si atacuri la adresa candidatilor contracandidati,  avand ca efect electoral influentarea votului in favoarea lui Dan Cosma si AUR.  Postarea, desi mascata sub forma unui comentariu politic,  este lipsita de obiectivitate si neutralitate jurnalistica,  utilizand un limbaj inflamator si apeluri directe la actiune,  constituind astfel propaganda electorala interzisa dupa incheierea perioadei de campanie.  Datele de contact ale advertiserului sunt: cosdan63@gmail.com si +407666663210.
    \item publicarea, dupa ora 18:00 pe 30.11.2024, a unei postari pe Facebook (ID \href{https://www.facebook.com/ads/library/?id=615181904175327}{615181904175327}) cu continut de propaganda electorala, care promoveaza candidatul Sorin Cristian Alic (AUR) pentru functia de senator. Postarea, realizata la comanda Partidului Alianta pentru Unirea Romanilor, contine imagini cu candidatul si membri ai partidului, insotite de indemnul explicit VOTEAZA AUR!, avand ca efect electoral influentarea voturilor in favoarea acestuia.  Prezenta numarului CMF 11240014 confirma natura de material electoral.  Postarea, cu o cheltuiala de sub 100 RON, a generat intre 3000 si 4000 de impresii, ajungand la o estimare de 100.000-500.000 de persoane, conform datelor din Ad Library.  Informatiile de contact ale advertiserului sunt disponibile: sorin.alic@yahoo.com si +40722501049.  Chiar daca data de incheiere a campaniei este mentionata dupa data alegerilor, contextul vizual si textual al postarii indica clar ca aceasta a fost publicata inainte de incheierea perioadei legale de campanie.
    \item publicarea unei postari pe Facebook (ID: \href{https://www.facebook.com/ads/library/?id=936673471712074}{936673471712074}), dupa ora 18:00 pe 30.11.2024, cu un buget de RON600-RON699, care atinge 35K-40K impresii,  continand critici vehemente la adresa coalitiei PSD-PNL-USR, asociind actiunile acestora cu alegerile parlamentare si indemnand implicit la vot impotriva lor, avand un efect electoral clar in favoarea AUR.  Postarea, cu textul "Romanii ar trebui sa taxeze coalitia toxica PSD-PNL-USR din doua motive...", utilizeaza un limbaj persuasiv si manipulativ,  incalcand prevederile legale privind propaganda electorala dupa incheierea perioadei legale.  Contact: iuliangros8@gmail.com, +40728044104, Bulevardul 22 Decembrie, bl. D3, Parter, Deva, Romania 330084, RO.
    \item publicarea, dupa ora 18:00 pe 30.11.2024, a unei postari pe Facebook (ID \href{https://www.facebook.com/ads/library/?id=9515737678442617}{9515737678442617}) cu continut electoral,  care, desi nu solicita explicit voturi pentru Marian Dumitru, presedintele Organizatiei Judetene AUR Calarasi, creeaza o legatura directa intre acesta si o promisiune de imbunatatire a transparentei in sedintele consiliilor locale.  Postarea, platita intre 700 si 799 RON, a atins 60.000-70.000 de impresii, avand ca tinta judetul Calarasi, si utilizeaza codul CMF 11240014, indicand clar natura electorala a mesajului.  Efectul electoral este evident, prin asocierea candidatului cu o solutie la o problema publica, sugerand ca votul pentru AUR va aduce transparenta.  Contact: redactie@calarasipress.ro, +40734642449, https://calarasipress.ro/.
\end{enumerate}

\vspace{0.5cm}

\subsection{Atac la Persoana Oficial}
Următoarele fapte contravenționale sunt sesizate împotriva acestei entități:

\begin{enumerate}[leftmargin=*, label=\arabic*.)]
    \item publicarea unei reclame platite pe Facebook (ID postare: \href{https://www.facebook.com/ads/library/?id=469467039491354}{469467039491354}), dupa ora 18:00 pe 30.11.2024, care face referire la Partidul Social Democrat (PSD), prezentand o analiza critica a structurii sale interne si a conducerii, cu un ton evident negativ si subiectiv, influentand negativ opinia publica fata de partid.  Reclama, cu o estimare a vizualizarilor de 1K-2K si un potential de peste 1 milion de persoane, depaseste limitele jurnalismului obiectiv, avand ca efect electoral o potentiala diminuare a voturilor pentru PSD.  Continutul articolului legat, accesibil prin link,  sustine acuzatiile de propaganda electorala, avand in vedere contextul electoral si cheltuielile financiare sub 100 RON, conform datelor din Ad Library.  Datele de contact ale advertiserului sunt asociatiakogaionon@gmail.com si +40744343554.
\end{enumerate}

\vspace{0.5cm}

\subsection{Ciprian Paraschiv}
Următoarele fapte contravenționale sunt sesizate împotriva acestei entități:

\begin{enumerate}[leftmargin=*, label=\arabic*.)]
    \item publicarea unei reclame platite pe Facebook (ID postare: \href{https://www.facebook.com/ads/library/?id=967965638717422}{967965638717422}), dupa ora 18:00 pe 30.11.2024, care promoveaza explicit partidul AUR, indemnand la votul pentru acesta cu fraza "Votam AUR", avand un efect electoral clar de influentare a votantilor in favoarea AUR. Reclama, cu o cheltuiala de RON300-RON399 si o estimare a audientei de 100.000-500.000 persoane,  indeplineste toate criteriile propagandei electorale conform articolului 36 (7) din LEGEA nr. 334 din 17 iulie 2006,  nefiind jurnalism sau opinie personala, ci o sustinere directa a unui candidat electoral.  Datele de contact ale advertiserului sunt partidauriasi@gmail.com si +40742353934.
\end{enumerate}

\vspace{0.5cm}

\subsection{Comunitatea Liberală 1848}
Următoarele fapte contravenționale sunt sesizate împotriva acestei entități:

\begin{enumerate}[leftmargin=*, label=\arabic*.)]
    \item publicarea pe Facebook, dupa ora 18:00 pe 30.11.2024, a unei postari platite (ID \href{https://www.facebook.com/ads/library/?id=410458798687837}{410458798687837}) intitulata "Scrisoare catre un votant al lui Georgescu", care, desi prezentata sub forma unei scrisori personale, vizeaza indirect candidatul Calin Georgescu si are potentialul de a influenta votul publicului.  Postarea, cu o cheltuiala de sub 100 RON, a atins peste 5000 de impresii, conform datelor din Ad Library, si utilizeaza un ton emotional pentru a-si exprima opinia despre candidatul mentionat, fara a respecta standardele jurnalismului obiectiv.  Efectul electoral este de a crea o perceptie negativa asupra lui Calin Georgescu, influentand astfel decizia de vot a publicului larg.  Contactul poate fi realizat la redactia@comunitatealiberala.ro sau +40723632641.
\end{enumerate}

\vspace{0.5cm}

\subsection{Cyrila SRL}
Următoarele fapte contravenționale sunt sesizate împotriva acestei entități:

\begin{enumerate}[leftmargin=*, label=\arabic*.)]
    \item publicarea unei reclame platite pe Facebook (ID postare: \href{https://www.facebook.com/ads/library/?id=389597150814992}{389597150814992}), dupa ora 18:00 pe 30.11.2024, care promoveaza candidatura lui Todoran Adrian, folosind imaginea acestuia si un mesaj explicit de indemn la vot ("Luptam cu toate FORTA pentru Valorile Familiei Crestine \#vot"). Reclama, cu o cheltuiala estimata intre 1500 si 2000 RON, a atins intre 150.000 si 175.000 de impresii, avand ca efect electoral influentarea votului in favoarea lui Todoran Adrian.  Aceasta actiune constituie propaganda electorala interzisa dupa incheierea perioadei legale de campanie, conform articolului 98 t).  Datele de contact ale advertiserului sunt disponibile in Ad Library: contact@cyrila.ro si +40747635571.
\end{enumerate}

\vspace{0.5cm}

\subsection{Dumitru Rujan}
Următoarele fapte contravenționale sunt sesizate împotriva acestei entități:

\begin{enumerate}[leftmargin=*, label=\arabic*.)]
    \item publicarea unei postari pe Facebook (ID: \href{https://www.facebook.com/ads/library/?id=1556211045019020}{1556211045019020}) cu caracter de propaganda electorala dupa incheierea perioadei legale de campanie, respectiv dupa ora 18:00 pe 30.11.2024,  postare care promoveaza explicit Partidul National Liberal (PNL) si indeamna la votarea acestuia.  Postarea, platita cu o suma sub 100 RON, a generat intre 1000 si 2000 de impresii, avand o estimare a acoperirii de 100.000 - 500.000 de persoane. Fraze precum Votul pentru PNL este votul pentru o Romanie moderna si Votati PNL pentru o Romanie puternica in Europa demonstreaza clar intentia de a influenta votul in favoarea PNL.  Contactul advertiserului este topliceanuflorin@yahoo.com si +40767242025.  Imaginea atasata postarii prezinta o persoana votand, ceea ce poate fi interpretat ca o aprobare implicita a actului de votare pentru un candidat.
\end{enumerate}

\vspace{0.5cm}

\subsection{FD}
Următoarele fapte contravenționale sunt sesizate împotriva acestei entități:

\begin{enumerate}[leftmargin=*, label=\arabic*.)]
    \item publicarea unei reclame platite pe Facebook (ID postare: \href{https://www.facebook.com/ads/library/?id=1622251322038541}{1622251322038541}), dupa ora 18:00 pe 30.11.2024, care promoveaza explicit Partidul Forta Dreptei, indemnand cetatenii sa voteze pentru acesta cu fraze precum Duminica votam Forta Dreptei! Duminica votam bine!, avand un efect electoral clar de influentare a votului in favoarea partidului mentionat.  Reclama, cu o cheltuiala de sub 100 RON, a atins o estimare de 10.000-15.000 de impresii si o estimare a audientei de 100.000-500.000 de persoane, conform datelor din Ad Library.  Informatiile de contact ale advertiserului sunt diaspora@pntmm.ro si +40770958865.  Aceasta actiune constituie propaganda electorala interzisa dupa incheierea perioadei legale de campanie, conform articolului 98 t).
    \item publicarea unei reclame platite pe Facebook dupa ora 18:00 pe 30.11.2024, cu ID-ul postarii \href{https://www.facebook.com/ads/library/?id=571327662183867}{571327662183867}, care promoveaza explicit partidul Forta Dreptei si indeamna la vot pentru acesta, folosind fraze precum "Sustineti tinerii, votati Forta Dreptei! Trimite in Parlament oameni cinstiti!", avand un efect electoral clar de influentare a votului in favoarea partidului mentionat. Reclama, cu o cheltuiala de sub 100 RON, a atins o audienta estimata intre 50.000 si 100.000 de persoane, conform datelor din Ad Library, si a fost publicata de catre Florin - Emanuel Dobria, cu datele de contact dobria.florinemanuel@yahoo.ro si +40769174551.  Aceasta actiune constituie propaganda electorala dupa incheierea perioadei legale, incalcand prevederile legale.
    \item publicarea unei reclame platite pe Facebook dupa ora 18:00 pe 30.11.2024 (verificati data exacta a incheierii campaniei electorale), cu ID-ul postarii \href{https://www.facebook.com/ads/library/?id=887235546850416}{887235546850416},  avand ca scop promovarea candidaturii sale la alegerile parlamentare pentru Forta Dreptei. Postarea, intitulata "MANIFEST pentru un Parlament al bunului simt" si continand numarul CMF 11240022, se adreseaza unui public larg (peste 1 milion de persoane, conform estimarilor), avand un efect electoral clar pozitiv pentru candidatul Palaz.  Reclama, cu o cheltuiala de 300-399 RON, a fost publicata pe Facebook si Instagram,  utilizand datele de contact miscareapopulara.constanta@gmail.com si +40720888888.  Aceasta actiune constituie propaganda electorala dupa incheierea perioadei legale, fiind astfel o incalcare grava a legislatiei electorale.
\end{enumerate}

\vspace{0.5cm}

\subsection{FRP}
Următoarele fapte contravenționale sunt sesizate împotriva acestei entități:

\begin{enumerate}[leftmargin=*, label=\arabic*.)]
    \item publicarea unei reclame platite pe Facebook dupa ora 18:00 pe 30.11.2024 (ID postare: \href{https://www.facebook.com/ads/library/?id=1525524132172391}{1525524132172391}), care promoveaza explicit candidatul Didi Dumitru pentru Senat in judetul Gorj din partea partidului Alternativa pentru Demnitatea Nationala (ADN), cu scopul clar de a influenta votul alegatorilor.  Postarea, care a avut o estimare a vizualizarilor intre 30.000 si 35.000 si o estimare a audientei intre 100.000 si 500.000, contine fraze precum "Votam Didi Dumitru" si "Votam Alternativa pentru Demnitatea Nationala - ADN!", reprezentand o incalcare evidenta a legii prin continuarea propagandei electorale dupa incheierea perioadei legale.  Reclama, pentru care s-au cheltuit intre 400 si 499 RON, a fost publicata de Federatia Romanilor de Pretutindeni, cu datele de contact disponibile in Ad Library: sedecoro@gmail.com si +40774533702.
    \item publicarea, dupa ora 18:00 pe 30.11.2024, a unei postari pe Facebook (ID \href{https://www.facebook.com/ads/library/?id=531685073178199}{531685073178199}) cu scopul de a influenta votul in favoarea candidatului Emanuel Cioaca (ADN) la alegerile parlamentare, folosind fraze precum "Votam masiv" si "Alaturati-va noua" intr-o campanie platita (RON300-RON399) care a atins 10.000-15.000 de impresii, cu o estimare a raspandirii de 500.000-1.000.000 de persoane, avand in vedere ca perioada de campanie electorala s-a incheiat,  constituie propaganda electorala interzisa.  Contact: sedecoro@gmail.com, +40774533702, Strada Italia 4, Chiajna , Ilfov 077040, RO.
    \item publicarea unei reclame electorale pe Facebook dupa ora 18:00 pe 30.11.2024 (data exacta a incheierii campaniei electorale trebuie verificata), cu ID-ul postarii \href{https://www.facebook.com/ads/library/?id=620144230341700}{620144230341700}, care promoveaza explicit candidatul Ioan Moldovan de la Alternativa pentru Demnitate Nationala (ADN) pentru Camera Deputatilor, pozitia 20, cu un buget de 200-299 RON, atingand 20.000-25.000 de impresii si o estimare a raspandirii de 100.000-500.000 de persoane.  Postarea contine un apel direct la vot ("Votam...") si numarul CMF11240063, elemente caracteristice propagandei electorale interzise dupa incheierea campaniei.  Efectul electoral este clar, vizand cresterea numarului de voturi pentru candidatul mentionat.  Informatiile de contact ale advertiserului sunt disponibile in Ad Library: sedecoro@gmail.com si +40774533702.
    \item publicarea dupa ora 18:00 pe 30.11.2024 a unei reclame electorale pe Facebook (ID postare: \href{https://www.facebook.com/ads/library/?id=993705712527586}{993705712527586}), cu cheltuieli estimate intre 400 si 499 RON, care promoveaza candidatul Costin Movila de la Alternativa pentru Demnitate Nationala (ADN) pentru Camera Deputatilor, pozitia 12, cu un mesaj explicit de indemn la vot ("Votam..."), avand ca efect electoral cresterea numarului de voturi pentru candidatul mentionat, fapt demonstrat de prezenta CMF-ului 11240063 si de datele din Ad Library privind acoperirea estimata de 100.000-500.000 de persoane.  Aceasta reclama constituie propaganda electorala dupa incheierea perioadei legale de campanie, fiind astfel ilegala. Contact: sedecoro@gmail.com, +40774533702, Strada Italia 4, Chiajna , Ilfov 077040, RO.
\end{enumerate}

\vspace{0.5cm}

\subsection{FRP TV}
Următoarele fapte contravenționale sunt sesizate împotriva acestei entități:

\begin{enumerate}[leftmargin=*, label=\arabic*.)]
    \item publicarea dupa ora 18:00 pe 30.11.2024 a unei reclame pe Facebook (ID postare: \href{https://www.facebook.com/ads/library/?id=420253954352755}{420253954352755}), cu cheltuieli intre 300 si 399 RON, care promoveaza candidata Diana Moraru pentru un post in Cluj, folosind hashtag-uri precum \#DianaMorarudeputat si \#Alegeriparlamentare2024, avand ca efect electoral influentarea votului in favoarea acesteia.  Reclama, desi nu mentioneaza explicit un partid, are un caracter promotional si se adreseaza publicului larg, depasind limitele activitatii jurnalistice, indeplinind astfel conditiile din articolul 36 (7) din LEGE nr. 334 din 17 iulie 2006.  Reclama a atins peste 30.000-35.000 de impresii, conform datelor din Ad Library, si a fost platita de catre Federatia Romanilor de Pretutindeni, cu contactul sedecoro@gmail.com si +40774533702.
    \item publicarea dupa ora 18:00 pe 30.11.2024 a unei reclame pe Facebook (ID postare: \href{https://www.facebook.com/ads/library/?id=920139606746512}{920139606746512}), cu un buget de 300-399 RON, care promoveaza Alternativa pentru Demnitate Nationala, cu mesajul "Anunt important", avand ca efect electoral cresterea vizibilitatii si a sprijinului pentru acest partid.  Postarea, lipsita de elemente jurnalistice, indeplineste toate criteriile materialului de propaganda electorala conform art. 36 (7) din LEGE nr. 334/2006, fiind difuzata dupa incheierea campaniei electorale, fapt ce constituie o incalcare grava a legislatiei in vigoare.  Reclama a atins o audienta estimata intre 100.000 si 500.000 de persoane, conform datelor din Ad Library, cu informatii de contact disponibile: sedecoro@gmail.com si +40774533702.
\end{enumerate}

\vspace{0.5cm}

\subsection{FreeInfo}
Următoarele fapte contravenționale sunt sesizate împotriva acestei entități:

\begin{enumerate}[leftmargin=*, label=\arabic*.)]
    \item publicarea unei reclame platite pe Facebook (ID postare: \href{https://www.facebook.com/ads/library/?id=435262922971000}{435262922971000}), dupa ora 18:00 pe 30.11.2024, care promoveaza explicit candidatul ANC pentru Parlamentul Romaniei si pe Calin Georgescu pentru Presedintia Romaniei, avand ca efect electoral influentarea voturilor in favoarea acestora.  Reclama, cu o cheltuiala estimata intre 200 si 299 RON, a atins o audienta de peste 1 milion de persoane, concentrandu-se pe Bucuresti si zonele limitrofe.  Textul reclamei, care indeamna la "concentrarea voturilor" catre candidatii mentionati, demonstreaza clar intentia de a influenta rezultatul alegerilor, indeplinind toate criteriile materialului de propaganda electorala conform articolului 36 (7) din LEGE nr. 334 din 17 iulie 2006.  Contactul advertiserului este mitimodoran@yahoo.com si +40774658934.
\end{enumerate}

\vspace{0.5cm}

\subsection{Fruncea}
Următoarele fapte contravenționale sunt sesizate împotriva acestei entități:

\begin{enumerate}[leftmargin=*, label=\arabic*.)]
    \item publicarea unei postari platite pe Facebook (ID: \href{https://www.facebook.com/ads/library/?id=386674257771104}{386674257771104}), dupa ora 18:00 pe 30.11.2024, cu mesajul "INCREDIBIL CE FACE PSD pentru a-l avea pe G. Simion in turul 2",  care constituie propaganda electorala interzisa dupa incheierea perioadei de campanie.  Postarea, cu o estimare a vizualizarilor intre 100.000 si 500.000, are ca scop influentarea negativa a opiniei publice fata de PSD si implicit, asupra votului alegatorilor, prin prezentarea unei afirmatii subiective si tendentioase, lipsita de neutralitate jurnalistica.  Cheltuielile pentru aceasta postare se situeaza intre 500 si 599 RON.  Informatiile de contact ale advertiserului sunt: fulgeralbastru99@gmail.com si +40747768451.
\end{enumerate}

\vspace{0.5cm}

\subsection{Gazeta de Nord-Vest}
Următoarele fapte contravenționale sunt sesizate împotriva acestei entități:

\begin{enumerate}[leftmargin=*, label=\arabic*.)]
    \item publicarea unei postari platite pe Facebook, dupa ora 18:00 pe 30.11.2024, cu ID-ul \href{https://www.facebook.com/ads/library/?id=3870108766578700}{3870108766578700}, care contine afirmatii defaimatoare la adresa candidatului Adrian Cozma, presedintele PNL Satu Mare, prezentate intr-un mod care sugereaza o intentie clara de a influenta votul alegatorilor.  Titlul postarii, "Camorra de Satu Mare partea a II-a. Grupul infractional organizat din jurul presedintelui PNL Adrian Cozma", este extrem de sugestiv si are un efect electoral negativ asupra candidatului mentionat.  Postarea, cu o estimare a vizualizarilor intre 45.000 si 50.000, a fost platita cu o suma cuprinsa intre 500 si 599 RON, conform datelor din Ad Library, si a vizat o audienta semnificativa din judetul Satu Mare.  Aceste actiuni constituie propaganda electorala dupa incheierea perioadei legale, avand ca scop influentarea votului in mod negativ asupra candidatului Adrian Cozma.  Datele de contact ale advertiserului sunt publicitate@gazetanord-vest.ro si +40747640578.
    \item publicarea unei postari platite pe Facebook (ID \href{https://www.facebook.com/ads/library/?id=427523277100495}{427523277100495}), dupa ora 18:00 pe 30.11.2024, cu titlul sugestiv "Camorra de Satu Mare" si referire la Adrian Cozma, Nicolae Cornea si Serban Zoro Calin, membri PNL, prezentandu-i intr-o lumina negativa,  avand un efect electoral negativ asupra PNL prin denigrarea unor membri ai partidului.  Postarea, desi nu solicita explicit votul pentru sau impotriva unui candidat, are potentialul de a influenta opinia publica prin asocierea acestor indivizi cu activitati dubioase, afectand astfel imaginea PNL.  Postarea, cu o estimare a vizualizarilor intre 50.000 si 60.000, a costat intre 800 si 899 RON, conform Ad Library, si a fost publicata de Gazeta de Nord-Vest, cu contactul publicitate@gazetanord-vest.ro si +40747640578.
\end{enumerate}

\vspace{0.5cm}

\subsection{INDEPENDENT}
Următoarele fapte contravenționale sunt sesizate împotriva acestei entități:

\begin{enumerate}[leftmargin=*, label=\arabic*.)]
    \item publicarea unei postari pe Facebook dupa ora 18:00 pe 30.11.2024 (verificare necesara a datei exacte), cu ID-ul \href{https://www.facebook.com/ads/library/?id=1120855423004477}{1120855423004477}, care promoveaza o propunere legislativa privind salariile soferilor de camion, continand un cod CMF11240003,  avand ca efect electoral cresterea popularitatii candidatului si influentarea votului in favoarea sa. Postarea, desi nu mentioneaza explicit un candidat, este asociata cu pagina de Facebook a lui Pop Sebastian Tiberiu, clasificata ca Candidat politic, si a fost finantata cu o suma cuprinsa intre 100 si 199 RON, ajungand la peste 1 milion de oameni, conform datelor din Ad Library.  Utilizarea codului CMF si tonul persuasiv al postarii, care vizeaza un public larg pe o platforma publica, demonstreaza intentia clara de a influenta votul, constituind astfel propaganda electorala ilegala dupa incheierea perioadei legale de campanie.  Datele de contact ale advertiserului sunt contact@sebastiantiberiupop.ro si +40760256257.
    \item publicarea unei postari pe Facebook (ID: \href{https://www.facebook.com/ads/library/?id=1130310645194385}{1130310645194385}) dupa ora 18:00 pe 30.11.2024, care contine propaganda electorala sub forma unui apel la vot impotriva aliantei PSD-PNL-USR, evidentiind avantajul numeric al acestei aliante si indemnand implicit la votarea altor candidati. Postarea, desi prezentata ca o opinie personala, are un efect electoral clar, vizand influentarea votului prin prezentarea unor informatii tendentioase si un apel emotional la actiune. Prezenta numarului CMF11240003 sugereaza o legatura directa cu campania electorala.  Cheltuielile pentru aceasta postare, conform Ad Library, sunt sub 100 RON, dar impactul asupra a 5000-6000 de impresii si o estimare a acoperirii de 100.000-500.000 de persoane, demonstreaza potentialul de influentare a votului.  Datele de contact ale advertiserului sunt contact@sebastiantiberiupop.ro si +40760256257.
    \item publicarea unei postari pe Facebook dupa ora 18:00 pe 30.11.2024 (conform datei de incheiere a postarii din arhiva de reclame), cu scopul de a influenta votul alegatorilor in favoarea sa, prin promovarea candidaturii sale la alegerile parlamentare si prin evidentierea unor realizari guvernamentale,  avand un efect electoral clar,  incalcand astfel prevederile legale privind perioada de campanie electorala. Postarea, cu ID-ul \href{https://www.facebook.com/ads/library/?id=1235337024405561}{1235337024405561}, contine mesaje precum "Pe 1 Decembrie, alege-ne",  directionate catre publicul larg,  si a beneficiat de o investitie financiara intre 200 si 299 RON, atingand intre 15.000 si 20.000 de impresii, cu o estimare a acoperirii de 100.000 - 500.000 de persoane.  Informatiile de contact ale advertiserului sunt topliceanuflorin@yahoo.com si +40767242025.
    \item publicarea unei postari pe Facebook (ID: \href{https://www.facebook.com/ads/library/?id=1242886000311327}{1242886000311327}), dupa ora 18:00 pe 30.11.2024, care contine propaganda electorala negativa indreptata impotriva candidatei Elena-Valeria Lasconi. Postarea, cu un buget de RON 100-199 si o audienta de 9.000-10.000 de impresii, utilizeaza un limbaj emotional puternic si argumente tendentioase pentru a discredita candidata si a-i influenta negativ voturile.  Fraze precum Moartea in Constitutia Romaniei, pruncucidere, si Fara Lasconi la Cotroceni! demonstreaza clar intentia de a influenta votul alegatorilor.  Aceasta postare, avand in vedere continutul sau si amploarea distributiei, constituie o incalcare clara a legislatiei privind propaganda electorala, fiind publicata dupa incheierea perioadei legale de campanie.  Contact: davidmondoc@gmail.com, +40741900294.
    \item publicarea unei postari platite pe Facebook (ID: \href{https://www.facebook.com/ads/library/?id=1245080840049294}{1245080840049294}) dupa ora 18:00 pe 30.11.2024, care, desi nu numeste explicit candidati, indeamna la nevotarea unor candidati care utilizeaza campanii bazate pe inteligenta artificiala ("NU VOTATI O ILUZIE CREATA DE INTELIGENTA ARTIFICIALA").  Postarea, cu un buget de 400-499 RON, a atins 15.000-20.000 de impresii, avand ca efect electoral influentarea negativa a votului prin dezinformare si manipulare.  Prezenta numarului CMF11240017 nu exonereaza de raspundere, intrucat mesajul principal al postarii este unul de propaganda electorala indirecta, incalcand prevederile legale privind perioada de campanie.  Contactul advertiserului este alexandru.rogobete@umft.ro si +40759852479.
    \item publicarea unei postari pe Facebook dupa ora 18:00 pe 30.11.2024, cu ID-ul \href{https://www.facebook.com/ads/library/?id=1252234892751479}{1252234892751479}, care contine mesajul inflamator Arde-i la vot!, avand ca efect electoral influentarea negativa a votului impotriva unor candidati nespecificati, dar vizibil directionat catre electorat,  printr-o campanie platita pe Facebook cu o estimare a audientei de peste 1 milion de persoane,  si cu cheltuieli sub 100 RON, conform datelor din Ad Library.  Contactul advertiserului este miscareapopulara.constanta@gmail.com si +40720888888.  Lipsa unui numar CMF nu exonereaza de raspundere, avand in vedere caracterul evident electoral al mesajului si intentia clara de a influenta votul.
    \item publicarea unei reclame platite pe Facebook (ID postare: \href{https://www.facebook.com/ads/library/?id=1254082252302541}{1254082252302541}) dupa ora 18:00 pe 30.11.2024, care promoveaza candidatura lui Manolache Vasile la Senat, folosind un limbaj persuasiv ce indeamna direct la vot ("cel mai frumos cadou... este increderea dumneavoastra exprimata prin vot", "Pe 1 decembrie, haideti sa punem impreuna bazele unei schimbari adevarate!"), avand un efect electoral clar pozitiv asupra candidaturii sale. Reclama, cu o cheltuiala de sub 100 RON, a atins intre 15.000 si 20.000 de impresii, avand o estimare a acoperirii de 100.000 - 500.000 de persoane, in principal in judetul Bistrita-Nasaud si zonele invecinate.  Aceasta actiune constituie propaganda electorala interzisa dupa incheierea perioadei legale de campanie.  Datele de contact ale Momentum Motion Pictures SRL sunt disponibile in arhiva de reclame Facebook: contact@momentum.com.ro si +40767277777.
    \item publicarea unei postari platite pe Facebook dupa ora 18:00 pe 30.11.2024 (ID postare: \href{https://www.facebook.com/ads/library/?id=1270220704180718}{1270220704180718}), care promoveaza explicit candidata Elena Lasconi de la USR, avand un efect electoral clar pozitiv asupra acesteia. Postarea, desi incepe cu o anecdota personala, are ca scop principal persuadarea electoratului sa voteze pentru Elena Lasconi si USR, printr-un apel direct la vot ("Pe 1 si pe 8 decembrie voi vota pentru... ELENA LASCONI, PRESEDINTE!!!").  Aceasta postare, avand in vedere contextul electoral si cheltuielile implicate (RON 100-RON 199), indeplineste toate criteriile materialului de propaganda electorala conform articolului 36 (7) din LEGEA nr. 334 din 17 iulie 2006, fiind publicata pe o platforma publica cu o raza de cuprindere de peste 1 milion de persoane, dupa incheierea perioadei legale de campanie electorala.  Contact: office@manastirea23august.com, +40724684771, https://www.manastirea23august.com/.
    \item publicarea unei postari platite pe Facebook dupa ora 18:00 pe 30.11.2024 (ID postare: \href{https://www.facebook.com/ads/library/?id=1289211355428130}{1289211355428130}), cu scopul de a-si promova candidatura la functia de deputat. Postarea, desi prezinta o propunere legislativa, este prezentata intr-un mod evident promotional, accentuand imaginea candidatului si platforma sa politica, mai degraba decat oferind o analiza obiectiva a politicii.  Efectul electoral este clar, avand in vedere ca postarea este platita si vizeaza o audienta larga, influentand indirect voturile prin asocierea candidatului cu o initiativa legislativa pozitiva.  Cheltuielile pentru publicitate se situeaza intre 100 si 199 RON, iar postarea a atins intre 5.000 si 6.000 de impresii, cu o estimare a acoperirii de 100.000 - 500.000 de persoane.  Datele de contact ale advertiserului sunt disponibile: contact@sebastiantiberiupop.ro si +40760256257.
    \item publicarea unei postari pe Facebook (ID: \href{https://www.facebook.com/ads/library/?id=1294526935229404}{1294526935229404}) dupa ora 18:00 pe 30.11.2024, care, desi mascata ca o stire, constituie propaganda electorala. Postarea mentioneaza candidatii Calin Georgescu si Lasconi, insinuand actiuni ale PSD menite sa influenteze rezultatul alegerilor. Lipsa obiectivitatii jurnalistice si afirmatiile nefondate, prezentate ca fapte, au scopul clar de a influenta alegatorii.  Plata postarii, vizand o audienta larga, consolideaza caracterul de propaganda electorala.  Afirmatiile precum Planul PSD era... si Vor reface alegerile... sunt menite sa influenteze votul, sugerand manipulare in interiorul PSD.  Postarea a generat intre 10.000 si 15.000 de impresii, cu o estimare a acoperirii de peste 1 milion de persoane, conform datelor din Ad Library.  Contactul advertiserului este fulgeralbastru99@gmail.com si +40747768451, cu adresa in Timisoara, Timis. Suma cheltuita pentru publicitate se situeaza intre 100 si 199 RON.
    \item publicarea unei postari pe Facebook cu ID-ul \href{https://www.facebook.com/ads/library/?id=1300565437767751}{1300565437767751}, dupa ora 18:00 pe 30.11.2024, care promoveaza candidatul Valeriu Iftime si echipa sa, folosind limbaj persuasiv si un apel clar la actiune ("Haideti sa votam \#EchipaValeriuIftime!", "Votam \#EchipaValeriuIftime! Votam schimbarea!"), avand ca efect electoral cresterea numarului de voturi pentru candidatul mentionat. Postarea, care a beneficiat de o cheltuiala de publicitate de sub 100 RON, a generat intre 3000 si 4000 de impresii si a avut o estimare a acoperirii de 100.000 - 500.000 de persoane, conform datelor din Ad Library.  Prezenta numarului CMF 11240002 confirma natura electorala a postarii.  Contactul advertiserului este ionut.voicu.bt@gmail.com si +40759112233.  Postarea nu se incadreaza in jurnalism, ci este o forma clara de propaganda electorala, interzisa dupa incheierea perioadei de campanie.
    \item publicarea unei postari pe Facebook cu ID-ul \href{https://www.facebook.com/ads/library/?id=1710189616492196}{1710189616492196}, dupa ora 18:00 pe 30.11.2024, care indeamna la votul pentru Elena Lasconi, folosind fraze precum "Singurul antidot al fricii este VOTUL. Un vot masiv pentru viitor, pentru democratie, pentru Romania!",  avand un efect electoral clar pozitiv asupra candidatei mentionate. Postarea, care a beneficiat de o cheltuiala de publicitate intre 100 si 199 RON, a atins o audienta estimata intre 35.000 si 40.000 de impresii, conform datelor din Ad Library.  Prezenta numarului CMF 11240015 nu exonereaza de raspundere, intrucat mesajul constituie propaganda electorala dupa incheierea perioadei legale de campanie, conform articolului 36 (7) din LEGE nr. 334 din 17 iulie 2006, si incalca prevederile articolului 98 t) din LEGEA nr. 208 din 20 iulie 2015.  Datele de contact ale advertiserului sunt: teodor.lazar@gmail.com si +40740269432.
    \item publicarea, dupa ora 18:00 pe 30.11.2024, a unei postari pe Facebook (ID \href{https://www.facebook.com/ads/library/?id=1740795043387919}{1740795043387919}) cu continut electoral, care promoveaza candidatura lui Manolache Vasile la Senat, folosind mesaje persuasive si apeluri la vot, avand un efect electoral pozitiv asupra candidaturii sale. Postarea, platita cu o suma sub 100 RON, a atins o audienta estimata intre 100.000 si 500.000 de persoane, conform datelor din Ad Library.  Continutul postarii, inclusiv hashtag-urile \#ManolacheVasileSenator, \#SchimbarePentruOameni, \#VotulConteaza, \#RomaniaMaiBuna, demonstreaza clar intentia de a influenta votul in favoarea candidatului.  Contactul Momentum Motion Pictures SRL este disponibil la contact@momentum.com.ro si +40767277777.
    \item publicarea unei reclame platite pe Facebook (ID postare: \href{https://www.facebook.com/ads/library/?id=1977180832709415}{1977180832709415}), dupa ora 18:00 pe 30.11.2024, care promoveaza votul pentru candidatii independenti, folosind mesaje persuasive precum Razboiul este intre PARTIDE si OAMENI si indemnul la actiune Sa iesim la vot si sa schimbam soarta Romaniei!, avand un efect electoral clar prin incercarea de a influenta alegatorii catre candidatii independenti. Reclama, cu o estimare a bugetului de 900-999 RON si o raza de cuprindere de 100.000-500.000 persoane, nu respecta prevederile legale privind propaganda electorala dupa incheierea perioadei legale de campanie.  Lipsa unui numar CMF nu exonereaza de raspundere, avand in vedere ca mesajul este clar electoral si vizeaza influentarea votului. Contact: contact@virgilpirvulescu.ro, +40758196870.
    \item publicarea unei postari pe Facebook dupa ora 18:00 pe 30.11.2024 (ID postare: \href{https://www.facebook.com/ads/library/?id=2041957979608071}{2041957979608071}), cu scopul de a influenta votul in favoarea Partidului National Liberal,  prin afirmatii precum Trebuie sa votam Partidul National Liberal si Hai sa nu lasam tara sa o ia razna!,  avand un efect electoral clar pozitiv asupra candidatului Mircea Minea si PNL. Postarea, care a beneficiat de o investitie financiara intre 200 si 299 RON, a atins o audienta estimata la peste 1 milion de persoane, conform datelor din Ad Library, si contine un numar CMF (11240002), indicand clar natura sa de material electoral.  Contactul advertiserului este mineachiajna@gmail.com si +40770114557. Aceasta postare constituie propaganda electorala interzisa dupa incheierea campaniei electorale, conform legislatiei in vigoare.
    \item publicarea unei postari pe Facebook (ID: \href{https://www.facebook.com/ads/library/?id=2121783591655239}{2121783591655239}), dupa ora 18:00 pe 30.11.2024, care, desi prezentata ca o analiza a lui Calin Georgescu, contine elemente de propaganda electorala.  Analiza detaliata a trecutului si a asociatiilor sale, prezentata intr-un mod subiectiv si partinitor, urmareste sa influenteze perceptia publica asupra candidatului.  Declaratia finala a autorului, "NU IL VOI VOTA!", desi pare o opinie personala, intareste imaginea negativa creata.  Postarea platita, cu o raza mare de cuprindere pe Facebook, sugereaza intentia de a influenta votul.  Avand in vedere cheltuielile de sub 100 RON, conform Ad Library, si o estimare a audientei de 100.000-500.000 de persoane, impactul potential este semnificativ.  Contactul autorului este cristian.raduca@gmail.com si +40722792301.
    \item publicarea unei reclame platite pe Facebook (ID postare: \href{https://www.facebook.com/ads/library/?id=2505706189625789}{2505706189625789}), dupa ora 18:00 pe 30.11.2024,  care promoveaza candidatura sa la Camera Deputatilor, mentionand explicit pozitia sa pe buletinul de vot (pozitia 43) si utilizand un numar CMF (14240080), avand ca efect electoral cresterea numarului de voturi pentru acesta. Reclama, cu un buget de sub 100 RON, a atins intre 1000 si 2000 de impresii si o estimare a audientei de 100.000 - 500.000 de persoane,  constituie propaganda electorala interzisa dupa incheierea campaniei electorale, conform prevederilor legale.  Fraze precum "Votul vostru pentru mine este un vot pentru dezvoltare" si "Votati pentru Iasi! Votati Ciprian Vasile Stegaru!" demonstreaza clar intentia de a influenta votul.  Informatiile de contact ale advertiserului sunt secretariat@deputatstegaru.com si +40755545973.
    \item publicarea unei reclame platite pe Facebook (ID postare: \href{https://www.facebook.com/ads/library/?id=2558185481042152}{2558185481042152}), dupa ora 18:00 pe 30.11.2024, cu imagini care o asociaza direct cu numarul 4 de pe buletinul de vot si un link catre site-ul sau, avand ca efect electoral influentarea voturilor in favoarea sa in judetul Arges. Reclama, desi cu text minimal, utilizeaza imagini si context pentru a indeplini toate criteriile propagandei electorale, vizand populatia din Arges, cu o cheltuiala de 500-599 RON si o estimare a impactului de 70.000-80.000 impresii.  Lipsa unui numar CMF nu exclude incriminarea, avand in vedere contextul si scopul evident electoral al postarii.
    \item publicarea unei reclame platite pe Facebook dupa ora 18:00 pe 30.11.2024 (ID postare: \href{https://www.facebook.com/ads/library/?id=2633125790192123}{2633125790192123}), care promoveaza candidatul independent Dorin Curtean pentru alegerile parlamentare, indemnand direct la votul pentru acesta ("Votati Dorin Curtean, pozitia 40 pe buletinul de vot!").  Reclama, cu o cheltuiala estimata intre 400 si 499 RON, a avut o acoperire de 30.000-35.000 de impresii si o estimare a audientei de 100.000-500.000 de persoane, vizand in principal judetul Bihor.  Efectul electoral al acestei reclame este clar, avand ca obiectiv influentarea votului in favoarea lui Dorin Curtean, constituind astfel propaganda electorala interzisa dupa incheierea campaniei electorale.  Datele de contact ale advertiserului sunt: hello@internationalstudio.ro si +40758908016.
    \item publicarea unei postari pe Facebook (ID: \href{https://www.facebook.com/ads/library/?id=2822429157957880}{2822429157957880}), dupa ora 18:00 pe 30.11.2024, cu continut electoral negativ, care vizeaza denigrarea unei figuri politice identificate ca BeneA13Ghinion, prin utilizarea unui limbaj inflamator si apeluri directe la respingerea acestuia. Postarea, platita intre 300 si 399 RON, a avut o acoperire estimata de 50.000-100.000 de persoane in judetul Bacau, avand ca efect electoral influentarea negativa a voturilor pentru candidatul vizat.  Fraze precum cel mai toxic sef de organizatie, viol in grup asupra moralei, caracatita perversa si apelul direct Afara cu BeneA13 Ghinion demonstreaza clar intentia de a influenta votul.  Datele de contact ale lui Lucian Sova sunt disponibile in arhiva de reclame Facebook: luciansova@yahoo.com si +40723168862.
    \item publicarea unei postari pe Facebook dupa ora 18:00 pe 30.11.2024, cu ID-ul \href{https://www.facebook.com/ads/library/?id=2984411375039498}{2984411375039498}, care promoveaza candidata Elena Lasconi prin hashtag-ul "\#ElenaLasconi" si indeamna la un "vot masiv" pentru aceasta, avand un efect electoral clar pozitiv asupra acesteia. Postarea, care a costat intre 300 si 399 RON, a atins intre 25.000 si 30.000 de impresii, cu o estimare a audientei de 100.000 - 500.000 de persoane, si contine numarul CMF 11240015, indicand clar natura sa de propaganda electorala.  Fraza "Singurul antidot al fricii este VOTUL" este o chemare directa la vot, iar mesajul "Un vot masiv pentru viitor, pentru democratie, pentru Romania!" este o indemnare explicita la vot pentru candidata mentionata.  Aceste elemente, coroborate cu data postarii, demonstreaza clar incalcarea legislatiei electorale.
    \item publicarea unei postari pe Facebook (ID: \href{https://www.facebook.com/ads/library/?id=420683510976832}{420683510976832}) dupa ora 18:00 pe 30.11.2024, care promoveaza candidatul Valeriu Iftime prin indemnul explicit Pe 1 decembrie, votam \#EchipaValeriuIftime! Votam schimbarea!, avand un efect electoral clar de influentare a votului in favoarea acestuia. Postarea, fiind o reclama platita pe Facebook, vizeaza o audienta larga, conform datelor din Ad Library, cu o cheltuiala de sub 100 RON si o estimare a audientei de 100.000-500.000 de persoane.  Lipsa unui numar CMF nu exonereaza de raspundere, avand in vedere caracterul evident de propaganda electorala al mesajului.  Contactul advertiserului este ionut.voicu.bt@gmail.com si +40759112233.
    \item publicarea unei postari platite pe Facebook (ID \href{https://www.facebook.com/ads/library/?id=453236837821057}{453236837821057}), dupa ora 18:00 pe 30.11.2024, cu scopul de a influenta votul alegatorilor in favoarea sa. Postarea, desi nu solicita explicit votul pentru un anumit candidat, utilizeaza diverse tehnici de promovare a imaginii lui Pop Sebastian Tiberiu si a viziunii sale politice, inclusiv hashtag-uri precum \#PopSebastianTiberiu si \#Alegeri2024, precum si un apel la actiune ("Da share daca si TU crezi ca Romania are nevoie de o altfel de politica!").  Avand in vedere cheltuielile de publicitate intre 100 si 199 RON,  ajungand la peste 1 milion de oameni, si contextul electoral, se poate concluziona ca obiectivul postarii este influentarea votului in favoarea lui Pop Sebastian Tiberiu.  Aceasta actiune constituie propaganda electorala dupa incheierea perioadei legale de campanie, conform articolului 98 t).
    \item publicarea unei reclame platite pe Facebook (ID postare: \href{https://www.facebook.com/ads/library/?id=490813700029217}{490813700029217}) dupa ora 18:00 pe 30.11.2024, care promoveaza candidatura lui Manolache Vasile la Senat,  prin indemnuri explicite la participarea la alegerile parlamentare si utilizarea de hashtag-uri promotionale (\#ManolacheVasileSenator, \#SchimbarePentruRomania, \#AlegeriParlamentare, \#ImplicareActiva), avand un efect electoral pozitiv asupra candidaturii sale.  Reclama, cu o estimare a acoperirii de 100.000-500.000 de persoane si un buget sub 100 RON,  indeplineste toate criteriile propagandei electorale conform articolului 36 (7) din LEGEA nr. 334 din 17 iulie 2006,  in special punctele a) si c),  fiind publicata dupa incheierea perioadei legale de campanie. Absenta unui numar CMF amplifica suspiciunea de incalcare a legislatiei.  Persoanele responsabile pot fi contactate la contact@momentum.com.ro sau +40767277777.
    \item publicarea unei postari pe Facebook dupa ora 18:00 pe 30.11.2024, cu ID-ul \href{https://www.facebook.com/ads/library/?id=560465680034710}{560465680034710}, care promoveaza candidata Elena Lasconi prin indemnul explicit la vot masiv,  "Un vot masiv pentru viitor, pentru democratie, pentru Romania!",  avand un efect electoral clar pozitiv asupra acesteia. Postarea, care a beneficiat de o cheltuiala de sub 100 RON, a atins intre 15.000 si 20.000 de impresii, conform datelor din Ad Library, si a vizat utilizatorii din judetul Bacau, cu varste intre 18 si 65 de ani. Prezenta numarului CMF 11240015 confirma natura electorala a postarii, demonstrand intentia de a influenta votul in favoarea candidatei mentionate.  Contactul Teodor Lazar poate fi stabilit prin email la teodor.lazar@gmail.com sau telefonic la +40740269432.
    \item publicarea unei postari platite pe Facebook (ID: \href{https://www.facebook.com/ads/library/?id=567876142503416}{567876142503416}), dupa ora 18:00 pe 30.11.2024, care promoveaza o initiativa legislativa despre evaluari psihologice obligatorii pentru candidati, avand ca efect electoral cresterea imaginii si a credibilitatii candidatului Sebastian Tiberiu Pop, influentand astfel votul alegatorilor. Postarea, desi pare neutra la prima vedere, indeplineste toate criteriile propagandei electorale conform articolului 36 (7) din LEGE nr. 334 din 17 iulie 2006, referindu-se direct la candidatura ("viitor deputat"), adresandu-se publicului larg si avand un obiectiv electoral clar.  Cheltuielile pentru aceasta postare, conform Ad Library, sunt sub 100 RON, ajungand la 1000-2000 de impresii si o estimare a acoperirii de 500.000-1.000.000 de persoane.  Informatiile de contact ale advertiserului sunt disponibile: contact@sebastiantiberiupop.ro si +40760256257.
    \item publicarea unei postari pe Facebook dupa ora 18:00 pe 30.11.2024, cu ID-ul \href{https://www.facebook.com/ads/library/?id=581415000943629}{581415000943629}, care promoveaza Partidul National Liberal (PNL) printr-un mesaj explicit de sustinere ("Eu aleg sa sustin echipa PNL si stabilitatea!") si indeamna la vot pentru acest partid, avand un efect electoral clar pozitiv pentru PNL. Postarea, platita cu o suma sub 100 RON, a atins o estimare de 6.000-7.000 de impresii, conform datelor din Ad Library, si a vizat utilizatori din judetul Constanta, cu varste cuprinse intre 18 si 65 de ani.  Aceasta postare constituie propaganda electorala interzisa dupa incheierea campaniei electorale, conform legislatiei in vigoare, si este o incalcare clara a articolului 98 t) din LEGEA nr. 208 din 20 iulie 2015.  Datele de contact ale lui Rafael Nichita sunt rafaelnichita77@gmail.com si +40722590573.
    \item publicarea unei postari platite pe Facebook (ID: \href{https://www.facebook.com/ads/library/?id=606719112020297}{606719112020297}) dupa ora 18:00 pe 30.11.2024, care, desi nu indeamna explicit la votarea sa, constituie propaganda electorala prin apelul personal la alegatori,  accentuand increderea acordata de peste 47.000 de mureseni si indemnand la participarea la vot cu fraza "iesi maine la vot!", avand un efect electoral pozitiv asupra campaniei sale. Postarea, cu o cheltuiala de sub 100 RON, a atins 2.000-3.000 de impresii si o estimare a acoperirii de 500.000-1.000.000 de persoane, conform datelor din Ad Library.  Contactul responsabil este mara\_toganel@yahoo.com, +40722886771.
    \item publicarea pe Facebook, dupa ora 18:00 pe 30.11.2024, a unei postari platite (ID \href{https://www.facebook.com/ads/library/?id=615078284376981}{615078284376981}) cu mesajul "Votezi USR sau votezi PNL = Votezi PSD!",  care constituie propaganda electorala interzisa dupa incheierea campaniei.  Mesajul vizeaza direct alegerile parlamentare, influentand negativ votul pentru USR si PNL si pozitiv pentru PSD, printr-o afirmatie nefondata si tendentioasa.  Postarea, avand o raza de cuprindere de peste 1 milion de persoane,  a fost directionata catre judetul Constanta, conform datelor din Ad Library, cu un buget sub 100 RON,  si utilizeaza datele de contact miscareapopulara.constanta@gmail.com si +40720888888.  Efectul electoral este clar, avand ca scop manipularea alegatorilor prin asocierea falsa a USR si PNL cu PSD.
    \item publicarea unei postari pe Facebook (ID \href{https://www.facebook.com/ads/library/?id=885034927154689}{885034927154689}) dupa ora 18:00 pe 30.11.2024, care, desi nu indeamna explicit la votul pentru sau impotriva unui candidat, face referire la candidata Elena-Valeria Lasconi si la Partidul National Liberal (PNL), creand o comparatie implicita si o critica subtila a strategiei de campanie a PNL in Brasov.  Concluzia postarii, "Halal schimbare! Valori europene!", amplifica potentialul de influentare a opiniei publice.  Postarea, platita cu o suma sub 100 RON, a atins intre 1000 si 2000 de impresii, avand un potential de impact electoral semnificativ, dat fiind contextul electoral si referirea directa la un candidat.  Contactul advertiserului este dragos.david5@gmail.com si +40730990611.  Aceasta postare, desi nu contine indemnuri directe, se incadreaza in definitia propagandei electorale prin influentarea indirecta a votului, incalcand astfel prevederile legale.
    \item publicarea unei postari platite pe Facebook (ID \href{https://www.facebook.com/ads/library/?id=889601543286900}{889601543286900}), dupa ora 18:00 pe 30.11.2024, care constituie propaganda electorala, avand ca efect influentarea votului impotriva candidatei Elena Lasconi si in favoarea candidatilor Mihail Neamtu si Samuel Zarnescu. Postarea, cu o estimare a audientei de peste 1 milion de persoane, contine afirmatii negative si tendentioase la adresa Elenei Lasconi, prezentand-o ca o amenintare la adresa valorilor crestine si asociind-o cu secularismul si neo-marxismul.  Aceasta postare, finantata cu o suma sub 100 RON,  incalca prevederile legale prin continuarea propagandei electorale dupa incheierea perioadei legale si prin influentarea directa a votului,  folosind un limbaj emotional si manipulativ,  in detrimentul Elenei Lasconi si in beneficiul candidatilor Mihail Neamtu si Samuel Zarnescu.  Datele de contact ale lui Peter Costea sunt disponibile in Ad Library: davidmondoc@gmail.com si +40741900294.
    \item publicarea unei reclame platite pe Facebook dupa ora 18:00 pe 30.11.2024, cu ID-ul \href{https://www.facebook.com/ads/library/?id=8951227288254053}{8951227288254053}, care, desi nu mentioneaza explicit numele lui Manolache Vasile, este puternic asociata cu acesta prin numele paginii Manolache Vasile - Senator si hashtag-urile utilizate (\#ManolacheVasileSenator, \#VotulEsteDatoriaNoastra, \#SchimbarePentruRomania, \#ImplicareActiva).  Mesajul, care indeamna la participarea la vot, are un efect electoral pozitiv indirect asupra candidaturii lui Manolache Vasile, intrucat creeaza o asociere favorabila si incurajeaza participarea la vot, beneficiind astfel indirect candidatul.  Reclama, cu o cheltuiala de sub 100 RON, a atins o audienta estimata intre 100.000 si 500.000 de persoane, conform datelor din Ad Library.  Contactul Momentum Motion Pictures SRL este contact@momentum.com.ro si +40767277777.  Desi nu este o propaganda electorala directa, contextul reclamei platite pe pagina candidatului si limbajul persuasiv utilizat demonstreaza intentia de a influenta comportamentul alegatorilor, incalcand astfel prevederile legale.
    \item publicarea unei postari platite pe Facebook (ID \href{https://www.facebook.com/ads/library/?id=913307383785120}{913307383785120}) dupa ora 18:00 pe 30.11.2024, cu mesajul  NU VOTATI O ILUZIE CREATA DE INTELIGENTA ARTIFICIALA!,  care, desi nu numeste explicit un candidat, are ca efect electoral influentarea negativa a votului impotriva candidatilor care utilizeaza continut generat de inteligenta artificiala in campaniile lor.  Prezenta numarului CMF11240017 indica clar o legatura cu propaganda electorala.  Postarea, avand o cheltuiala de sub 100 RON si o estimare a audientei de 500.000 - 1.000.000 de persoane,  se adreseaza publicului larg si are ca obiectiv clar influentarea votului, incalcand astfel prevederile legale privind continuarea propagandei electorale dupa incheierea perioadei legale.  Contactul advertiserului este alexandru.rogobete@umft.ro si +40759852479.
    \item publicarea unei postari pe Facebook (ID: \href{https://www.facebook.com/ads/library/?id=928392119243935}{928392119243935}) dupa ora 18:00 pe 30.11.2024, care promoveaza candidatul Calin Georgescu si denigreaza implicit un candidat USR, avand un efect electoral clar prin indemnul la vot pentru un anumit candidat. Postarea, platita cu o suma cuprinsa intre 1000 si 1500 RON, conform Ad Library, a atins intre 60.000 si 70.000 de impresii, avand o estimare a acoperirii de 100.000-500.000 de persoane.  Fraza Voi vota Calin Georgescu, nicidecum un candidat USR!, urmata de justificarea alegerii, constituie o propaganda electorala directa, interzisa dupa incheierea perioadei de campanie.  Contactul poate fi realizat prin email la contact@mirceacirt.ro sau telefonic la +40721288000.
    \item publicarea dupa ora 18:00 pe 30.11.2024 a unei postari pe Facebook (ID \href{https://www.facebook.com/ads/library/?id=946591617530257}{946591617530257}) cu caracter de propaganda electorala, care promoveaza candidatura lui Manolache Vasile la Senat, indemnand la vot in favoarea sa prin sublinierea necesitatii imbunatatirii sistemului sanitar.  Postarea, cu un buget de sub 100 RON, a generat intre 8.000 si 9.000 de impresii si o estimare a acoperirii de 100.000 - 500.000 de persoane, avand un efect electoral clar prin influentarea votului in favoarea candidatului.  Mesajul "Pe 1 decembrie, hai sa votam pentru lideri care sa puna sanatatea pe primul loc" si hashtag-urile "\#ManolacheVasileSenator \#SanatatePentruRomania \#SpitaleModerne \#ViitorMaiBun" demonstreaza intentia de a influenta votul.  Contact: contact@momentum.com.ro, +40767277777.
\end{enumerate}

\vspace{0.5cm}

\subsection{Info360}
Următoarele fapte contravenționale sunt sesizate împotriva acestei entități:

\begin{enumerate}[leftmargin=*, label=\arabic*.)]
    \item publicarea unei postari platite pe Facebook (ID postare: \href{https://www.facebook.com/ads/library/?id=362792233522175}{362792233522175}), dupa ora 18:00 pe 30.11.2024, cu continut care prezinta intr-un mod tendentios si manipulativ o declaratie a unui membru AUR, influentand negativ perceptia publicului asupra partidului.  Postarea, cu o cheltuiala de 200-299 RON, a atins 40.000-45.000 de impresii si o estimare a audientei de 50.000-100.000 de persoane, avand ca obiectiv influentarea votului prin crearea unei imagini negative a AUR. Titlul sugestiv Manelistul Dani Mocanu nu voteaza AUR si lipsa neutralitatii jurnalistice demonstreaza intentia de a influenta alegerile.  Informatiile de contact ale advertiserului sunt: info3983@info360.online si +40758403724, adresa: Bucuresti, Bucuresti, si website: https://www.facebook.com/info360romania/.
\end{enumerate}

\vspace{0.5cm}

\subsection{Informatia ta}
Următoarele fapte contravenționale sunt sesizate împotriva acestei entități:

\begin{enumerate}[leftmargin=*, label=\arabic*.)]
    \item publicarea unei postari pe Facebook (ID \href{https://www.facebook.com/ads/library/?id=924717059235439}{924717059235439}), dupa ora 18:00 pe 30.11.2024, cu un buget de sub 100 RON, care, desi prezentata sub forma unei stiri, promoveaza indirect PNL prin asocierea unui proiect pozitiv (extinderea autostrazii) cu partidul. Postarea, avand o acoperire larga pe Facebook si fiind platita, are ca scop influentarea electoratului, chiar daca nu solicita explicit votul. Frazarea manipulatoare, care subliniaza rolul PNL si omiterea altor contributii, demonstreaza intentia de a influenta votul in favoarea PNL si a deputatului Ioan Balan.  Numarul mare de impresii (7000-8000) si potentiala acoperire de 100.000-500.000 de persoane, conform datelor din Ad Library, amplifica efectul electoral.  Contactul poate fi realizat la anunturi@informatiata.ro sau +40745691121.
\end{enumerate}

\vspace{0.5cm}

\subsection{Kovacs Mihály Levente}
Următoarele fapte contravenționale sunt sesizate împotriva acestei entități:

\begin{enumerate}[leftmargin=*, label=\arabic*.)]
    \item publicarea unei postari platite pe Facebook dupa ora 18:00 pe 30.11.2024 (ID postare: \href{https://www.facebook.com/ads/library/?id=518887677800234}{518887677800234}), cu mesajul "Ha Magyari Albert 96 evesen is elmegy szavazni, akkor vasarnap neked is a szavazofulkeben a helyed!", care, desi nu mentioneaza explicit un candidat, are un efect electoral indirect prin incurajarea participarii la vot, apropiindu-se de ziua alegerilor.  Mesajul, avand in vedere contextul si plata pentru promovare, vizeaza o audienta larga si poate fi interpretat ca o tentativa de influentare a votului, chiar daca nu nominalizeaza un candidat specific.  Chemarea la vot, in contextul apropiat de ziua alegerilor, constituie o forma de propaganda electorala, incalcand prevederile legale.  Postarea a generat intre 10.000 si 15.000 de impresii, conform datelor din Ad Library, cu o cheltuiala de sub 100 RON.  Datele de contact ale advertiserului sunt krisztakiss8@gmail.com si +40771707322.
\end{enumerate}

\vspace{0.5cm}

\subsection{MOMENTUM}
Următoarele fapte contravenționale sunt sesizate împotriva acestei entități:

\begin{enumerate}[leftmargin=*, label=\arabic*.)]
    \item publicarea, dupa ora 18:00 pe 30.11.2024, a unei postari pe Facebook (ID: \href{https://www.facebook.com/ads/library/?id=1075210477421694}{1075210477421694}) cu scop electoral, care promoveaza candidatul la Senat Manolache Vasile, folosind mesaje pozitive si un apel la actiune ("Send message"), avand un efect electoral clar de influentare a votului in favoarea acestuia. Postarea, cu o cheltuiala de sub 100 RON, a atins intre 1000 si 2000 de impresii si o estimare a audientei de 100.000-500.000 de persoane, conform datelor din Ad Library.  Continutul postarii, inclusiv hashtag-urile "\#ManolacheVasileSenator \#SchimbarePentruOameni \#VotulConteaza \#RomaniaMaiBuna", demonstreaza clar intentia de a influenta votul, incalcand prevederile legale privind continuarea propagandei electorale dupa incheierea perioadei legale.  Informatiile de contact ale Momentum Motion Pictures SRL sunt disponibile in Ad Library: contact@momentum.com.ro si +40767277777.
    \item publicarea, dupa ora 18:00 pe 30.11.2024, a unei reclame platite pe Facebook (ID postare: \href{https://www.facebook.com/ads/library/?id=1226606971778404}{1226606971778404}) care promoveaza candidatura lui Manolache Vasile la alegerile pentru Senat in judetul Bistrita-Nasaud, folosind un mesaj persuasiv ("Hai sa votam pentru un viitor diferit", "Ignoranta nu rezolva problemele  implicarea si votul da!") cu scopul clar de a influenta votul alegatorilor.  Reclama, cu o estimare a audientei de 100.000-500.000 de persoane si un buget sub 100 RON, indeplineste toate criteriile propagandei electorale conform articolului 36 (7) din LEGE nr. 334 din 17 iulie 2006, fiind publicata dupa incheierea perioadei legale de campanie electorala.  Contact: contact@momentum.com.ro, +40767277777, adresa: Bucuresti, Romania.
    \item publicarea unei postari platite pe Facebook (ID postare: \href{https://www.facebook.com/ads/library/?id=1746697239416114}{1746697239416114}), dupa ora 18:00 pe 30.11.2024, care promoveaza implicit candidatura lui Manolache Vasile la Senat, folosind limbaj emotional si asociind valori pozitive cu acesta.  Postarea, cu o cheltuiala sub RON 100, a avut o acoperire estimata de 100K-500K persoane si 10K-15K impresii, avand ca efect electoral influentarea votului in favoarea candidatului.  Prezenta hashtag-urilor "\#ManolacheVasileSenator" si "\#RespectPentruOameni" precum si denumirea paginii "Manolache Vasile - Senator" consolideaza legatura directa intre postare si campania electorala a candidatului.  Continutul postarii, care indeamna la vot pentru un viitor bazat pe "familie, respect si constiinta", constituie o forma de propaganda electorala, nefiind o simpla exprimare a unei opinii personale.  Datele de contact ale Momentum Motion Pictures SRL sunt disponibile in Ad Library: contact@momentum.com.ro si +40767277777.
    \item publicarea unei reclame platite pe Facebook dupa ora 18:00 pe 30.11.2024 (ID postare: \href{https://www.facebook.com/ads/library/?id=1770015770414903}{1770015770414903}), care, desi nu il mentioneaza explicit pe Manolache Vasile, il promoveaza indirect prin utilizarea numelui sau in hashtag-uri (\#ManolacheVasileSenator) si prin indemnul insistent la participarea la vot, avand ca efect electoral cresterea numarului de voturi pentru acesta.  Reclama, cu un buget de sub 100 RON, a vizat in principal judetul Bistrita-Nasaud, atingand o estimare de 100.000-500.000 de impresii.  Textul reclamei ("Votul romanilor...Duminica, 1 decembrie, mergeti, oameni buni, la vot!")  si contextul electoral clar indica intentia de a influenta votul in favoarea candidatului.  Contact: contact@momentum.com.ro, +40767277777.
    \item publicarea unei reclame platite pe Facebook dupa ora 18:00 pe 30.11.2024 (ID postare: \href{https://www.facebook.com/ads/library/?id=3386519718323573}{3386519718323573}), care promoveaza participarea la alegeri si este strans asociata cu campania senatoriala a lui Manolache Vasile, avand un efect electoral pozitiv asupra acestuia.  Postarea, desi nu il sustine explicit, indeplineste criteriile articolului 36 (7) din LEGE nr. 334 din 17 iulie 2006, referindu-se la un candidat, fiind utilizata in perioada electorala, avand un obiectiv electoral si depasind limitele activitatii jurnalistice.  Mesajul "Mergeti la vot!" si asocierea cu pagina lui Manolache Vasile constituie o tentativa clara de influentare a votului.  Reclama a avut o estimare a impactului de 100K-500K persoane, cu cheltuieli sub 100 RON, conform datelor din Ad Library.  Datele de contact ale Momentum Motion Pictures SRL sunt disponibile in Ad Library: contact@momentum.com.ro si +40767277777.
    \item publicarea unei postari pe Facebook (ID \href{https://www.facebook.com/ads/library/?id=467138062648657}{467138062648657}) dupa ora 18:00 pe 30.11.2024, care promoveaza candidatura lui Manolache Vasile la Senat in judetul Bistrita-Nasaud, folosind limbaj emotional si apeluri directe la vot. Postarea, cu o estimare a impactului de 100.000-500.000 de persoane,  si un buget sub 100 RON, indeplineste toate criteriile propagandei electorale conform articolului 36 (7) din LEGEA nr. 334 din 17 iulie 2006, avand ca obiectiv influentarea votului in favoarea candidatului.  Fraze precum cel mai frumos cadou, viitor mai bun, jug, si schimbare adevarata sunt utilizate pentru a manipula emotiile electoratului.  Contact: contact@momentum.com.ro, +40767277777, Bucharest, Romania.
    \item publicarea unei reclame platite pe Facebook (ID postare: \href{https://www.facebook.com/ads/library/?id=541496618789067}{541496618789067}), dupa ora 18:00 pe 30.11.2024, care promoveaza participarea la alegeri si este asociata cu candidatul Manolache Vasile, senator, avand un efect electoral pozitiv prin indemnul la vot.  Reclama, desi nu solicita explicit votul pentru Manolache Vasile, utilizeaza un limbaj persuasiv ("Romania are nevoie de oameni care cred in schimbare"),  hashtag-uri legate de campania sa (\#ManolacheVasileSenator, \#SchimbarePentruRomania) si un apel direct la actiune ("vino la urne"),  indeplinind astfel criteriile materialului de propaganda electorala conform art. 36 (7) din Legea nr. 334/2006.  Cheltuielile pentru reclama au fost sub 100 RON, conform datelor din Ad Library,  ajungand la o audienta estimata intre 100 si 1000 de persoane.  Contact: contact@momentum.com.ro, +40767277777.
    \item publicarea unei reclame platite pe Facebook dupa ora 18:00 pe 30.11.2024 (ID postare: \href{https://www.facebook.com/ads/library/?id=562051423431759}{562051423431759}), care promoveaza candidatura lui Manolache Vasile la Senat, folosind mesaje persuasive precum momentul sa demonstram ca schimbarea este posibila si votam pentru o Romania mai buna, cu scopul clar de a influenta votul in favoarea acestuia.  Reclama, cu o cheltuiala de sub 100 RON, a vizat o audienta de 1000-2000 de impresii, cu o estimare a acoperirii de 100.000-500.000 de persoane, in principal in judetul Bistrita-Nasaud, conform datelor din Facebook Ad Library.  Contact: contact@momentum.com.ro, +40767277777.  Aceasta actiune constituie propaganda electorala ilegala dupa incheierea perioadei legale de campanie.
    \item publicarea, dupa ora 18:00 pe 30.11.2024, a unei postari pe Facebook (ID \href{https://www.facebook.com/ads/library/?id=568792879079895}{568792879079895}) cu caracter de propaganda electorala, care promoveaza candidatul Manolache Vasile la Senat, folosind mesaje despre valori traditionale si familie pentru a influenta votul alegatorilor. Postarea, platita cu o suma sub 100 RON, a atins o audienta estimata intre 100.000 si 500.000 de persoane, conform datelor din Ad Library.  Efectul electoral al postarii este clar, avand ca obiectiv influentarea pozitiva a opiniei publice fata de candidat si incurajarea votului pentru acesta.  Fraze precum Pe 1 decembrie, hai sa votam pentru un viitor construit pe valori reale  familie, respect si constiinta demonstreaza intentia clara de a influenta votul.  Contactul Momentum Motion Pictures SRL este disponibil la contact@momentum.com.ro si +40767277777.
    \item publicarea unei postari pe Facebook (ID: \href{https://www.facebook.com/ads/library/?id=881142170865396}{881142170865396}), dupa ora 18:00 pe 30.11.2024, care promoveaza candidatura lui Manolache Vasile la Senat in judetul Bistrita-Nasaud. Postarea, cu un buget de sub 100 RON, a atins intre 1000 si 2000 de impresii, utilizand o alegorie si un apel direct la vot pentru candidatul mentionat, avand ca efect electoral influentarea voturilor in favoarea acestuia.  Continutul persuasiv, lipsa caracterului jurnalistic si directionarea catre un electorat specific, demonstreaza intentia clara de a influenta alegerile.  Informatiile de contact ale Momentum Motion Pictures SRL sunt disponibile in Ad Library: contact@momentum.com.ro si +40767277777.
\end{enumerate}

\vspace{0.5cm}

\subsection{Maszol.ro}
Următoarele fapte contravenționale sunt sesizate împotriva acestei entități:

\begin{enumerate}[leftmargin=*, label=\arabic*.)]
    \item publicarea unei postari platite pe Facebook, dupa ora 18:00 pe 30.11.2024, cu ID-ul \href{https://www.facebook.com/ads/library/?id=1302675587576771}{1302675587576771}, care, desi se prezinta ca o stire, face referire la candidatul Kelemen Hunor intr-un context negativ, influentand astfel perceptia alegatorilor.  Titlul stirii, Kelemen Hunor despre renumararea voturilor: ulei pe foc, are un efect electoral negativ, avand potentialul de a-i determina pe unii alegatori sa nu voteze pentru acesta.  Postarea, cu o cheltuiala de 100-199 RON, a atins 15.000-20.000 de impresii, avand o estimare a acoperirii de 500.000-1.000.000 de persoane.  Desi nu exista un indemn explicit la vot, contextul apropiat de alegeri si prezentarea subiectiva a informatiei constituie o incalcare a legii.  Datele de contact ale Maszol.ro sunt disponibile pe pagina lor de Facebook si in informatiile furnizate de Facebook Ad Library.
    \item publicarea unei reclame platite pe Facebook (ID postare: \href{https://www.facebook.com/ads/library/?id=1367534737561445}{1367534737561445}), dupa ora 18:00 pe 30.11.2024, care, desi nu sustine explicit un anumit candidat, exercita o influenta subtila asupra alegatorilor prin indemnul la o prezenta masiva la urne, folosind un limbaj alarmist ("A parlamenti valasztasokon nagy reszvetelre kell szamitani es, ha nem megyunk el elegen szavazni kieshetunk a parlamentbol. Akkor pedig szabad predajava valhatunk a sotet eroknek.").  Efectul electoral al acestei postari este de a mobiliza alegatorii, fara a specifica insa pentru cine.  Reclama, cu o cheltuiala de 100-199 RON, a atins 15.000-20.000 de impresii, avand o estimare a acoperirii de 500.000-1.000.000 de persoane.  Datele de contact ale Maszol.ro sunt disponibile in Ad Library: kiado@maszol.ro si +407400001818.
    \item publicarea unei postari platite pe Facebook (ID: \href{https://www.facebook.com/ads/library/?id=596242489521990}{596242489521990}), dupa ora 18:00 pe 30.11.2024, cu continut ce, desi prezentat sub forma unui interviu jurnalistic cu Kelemen Hunor, influenteaza indirect votul alegatorilor. Discutia despre sansele lui C\textbackslash{}u0103lin Georgescu si Elena Lasconi, precum si impactul alegerilor asupra comunitatii maghiare din Transilvania, constituie propaganda electorala, avand ca efect electoral mobilizarea unui anumit segment de alegatori. Titlul sugestiv, "Politikai foldindulas: velunk vagy nelkulunk?", amplifica acest efect.  Postarea, avand in vedere cheltuielile intre 100 si 199 RON si o estimare a audientei de 500.000 - 1.000.000 de persoane, a avut un impact semnificativ, incalcand prevederile legale privind interzicerea propagandei electorale dupa incheierea perioadei legale.  Contact: kiado@maszol.ro, +40740001818, https://www.maszol.ro/.
\end{enumerate}

\vspace{0.5cm}

\subsection{Moldova Invest}
Următoarele fapte contravenționale sunt sesizate împotriva acestei entități:

\begin{enumerate}[leftmargin=*, label=\arabic*.)]
    \item publicarea, dupa ora 18:00 pe 30.11.2024, a unei postari pe Facebook (ID \href{https://www.facebook.com/ads/library/?id=1085063749553354}{1085063749553354}) cu imaginea si titlul primarului Ion Ticusor Vasiliu, din comuna Petricani, Neamt, intr-un context care, desi nu face apel direct la vot, creeaza o asociere pozitiva a acestuia cu o imagine nationala, avand un efect electoral indirect.  Postarea, platita cu o suma sub 100 RON, a atins o audienta estimata intre 100.000 si 500.000 de persoane, conform datelor din Ad Library.  Utilizarea imaginii si titlului primarului, intr-o postare publicitara platita,  constituie o forma de propaganda electorala indirecta, influentand perceptia publicului asupra sa, in beneficiul sau, in contextul apropiat de perioadele electorale.  Contact: contact@moldovainvest.newsmoldova.ro, +40772021599, str Universitatii nr 28 A, suceava, suceava 720225, RO.
    \item publicarea unei reclame platite pe Facebook, cu ID-ul \href{https://www.facebook.com/ads/library/?id=1257084132223874}{1257084132223874}, dupa ora 18:00 pe 30.11.2024, care promoveaza un articol ce analizeaza averea si interesele candidatului Ciprian Serban de la PSD pentru Camera Deputatilor, avand un efect electoral clar prin influentarea opiniei publice inainte de alegeri.  Reclama, desi prezentata ca o analiza, nu respecta standardele jurnalistice de obiectivitate si neutralitate, fiind lipsita de perspective multiple si avand un scop evident de a influenta votul.  Avand in vedere suma cheltuita (sub 100 RON), numarul de impresii (sub 1000) si aria geografica tinta (judetul Neamt), se poate deduce o intentie directa de a influenta votul in favoarea candidatului mentionat.  Datele de contact ale Moldova Invest sunt disponibile in arhiva reclamelor Facebook: contact@moldovainvest.newsmoldova.ro si +40772021599.
    \item publicarea, dupa ora 18:00 pe 30.11.2024, a unei postari platite pe Facebook (ID \href{https://www.facebook.com/ads/library/?id=447999341393846}{447999341393846}) care, desi aparent un mesaj de felicitare pentru Ziua Nationala, contine imaginea si titlul primarului Stelian Turcu, creand o asociere pozitiva indirecta cu acesta.  Efectul electoral este potential pozitiv pentru Turcu, intrucat mesajul, difuzat catre o audienta tinta din judetul Iasi (conform datelor din Ad Library),  vizeaza consolidarea imaginii sale publice.  Cheltuielile pentru aceasta postare au fost sub 100 RON, ajungand la o audienta estimata intre 100.000 si 500.000 de persoane.  Contactul Moldova Invest este contact@moldovainvest.newsmoldova.ro si +40772021599.
    \item publicarea unei postari platite pe Facebook (ID \href{https://www.facebook.com/ads/library/?id=458352563600468}{458352563600468}) dupa ora 18:00 pe 30.11.2024, care, desi prezentata sub forma unui articol de stiri, contine informatii despre finantarea campaniei lui Alexandru Muraru, presedintele PNL Iasi, si a deputatului PNL Florin Alexe, cu scopul de a influenta negativ perceptia publicului asupra acestora si a PNL.  Postarea, cu un buget de sub 100 RON, a atins intre 100.000 si 500.000 de persoane, utilizand un ton negativ si concentrandu-se pe imprumuturi de la persoane cu legaturi politice, subminand astfel increderea publicului in transparenta financiara a PNL.  Aceasta actiune constituie propaganda electorala, avand in vedere ca influenteaza indirect votul, chiar daca nu indeamna explicit la votarea sau ne-votarea unui candidat.  Contact: contact@moldovainvest.newsmoldova.ro, +40772021599, https://moldovainvest.newsmoldova.ro/.
    \item publicarea, dupa ora 18:00 pe 30.11.2024, a unei postari pe Facebook (ID: \href{https://www.facebook.com/ads/library/?id=509034284929292}{509034284929292}) cu continut electoral, care promoveaza candidatul Ciprian Constantin Serban (PSD) la Camera Deputatilor. Postarea, cu titlul Voteaza calea sigura pentru Neamt, are un efect electoral clar, vizand influentarea votului in favoarea candidatului PSD prin prezentarea sa intr-o lumina pozitiva si evidentierea realizarilor sale politice.  Aceasta postare, platita cu o suma sub 100 RON, a atins o estimare de 100.000 - 500.000 de persoane in judetul Neamt, conform datelor din Ad Library.  Contactul Moldova Invest poate fi realizat la adresa de email contact@moldovainvest.newsmoldova.ro sau la numarul de telefon +40772021599.  Lipsa obiectivitatii si neutralitatii, caracteristice jurnalismului, precum si prezentarea unilaterala a informatiilor, demonstreaza intentia clara de a influenta votul, constituind astfel propaganda electorala interzisa dupa incheierea campaniei electorale.
\end{enumerate}

\vspace{0.5cm}

\subsection{Momentum Motion Pictures SRL}
Următoarele fapte contravenționale sunt sesizate împotriva acestei entități:

\begin{enumerate}[leftmargin=*, label=\arabic*.)]
    \item publicarea unei reclame platite pe Facebook (ID postare: \href{https://www.facebook.com/ads/library/?id=1123391206457102}{1123391206457102}), dupa ora 18:00 pe 30.11.2024, cu mesajul "Pe 1 decembrie, mergem sa votam pentru lideri care inteleg ca educatia este fundatia unui viitor mai bun",  in contextul apropiat de alegeri, avand un efect electoral pozitiv asupra candidatului Manolache Vasile, promovat indirect prin asocierea valorii educatiei cu un vot favorabil.  Reclama, desi abordeaza tema educatiei, serveste drept instrument de influentare a votului, indemnand la sustinerea unor candidati care impartasesc aceasta valoare.  Cheltuielile pentru reclama au fost sub 100 RON, atingand o audienta estimata intre 100.000 si 500.000 de persoane, conform datelor din Ad Library.  Contact: contact@momentum.com.ro, +40767277777.
    \item publicarea unei reclame platite pe Facebook dupa ora 18:00 pe 30.11.2024, care, desi nu indeamna explicit la votarea lui Manolache Vasile, creeaza o asociere pozitiva cu candidatura sa prin utilizarea numelui si imaginii sale intr-un mesaj care incurajeaza participarea la vot.  Efectul electoral este o crestere indirecta a numarului de voturi pentru Manolache Vasile, prin asocierea pozitiva a acestuia cu un mesaj care promoveaza participarea la vot.  Postarea, cu ID-ul \href{https://www.facebook.com/ads/library/?id=1160822938797308}{1160822938797308},  a fost vizualizata de sub 1000 de persoane, conform datelor din Ad Library, si a costat mai putin de 100 RON.  Contactul este contact@momentum.com.ro si +40767277777.  Tinta reclamei a fost populatia din Bistrita-Nasaud, cu varste cuprinse intre 29 si 65 de ani.  Aceasta actiune constituie propaganda electorala interzisa dupa incheierea campaniei electorale.
    \item publicarea unei reclame platite pe Facebook dupa ora 18:00 pe 30.11.2024, cu ID-ul postarii \href{https://www.facebook.com/ads/library/?id=1210400386734563}{1210400386734563}, care promoveaza candidatura lui Manolache Vasile la Senat, avand un efect electoral pozitiv prin indemnul explicit la vot ("votul tau pentru Parlament conteaza enorm!", "vino la urne si fii parte din schimbare!").  Reclama, cu o cheltuiala de sub 100 RON, a atins o audienta estimata intre 100.000 si 500.000 de persoane, conform datelor din Ad Library, si a fost publicata pe Facebook si Instagram.  Contactul Momentum Motion Pictures SRL este contact@momentum.com.ro si +40767277777.  Aceasta actiune constituie propaganda electorala interzisa dupa incheierea perioadei legale de campanie.
    \item publicarea, dupa ora 18:00 pe 30.11.2024, a unei reclame platite pe Facebook (ID postare: \href{https://www.facebook.com/ads/library/?id=1234037750987811}{1234037750987811}), care promoveaza candidatul Manolache Vasile la Senat,  printr-un mesaj care, desi nu solicita explicit votul pentru acesta,  il asociaza indirect cu o promisiune de imbunatatire a sistemului sanitar si contine un apel la vot ("Pe 1 decembrie, mergem sa votam...").  Efectul electoral al acestei postari este clar, avand ca obiectiv influentarea votului in favoarea lui Manolache Vasile.  Mesajul,  cu hashtag-uri precum \#ManolacheVasileSenator, \#SanatatePentruToti,  \#SchimbareinSanatate, \#ViitorFaraCoruptie,  este conceput pentru a crea o imagine pozitiva a candidatului si a-i spori sansele de succes electoral.  Reclama, cu o cheltuiala de sub 100 RON, a atins intre 1000 si 2000 de impresii, cu o estimare a acoperirii de 100.000 - 500.000 de persoane, conform datelor din Ad Library.  Informatiile de contact ale Momentum Motion Pictures SRL sunt disponibile in Ad Library: contact@momentum.com.ro si +40767277777.
    \item publicarea unei reclame platite pe Facebook (ID postare: \href{https://www.facebook.com/ads/library/?id=1261718045026800}{1261718045026800}), dupa ora 18:00 pe 30.11.2024, care promoveaza candidatura lui Manolache Vasile la alegerile senatoriale, folosind un limbaj persuasiv si apeluri directe la vot ("cel mai frumos cadou pe care il pot primi este increderea dumneavoastra exprimata prin vot", "Pe 1 decembrie, haideti sa punem impreuna bazele unei schimbari adevarate!").  Reclama, cu o estimare a bugetului de RON 100-199 si o raza de cuprindere de 100.000-500.000 de persoane, are ca efect electoral influentarea votului in favoarea candidatului mentionat.  Lipsa obiectivitatii si neutralitatii caracteristice jurnalismului, precum si contextul pre-electoral, confirma natura de propaganda electorala a postarii.  Informatiile de contact ale Momentum Motion Pictures SRL sunt disponibile in Ad Library: contact@momentum.com.ro, +40767277777, Bucharest, Bucharest, https://momentum.com.ro/.
    \item publicarea unei reclame platite pe Facebook (ID postare: \href{https://www.facebook.com/ads/library/?id=423394820841845}{423394820841845}), dupa ora 18:00 pe 30.11.2024, cu mesajul  Stiti ce ar trebui sa se intample duminica? Oamenii sa isi exercite dreptul de vot liber, fara presiuni, fara amenintari si fara influente nedrepte.,  mesaj ce, desi aparent neutru,  in contextul electoral si al asocierii cu numele senatorului Manolache Vasile (prezent in hashtag-uri), are un efect electoral pozitiv,  incurajand implicit votul liber si criticand practicile de constrangere electorala.  Reclama, cu o cheltuiala de sub 100 RON, a vizat o audienta de 100K-500K persoane, concentrata in judetele Bistrita-Nasaud, Mures, Cluj si Maramures, conform datelor din Ad Library.  Contact: contact@momentum.com.ro, +40767277777.  Aceasta actiune constituie propaganda electorala indirecta, influentand votul in favoarea lui Manolache Vasile, incalcand astfel prevederile legale.
    \item publicarea unei reclame platite pe Facebook, cu ID-ul \href{https://www.facebook.com/ads/library/?id=542177428693423}{542177428693423}, dupa ora 18:00 pe 30.11.2024, care, desi nu face o sustinere explicita a candidatului Manolache Vasile, il mentioneaza in context electoral, intr-un mesaj care promoveaza votul liber, dar care, prin contextul reclamei platite si prin vizarea unei anumite demografice, are un efect electoral pozitiv pentru acesta.  Mesajul, desi aparent neutru, critica implicit practicile care ar putea suprima votul pentru anumiti candidati, beneficiind indirect pe Manolache Vasile.  Reclama, cu o cheltuiala de sub 100 RON, a atins o estimare de 100.000 - 500.000 de persoane, conform datelor din Ad Library, si poate fi contactata la contact@momentum.com.ro sau +40767277777.
    \item difuzarea unei reclame platite pe Facebook dupa ora 18:00 pe 30.11.2024, cu ID-ul postarii \href{https://www.facebook.com/ads/library/?id=874335204687626}{874335204687626}, care promoveaza candidatul Manolache Vasile la alegerile parlamentare, prezentandu-l intr-o lumina pozitiva si facand promisiuni electorale, avand ca efect electoral influentarea voturilor in favoarea acestuia.  Reclama, cu un buget de sub 100 RON, a atins o audienta estimata intre 100.000 si 500.000 de persoane, conform datelor din Facebook Ad Library.  Continutul postarii, care subliniaza angajamentul lui Manolache Vasile fata de comunitate si utilizarea hashtag-ului \#alegeriparlamentare2024, demonstreaza clar intentia de a influenta votul, chiar daca nu solicita explicit votul pentru acesta.  Informatiile de contact ale Momentum Motion Pictures SRL sunt disponibile in Facebook Ad Library: contact@momentum.com.ro si +40767277777.
    \item publicarea, dupa ora 18:00 pe 30.11.2024, a unei postari pe Facebook (ID \href{https://www.facebook.com/ads/library/?id=879362090933456}{879362090933456}) cu scop electoral, care promoveaza candidatura lui Manolache Vasile la Senat. Postarea, cu o cheltuiala sub RON 100 si o estimare a audientei de 100.000-500.000 de persoane, utilizeaza statistici despre speranta de viata pentru a sublinia deficientele sistemului sanitar, sugerand implicit ca Manolache Vasile ar putea aduce imbunatatiri.  Apelul la vot din 1 decembrie, combinat cu mentionarea proeminenta a numelui candidatului, are ca obiectiv clar influentarea alegatorilor.  Continutul si directionarea postarii se aliniaza definitiei propagandei electorale conform articolului 36 (7) din LEGEA nr. 334 din 17 iulie 2006, iar publicarea acesteia dupa incheierea perioadei oficiale de campanie constituie o incalcare grava.  Informatiile de contact ale Momentum Motion Pictures SRL sunt disponibile in Ad Library: contact@momentum.com.ro si +40767277777.
\end{enumerate}

\vspace{0.5cm}

\subsection{MozaiQ}
Următoarele fapte contravenționale sunt sesizate împotriva acestei entități:

\begin{enumerate}[leftmargin=*, label=\arabic*.)]
    \item publicarea unei postari platite pe Facebook (ID: \href{https://www.facebook.com/ads/library/?id=1125773355569863}{1125773355569863}), dupa ora 18:00 pe 30.11.2024, care prezinta extrase din platformele unor partide politice referitoare la drepturile femeilor si LGBTQ+, cu citate de la candidati la alegerile parlamentare, avand un efect electoral clar prin influentarea opiniei publice.  Postarea, cu o cheltuiala de 400-499 RON si o audienta estimata la peste 1 milion de persoane,  nu se incadreaza in exceptiile legale, deoarece, desi pare informativa, selectia si prezentarea citatelor pot fi interpretate ca o propaganda subtila, influentand votul.  Citatele de la Florina Presada (SENS), Diana Sosoaca (SOS), Cosette Chichirau (DREPT), Cristian Terhes (PNCR), Szabo Odon (UDMR) si Alexandra Pacuraru (ADN) demonstreaza legatura directa cu alegerile parlamentare.  Lipsa unui numar CMF nu exclude incriminarea, dat fiind efectul electoral al continutului.  Contact: vladviski@mozaiqlgbt.ro, +40730924433, https://www.mozaiqlgbt.ro/.
\end{enumerate}

\vspace{0.5cm}

\subsection{NEWBIZ ONLINE PUBLISHING S.R.L.}
Următoarele fapte contravenționale sunt sesizate împotriva acestei entități:

\begin{enumerate}[leftmargin=*, label=\arabic*.)]
    \item publicarea, dupa ora 18:00 pe 30.11.2024, a unei postari pe Facebook (ID \href{https://www.facebook.com/ads/library/?id=1288484149018085}{1288484149018085}) cu continut propagandistic electoral, care prezinta o analiza subiectiva si tendentioasa a succesului electoral al candidatului Calin Georgescu, utilizand termeni precum outsider extremist si taifun, cu scopul de a influenta negativ perceptia publicului asupra acestuia.  Postarea, cu o estimare a audientei de peste 1 milion de persoane si un buget de 100-199 RON, depaseste limitele jurnalismului obiectiv si are un efect electoral clar negativ asupra lui Calin Georgescu.  Titlul si continutul postarii, precum si imaginea atasata, contribuie la crearea unei imagini negative a candidatului, influentand decizia de vot a publicului.  Contact: sergiu@panorama.ro, +40771101657.
\end{enumerate}

\vspace{0.5cm}

\subsection{Opinii Independente}
Următoarele fapte contravenționale sunt sesizate împotriva acestei entități:

\begin{enumerate}[leftmargin=*, label=\arabic*.)]
    \item publicarea unei postari platite pe Facebook (ID: \href{https://www.facebook.com/ads/library/?id=1281676142849091}{1281676142849091}), dupa ora 18:00 pe 30.11.2024, care promoveaza Partidul Romania in Actiune (PRA) prin evidentierea implicarii lui Mircea Geoana, avand un efect electoral pozitiv asupra PRA. Postarea, cu o estimare a vizualizarilor de 10.000-50.000, indeplineste criteriile propagandei electorale conform articolului 36 (7) din LEGE nr. 334 din 17 iulie 2006, prin referirea directa la un candidat (implicit, prin asocierea cu PRA), adresarea catre publicul larg si obiectivul electoral clar.  Lipsa unui numar CMF nu diminueaza caracterul promotional al postarii.  Publicarea dupa incheierea campaniei electorale constituie o incalcare a articolului 98 t).  Postarea, accesibila la [link-ul postarii], contine imagini si text care promoveaza pozitiv PRA si Mircea Geoana, influentand opinia publica in favoarea partidului.  Suma cheltuita pentru publicitate este sub 100 RON.  Datele de contact ale Opinii Independente sunt office@opinii-independente.ro si +40723074107.
\end{enumerate}

\vspace{0.5cm}

\subsection{PNCR}
Următoarele fapte contravenționale sunt sesizate împotriva acestei entități:

\begin{enumerate}[leftmargin=*, label=\arabic*.)]
    \item publicarea unei postari pe Facebook (ID: \href{https://www.facebook.com/ads/library/?id=1522939945077241}{1522939945077241}) dupa ora 18:00 pe 30.11.2024, care promoveaza candidatura sa la Camera Deputatilor din partea PNCR,  continand un mesaj explicit de sustinere a candidaturii sale si indemn la vot,  precum si mentionarea unui numar CMF (CMF11240003), fapt ce constituie propaganda electorala dupa incheierea perioadei legale. Postarea, cu o cheltuiala de 100-199 RON, a generat intre 6.000 si 7.000 de impresii, ajungand la o estimare de 100.000 - 500.000 de persoane, avand ca efect electoral influentarea votului in favoarea candidatului.  Informatiile de contact ale advertiserului sunt disponibile: contact@sebastiantiberiupop.ro si +40760256257.
    \item publicarea unei postari pe Facebook (ID: \href{https://www.facebook.com/ads/library/?id=1924264031390612}{1924264031390612}) dupa ora 18:00 pe 30.11.2024, cu scop electoral,  continand  mesaje care promoveaza indirect PNCR si candidatul Cristian Terhes prin utilizarea hashtag-urilor "\#pncr\#cristianterhes" si un numar CMF (CMF11240003),  avand un efect electoral pozitiv asupra PNCR. Postarea, platita intre 200 si 299 RON, a atins intre 10.000 si 15.000 de impresii si o estimare a audientei de 100.000 - 500.000 de persoane,  constituie propaganda electorala interzisa dupa incheierea perioadei de campanie.  Continutul, desi nu solicita explicit votul,  critica indirect situatia politica actuala si promoveaza valori care se pot asocia cu PNCR, influentand astfel opinia publica in favoarea partidului.  Informatiile de contact ale advertiserului sunt contact@sebastiantiberiupop.ro si +40760256257.
    \item publicarea unei postari platite pe Facebook (ID postare: \href{https://www.facebook.com/ads/library/?id=392299580540497}{392299580540497}) dupa ora 18:00 pe 30.11.2024, care promoveaza candidatul Pop Sebastian Tiberiu pentru Camera Deputatilor din partea PNCR, prezentand initiativa sa legislativa privind educatia civica intr-un mod care depaseste limitele jurnalismului si are ca scop influentarea votului. Postarea, care a avut intre 4.000 si 5.000 de impresii si o estimare a acoperirii de 100.000 - 500.000 de persoane, contine hashtag-uri precum \#ViitorulEAlTau, \#TineriiVoteaza, \#PutereaVotului si \#PNCR, evidentiind clar intentia electorala.  Suma cheltuita pentru aceasta postare a fost intre 100 si 199 RON, conform informatiilor din Ad Library.  Contactul advertiserului este contact@sebastiantiberiupop.ro si +40760256257.  Prezentarea initiativei legislative, desi pozitiva in sine, este utilizata ca instrument de campanie electorala, incalcand astfel prevederile legale privind continuarea propagandei electorale dupa incheierea perioadei legale.
    \item publicarea unei reclame platite pe Facebook (ID postare: \href{https://www.facebook.com/ads/library/?id=3992315094428233}{3992315094428233}) dupa ora 18:00 pe 30.11.2024, care promoveaza candidatul PNCR de pe pozitia 12 de pe buletinul de vot, utilizand imaginea acestuia si mesaje care indeamna la vot pentru acesta. Reclama, cu un buget de sub 100 RON, a generat intre 7.000 si 8.000 de impresii si a avut o estimare a acoperirii de 100.000 - 500.000 de persoane.  Prezenta numarului CMF 1124003 confirma natura electorala a materialului.  Textul reclamei, cu fraze precum Iesi la vot!, Alege schimbarea, si accentul pe importanta alegerilor parlamentare, demonstreaza clar intentia de a influenta votul in favoarea candidatului PNCR.  Aceasta actiune constituie propaganda electorala interzisa dupa incheierea campaniei electorale, conform legislatiei in vigoare.  Datele de contact ale advertiserului sunt: contact@sebastiantiberiupop.ro si +40760256257.
    \item publicarea, dupa ora 18:00 pe 30.11.2024, a unei postari pe Facebook (ID: \href{https://www.facebook.com/ads/library/?id=452313740908308}{452313740908308}) cu caracter de propaganda electorala, care promoveaza candidatura lui Sebastian Tiberiu Pop pentru Camera Deputatilor din partea PNCR. Postarea, care a beneficiat de o investitie financiara intre 100 si 199 RON,  contine un mesaj explicit de sustinere a candidaturii, invitand la ascultarea unui podcast electoral si utilizand un apel la actiune direct ("Pe 1 decembrie, sa aratam ca vocea noastra conteaza!"). Prezenta codului CMF11240003 confirma natura sa de material de campanie electorala.  Avand in vedere ca perioada de campanie electorala s-a incheiat, aceasta postare constituie o incalcare clara a legii, avand un efect electoral evident prin promovarea candidatului si incercarea de a influenta votul alegatorilor.  Datele de contact ale advertiserului sunt contact@sebastiantiberiupop.ro si +40760256257. Postarea a generat intre 5000 si 6000 de impresii, cu o estimare a acoperirii de 100.000 - 500.000 de persoane.
    \item publicarea, dupa ora 18:00 pe 30.11.2024, a unei postari pe Facebook (ID \href{https://www.facebook.com/ads/library/?id=542786071898985}{542786071898985}), cu cheltuieli de publicitate sub 100 RON, ce contine imagini cu candidatul distribuind materiale electorale si indemnuri explicite la vot pentru PNCR si candidatul Sebastian - Tiberiu Pop, avand ca efect electoral influentarea voturilor in favoarea acestuia.  Postarea, cu fraze precum Voteaza schimbarea reala!, Impreuna ducem Pecica in Parlament!, si imaginile atasate, depaseste limitele activitatii jurnalistice si constituie propaganda electorala interzisa dupa incheierea campaniei electorale, conform articolului 36 (7) din LEGE nr. 334 din 17 iulie 2006.  Postarea a atins o audienta estimata la peste 1 milion de persoane, conform datelor din Ad Library, cu contactul advertiserului disponibil la contact@sebastiantiberiupop.ro si +40760256257.
    \item publicarea unei postari platite pe Facebook (ID \href{https://www.facebook.com/ads/library/?id=570608735909461}{570608735909461}), dupa ora 18:00 pe 30.11.2024, cu continut ce promoveaza indirect Partidul National Conservator Roman (\#PNCR) prin sustinerea unor politici economice specifice, avand ca efect electoral influentarea votului in favoarea partidului mentionat.  Postarea, desi prezentata sub forma unui articol de opinie, contine hashtag-uri politice (\#PNCR, \#SolutiiNuProbleme, etc.) si un numar CMF (CMF11240003), indicand clar natura sa electorala.  Avand in vedere cheltuielile de sub 100 RON,  ajungand la 7000-8000 de impresii si o estimare a acoperirii de 100.000-500.000 de persoane,  se poate demonstra intentia clara de a influenta electoratul.  Continutul postarii, care promoveaza legi specifice pentru sprijinirea producatorilor locali, este direct legat de agenda politica a PNCR, constituind astfel propaganda electorala interzisa dupa incheierea perioadei de campanie.
    \item publicarea unei postari platite pe Facebook (ID postare: \href{https://www.facebook.com/ads/library/?id=598402542630899}{598402542630899}), dupa ora 18:00 pe 30.11.2024, care, desi abordeaza superficial tema educatiei civice, constituie propaganda electorala. Postarea mentioneaza explicit candidatura lui Pop Sebastian Tiberiu pentru Camera Deputatilor din partea PNCR, utilizeaza numeroase hashtag-uri cu conotatii electorale (\#ViitorulEAlTau, \#TineriiVoteaza, etc.) si include un numar CMF (CMF11240003), indicand clar natura sa de material de campanie.  Efectul electoral este evident, avand in vedere ca postarea urmareste sa creeze o asociere pozitiva intre candidat si o initiativa de politica publica, influentand astfel votul alegatorilor.  Avand in vedere cheltuielile de publicitate sub 100 RON, conform Ad Library, si o estimare a audientei de peste 1 milion de persoane, impactul acestei postari este semnificativ.  Contactul advertiserului este contact@sebastiantiberiupop.ro si +40760256257.
    \item publicarea unei postari pe Facebook (ID: \href{https://www.facebook.com/ads/library/?id=937272021083576}{937272021083576}) dupa ora 18:00 pe 30.11.2024, care promoveaza candidatura lui Pop Sebastian Tiberiu pentru Camera Deputatilor din partea PNCR. Postarea, care a beneficiat de o investitie financiara de sub 100 RON, a atins o audienta de peste 1 milion de persoane, utilizand hashtag-uri precum \#PutereaVotului si \#TineriiVoteaza pentru a influenta votul.  Prezenta numarului CMF11240003 confirma natura sa de propaganda electorala. Continutul, desi abordeaza educatia civica, este prezentat intr-un cadru de campanie politica, lipsit de neutralitatea necesara unui material jurnalistic sau educational. Efectul electoral este clar, avand ca obiectiv cresterea numarului de voturi pentru candidatul mentionat.  Datele de contact ale advertiserului sunt contact@sebastiantiberiupop.ro si +40760256257.
    \item publicarea unei postari platite pe Facebook (ID: \href{https://www.facebook.com/ads/library/?id=946373124019157}{946373124019157}), dupa ora 18:00 pe 30.11.2024, care, desi prezentata ca o propunere legislativa, constituie propaganda electorala. Postarea, cu cheltuieli sub 100 RON, vizeaza promovarea candidatului Sebastian - Tiberiu Pop, afiliat PNCR, prin accentuarea competentei si responsabilitatii sale.  Efectul electoral este evident, avand in vedere ca postarea este directionata catre un public larg, utilizand un limbaj persuasiv si apeland la emotii.  Mesajul implicit este ca sustinerea propunerii legislative echivaleaza cu sustinerea candidaturii.  Prezenta hashtag-urilor \#PNCR si imaginea atasata confirma afilierea politica.  Contact: contact@sebastiantiberiupop.ro, +40760256257.  Numarul de impresii este estimat la 1000-2000.
\end{enumerate}

\vspace{0.5cm}

\subsection{PNL}
Următoarele fapte contravenționale sunt sesizate împotriva acestei entități:

\begin{enumerate}[leftmargin=*, label=\arabic*.)]
    \item publicarea unei reclame platite pe Facebook cu ID-ul \href{https://www.facebook.com/ads/library/?id=1032187598679458}{1032187598679458}, dupa ora 18:00 pe 30.11.2024, care promoveaza echipa \#EchipaValeriuIftime,  avand un efect electoral clar prin promovarea indirecta a candidatului Valeriu Iftime la alegerile locale pentru Primaria Leorda.  Prezenta numarului CMF 11240002 confirma natura electorala a postarii, iar imaginile atasate intaresc mesajul de sustinere.  Reclama, cu o cheltuiala de 200-299 RON, a atins 15.000-20.000 de impresii, avand o estimare a acoperirii de 100.000-500.000 de persoane.  Persoanele responsabile pot fi contactate la contact@pnlbotosani.ro sau +40746995316.
    \item publicarea, dupa ora 18:00 pe 30.11.2024, a unei postari pe Facebook (ID: \href{https://www.facebook.com/ads/library/?id=1098833928367867}{1098833928367867}) cu caracter de propaganda electorala, care promoveaza explicit candidatul Valeriu Iftime si Partidul National Liberal (PNL), indemnand la vot pentru acesta. Postarea, cu un buget de sub 100 RON, a atins o audienta estimata intre 100.000 si 500.000 de persoane, utilizand fraze precum Echipa Valeriu Iftime, PNL, si Pozitia 1 pe buletinul de vot, avand ca efect electoral influentarea votului in favoarea PNL.  Informatiile de contact ale advertiserului sunt office@voceabotosani.ro si +40759112233.  Aceasta postare constituie propaganda electorala interzisa dupa incheierea perioadei legale de campanie, conform articolului 98 t).
    \item publicarea unei postari pe Facebook (ID: \href{https://www.facebook.com/ads/library/?id=1205923177168461}{1205923177168461}) dupa ora 18:00 pe 30.11.2024, care constituie propaganda electorala, avand in vedere ca promoveaza candidatura sa la alegerile parlamentare si indeamna la vot pentru echipa PNL, folosind un CMF (11240002),  cu un buget de 100-199 RON,  ajungand la 8000-9000 de impresii si o estimare a audientei de 50.000-100.000 persoane. Postarea, cu textul "Acestea sunt misiunile mele ca viitor deputat. Dar pentru a reusi, am nevoie de increderea voastra! Duminica, votati echipa PNL!",  este o forma clara de propaganda electorala, avand ca efect electoral influentarea voturilor in favoarea candidatei si a partidului PNL.  Aceasta postare incalca prevederile legale privind incetarea campaniei electorale, fiind publicata dupa incheierea perioadei legale de campanie.
    \item publicarea unei postari platite pe Facebook dupa ora 18:00 pe 30.11.2024, cu ID-ul \href{https://www.facebook.com/ads/library/?id=1267675860939708}{1267675860939708}, care promoveaza PNL si Ilie Bolojan cu mesajul "PNL si Ilie Bolojan  Alegerea pentru un Parlament al faptelor!", avand un efect electoral clar de a influenta votul in favoarea acestora. Postarea, desi nu contine un numar CMF, indeplineste toate criteriile propagandei electorale conform articolului 36 (7) din LEGE nr. 334 din 17 iulie 2006, adresandu-se unui public larg prin intermediul unei reclame platite pe Facebook, cu o cheltuiala de sub 100 RON, ajungand la 10.000-15.000 de impresii si o estimare a acoperirii de 100.000-500.000 de persoane.  Informatiile de contact ale advertiserului sunt mihaitanegura@yahoo.com si +40723177715.
    \item publicarea unei postari platite pe Facebook (ID postare: \href{https://www.facebook.com/ads/library/?id=1298743991259471}{1298743991259471}) dupa ora 18:00 pe 30.11.2024, care promoveaza candidatii PNL Aneta Matei, Toma Petcu, Maria Gabriela Horga, Alexandru Andrei si Alina Meclea, cu un buget de publicitate sub 100 RON, atingand 2000-3000 de impresii si o estimare a audientei de 5000-10000 de persoane. Postarea contine CMF 11240002, clar identificand-o ca material de propaganda electorala.  Imaginea si textul postarii au ca scop influentarea votului in favoarea candidatilor PNL, reprezentand o incalcare clara a legii, avand in vedere ca perioada de campanie electorala s-a incheiat.  Datele de contact ale advertiserului sunt: cosmin.radoi@blueads.ro si +40726405495.
    \item publicarea unei reclame electorale platite pe Facebook (ID postare: \href{https://www.facebook.com/ads/library/?id=1460906211256435}{1460906211256435}) dupa ora 18:00 pe 30.11.2024,  promovand candidatii Gabriel Pana si Maria Gabriela Horga,  afiliati PNL, cu un buget de sub 100 RON,  ajungand la 5.000-10.000 de impresii,  prin intermediul imaginii care ii prezinta pe candidati cu titlurile lor (Primar si Senator) si sigla PNL,  avand ca efect electoral cresterea vizibilitatii si a sprijinului pentru acestia.  Reclama contine numarul CMF 11240002, indicand clar natura sa electorala.  Aceasta actiune constituie propaganda electorala interzisa dupa incheierea perioadei legale de campanie, conform articolului 98 t).  Datele de contact ale advertiserului sunt: cosmin.radoi@blueads.ro si +40726405495.
    \item publicarea unei reclame platite pe Facebook cu ID-ul \href{https://www.facebook.com/ads/library/?id=1517892218924770}{1517892218924770}, dupa ora 18:00 pe 30.11.2024,  cu mesajul "Agricultura. Motorul judetului Botosani! \#continuamSchimbarea CMF 11240002",  care constituie propaganda electorala interzisa dupa incheierea perioadei de campanie.  Reclama, cu un buget sub 100 RON, a vizat o audienta extinsa in judetul Botosani si judetele invecinate, avand ca efect electoral promovarea imaginii PNL si influentarea votului in favoarea candidatilor sai. Prezenta numarului CMF confirma natura electorala a reclamei.  Contactul advertiserului este ionut.voicu.bt@gmail.com si +40756823832.
    \item publicarea unei reclame platite pe Facebook dupa ora 18:00 pe 30.11.2024 (data trebuie verificata in raport cu calendarul electoral), cu mesajul "Hai la vot pe 1 decembrie! Votam echipa PNL - Valeriu Iftime!", care indeamna direct la votarea candidatului Valeriu Iftime si a partidului PNL, avand un efect electoral clar pozitiv pentru acestia.  Reclama, cu ID-ul \href{https://www.facebook.com/ads/library/?id=1606080077000792}{1606080077000792}, a fost difuzata pe Facebook si Instagram, avand o estimare a bugetului de 200-299 RON si o estimare a vizualizarilor de 15.000-20.000.  Aceasta actiune constituie propaganda electorala dupa incheierea perioadei legale de campanie, conform articolului 98 t), intrucat indeamna la vot pentru un anumit candidat in ziua alegerilor.  Informatiile de contact ale advertiserului sunt contact@pnlbotosani.ro si +40759112233.
    \item publicarea unei postari pe Facebook (ID: \href{https://www.facebook.com/ads/library/?id=1674562483439492}{1674562483439492}) dupa ora 18:00 pe 30.11.2024, care contine propaganda electorala pentru PNL,  cu un buget de 100-199 RON,  ajungand la peste 1 milion de oameni,  prin citarea lui Adrian Ioan Vestea si afirmatii precum PNL a demonstrat ca stie si poate moderniza Romania si Luptam pentru Romania, luptam pentru PNL!,  avand ca efect electoral influentarea votului in favoarea PNL. Prezenta numarului CMF 11240002 confirma natura electorala a postarii.  Postarea, desi contine o scurta anecdota personala, este in principal o reclama politica care depaseste limitele jurnalismului, conform articolului 36 (7) din LEGE nr. 334 din 17 iulie 2006. Informatiile de contact ale advertiserului sunt tlmecomanagement@gmail.com si +40712282326.
    \item publicarea unei reclame platite pe Facebook (ID postare: \href{https://www.facebook.com/ads/library/?id=1718321198989578}{1718321198989578}), dupa ora 18:00 pe 30.11.2024, cu un buget de 900-999 RON, care promoveaza imaginea sa pozitiva si, implicit, influenteaza votul alegatorilor din judetul Ialomita. Postarea, desi nu solicita explicit voturi, subliniaza importanta turismului pentru bunastarea judetului si necesitatea unor oameni pregatiti, mesaj care se poate interpreta ca o promovare subtila a competentelor sale in contextul apropiat al alegerilor parlamentare.  Avand in vedere afilierea sa la PNL, functia de senator si presedinte al Comisiei pentru cultura si media din Senat, precum si tinta geografica a reclamei, se poate concluziona ca scopul este influentarea votului in favoarea sa.  Reclama a atins 50.000-100.000 de persoane, conform datelor din Ad Library.  Contact: gavriladanielag@gmail.com, +40760563856.
    \item publicarea, dupa ora 18:00 pe 30.11.2024, a unei postari pe Facebook (ID \href{https://www.facebook.com/ads/library/?id=439880395442886}{439880395442886}) cu caracter de propaganda electorala, care promoveaza candidatul PNL la Senat, Maria Gabriela Horga, precum si alti candidati PNL,  prin evidentierea realizarilor lor si a apartenentei la PNL, avand ca efect electoral influentarea pozitiva a votului in favoarea acestora. Postarea, care a generat intre 80.000 si 90.000 de impresii,  contine imagini ale candidatilor si mentioneaza explicit realizarile candidatei Horga, inclusiv numarul de legi votate in mandatul anterior,  cu scopul clar de a-i influenta pozitiv pe alegatori.  Prezenta numarului CMF 11240002 confirma natura de propaganda electorala a postarii.  Postarea a costat intre 100 si 199 RON si a fost platita de Forges Policy, cu datele de contact disponibile in Ad Library: contact@forgespolicy.ro si +40790663007.
    \item publicarea, dupa ora 18:00 pe 30.11.2024, a unei postari pe Facebook (ID \href{https://www.facebook.com/ads/library/?id=539277882281285}{539277882281285}) cu caracter de propaganda electorala, care promoveaza candidatul Valeriu Iftime de la PNL,  in mod evident, prin prezentarea sa intr-o lumina pozitiva si prin asocierea sa cu valori traditionale,  cu scopul de a influenta votul alegatorilor.  Postarea, care a beneficiat de o suma de bani sub 100 RON, conform Ad Library, si a atins o audienta estimata la 100.000-500.000 de persoane, contine un numar CMF (11240002), indicand clar natura sa de propaganda electorala.  Textul postarii,  "echipa pe care o propun la alegerile parlamentare de duminica aceasta", constituie o indemnare directa la vot.  Aceasta actiune este ilegala, deoarece are loc dupa incheierea perioadei legale de campanie electorala, conform prevederilor legale.  Datele de contact ale Vocea Botosani sunt office@voceabotosani.ro si +40759112233.
    \item publicarea unei reclame platite pe Facebook dupa ora 18:00 pe 30.11.2024, cu ID-ul \href{https://www.facebook.com/ads/library/?id=543310025281526}{543310025281526}, care promoveaza pozitiv pe Ilie Bolojan, presedintele PNL, fara a indemna explicit la vot, dar creand o impresie favorabila in preajma alegerilor.  Reclama, cu o cheltuiala de 100-199 RON, a atins 20.000-25.000 de impresii, vizand in principal judetul Hunedoara, avand ca efect electoral potential influentarea perceptiei electoratului asupra lui Ilie Bolojan si a PNL.  Fraza Astazi este momentul! sugereaza o legatura directa cu contextul electoral actual.  Informatiile din Ad Library indica Ovidiu Vlad (ovivlad@yahoo.com, +40721300733) ca persoana de contact.
    \item publicarea unei postari pe Facebook cu ID-ul \href{https://www.facebook.com/ads/library/?id=561393536849758}{561393536849758}, dupa ora 18:00 pe 30.11.2024, care constituie propaganda electorala interzisa dupa incheierea perioadei de campanie. Postarea, care a costat intre 600 si 699 RON,  se refera direct la candidata Alina Gorghiu si la PNL, indemnand la vot cu fraza "Duminica, votati echipa PNL!", avand un efect electoral clar pozitiv pentru candidata. Prezenta codului CMF 11240002 confirma natura sa de material de propaganda electorala.  Postarea, vizibila pe Facebook, a atins intre 90.000 si 100.000 de impresii, conform datelor din Ad Library,  si a avut o estimare a acoperirii de 100.000-500.000 de persoane.  Informatiile de contact ale candidatei sunt disponibile public: senator@alinagorghiu.ro si +40765055855.
    \item publicarea, dupa ora 18:00 pe 30.11.2024, a unei postari pe Facebook (ID: \href{https://www.facebook.com/ads/library/?id=572535765367817}{572535765367817}), cu un buget de 500-599 RON, care promoveaza candidatii PNL Hubert Thuma si Nicoleta Pauliuc, indemnand la votul pentru acestia si indicand pozitia lor pe buletinul de vot. Postarea, care a generat intre 10.000 si 15.000 de impresii si a avut o estimare a acoperirii de 500.000 - 1.000.000 de persoane, contine numarul CMF 11240002, indicand clar caracterul de propaganda electorala.  Efectul electoral este evident, avand ca obiectiv influentarea votului in favoarea candidatilor PNL.  Fraze precum "votam pentru Romania moderna", "Ilfovenii voteaza PNL", si "pozitia 1 pe buletinul de vot" demonstreaza intentia clara de a influenta votul.  Contact: pnl.ilfov@pnl.ro, +40765258996.
    \item publicarea unei reclame platite pe Facebook (ID postare: \href{https://www.facebook.com/ads/library/?id=585218097425254}{585218097425254}) dupa ora 18:00 pe 30.11.2024, care promoveaza direct candidatul Ilie Bolojan si lista PNL Bucuresti pentru alegerile parlamentare, cu un buget de 1.500-2.000 RON, atingand peste 1 milion de oameni.  Postarea contine un apel explicit la vot ("Voteaza lista PNL Bucuresti pentru alegerile \#Parlamentare!") si numarul CMF11240002, indicand clar natura sa de propaganda electorala.  Aceasta actiune, avand ca efect electoral cresterea numarului de voturi pentru PNL, constituie o incalcare clara a legii, deoarece continua propaganda electorala dupa incheierea perioadei legale.  Informatiile de contact ale Partidului National Liberal sunt disponibile la comunicare@pnl.ro si +40743831906.
    \item publicarea, dupa ora 18:00 pe 30.11.2024, a unei postari pe Facebook (ID \href{https://www.facebook.com/ads/library/?id=884598480552277}{884598480552277}) cu caracter de propaganda electorala, care promoveaza explicit candidatii Valeriu Iftime, Eduard Mititelu, Buliga Luciana, Dumitriu Alexandru, Jitaru Ovidiu si Ionela Aiftinca de la PNL,  incurajand votul pentru acestia. Postarea, platita cu o suma sub 100 RON,  a avut o estimare a audientei de 100.000-500.000 de persoane, concentrata in judetul Botosani si zonele limitrofe, conform datelor din Ad Library.  Prezenta numarului CMF 11240002 confirma natura electorala a postarii.  Textul postarii, cu fraze precum votam \#EchipaValeriuIftime si indemnul direct la vot pentru echipa PNL, demonstreaza clar intentia de a influenta votul in favoarea candidatilor mentionati.  Aceasta actiune constituie o incalcare a legii, avand un efect electoral direct si vizibil.
    \item publicarea, dupa ora 18:00 pe 30.11.2024, a unei postari pe Facebook (ID \href{https://www.facebook.com/ads/library/?id=925633439092773}{925633439092773}) cu caracter de propaganda electorala, care promoveaza explicit candidatul Valeriu Iftime si Partidul National Liberal (PNL), indemnand cetatenii sa voteze pentru acesta. Postarea, platita cu o suma sub 100 RON, a atins o audienta estimata intre 2.000 si 3.000 de impresii, cu o raza de acoperire de 100.000 - 500.000 de persoane, in judetele Botosani, Iasi si Suceava.  Fraze precum "Avem nevoie de toata mobilizarea pentru alegerile parlamentare," "Doar alegand \#EchipaValeriuIftime, echipa PNL, putem schimba ceva," si "Pozitia 1 pe buletinul de vot" demonstreaza clar intentia de a influenta votul in favoarea PNL. Prezenta numarului CMF 11240002 confirma natura electorala a postarii.  Aceasta actiune constituie o incalcare flagranta a legislatiei electorale, avand in vedere ca perioada de campanie electorala s-a incheiat.  Datele de contact ale advertiserului sunt office@voceadorohoi.ro si +40759112233.
    \item publicarea unei postari pe Facebook (ID: \href{https://www.facebook.com/ads/library/?id=8796733477041239}{8796733477041239}) dupa ora 18:00 pe 30.11.2024, care promoveaza Partidul National Liberal (PNL) si indeamna la votarea pozitiei 4 pe buletinul de vot, avand un efect electoral clar prin promovarea candidatilor PNL Alin Calinescu, Mihai Cotet, Alina Gorghiu si Radu Perianu. Postarea, care contine numarul CMF: 11240002, se incadreaza in definitia materialului de propaganda electorala conform art. 36 (7) din LEGE nr. 334/2006, avand in vedere ca se refera direct la candidati PNL, este utilizata dupa incheierea campaniei electorale, are un obiectiv electoral evident si se adreseaza publicului larg din judetul Arges, cu o cheltuiala de 300-399 RON, ajungand la 60.000-70.000 de impresii, conform datelor din Ad Library.  Contact: senator@alinagorghiu.ro, +40765055855.
\end{enumerate}

\vspace{0.5cm}

\subsection{PRA}
Următoarele fapte contravenționale sunt sesizate împotriva acestei entități:

\begin{enumerate}[leftmargin=*, label=\arabic*.)]
    \item publicarea unei reclame platite pe Facebook dupa ora 18:00 pe 30.11.2024, cu ID-ul postarii \href{https://www.facebook.com/ads/library/?id=1081252570411419}{1081252570411419}, care promoveaza indirect Partidul Romania in Actiune prin asocierea acestuia cu Mircea Geoana si Cristian Lungu.  Postarea, desi nu solicita explicit voturi, prezinta informatii intr-un mod care sugereaza o intentie de a influenta electoratul, avand in vedere contextul, formatul de reclama platita si lipsa analizei critice.  Efectul electoral este potential pozitiv pentru PRA, avand in vedere notorietatea lui Mircea Geoana.  Reclama a avut o estimare a impactului de 2000-3000 de impresii, cu o estimare a acoperirii de 100.000-500.000 de persoane, costul fiind sub 100 RON.  Datele de contact ale advertiserului sunt office@opinii-independente.ro si +40723074107.
    \item publicarea unei reclame platite pe Facebook (ID postare: \href{https://www.facebook.com/ads/library/?id=1244915983757890}{1244915983757890}), dupa ora 18:00 pe 30.11.2024, care il prezinta pe Mircea Geoana sustinand Partidul Romania in Actiune, avand un efect electoral clar de influentare a votului in favoarea partidului respectiv.  Reclama, cu o cheltuiala de 1000-1500 RON, a atins 45.000-50.000 de impresii, vizand o audienta larga din judetul Mehedinti.  Continutul postarii, lipsit de obiectivitate jurnalistica si avand ca scop explicit promovarea unui partid participant la alegerile parlamentare, constituie propaganda electorala interzisa dupa incheierea perioadei legale de campanie.  Datele de contact ale Partidului Romania in Actiune sunt disponibile la adresa https://partidulromaniainactiunemehedinti.info/.
    \item publicarea unei postari platite pe Facebook (ID: \href{https://www.facebook.com/ads/library/?id=1328605544774777}{1328605544774777}) dupa ora 18:00 pe 30.11.2024, care promoveaza Partidul Romania in Actiune Mehedinti si pe Mircea Geoana, avand ca efect electoral influentarea voturilor in favoarea partidului. Postarea, cu o cheltuiala de 700-799 RON, a atins 30.000-35.000 de impresii si o estimare a acoperirii de 100.000-500.000 de persoane in judetul Mehedinti.  Textul postarii Mircea Geoana indeamna romanii sa voteze Partidul Romania in Actiune Mehedinti! si hashtag-urile utilizate (\#mirceageoana \#AlegeriParlamentare) demonstreaza clar intentia de a influenta votul.  Informatiile de contact ale advertiserului sunt disponibile in Ad Library: part9419@partidulromaniainactiunemehedinti.info si +40743302640.
    \item publicarea, dupa ora 18:00 pe 30.11.2024, a unei postari pe Facebook (ID: \href{https://www.facebook.com/ads/library/?id=1356474952001168}{1356474952001168}) cu caracter de propaganda electorala, care promoveaza candidatura domnului Sandor Ovidiu pentru functia de senator. Postarea, care a beneficiat de o investitie financiara de 1000-1500 RON,  si a atins o audienta estimata la 100.000-500.000 de persoane,  contine elemente de propaganda electorala, inclusiv mentionarea sprijinului lui Mircea Geoana,  si depaseste limitele jurnalismului, avand ca obiectiv influentarea votului in favoarea candidatului PRA.  Prezenta codului CMF: 11240040 confirma natura electorala a postarii.  Contact: part9419@partidulromaniainactiunemehedinti.info, +40743302640, https://partidulromaniainactiunemehedinti.info/.
    \item publicarea unei reclame platite pe Facebook (ID postare: \href{https://www.facebook.com/ads/library/?id=1531735944371091}{1531735944371091}) dupa ora 18:00 pe 30.11.2024, care promoveaza Partidul Romania in Actiune Mehedinti si il prezinta pe Mircea Geoana, avand ca efect electoral influentarea votului in favoarea partidului.  Reclama, cu o cheltuiala de 1000-1500 RON, a atins 50.000-60.000 de impresii, conform datelor din Ad Library, si a vizat utilizatorii din Judetul Mehedinti, cu varste cuprinse intre 18 si 65 de ani.  Textul reclamei, Mircea Geoana indeamna romanii sa voteze Partidul Romania in Actiune Mehedinti!, constituie o indemnare directa la vot, incalcand prevederile legale privind propaganda electorala dupa incheierea perioadei legale.  Datele de contact ale advertiserului sunt: part9419@partidulromaniainactiunemehedinti.info si +40743302640.
    \item publicarea unei reclame platite pe Facebook (ID postare: \href{https://www.facebook.com/ads/library/?id=434035689751723}{434035689751723}) dupa ora 18:00 pe 30.11.2024, care indeamna la votarea Partidului Romania in Actiune (pozitia 10 pe buletinul de vot pentru Camera Deputatilor), avand un efect electoral clar prin indemnul direct la vot si utilizarea unor resurse financiare semnificative (RON 1K-RON 1.5K) pentru o audienta estimata la 50K-60K impresii.  Textul reclamei: "Mircea Geoana indeamna romanii sa voteze Partidul Romania in Actiune Mehedinti! Votati pozitia 10 de pe buletinul de vot la Camera Deputatilor!" constituie propaganda electorala interzisa dupa incheierea campaniei electorale, conform articolului 36 (7) din LEGE nr. 334 din 17 iulie 2006, intrucat indeplineste toate conditiile prevazute de lege: se refera direct la un candidat (PRA), este utilizat dupa incheierea campaniei, are obiectiv electoral si se adreseaza publicului larg, depasind limitele activitatii jurnalistice.  Datele de contact ale advertiserului sunt: part9419@partidulromaniainactiunemehedinti.info si +40743302640.
    \item publicarea, dupa ora 18:00 pe 30.11.2024, a unei postari platite pe Facebook (ID \href{https://www.facebook.com/ads/library/?id=2511869432349035}{2511869432349035}), care, desi prezentata ca articol de stiri, constituie propaganda electorala indirecta. Postarea, cu o cheltuiala de sub 100 RON, a atins 2000-3000 de impresii, avand ca tinta un public din judetul Mehedinti, cu varste intre 18 si 65 de ani, si promoveaza Partidul Romania in Actiune prin asocierea acestuia cu Mircea Geoana, prezentandu-l intr-o lumina pozitiva si relatand despre revenirea lui Cristian Lungu in politica.  Efectul electoral este clar, vizand influentarea votului prin asocierea cu o personalitate marcanta, fara a solicita explicit votul, dar creand o imagine favorabila partidului.  Legatura dintre Geoana si PRA, prezentata in articolul de pe opini-independente.ro, este elementul cheie al propagandei.
    \item publicarea unei postari pe Facebook (ID: \href{https://www.facebook.com/ads/library/?id=3263257220484128}{3263257220484128}) dupa ora 18:00 pe 30.11.2024, care promoveaza explicit Partidul Romania in Actiune (PRA), candidata Elena Condei (CMF: 11240040), si indeamna la vot pentru acest partid, folosind fraze precum "Votati Partidul Romania in Actiune!", "Votati pentru integritate!", si "Alegeti pe cei care va asculta si va reprezinta cu adevarat!".  Postarea, cu o cheltuiala de 500-599 RON, a atins 35.000-40.000 de impresii, avand ca efect electoral influentarea votului in favoarea PRA.  Contactul advertiserului este contact@condeielena.ro si +40770745585.  Aceasta postare constituie propaganda electorala interzisa dupa incheierea perioadei de campanie, conform articolului 98 t).
    \item publicarea unei postari platite pe Facebook (ID \href{https://www.facebook.com/ads/library/?id=465465409901525}{465465409901525}) dupa ora 18:00 pe 30.11.2024, care promoveaza indirect Partidul Romania in Actiune prin asocierea pozitiva cu Mircea Geoana si Cristian Lungu, avand ca efect electoral influentarea opiniei publice in favoarea partidului si a candidatilor sai. Postarea, cu o estimare a audientei de 10.000-50.000 de persoane, depaseste limitele jurnalismului obiectiv, prezentand informatii intr-un mod favorabil, fara a include perspective contrare.  Cheltuielile pentru aceasta postare au fost sub 100 RON, conform Ad Library, iar contactul este office@opinii-independente.ro si +40723074107.  Aceasta actiune constituie propaganda electorala interzisa dupa incheierea perioadei legale de campanie.
    \item publicarea unei postari platite pe Facebook (ID \href{https://www.facebook.com/ads/library/?id=499917209723906}{499917209723906}) dupa ora 18:00 pe 30.11.2024, care contine o indemnare explicita la vot din partea lui Mircea Geoana in favoarea Partidului Romania in Actiune.  Postarea, cu o estimare a audientei de peste 1 milion de persoane si un buget de 1000-1500 RON, are ca obiectiv clar influentarea votului in favoarea partidului, indeplinind toate criteriile materialului de propaganda electorala conform articolului 36 (7) din LEGEA nr. 334 din 17 iulie 2006.  Fraza "Mircea Geoana indeamna romanii sa voteze Partidul Romania in Actiune!" constituie o incalcare evidenta a legii, avand in vedere ca este o forma de propaganda electorala dupa incheierea perioadei legale de campanie.  Contact: part9419@partidulromaniainactiunemehedinti.info, +40743302640, https://partidulromaniainactiunemehedinti.info/.
    \item publicarea unei postari platite pe Facebook dupa ora 18:00 pe 30.11.2024, care indeamna la votarea Partidului Romania in Actiune Mehedinti, folosind imaginea lui Mircea Geoana. Postarea, cu ID-ul \href{https://www.facebook.com/ads/library/?id=7967423503360752}{7967423503360752},  indeplineste toate criteriile propagandei electorale conform articolului 36 (7) din LEGE nr. 334 din 17 iulie 2006, avand ca obiectiv influentarea votului printr-o adresare directa catre publicul larg, depasind limitele jurnalismului.  Efectul electoral este clar, vizand cresterea numarului de voturi pentru PRA.  Cheltuielile estimate se situeaza intre 700 si 799 RON, iar postarea a atins intre 30.000 si 35.000 de impresii, cu o estimare a audientei de 100.000 - 500.000 persoane.  Informatiile de contact ale advertiserului sunt disponibile in Ad Library: email part9419@partidulromaniainactiunemehedinti.info, telefon +40743302640, adresa Drobeta-Turnu Severin, Mehedinti, website https://partidulromaniainactiunemehedinti.info/.
\end{enumerate}

\vspace{0.5cm}

\subsection{PSD}
Următoarele fapte contravenționale sunt sesizate împotriva acestei entități:

\begin{enumerate}[leftmargin=*, label=\arabic*.)]
    \item publicarea unei reclame platite pe Facebook dupa ora 18:00 pe 30.11.2024, cu ID-ul postarii \href{https://www.facebook.com/ads/library/?id=1120928716120960}{1120928716120960}, care promoveaza imaginea sa pozitiva ca deputat PSD in judetul Ialomita, avand un buget de sub 100 RON,  ajungand la 25.000-30.000 de impresii, cu o estimare a audientei de 100.000-500.000 de persoane,  si un link catre profilul sau oficial de pe site-ul Camerei Deputatilor,  avand drept efect electoral influentarea perceptiei electoratului asupra candidaturii sale.  Aceasta postare, desi nu solicita explicit voturi, constituie propaganda electorala prin asocierea imaginii sale cu functia sa politica, incalcand prevederile legale privind perioada de campanie electorala.  Contactul reclamantei este dumitrescuralucagiorgiana1@gmail.com si +40727704373.
    \item publicarea unei postari pe Facebook dupa ora 18:00 pe 30.11.2024, cu ID-ul \href{https://www.facebook.com/ads/library/?id=1295415661810762}{1295415661810762}, care promoveaza Partidul Social Democrat (PSD) si indeamna direct la votarea acestuia in alegerile locale. Postarea, cu un buget de 200-299 RON, a atins o audienta estimata la 100.000-500.000 de persoane, avand un efect electoral clar prin indemnul direct la vot pentru PSD.  Fraza Va invit duminica sa ii votati pentru ca votandu-i pe ei, ajutati comunitatile Maramuresului! Duminica votam echilibrat, votam PSD! constituie o incalcare evidenta a legislatiei electorale, fiind o forma de propaganda electorala interzisa dupa incheierea campaniei.  Contactul advertiserului este babici\_ama@yahoo.com si +40747223007.
    \item publicarea unei reclame platite pe Facebook dupa ora 18:00 pe 30.11.2024 (ID postare: \href{https://www.facebook.com/ads/library/?id=1380816302896257}{1380816302896257}), care promoveaza candidata PSD Victoria Stoiciu la functia de senator in Vaslui, cu un efect electoral pozitiv, prin prezentarea acesteia ca invitata noastra speciala si evidentierea rolului sau de consilier de stat si candidata PSD.  Reclama, cu o estimare a audientei de 100.000-500.000 de persoane,  a fost platita cu o suma sub 100 RON, conform datelor din Ad Library, si are ca scop influentarea votului in favoarea candidatei mentionate.  Datele de contact ale advertiserului sunt bogdanbzv@yahoo.ro si +40760716545.
    \item publicarea unei postari platite pe Facebook dupa ora 18:00 pe 30.11.2024 (ID postare: \href{https://www.facebook.com/ads/library/?id=1723851535114433}{1723851535114433}), cu mesajul explicit Te asteptam pe 1 decembrie la vot!,  care indeamna la vot pentru Partidul Social Democrat, avand un efect electoral pozitiv pentru PSD. Postarea, cu o cheltuiala de 900-999 RON, a atins 40.000-45.000 de impresii, vizand populatia din judetul Vrancea, constituie propaganda electorala interzisa dupa incheierea perioadei legale de campanie.  Contact: vrancea@psd.ro, +40749034652, https://psd.ro/.  Aceasta postare incalca prevederile legale privind continuarea propagandei electorale dupa incheierea acesteia, influentand votul alegatorilor.
    \item publicarea unei postari platite pe Facebook (ID: \href{https://www.facebook.com/ads/library/?id=1775434133288283}{1775434133288283}) dupa ora 18:00 pe 30.11.2024, care promoveaza o initiativa a Partidului Social Democrat (PSD) fara neutralitate jurnalistica, avand ca efect electoral influentarea pozitiva a voturilor pentru PSD. Postarea, cu un buget de sub 100 RON, a atins intre 1000 si 2000 de impresii, vizand seniorii din judetul Alba.  Textul postarii, care evidentiaza beneficii pentru pensionari datorita unei masuri guvernamentale PSD, constituie propaganda electorala, incalcand prevederile legale privind perioada de campanie electorala.  Contact: ofice@mediasmartsro.ro, +40752099270, https://www.mediasmartsro.ro/.
    \item publicarea dupa ora 18:00 pe 30.11.2024 a unei postari pe Facebook (ID: \href{https://www.facebook.com/ads/library/?id=1995817070930630}{1995817070930630}), cu un buget de RON 100-199, care promoveaza explicit candidatii PSD Larisa Blanari, Mirela Adomnicai, George Mandruta si Cristian Sologon la alegerile parlamentare din Suceava, avand ca efect electoral influentarea votului in favoarea acestora. Postarea, sub forma unui interviu, nu respecta standardele jurnalismului obiectiv, prezentand doar aspecte pozitive ale candidatilor si ignorand perspective alternative.  Targeting-ul geografic (judetul Suceava) si estimarea de reach (100K-500K) demonstreaza intentia clara de a influenta electoratul.  Continutul postarii indeplineste criteriile de propaganda electorala conform articolului 36 (7) din LEGE nr. 334 din 17 iulie 2006, fiind astfel o incalcare flagranta a legii.
    \item publicarea unei postari pe Facebook (ID: \href{https://www.facebook.com/ads/library/?id=2252869808418948}{2252869808418948}) dupa ora 18:00 pe 30.11.2024, care promoveaza Partidul Social Democrat (PSD) si candidatul sau, Marcel Ciolacu, in mod evident, folosind un limbaj persuasiv si prezentand selectiv informatii pentru a influenta votul alegatorilor. Postarea, platita cu o suma sub 100 RON, a atins o audienta estimata intre 100.000 si 500.000 de persoane, conform datelor din Ad Library.  Continutul postarii, care prezinta realizarile PSD in contrast cu esecurile percepute ale altor partide (USR si PNL), are un efect electoral clar, avand ca obiectiv influentarea votului in favoarea PSD.  Mentionarea explicita a PSD si a lui Marcel Ciolacu, combinata cu indemnul direct la vot ("Pe 1 Decembrie, voteaza PSD!"), demonstreaza intentia de a influenta alegerile parlamentare.  Desi postarea contine unele informatii factuale, prezentarea lor este lipsita de obiectivitate jurnalistica, fiind evident conceputa pentru a manipula alegatorii.  Contactul advertiserului este laurentiumarin\_vn@yahoo.com si +40744666633.
    \item publicarea unei postari platite pe Facebook (ID postare: \href{https://www.facebook.com/ads/library/?id=2259221664463522}{2259221664463522}) dupa ora 18:00 pe 30.11.2024, care indeamna la vot pentru candidatii Partidului Social Democrat (PSD) la Senat si Camera Deputatilor. Postarea, cu un buget de 100-199 RON, a atins 10.000-15.000 de impresii si o estimare a audientei de 100.000-500.000 de persoane, contine un apel explicit la vot pentru PSD ("Va indemn sa iesiti la vot si sa sustineti candidatii Partidului Social Democrat"), indeplinind toate criteriile propagandei electorale conform articolului 36 (7) din LEGE nr. 334 din 17 iulie 2006. Prezenta numarului CMF 11240017 confirma natura electorala a postarii.  Efectul electoral este clar, avand ca obiectiv influentarea votului in favoarea PSD.  Informatiile de contact ale advertiserului sunt verginelgireada@gmail.com si +40755272595.
    \item publicarea, dupa ora 18:00 pe 30.11.2024, a unei postari pe Facebook (ID \href{https://www.facebook.com/ads/library/?id=2948803988603171}{2948803988603171}) cu continut de propaganda electorala, care promoveaza candidatura Danielei Mihai pentru functia de senator din partea PSD. Postarea, cu un buget de 1000-1500 RON, a atins 150.000-175.000 de impresii si contine elemente specifice de propaganda electorala, cum ar fi imaginea candidatei, sloganuri precum "Voteaza Calea Sigura pentru Calarasi", si numarul CMF 11240017. Obiectivul evident al postarii este influentarea votului in favoarea Danielei Mihai, incalcand astfel prevederile legale privind continuarea propagandei electorale dupa incheierea perioadei legale.  Contact: infopunkt@infopunkt.ro, +40727256475, Strada Bucuresti 254, Calarasi 910058, RO.
    \item publicarea, dupa ora 18:00 pe 30.11.2024, a unei postari pe Facebook (ID \href{https://www.facebook.com/ads/library/?id=460612109841865}{460612109841865}) cu caracter de propaganda electorala, care promoveaza candidatura lui ION SAMOILA pentru PSD, folosind un apel explicit la vot ("VOTEAZA CALEA SIGURA PENTRU CALARASI"), imaginea candidatului si sigla partidului, precum si hashtag-uri care sustin campania (\#CaleaSiguraPentruCalarasi \#CaleaSiguraPentruRomania). Postarea, platita cu o suma cuprinsa intre 100 si 199 RON, a avut o vizibilitate estimata intre 15.000 si 20.000 de impresii, cu o estimare a audientei de 100.000 - 500.000 de persoane in judetul Calarasi. Prezenta numarului CMF 11240017 confirma natura materialului de campanie. Obiectivul evident al postarii este influentarea votului in favoarea lui Ion Samoila, reprezentand o incalcare clara a legislatiei electorale.
    \item publicarea unei postari pe Facebook (ID \href{https://www.facebook.com/ads/library/?id=465702052796693}{465702052796693}) dupa ora 18:00 pe 30.11.2024, care constituie propaganda electorala interzisa. Postarea, cu un buget de 1000-1500 RON, a atins 70.000-80.000 de impresii si o estimare de 100.000-500.000 de persoane, promovand explicit candidatul Gheorghe Simon (PSD) pentru alegerile parlamentare, indicand pozitia sa pe buletinul de vot (\#2) si utilizand un limbaj emotional pentru a influenta votul.  Aceasta actiune contravine prevederilor legale privind incetarea campaniei electorale si interzicerea propagandei electorale dupa incheierea acesteia.  Contactul este primariaviseu@yahoo.com si +40758142897.
    \item publicarea unei postari pe Facebook (ID \href{https://www.facebook.com/ads/library/?id=489518217471492}{489518217471492}) dupa ora 18:00 pe 30.11.2024, care constituie propaganda electorala pentru Partidul Social Democrat (PSD), avand ca efect electoral influentarea votului in favoarea PSD. Postarea, cu un buget de sub 100 RON, a atins intre 1000 si 2000 de impresii, avand o estimare a acoperirii de 50.000 - 100.000 de persoane.  Textul postarii contine indemnuri directe la vot ("Voteaza pe 1 Decembrie"), asociaza PSD cu stabilitate si dezvoltare, si prezinta imagini cu logo-ul PSD si pozitia pe buletinul de vot, toate acestea constituind elemente de propaganda electorala interzisa dupa incheierea perioadei legale de campanie.  Contactul advertiserului este aditarca@gmail.com si +40723209623.
    \item publicarea unei postari platite pe Facebook (ID \href{https://www.facebook.com/ads/library/?id=515979414777457}{515979414777457}) dupa ora 18:00 pe 30.11.2024, care promoveaza candidatii Partidului Social Democrat (PSD) la alegerile parlamentare, folosind un limbaj persuasiv si indemnuri la vot. Postarea, cu o cheltuiala de 400-499 RON, a atins 35.000-40.000 de impresii, vizand judetul Bacau.  Efectul electoral este clar, avand ca obiectiv influentarea votului in favoarea PSD. Fraze precum Votati Responsabil la Parlamentare, Trebuie sa iesim la vot si sa votam echilibrul si pe cei care au dovedit... candidatii echipei social-democrate si hashtag-urile \#EchipaPSD si \#PentruBacau demonstreaza intentia de a influenta votul.  Aceasta postare constituie propaganda electorala dupa incheierea campaniei, fiind astfel ilegala.  Datele de contact ale reclamantei sunt cristina.pravat@outlook.com si +40735598429.
    \item publicarea, dupa ora 18:00 pe 30.11.2024, a unei postari pe Facebook (ID \href{https://www.facebook.com/ads/library/?id=607073918491384}{607073918491384}) cu caracter de propaganda electorala, care promoveaza candidatura Larisei Blanari, candidat PSD la Camera Deputatilor,  prin prezentarea opiniilor sale si accentuarea angajamentului sau fata de alegatori.  Postarea, platita cu o suma sub 100 RON, a atins o audienta estimata intre 2000 si 3000 de impresii, cu o estimare a raspandirii intre 100.000 si 500.000 de persoane, in principal in judetul Suceava.  Textul postarii, in special fraza "Eu nu sunt de vanzare", are un efect electoral clar, vizand influentarea votului in favoarea candidatei.  Lipsa obiectivitatii jurnalistice si contextul publicitar platit demonstreaza intentia de a influenta procesul electoral.  Contactul advertiserului este anunturi@informatiata.ro si +40745691121.
    \item publicarea unei reclame platite pe Facebook (ID postare: \href{https://www.facebook.com/ads/library/?id=789982992533253}{789982992533253}), dupa ora 18:00 pe 30.11.2024, care promoveaza imaginea si profilul sau ca senator PSD din judetul Ilfov, avand un efect electoral pozitiv prin cresterea vizibilitatii si a credibilitatii sale in randul alegatorilor. Reclama, cu o estimare a acoperirii de peste 1 milion de persoane si un buget de 1000-1500 RON, indeplineste toate criteriile materialului de propaganda electorala conform articolului 36 (7) din LEGEA nr. 334 din 17 iulie 2006, referindu-se direct la un candidat clar identificat, fiind utilizata in perioada post-campanie electorala, avand un obiectiv electoral evident si adresandu-se publicului larg, depasind limitele activitatii jurnalistice.  Persoana responsabila pentru aceasta incalcare este Mihai Alfred Laurentiu, cu datele de contact disponibile in arhiva Facebook Ads Library: mihai.alfred@yahoo.com si +40720220010.
    \item publicarea unei postari pe Facebook (ID: \href{https://www.facebook.com/ads/library/?id=930934385156486}{930934385156486}) dupa ora 18:00 pe 30.11.2024, care promoveaza candidatul Vasile Roman (PSD) la alegerile parlamentare,  cu un buget de sub 100 RON,  avand ca efect electoral cresterea numarului de voturi pentru acesta. Postarea, adresata publicului larg prin intermediul unei reclame platite pe Facebook, contine o solicitare explicita de vot pentru Vasile Roman, precizand pozitia sa pe buletinul de vot (pozitia 2) si include numarul CMF 11240017, indicand clar natura sa de propaganda electorala.  Aceasta postare, avand in vedere continutul sau si contextul post-electoral, constituie o incalcare clara a legislatiei.  Datele de contact ale advertiserului sunt: romanvasile10@yahoo.com si +40754535452.  Postarea a atins intre 4000 si 5000 de impresii, cu o estimare a audientei de 10.000 - 50.000 de persoane.
\end{enumerate}

\vspace{0.5cm}

\subsection{Quantum TV}
Următoarele fapte contravenționale sunt sesizate împotriva acestei entități:

\begin{enumerate}[leftmargin=*, label=\arabic*.)]
    \item publicarea unei postari platite pe Facebook (ID \href{https://www.facebook.com/ads/library/?id=618360727180200}{618360727180200}), dupa ora 18:00 pe 30.11.2024, care, desi prezentata sub forma unei stiri, contine informatii tendentioase ce favorizeaza PSD si denigreaza USR. Postarea mentioneaza actiunile lui Laurentiu Nistor, presedintele PSD Hunedoara, si ale unui ministru USR (neidentificat nominal), prezentand actiunile PSD intr-o lumina pozitiva si pe cele ale USR intr-o lumina negativa, cu scopul evident de a influenta votul alegatorilor.  Efectul electoral al acestei postari este clar: influentarea opiniei publice in favoarea PSD si impotriva USR.  Continutul manipulativ, distribuirea pe o platforma publica precum Facebook si plata pentru promovare (peste 100 RON, conform Ad Library) demonstreaza intentia de a influenta alegerile.  Desi nu se solicita direct votul pentru un anumit candidat, impactul asupra alegerilor este evident.
\end{enumerate}

\vspace{0.5cm}

\subsection{REPER}
Următoarele fapte contravenționale sunt sesizate împotriva acestei entități:

\begin{enumerate}[leftmargin=*, label=\arabic*.)]
    \item publicarea unei reclame platite pe Facebook (ID postare: \href{https://www.facebook.com/ads/library/?id=1258668132017843}{1258668132017843}) dupa ora 18:00 pe 30.11.2024, cu mesajul Dambovita voteaza REPER pe 1 decembrie!, care promoveaza in mod direct candidatii Moleavin Adrian si Dogaru Georgeta pentru Senat, din partea partidului REPER, avand ca efect electoral influentarea votului in favoarea acestora.  Reclama, care a avut o estimare a acoperirii de 100.000-500.000 de persoane,  contine un numar CMF (11240060), specific materialelor de campanie electorala, si a fost platita cu o suma sub 100 RON.  Informatiile de contact ale advertiserului sunt: steluta.munteanu@partidulreper.ro si +40720412010.  Aceasta actiune constituie propaganda electorala interzisa dupa incheierea perioadei legale de campanie, conform legislatiei in vigoare.
    \item publicarea dupa ora 18:00 pe 30.11.2024 a unei reclame platite pe Facebook (ID postare: \href{https://www.facebook.com/ads/library/?id=1260800031733382}{1260800031733382}), cu mesajul "Georgeta Dogaru - candidata la Senat pentru Dambovita din partea partidului REPER. Votati REPER pe 1 decembrie in Dambovita!",  avand ca efect electoral promovarea directa a candidatei Georgeta Dogaru si a partidului REPER, influentand votul alegatorilor din judetul Dambovita.  Reclama, cu o cheltuiala de sub 100 RON, a vizat o audienta estimata intre 50.000 si 100.000 de persoane, in principal femei cu varste intre 38 si 65 de ani din judetul Dambovita, conform datelor din Ad Library.  Contact: steluta.munteanu@partidulreper.ro, +40720412010.  Mesajul, lipsit de caracter jurnalistic, constituie propaganda electorala clara, interzisa dupa incheierea campaniei electorale.
    \item publicarea unei postari pe Facebook (ID: \href{https://www.facebook.com/ads/library/?id=1271996963928121}{1271996963928121}) dupa ora 18:00 pe 30.11.2024, care promoveaza candidatii REPER Gheorghina Marian si Florin Barbur, indemnand alegatorii sa voteze pozitia 10 pe buletinul de vot. Postarea, platita intre 100 si 199 RON, a generat intre 6000 si 7000 de impresii si a vizat o audienta de 100.000 - 500.000 de persoane, avand un efect electoral clar prin influentarea votului in favoarea candidatilor REPER. Prezenta numarului CMF 11240060 confirma natura electorala a postarii, care depaseste limitele jurnalismului prin indemnul direct la vot.  Contact: andrei@atelieru.ro, +40755080070, Strada Garlei nr. 176, sector 1, Bucuresti.
    \item publicarea unei postari platite pe Facebook (ID: \href{https://www.facebook.com/ads/library/?id=1289488752083021}{1289488752083021}) dupa ora 18:00 pe 30.11.2024, care promoveaza Partidul REPER pentru alegerile parlamentare din 1 Decembrie 2024, indemnand direct alegatorii sa voteze pozitia 10 pe buletinul de vot. Postarea, cu un buget de 300-399 RON, a atins 25.000-30.000 de impresii, avand ca efect electoral influentarea votului in favoarea Partidului REPER.  Textul postarii ("Voteaza pozitia 10", "SUSTINE PARTIDUL REPER IN PARLAMENT") si imaginea atasata constituie dovezi clare ale intentiei de a influenta votul, incalcand prevederile legale privind propaganda electorala dupa incheierea perioadei legale. Contact: andrei@atelieru.ro, +40755080070, https://atelieru.ro/.
    \item publicarea unei reclame electorale platite pe Facebook (ID postare: \href{https://www.facebook.com/ads/library/?id=1304837790688838}{1304837790688838}) dupa ora 18:00 pe 30.11.2024, care promoveaza candidatul IOSIFESCU VICTOR-ION pentru Camera Deputatilor in judetul Dambovita, utilizand imaginea acestuia si indemnul explicit Voteaza reper, avand ca efect electoral influentarea votului in favoarea candidatului si partidului REPER.  Reclama, cu un buget de sub 100 RON, a atins intre 1000 si 2000 de impresii, avand o estimare a acoperirii de 50.000-100.000 de persoane, conform datelor din Ad Library.  Informatiile de contact ale advertiserului sunt: steluta.munteanu@partidulreper.ro si +40720412010.  Aceasta actiune constituie propaganda electorala interzisa dupa incheierea campaniei electorale, conform legislatiei in vigoare.
    \item publicarea, dupa ora 18:00 pe 30.11.2024, a unei postari pe Facebook (ID \href{https://www.facebook.com/ads/library/?id=2987748928057189}{2987748928057189}) cu scop electoral, care promoveaza candidatul Florin Barbur din partea partidului REPER pentru alegerile parlamentare din Maramures. Postarea, care a beneficiat de o investitie financiara intre 200 si 299 RON, contine imaginea candidatului, indemnuri explicite la vot ("Voteaza pozitia 10 pe buletinul de vot!", "Te indemn sa votezi cu incredere partidul REPER"), si numarul CMF 11240060, toate acestea avand ca efect influentarea votului in favoarea candidatului mentionat.  Postarea a avut o estimare a vizualizarilor intre 10.000 si 15.000, ajungand la o audienta estimata intre 100.000 si 500.000 de persoane.  Contactul Atelierul de internet SRL poate fi realizat la adresa de email andrei@atelieru.ro sau la numarul de telefon +40755080070.
    \item publicarea, dupa ora 18:00 pe 30.11.2024, a unei postari pe Facebook (ID \href{https://www.facebook.com/ads/library/?id=456549750804666}{456549750804666}) cu caracter de propaganda electorala, care promoveaza candidatul Florin Barbur pentru Camera Deputatilor din partea partidului REPER. Postarea, care a beneficiat de o investitie financiara intre 100 si 199 RON,  contine un mesaj explicit de sustinere a candidatului, inclusiv un apel direct la vot ("Voteaza Partidul reper") si mentionarea datei alegerilor ("1 Decembrie"), avand ca efect electoral influentarea votului in favoarea lui Florin Barbur si a partidului REPER.  Prezenta codului CMF 11240060 confirma natura electorala a postarii.  Postarea a atins o audienta estimata intre 15.000 si 20.000 de impresii, cu o estimare a potentialului de acoperire intre 100.000 si 500.000 de persoane.  Datele de contact ale Atelierul de internet SRL sunt disponibile in Ad Library: andrei@atelieru.ro si +40755080070.
    \item publicarea unei reclame platite pe Facebook (ID postare: \href{https://www.facebook.com/ads/library/?id=486555060504441}{486555060504441}), dupa ora 18:00 pe 30.11.2024, cu mesajul "Dambovita voteaza REPER pe 1 decembrie!",  avand ca efect electoral influentarea votului in favoarea partidului REPER.  Reclama, desi nu contine un numar CMF, indeplineste toate criteriile propagandei electorale: se refera direct la un candidat (REPER), se adreseaza publicului larg din judetul Dambovita, are un obiectiv electoral clar (influentarea votului) si depaseste limitele activitatii jurnalistice.  Cheltuielile pentru reclama au fost sub 100 RON, conform Ad Library,  ajungand la o estimare a audientei de 100.000-500.000 de persoane.  Datele de contact ale advertiserului sunt disponibile in Ad Library: steluta.munteanu@partidulreper.ro si +40720412010.
    \item publicarea unei reclame platite pe Facebook (ID postare: \href{https://www.facebook.com/ads/library/?id=559325903408021}{559325903408021}) dupa ora 18:00 pe 30.11.2024, care promoveaza candidatura lui Stefan Cimpean la alegerile parlamentare. Reclama, cu o cheltuiala estimata intre 200 si 299 RON,  a vizat o audienta de 40.000-45.000 de impresii si o estimare a acoperirii de 50.000-100.000 de persoane, in principal in judetul Suceava,  tinand cont de mentionarea explicita a candidaturii domnului Cimpean si a hashtag-urilor legate de alegeri (\#Alegeri2024, \#AlegeriParlamentare), are ca efect electoral clar promovarea candidaturii acestuia si influentarea votului.  Contactul advertiserului este stefan.cimpan@partidulreper.ro si +40731090557.
    \item publicarea unei reclame platite pe Facebook dupa ora 18:00 pe 30.11.2024 (ID postare: \href{https://www.facebook.com/ads/library/?id=584385034149116}{584385034149116}), care promoveaza direct candidatul Adrian Moleavin si Partidul REPER la alegerile parlamentare din 1 decembrie, cu fraza explicita Votati REPER pe 1 decembrie, avand un efect electoral clar pozitiv pentru REPER.  Reclama, desi abordeaza dezvoltarea turismului, o face intr-un mod evident propagandistic, lipsit de obiectivitate jurnalistica, si este directionata catre un public tinta specific (femei cu varste intre 30 si 63 de ani din judetul Dambovita), conform datelor din Ad Library.  Cheltuielile pentru reclama sunt sub 100 RON, conform Ad Library, iar postarea a atins o estimare de 50.000-100.000 de persoane.  Informatiile de contact ale advertiserului sunt disponibile in Ad Library: steluta.munteanu@partidulreper.ro si +40720412010.
    \item publicarea, dupa ora 18:00 pe 30.11.2024, a unei reclame platite pe Facebook (ID postare: \href{https://www.facebook.com/ads/library/?id=862108969407622}{862108969407622}), care promoveaza candidatura lui FLORIN BARBUR pentru Camera Deputatilor din partea partidului REPER, prezentand imaginea acestuia si indemnand la vot cu fraze precum Voteaza cu incredere! Voteaza informat! Voteaza responsabil!, avand ca efect electoral influentarea pozitiva a voturilor pentru candidatul mentionat.  Reclama, care a avut o estimare a vizualizarilor intre 8.000 si 9.000 si o estimare a audientei de 100.000 - 500.000 de persoane, contine numarul CMF 11240060, specific materialelor electorale, si a costat intre 100 si 199 RON.  Aceasta actiune constituie propaganda electorala dupa incheierea perioadei legale, fiind interzisa conform articolului 98 t).  Datele de contact ale Atelierul de internet SRL sunt: andrei@atelieru.ro si +40755080070.
    \item publicarea unei postari pe Facebook (ID: \href{https://www.facebook.com/ads/library/?id=927513996021099}{927513996021099}) dupa ora 18:00 pe 30.11.2024, care promoveaza candidatul Florin Barbur pentru Camera Deputatilor din partea partidului REPER (pozitia 10 pe buletinul de vot),  prin utilizarea unui limbaj persuasiv si a unor afirmatii care au ca scop influentarea votului. Postarea, care a avut intre 10.000 si 15.000 de impresii si o estimare a acoperirii de 100.000-500.000 de persoane, contine un cod CMF (11240060), specific materialelor electorale, si indeamna direct la votarea candidatului.  Cheltuielile pentru aceasta postare s-au situat intre 100 si 199 RON.  Aceste actiuni constituie propaganda electorala dupa incheierea perioadei legale, incalcand prevederile legale.  Datele de contact ale Atelierul de internet SRL sunt: andrei@atelieru.ro si +40755080070, adresa: Strada Garlei nr. 176, sector 1, Bucuresti.
    \item publicarea unei postari pe Facebook (ID: \href{https://www.facebook.com/ads/library/?id=932185282168985}{932185282168985}) dupa ora 18:00 pe 30.11.2024, care constituie propaganda electorala, avand ca efect electoral influentarea votului in favoarea partidului REPER. Postarea, cu un buget de sub 100 RON,  contine indemnul explicit Voteaza cu incredere, nu cu frica!,  precum si pozitia partidului pe buletinul de vot (\#19),  adresandu-se unui public larg prin intermediul unei campanii platite pe Facebook,  ajungand la 4.000-5.000 de impresii si avand o estimare a acoperirii de 100.000-500.000 de persoane.  Prezenta numarului CMF11240060 confirma natura oficiala a materialului de propaganda electorala.  Aceste elemente demonstreaza clar intentia de a influenta votul in favoarea REPER, incalcand prevederile legale privind continuarea propagandei electorale dupa incheierea perioadei legale. Contact: campaniereper@gmail.com, +40757521109, Calea Grivitei 73-75, Sector 1, Bucuresti, Romania 010734, RO.
\end{enumerate}

\vspace{0.5cm}

\subsection{RMDSZ}
Următoarele fapte contravenționale sunt sesizate împotriva acestei entități:

\begin{enumerate}[leftmargin=*, label=\arabic*.)]
    \item publicarea unei postari pe Facebook (ID: \href{https://www.facebook.com/ads/library/?id=1090345325891072}{1090345325891072}), dupa ora 18:00 pe 30.11.2024, care constituie propaganda electorala, avand in vedere continutul sau explicit de indemn la vot pentru RMDSZ, utilizarea unui limbaj emotional ("Veszelyben vagyunk!") si un apel direct la actiune ("SZAVAZZ TE IS!").  Postarea, platita cu o suma sub 100 RON, a atins o audienta estimata intre 100.000 si 500.000 de persoane, in principal din judetele cu populatie maghiara semnificativa, conform datelor din Ad Library. Obiectivul electoral este evident, vizand influentarea votului in favoarea RMDSZ.  Contact: kommunikacio@rmdsz.ro, +40742125468.
    \item publicarea unei postari platite pe Facebook (ID postare: \href{https://www.facebook.com/ads/library/?id=1112562726903906}{1112562726903906}), dupa ora 18:00 pe 30.11.2024, care indeamna la vot intr-un mod persuasiv, vizand in mod specific comunitatea maghiara din Romania.  Postarea, desi nu mentioneaza candidati anumiti, are un efect electoral clar, avand ca obiectiv mobilizarea electoratului maghiar. Fraze precum Azert szavazok, mert rajtam mulik, hogy lesz-e a magyaroknak kepviselete (I vote because it depends on me whether Hungarians will have representation) demonstreaza intentia de a influenta votul.  Chemarea la actiune, combinata cu bugetul alocat (RON 100-199) si numarul mare de impresii (10K-15K), confirma natura propagandistica a postarii, incalcand astfel prevederile legale.  Contact: kommunikacio@rmdsz.ro, +40742125468.
    \item publicarea unei postari platite pe Facebook (ID postare: \href{https://www.facebook.com/ads/library/?id=1120849286226912}{1120849286226912}), dupa ora 18:00 pe 30.11.2024, care indeamna la vot comunitatea maghiara din Romania, accentuand amenintari percepute la adresa intereselor lor.  Postarea, desi nu mentioneaza candidati specifici, are un efect electoral clar, vizand mobilizarea unui bloc electoral specific.  Tintarea geografica a judetelor Covasna, Harghita si Mures, cu populatie maghiara semnificativa, confirma intentia de a influenta alegerile.  Textul postarii, cu fraze precum "Veszelyben vagyunk!", "Rajtad mulik" si "nem veszithetnek",  constituie o forma de propaganda electorala, indeplinind criteriile articolului 36 (7) din LEGEA nr. 334 din 17 iulie 2006.  Cheltuielile pentru aceasta postare, conform Ad Library, sunt sub 100 RON, dar impactul potential asupra alegerilor este semnificativ, avand in vedere numarul de impresii (6K-7K) si estimarea de reach (100K-500K).  Contactul RMDSZ este disponibil la kommunikacio@rmdsz.ro si +40742125468.
    \item publicarea unei reclame platite pe Facebook dupa ora 18:00 pe 30.11.2024, cu ID-ul postarii \href{https://www.facebook.com/ads/library/?id=1133970618342201}{1133970618342201}, care, desi nu mentioneaza explicit un candidat, constituie propaganda electorala prin apelul emotional la comunitatea maghiara din Romania, indemnandu-i sa voteze pentru a-si proteja interesele.  Reclama, cu o cheltuiala de sub 100 RON, a atins intre 2000 si 3000 de impresii si o estimare a audientei de 100.000 - 500.000 de persoane, avand ca efect electoral mobilizarea electoratului maghiar.  Textul reclamei, "Veszelyben vagyunk! (...) Rajtad mulik (...) Csak egyutt tudjuk megvedeni...",  constituie un apel direct la actiune, influentand votul fara a respecta neutralitatea necesara.  Contactul este biharmegye@rmdsz.ro si +40771314601.
    \item publicarea unei postari platite pe Facebook (ID postare: \href{https://www.facebook.com/ads/library/?id=1589527065267903}{1589527065267903}) dupa ora 18:00 pe 30.11.2024, care, desi nu mentioneaza explicit alegerile parlamentare curente, indeamna la vot intr-un mod care favorizeaza implicit RMDSZ, avand in vedere afilierea politica a Biro Rozalia, persoana mentionata in postare.  Mesajul Sa ne adunam duminica! Spuneti prietenilor si cunoscutilor si cu minte treaza, cu inima maghiara, haideti sa mergem sa votam! are un efect electoral clar, vizand mobilizarea electoratului maghiar.  Chemarea la vot, combinata cu parametrii de targetare si afilierea politica a vorbitorului, sugereaza o incercare de a influenta participarea la vot in favoarea RMDSZ.  Suma cheltuita pentru aceasta postare este estimata la 2000-2500 RON, ajungand la 60.000-70.000 de impresii, conform datelor din Ad Library.  Datele de contact ale advertiserului sunt disponibile in Ad Library: office@heusmedia.ro si +40771314601.
    \item publicarea, dupa ora 18:00 pe 30.11.2024, a unei postari pe Facebook (ID: \href{https://www.facebook.com/ads/library/?id=2632021697008160}{2632021697008160}) cu continut electoral, care promoveaza candidatura lui PETRU FARAGO pentru Camera Deputatilor, specificand pozitia sa pe buletinul de vot (Pozitia 1), avand ca efect electoral cresterea numarului de voturi pentru acesta. Postarea, care a beneficiat de o cheltuiala de publicitate de sub 100 RON, a atins intre 2000 si 3000 de impresii, cu o estimare a acoperirii de 10.000 - 50.000 de persoane, in judetul Arad, conform datelor din Facebook Ad Library.  Prezenta codului CMF (COD AEP: 13240004) confirma natura electorala a postarii.  Indiferent de data de incheiere a campaniei electorale, continutul postarii constituie propaganda electorala, conform articolului 36 (7) din LEGE nr. 334 din 17 iulie 2006, intrucat se refera direct la un candidat, este utilizata in perioada campaniei electorale, are un obiectiv electoral clar si se adreseaza publicului larg.  Datele de contact ale advertiserului sunt disponibile in Facebook Ad Library: faragopeter@rmdsz.ro si +40728968056.
    \item publicarea unei reclame platite pe Facebook (ID postare: \href{https://www.facebook.com/ads/library/?id=2723665604473614}{2723665604473614}), dupa ora 18:00 pe 30.11.2024, cu mesajul "Alin Nicusca: Kozossegunk jovoje forog kockan. Vasarnap szavazzunk, mert kesobb nem tudjuk befolyasolni a donteseket!", care indeamna la vot pentru un candidat neidentificat explicit, dar puternic sugerat prin contextul regional si temporal, avand un efect electoral clar pozitiv asupra lui Alin Nicusca.  Reclama, vizand judetele Covasna, Harghita si Mures, a atins o audienta estimata intre 100.000 si 500.000 de persoane, cu cheltuieli sub 100 RON, conform datelor din Ad Library.  Aceasta actiune constituie propaganda electorala dupa incheierea perioadei legale, incalcand prevederile legale.  Datele de contact ale RMDSZ sunt disponibile pe pagina lor de Facebook si includ adresa de email kommunikacio@rmdsz.ro si numarul de telefon +40742125468.
    \item publicarea unei postari platite pe Facebook (ID postare: \href{https://www.facebook.com/ads/library/?id=431026766749165}{431026766749165}), dupa ora 18:00 pe 30.11.2024, care constituie propaganda electorala prin indemnul explicit la vot adresat comunitatii maghiare din judetul Bihor, fara a mentiona candidati specifici, dar cu scopul clar de a influenta rezultatul alegerilor parlamentare.  Postarea, cu o cheltuiala de sub 100 RON, a atins intre 2000 si 3000 de impresii si o estimare a audientei de 50.000-100.000 persoane, utilizand un limbaj persuasiv care subliniaza consecintele negative ale neparticiparii la vot.  Aceasta actiune se incadreaza in definitia propagandei electorale conform articolului 36 (7) din LEGEA nr. 334 din 17 iulie 2006, avand ca efect electoral cresterea participarii la vot in randul unui anumit segment al populatiei.  Informatiile de contact ale RMDSZ Bihar Megye sunt disponibile la biharmegye@rmdsz.ro si +40771314601.
    \item publicarea unei postari platite pe Facebook cu ID-ul \href{https://www.facebook.com/ads/library/?id=467703365924245}{467703365924245}, dupa ora 18:00 pe 30.11.2024, care contine mesajul "VASARNAP SZAVAZOK!",  incurajand votul pentru RMDSZ si subliniind importanta participarii electorale pentru comunitatea maghiara din Romania.  Efectul electoral este clar, avand ca obiectiv influentarea votului in favoarea RMDSZ.  Postarea, desi nu mentioneaza candidati individuali, are un caracter propagandistic evident, vizand o audienta larga prin intermediul unei campanii platite (RON100-RON199), cu o estimare a impactului de peste 1 milion de persoane, tinand cont de localizarea geografica specifica.  Textul insotitor accentueaza riscurile pentru comunitatea maghiara si necesitatea votului pentru a asigura reprezentarea acesteia.
    \item publicarea unei postari platite pe Facebook (ID \href{https://www.facebook.com/ads/library/?id=486037067224641}{486037067224641}), dupa ora 18:00 pe 30.11.2024, care indeamna la vot fara a specifica un candidat, dar in contextul campaniei electorale pentru alegerile parlamentare, avand un efect electoral pozitiv pentru RMDSZ. Postarea, cu o cheltuiala de 600-699 RON, a vizat o audienta de 100.000-500.000 persoane in judetele Covasna, Harghita si Mures, utilizand imaginea lui Bagossy Norbi pentru a influenta electoratul maghiar.  Aceasta actiune constituie propaganda electorala dupa incheierea perioadei legale, incalcand prevederile legale.  Contact: kommunikacio@rmdsz.ro, +40742125468.
    \item publicarea unei reclame politice platite pe Facebook cu mesajul "Holnap nem veszithetunk", difuzata dupa ora 18:00 pe 30.11.2024, cu scopul de a influenta votul in favoarea RMDSZ in alegerile parlamentare.  Mesajul, desi nu mentioneaza un candidat specific, creeaza o urgenta si un apel implicit la actiune, vizand o audienta larga in judetul Mures, conform datelor de targetare.  Reclama, cu un buget de 100-199 RON, a generat 25.000-30.000 de impresii, avand potentialul de a influenta semnificativ rezultatul alegerilor.  Contact: rmdszms@rmdsz.ro, +40754205269, https://www.rmdsz.ro/. ID-ul postarii pe Facebook este \href{https://www.facebook.com/ads/library/?id=503980579343231}{503980579343231}.
    \item publicarea unei postari platite pe Facebook (ID: \href{https://www.facebook.com/ads/library/?id=583730777346361}{583730777346361}), dupa ora 18:00 pe 30.11.2024, cu mesajul "December 1-jen a magyarok nem veszithetnek", care indeamna la vot pentru partidul RMDSZ, avand un efect electoral clar pozitiv pentru acest partid, prin mobilizarea electoratului maghiar din Romania. Postarea, vizibila pe Facebook, a avut o acoperire extinsa, conform datelor furnizate, in mai multe judete din Romania, demonstrand intentia de a influenta rezultatul alegerilor parlamentare.  Aceasta postare, desi nu mentioneaza candidati specifici, constituie propaganda electorala conform articolului 36 (7) din LEGEA nr. 334 din 17 iulie 2006, intrucat se refera direct la un partid politic (RMDSZ), este utilizata in perioada post-campanie electorala (dupa incheierea acesteia), are un obiectiv electoral clar si se adreseaza publicului larg. Suma cheltuita pentru aceasta postare este sub 100 RON.
\end{enumerate}

\vspace{0.5cm}

\subsection{Radio HIT FM Alba}
Următoarele fapte contravenționale sunt sesizate împotriva acestei entități:

\begin{enumerate}[leftmargin=*, label=\arabic*.)]
    \item promovarea platita pe Facebook a unui articol care prezinta intr-o lumina favorabila candidatul la Senat Corneliu Muresan, contrastandu-l cu alti candidati, dupa ora 18:00 pe 30.11.2024.  Postarea, cu ID-ul \href{https://www.facebook.com/ads/library/?id=1103675518078138}{1103675518078138},  a fost vizualizata de 6.000-7.000 de persoane, avand o estimare a acoperirii de 50.000-100.000,  si a costat mai putin de 100 RON.  Efectul electoral este clar, avand ca obiectiv influentarea votului in favoarea lui Corneliu Muresan prin sublinierea legaturii sale cu judetul Alba.  Utilizarea unei platforme publice precum Facebook, combinata cu plata pentru promovare, demonstreaza intentia de a influenta alegerile.  Textul postarii, care mentioneaza explicit candidatul si il prezinta intr-o maniera pozitiva, in contextul apropiat de incheierea campaniei electorale, constituie o incalcare clara a legislatiei.
\end{enumerate}

\vspace{0.5cm}

\subsection{SC INDISCRET MEDIA SRL}
Următoarele fapte contravenționale sunt sesizate împotriva acestei entități:

\begin{enumerate}[leftmargin=*, label=\arabic*.)]
    \item publicarea pe Facebook, dupa ora 18:00 pe 30.11.2024, a unei postari platite (ID \href{https://www.facebook.com/ads/library/?id=1589127498477281}{1589127498477281}) care, desi nu sustine explicit un candidat, prezinta rezultatele electorale ale lui Calin Georgescu intr-un cadru tendentios, sugerand manipulare sau vot de protest prin intrebari capcana precum Asistam la o manipulare de proportii sau voturile primite de Calin Georgescu sunt doar o reactie a unei parti a electoratului care si-a exprimat astfel nemultumirea fata de clasa politica?!.  Acest lucru are un efect electoral negativ asupra lui Calin Georgescu, influentand indirect opinia publica printr-o prezentare subiectiva a faptelor. Postarea, avand o raza de cuprindere de 500.000-1.000.000 de persoane, conform datelor din Ad Library, si o cheltuiala de peste 100 RON, constituie propaganda electorala, incalcand prevederile legale.  Contactul SC INDISCRET MEDIA SRL este indiscretinoltenia@gmail.com si +40251522011.
    \item publicarea pe Facebook, dupa ora 18:00 pe 30.11.2024, a unei postari platite (ID \href{https://www.facebook.com/ads/library/?id=970553078443769}{970553078443769}) cu continut ce prezinta rezultatele electorale din judetul Dolj intr-un mod subiectiv si tendentios, utilizand expresii precum cat de eroic a ramas judetul dupa scrutinul de duminica, cu scopul de a influenta opinia publica si comportamentul electoral al cetatenilor.  Postarea vizeaza mai multi candidati, mentionand explicit partidele PSD, AUR, USR, PNL si candidatul independent Calin Georgescu, fara a-i denumi direct, dar contextul fiind evident.  Efectul electoral al postarii este de a crea o perceptie favorabila sau nefavorabila asupra rezultatelor electorale, influentand astfel voturile viitoare.  Postarea a generat intre 9.000 si 10.000 de impresii, avand o estimare a acoperirii de 500.000-1.000.000 de persoane, conform datelor din Ad Library.  Cheltuielile pentru aceasta postare au fost sub 100 RON.  Datele de contact ale advertiserului sunt: indiscretinoltenia@gmail.com, +40251522011, strada Mihail Kogalniceanu, nr.20, Craiova, Dolj 200390, RO.
\end{enumerate}

\vspace{0.5cm}

\subsection{SOS}
Următoarele fapte contravenționale sunt sesizate împotriva acestei entități:

\begin{enumerate}[leftmargin=*, label=\arabic*.)]
    \item publicarea, dupa ora 18:00 pe 30.11.2024, a unei postari pe Facebook (ID: \href{https://www.facebook.com/ads/library/?id=1114646240320099}{1114646240320099}) cu caracter de propaganda electorala, care promoveaza explicit candidatul George Grasu pentru Camera Deputatilor din partea Partidului SOS Romania, Filiala Dambovita, indicandu-i pozitia pe buletinul de vot (pozitia 8).  Postarea, platita cu o suma sub 100 RON, a generat intre 6.000 si 7.000 de impresii si a vizat o audienta estimata intre 100.000 si 500.000 de persoane, in principal in judetul Dambovita.  Prezenta numarului CMF 11240027 confirma natura sa de material electoral. Obiectivul electoral este evident, prin indemnul direct la vot pentru candidatul mentionat.  Aceasta actiune constituie o incalcare clara a legii, deoarece postarea indeplineste toate criteriile propagandei electorale dupa incheierea perioadei legale de campanie, conform articolului 98 t).  Datele de contact ale advertiserului sunt dambovita@sosro.ro si +40723957245.
    \item publicarea unei postari pe Facebook (ID: \href{https://www.facebook.com/ads/library/?id=1199837111083382}{1199837111083382}) dupa ora 18:00 pe 30.11.2024, care promoveaza candidatura lui Manolache Vasile la Senat in judetul Bistrita-Nasaud. Postarea, cu un buget de 200-299 RON, a generat 40.000-45.000 de impresii si are o estimare a acoperirii de 100.000-500.000 de persoane, utilizand limbaj emotional pozitiv si evidentiind platforma candidatului, creand astfel o asociere pozitiva cu acesta si influentand implicit alegerile electorale. Prezenta numarului CMF: 11240027 confirma natura de propaganda electorala.  Contact: contact@momentum.com.ro, +40767277777, Bucharest, Romania.
    \item publicarea, dupa ora 18:00 pe 30.11.2024, a unei postari platite pe Facebook (ID: \href{https://www.facebook.com/ads/library/?id=1781215229299135}{1781215229299135}), care promoveaza indirect candidatul MANOLACHE VASILE, senator din partea partidului SOS, prin asocierea imaginii si numelui sau cu o initiativa de sustinere a afacerilor locale.  Postarea, cu o cheltuiala de sub 100 RON, a atins intre 20.000 si 25.000 de impresii, avand o estimare a acoperirii de 100.000 - 500.000 de persoane.  Efectul electoral este evident, prin crearea unei imagini pozitive a candidatului asociata cu o tema relevanta pentru electorat.  Contactul Momentum Motion Pictures SRL este contact@momentum.com.ro si +40767277777.
    \item publicarea, dupa ora 18:00 pe 30.11.2024, a unei postari electorale pe Facebook (ID: \href{https://www.facebook.com/ads/library/?id=569215805798224}{569215805798224}), care promoveaza candidatura lui Manolache Vasile (Partidul S.O.S. Romania) la Senat in judetul Bistrita-Nasaud, folosind mesaje cu scop electoral ("Sprijinim afacerile locale pentru o economie puternica!") si un numar CMF (11240027),  avand un efect electoral pozitiv asupra candidatului. Postarea, cu o cheltuiala de 300-399 RON, a atins 80.000-90.000 de impresii, conform Ad Library, si a vizat o audienta de 100.000-500.000 de persoane, demonstrand intentia clara de a influenta votul.  Contact: contact@momentum.com.ro, +40767277777.
\end{enumerate}

\vspace{0.5cm}

\subsection{Sebastian - Tiberiu Pop}
Următoarele fapte contravenționale sunt sesizate împotriva acestei entități:

\begin{enumerate}[leftmargin=*, label=\arabic*.)]
    \item publicarea unei reclame platite pe Facebook (ID postare: \href{https://www.facebook.com/ads/library/?id=4137407953158994}{4137407953158994}), dupa ora 18:00 pe 30.11.2024, care promoveaza candidatul Liviu Lazar la Senat din partea PNCR, prin intermediul unui videoclip YouTube. Reclama, cu o estimare a audientei de 500.000 - 1.000.000 de persoane si o cheltuiala de 100-199 RON, indeplineste toate criteriile propagandei electorale, avand ca scop influentarea votului in favoarea candidatului mentionat. Titlul reclamei ("PODCAST ELECTORAL | Liviu Lazar - candidat PNCR pentru Senatul Romaniei") si continutul acesteia, care directioneaza utilizatorii catre un videoclip de campanie, constituie dovezi clare ale intentiei de a influenta procesul electoral.  Contact: contact@sebastiantiberiupop.ro, +40760256257.
\end{enumerate}

\vspace{0.5cm}

\subsection{Surse Interne}
Următoarele fapte contravenționale sunt sesizate împotriva acestei entități:

\begin{enumerate}[leftmargin=*, label=\arabic*.)]
    \item publicarea unei postari platite pe Facebook (ID: \href{https://www.facebook.com/ads/library/?id=1106803711113491}{1106803711113491}), dupa ora 18:00 pe 30.11.2024, care constituie propaganda electorala interzisa. Postarea, cu un buget de sub 100 RON, a atins 2000-3000 de impresii, vizand electoratul din judetul Constanta, promoveaza Partidul Social Democrat (PSD) intr-un mod evident partinitor, folosind imagini si text ("Bravo PSD Constanta!", "Noul PSD OPRIT!") pentru a influenta votul in favoarea PSD si impotriva PNL, AUR si USR, prezentate ca o "alianta de austeritate".  Aceasta postare, prin continutul sau si prin modul de prezentare, depaseste limitele jurnalismului obiectiv si constituie propaganda electorala, incalcand prevederile legale privind perioada de campanie electorala.  Contactul este lejuzeba@cyclelove.cc si +40720319090.
\end{enumerate}

\vspace{0.5cm}

\subsection{Tele 3 Media}
Următoarele fapte contravenționale sunt sesizate împotriva acestei entități:

\begin{enumerate}[leftmargin=*, label=\arabic*.)]
    \item publicarea unei postari platite pe Facebook (ID: \href{https://www.facebook.com/ads/library/?id=1117848819974618}{1117848819974618}) dupa ora 18:00 pe 30.11.2024, care promoveaza Partidul Forta Dreptei cu mesajul "Echipa Forta Dreptei aduce impreuna profesionisti dedicati pentru un viitor mai bun", avand un efect electoral pozitiv asupra partidului mentionat. Postarea, cu o cheltuiala de RON1.5K - RON2K, a vizat o audienta larga in judetul Gorj, Romania, conform datelor de targetare.  Lipsa obiectivitatii jurnalistice si intentia clara de a influenta votul, fac din aceasta postare material de propaganda electorala, contrar prevederilor legale.  Contact: office@tele3.ro, +40774658934, https://tele3.ro/.
\end{enumerate}

\vspace{0.5cm}

\subsection{Tomis News}
Următoarele fapte contravenționale sunt sesizate împotriva acestei entități:

\begin{enumerate}[leftmargin=*, label=\arabic*.)]
    \item publicarea, dupa ora 18:00 pe 30.11.2024, a unei postari platite pe Facebook (ID-ul postarii: \href{https://www.facebook.com/ads/library/?id=1678663123084033}{1678663123084033}), cu continut ce promoveaza indirect o perspectiva politica favorabila lui Daniel Constantin Moraru, presedintele UNPR - Regiunea Sud-Est.  Postarea, desi nu solicita explicit voturi pentru un anumit candidat, contine afirmatii precum aceasta halucinatie electorala si 2025 nu trebuie sa fie ceva de speriat. Va fi asa cum ni-l vom face noi!, care pot fi interpretate ca incercari de a influenta opinia publica intr-un mod care ar putea beneficia indirect anumite forte politice.  Utilizarea unei platforme publice platite, combinata cu citarea lui Moraru intr-un context care nu este strict jurnalistic, sugereaza o intentie de a influenta electoratul.  Cheia este contextul: postarea platita, cu o suma de sub 100 RON, a atins intre 3000 si 4000 de impresii, cu o estimare a acoperirii de 100.000 - 500.000 de persoane, conform datelor din Ad Library.  Contactul Tomis News este office@tomisnews.ro si +40731585221.
\end{enumerate}

\vspace{0.5cm}

\subsection{Tomis Tv}
Următoarele fapte contravenționale sunt sesizate împotriva acestei entități:

\begin{enumerate}[leftmargin=*, label=\arabic*.)]
    \item publicarea dupa ora 18:00 pe 30.11.2024 a unei postari pe Facebook (ID \href{https://www.facebook.com/ads/library/?id=1114621856207994}{1114621856207994}), care promoveaza candidatul Dumitru Caragheorghe la Senat din partea partidului DREPT, prezentand o discutie despre initiativele partidului intr-un mod evident pozitiv si cu un apel la actiune ("Merita sa urmariti!").  Postarea, cu o estimare a audientei de 100.000-500.000 de persoane, constituie propaganda electorala, intrunind toate criteriile articolului 36 (7) din LEGEA nr. 334 din 17 iulie 2006, avand ca efect electoral influentarea pozitiva a voturilor pentru candidatul mentionat.  Avand in vedere ca perioada de campanie electorala s-a incheiat, aceasta postare reprezinta o incalcare clara a articolului 98 t).  Suma cheltuita pentru aceasta reclama este sub 100 RON, conform datelor din Ad Library, iar datele de contact ale advertiserului sunt disponibile: h\_dorian@yahoo.com si +40752181038.
    \item publicarea, dupa ora 18:00 pe 30.11.2024, a unei postari pe Facebook (ID: \href{https://www.facebook.com/ads/library/?id=1618099118787045}{1618099118787045}) cu caracter de propaganda electorala, promovand candidatura lui Mohammad Murad pentru AUR la alegerile parlamentare. Postarea, platita cu o suma sub 100 RON, conform Ad Library, a atins o audienta de 2000-3000 de impresii, avand ca efect electoral promovarea candidatului si a partidului AUR.  Continutul postarii, care prezinta un interviu cu Mohammad Murad, discutand despre rezultatele alegerilor prezidentiale si propunerile AUR pentru alegerile parlamentare, depaseste limitele jurnalismului obiectiv si are ca scop influentarea votului.  Contactul Tomis Tv este h\_dorian@yahoo.com si +40752181038, adresa fiind strada Daciei 28, Topraisar, Constanta 900057, RO.
    \item publicarea, dupa ora 18:00 pe 30.11.2024, a unei postari pe Facebook (ID \href{https://www.facebook.com/ads/library/?id=864861295517511}{864861295517511}) cu caracter de propaganda electorala, care promoveaza candidatul PNL la Senat, Silviu Cosa, prezentandu-i planurile si analizand rezultatele recente ale partidului intr-un mod favorabil, cu scopul clar de a influenta votul alegatorilor.  Postarea, platita si cu o estimare a audientei de 100.000-500.000 de persoane, indeplineste toate criteriile articolului 36 (7) din LEGEA nr. 334 din 17 iulie 2006, referindu-se direct la un candidat, adresandu-se publicului larg si avand un obiectiv electoral evident.  Chiar daca nu solicita explicit votul pentru Cosa, prezentarea pozitiva si contextul electoral fac din aceasta postare o forma de propaganda interzisa dupa incheierea campaniei electorale.  Contactul Tomis Tv este h\_dorian@yahoo.com si +40752181038, iar cheltuielile de publicitate au fost sub 100 RON.
\end{enumerate}

\vspace{0.5cm}

\subsection{Total Impact}
Următoarele fapte contravenționale sunt sesizate împotriva acestei entități:

\begin{enumerate}[leftmargin=*, label=\arabic*.)]
    \item publicarea dupa ora 18:00 pe 30.11.2024 a unei postari pe Facebook (ID \href{https://www.facebook.com/ads/library/?id=1262305838438880}{1262305838438880}), cu un buget de sub 100 RON, care promoveaza candidata Karmen - Elena Mitrana,  avand ca efect electoral influentarea voturilor in favoarea acesteia. Postarea, desi prezentata sub forma unui discurs personal, are un scop evident electoral, folosind afirmatii precum Sunt aici pentru a rezolva aceasta problema si Sunt suficient de puternica sa intru in politica,  pentru a-si promova candidatura si a atrage voturi.  Aceasta postare, avand in vedere ca se adreseaza publicului larg prin intermediul unei reclame platite pe Facebook, depaseste limitele activitatii jurnalistice si constituie material de propaganda electorala dupa incheierea perioadei legale, conform articolului 36 (7) din LEGE nr. 334 din 17 iulie 2006.  Postarea a atins intre 5.000 si 6.000 de impresii, cu o estimare a acoperirii de 100.000 - 500.000 de persoane, in judetul Teleorman. Informatiile de contact ale advertiserului sunt: contact@totalimpact.ro si +40748955922.
    \item publicarea unei reclame platite pe Facebook (ID postare: \href{https://www.facebook.com/ads/library/?id=1923042014830768}{1923042014830768}), dupa ora 18:00 pe 30.11.2024, care promoveaza candidatura lui Constantin Fierascu pentru Camera Deputatilor din partea USR in Teleorman. Reclama, cu o estimare a audientei de 100.000 - 500.000 de persoane, utilizeaza imagini si text ("Teleorman in Parlament!") menite sa influenteze votul in favoarea candidatului, constituind astfel propaganda electorala dupa incheierea perioadei legale de campanie.  Textul reclamei mentioneaza explicit numele candidatului si afilierea sa politica, indeplinind criteriile articolului 36 (7) din LEGE nr. 334 din 17 iulie 2006.  Efectul electoral este clar, avand ca obiectiv cresterea numarului de voturi pentru Constantin Fierascu.  Datele de contact ale Total Impact sunt disponibile in arhiva reclamei: contact@totalimpact.ro si +40748955922. Suma cheltuita este sub 100 RON.
    \item publicarea unei reclame platite pe Facebook (ID postare: \href{https://www.facebook.com/ads/library/?id=3810306719208323}{3810306719208323}), dupa ora 18:00 pe 30.11.2024, cu un buget sub 100 RON, care promoveaza imaginea lui Costel Barbu, folosind un limbaj sugestiv de campanie electorala si evidentiind realizari si promisiuni specifice unei campanii electorale, avand ca efect electoral influentarea pozitiva a opiniei publice asupra candidatului, cu scopul de a-i creste numarul de voturi.  Reclama, adresata populatiei din judetul Teleorman, cu o estimare a audientei de 100.000 - 500.000 persoane,  contine fraze precum "Nu sunt doar promisiuni de campanie. Sunt fapte",  indicand clar o intentie electorala.  Datele de contact ale advertiserului sunt disponibile in Ad Library: contact@totalimpact.ro si +40748955922.
    \item publicarea unei postari platite pe Facebook (ID \href{https://www.facebook.com/ads/library/?id=492436650516086}{492436650516086}), dupa ora 18:00 pe 30.11.2024, cu un buget sub 100 RON, care, desi nu mentioneaza explicit candidati, analizeaza contextul politic actual si sugereaza indirect necesitatea schimbarii, influentand astfel optiunile electorale ale cetatenilor din judetul Teleorman.  Postarea, vizand o audienta larga (100K-500K),  foloseste un limbaj general, dar cu o tenta subiectiva, apropiindu-se mai mult de propaganda decat de jurnalism obiectiv.  Efectul electoral este indirect, dar semnificativ, prin crearea unei atmosfere de nemultumire fata de situatia politica existenta, incurajand implicit votul pentru o alternativa neprecizata.  Lipsa unui numar CMF nu exonereaza de raspundere, avand in vedere contextul si intentia clara de a influenta alegerile.
    \item publicarea unei postari pe Facebook (ID: \href{https://www.facebook.com/ads/library/?id=876541257633872}{876541257633872}) dupa ora 18:00 pe 30.11.2024, care promoveaza imaginea lui Costel Barbu, asociind numele sau cu un proiect de dezvoltare important in comuna Suhaia, finantat din fonduri guvernamentale.  Postarea, care a beneficiat de o campanie publicitara platita (peste 100 RON), cu o estimare a audientei de 100.000 - 500.000 de persoane,  contine un numar CMF (11240002), indicand clar natura sa de material electoral.  Desi nu solicita explicit votul pentru Costel Barbu, efectul electoral este evident, avand ca obiectiv influentarea pozitiva a opiniei publice asupra acestuia.  Imaginile atasate prezinta proiectul ca o realizare a lui Barbu, amplificand mesajul propagandistic.  Contact: contact@totalimpact.ro, +40748955922.
\end{enumerate}

\vspace{0.5cm}

\subsection{Trendmakers}
Următoarele fapte contravenționale sunt sesizate împotriva acestei entități:

\begin{enumerate}[leftmargin=*, label=\arabic*.)]
    \item publicarea unei reclame platite pe Facebook (ID postare: \href{https://www.facebook.com/ads/library/?id=907338324823558}{907338324823558}), dupa ora 18:00 pe 30.11.2024, care promoveaza candidatul Molnar Andras prin indemnuri directe la vot ("Vasarnap tamogassanak minket szavazatukkal", "Duminica sprijiniti-ne cu votul dumneavoastra"), avand un efect electoral clar prin influentarea votantilor in favoarea acestuia. Reclama, cu o cheltuiala de RON 400-499,  se adreseaza publicului larg, utilizand atat limba romana, cat si limba maghiara, si nu respecta standardele jurnalismului obiectiv, constituind astfel material de propaganda electorala interzisa dupa incheierea perioadei legale de campanie.  Contact: office@trendmakers.ro, +40744612046.
\end{enumerate}

\vspace{0.5cm}

\subsection{UDMR}
Următoarele fapte contravenționale sunt sesizate împotriva acestei entități:

\begin{enumerate}[leftmargin=*, label=\arabic*.)]
    \item publicarea unei reclame platite pe Facebook dupa ora 18:00 pe 30.11.2024, cu ID-ul \href{https://www.facebook.com/ads/library/?id=1090600686123484}{1090600686123484}, care promoveaza candidatul Demeter Andras Istvan la Senat pentru UDMR, folosind hashtag-uri precum "\#echipapentruilfovsenat", "\#GyozzonAJozanEsz", "\#AlegeVoceaRatiunii", si "\#UDMR", avand un efect electoral clar de influentare a votului in favoarea acestuia. Reclama, cu o estimare a acoperirii de peste 1 milion de persoane, depaseste limitele activitatii jurnalistice si are ca obiectiv influentarea votului, conform articolului 36 (7) din LEGEA nr. 334 din 17 iulie 2006.  Cheltuielile pentru reclama sunt sub 100 RON, conform datelor din Ad Library, iar datele de contact ale advertiserului sunt demeterandras@rmdsz.ro si +40745600964.
    \item publicarea unei reclame platite pe Facebook (ID postare: \href{https://www.facebook.com/ads/library/?id=1118190939907017}{1118190939907017}), dupa ora 18:00 pe 30.11.2024, care promoveaza implicit candidatii UDMR la alegerile parlamentare printr-un mesaj care indeamna la vot pentru reprezentanti pro-europeni, contrastand cu partidele extremiste.  Efectul electoral este clar, avand ca obiectiv influentarea votului in favoarea UDMR.  Mesajul, desi nu numeste candidati explicit,  foloseste logo-ul UDMR si indeamna la vot ("Pe 1 decembrie, sprijiniti-ne cu votul dumneavoastra"), indeplinind criteriile materialului de propaganda electorala conform articolului 36 (7) din LEGE nr. 334 din 17 iulie 2006.  Reclama a avut o cheltuiala de 400-499 RON si a atins 45.000-50.000 de impresii, conform datelor din Ad Library.  Contact: office@temesmegyeirmdsz.ro, +40752009342.
    \item publicarea postarii cu ID-ul \href{https://www.facebook.com/ads/library/?id=1335064637876078}{1335064637876078} pe Facebook dupa ora 18:00 pe 30.11.2024, care promoveaza candidatul Andras Istvan Demeter pentru alegerile parlamentare,  utilizand imagini si texte persuasive menite sa influenteze votul alegatorilor. Postarea, cu un buget de sub 100 RON, a atins peste 1 milion de utilizatori, contine codul CMF 3324005 si indica pozitia 4 a candidatului pe buletinul de vot.  Textul "Alege vocea ratiunii! Romania pe primul loc!" si prezentarea vizuala a candidatului sunt elemente clare de propaganda electorala, interzise dupa incheierea perioadei legale de campanie.  Informatiile de contact ale advertiserului sunt disponibile in Ad Library: demeterandras@rmdsz.ro si +40745600964.
    \item publicarea unei postari pe Facebook (ID: \href{https://www.facebook.com/ads/library/?id=1524370898242614}{1524370898242614}) dupa ora 18:00 pe 30.11.2024, care promoveaza implicit partidul RMDSZ si indeamna la vot in favoarea acestuia, avand un efect electoral clar. Postarea, cu un buget de 100-199 RON, a atins 4.000-5.000 de impresii si o estimare a acoperirii de 10.000-50.000 de persoane in judetul Timis.  Continutul postarii, lipsit de obiectivitate jurnalistica, prezinta un punct de vedere partizan, promovand beneficiile RMDSZ fara a prezenta perspective alternative.  Informatiile de contact ale advertiserului sunt office@temesmegyeirmdsz.ro si +40752009342.  Aceasta actiune constituie propaganda electorala ilegala, avand in vedere ca se desfasoara dupa incheierea perioadei legale de campanie.
    \item publicarea, dupa ora 18:00 pe 30.11.2024, a unei postari pe Facebook (ID: \href{https://www.facebook.com/ads/library/?id=1631706347773457}{1631706347773457}) cu scop electoral, care promoveaza candidatura lui PETRU FARAGO pentru Camera Deputatilor,  utilizand imaginea acestuia si sigla UDMR,  precum si codul AEP 13240004,  avand un efect electoral pozitiv asupra candidaturii sale. Postarea, platita cu o suma sub 100 RON, a generat intre 3000 si 4000 de impresii si a avut o estimare a acoperirii de 10.000 - 50.000 de persoane.  Informatiile de contact ale advertiserului sunt: bold.barbara98@gmail.com si +40759828198, cu adresa Episcopiei, 32, Arad, Arad 310084, RO si website-ul https://www.facebook.com/rmdszarad/.  Aceasta postare constituie propaganda electorala interzisa dupa incheierea campaniei electorale, conform legislatiei in vigoare.
    \item publicarea unei reclame platite pe Facebook (ID postare: \href{https://www.facebook.com/ads/library/?id=2968688483294798}{2968688483294798}), dupa ora 18:00 pe 30.11.2024, cu hashtag-uri precum \#UDMR, \#echipapentruilfovdeputati si \#AlegeVoceaRatiunii, care promoveaza direct candidatul UDMR la alegerile parlamentare din Ilfov, avand un efect electoral pozitiv pentru UDMR.  Reclama, cu o cheltuiala de sub 100 RON, a atins peste 1 milion de oameni, conform datelor din Ad Library, si utilizeaza un video pentru a-si transmite mesajul.  Contactul advertiserului este demeterandras@rmdsz.ro si +40745600964.  Aceasta reclama constituie propaganda electorala interzisa dupa incheierea perioadei de campanie, conform articolului 98 t), intrucat are ca scop influentarea votului in favoarea UDMR.
    \item publicarea, dupa ora 18:00 pe 30.11.2024, a unei postari pe Facebook (ID: \href{https://www.facebook.com/ads/library/?id=3942630265974085}{3942630265974085}), cu un buget de sub 100 RON, care contine propaganda electorala pentru UDMR,  folosind hashtag-uri precum "\#alegevocearatiunii" si "\#echipaudmrilfov" si un numar CMF (1324000),  avand ca efect electoral influentarea votului in favoarea UDMR.  Postarea, cu un mesaj direct ("Cetateanul sa nu mai fie sanctionat prin impozitari care sfideaza ratiunea. Statul sa serveasca, nu sa umileasca!"),  se adreseaza publicului larg, depasind limitele jurnalismului si avand ca obiectiv clar influentarea votului.  Postarea a atins o audienta estimata la peste 1 milion de persoane, conform datelor din Ad Library, cu informatii de contact disponibile (demeterandras@rmdsz.ro, +40745600964, https://rmdsz.ro/).
    \item publicarea, dupa ora 18:00 pe 30.11.2024, a unei postari pe Facebook (ID \href{https://www.facebook.com/ads/library/?id=865550199077924}{865550199077924}) cu continut electoral,  care promoveaza candidatul Demeter Andras Istvan si partidul UDMR, folosind hashtag-uri precum \#echipaudmrilfov si \#UDMR,  si mesaje subliminale ca Cetateanul nu este postas, curier sau copiator de acte,  avand ca efect electoral influentarea votului in favoarea candidatului mentionat. Postarea, care a avut o estimare a audientei de peste 1 milion de persoane si a costat intre 100 si 199 RON,  constituie propaganda electorala interzisa dupa incheierea perioadei legale de campanie, conform articolului 98 t).  Datele de contact ale advertiserului sunt demeterandras@rmdsz.ro si +40745600964.
    \item publicarea unei reclame platite pe Facebook (ID postare: \href{https://www.facebook.com/ads/library/?id=931203648367320}{931203648367320}), dupa ora 18:00 pe 30.11.2024, care promoveaza direct candidatii UDMR pentru alegerile parlamentare din Ilfov, folosind slogane precum "Alege vocea ratiunii! Romania pe primul loc!" si indicand pozitia lor pe buletinul de vot ("Pozitia 4 pe buletinul de vot").  Reclama, cu o estimare a acoperirii de peste 1 milion de persoane, depaseste limitele activitatii jurnalistice, avand ca obiectiv clar influentarea votului in favoarea UDMR.  Aceasta actiune constituie propaganda electorala interzisa dupa incheierea perioadei de campanie.  Reclama a fost platita cu o suma sub 100 RON, conform datelor din Ad Library, si a fost accesibila prin link-ul [https://www.facebook.com/ads/library/?active\_status=active\&ad\_type=political\_and\_issue\_ads\&country=RO\&is\_targeted\_country=false\&media\_type=all\&q=UDMR\&search\_type=keyword\_unordered].  Persoana de contact este Demeter Andras Istvan, demeterandras@rmdsz.ro, +40745600964.
\end{enumerate}

\vspace{0.5cm}

\subsection{USR}
Următoarele fapte contravenționale sunt sesizate împotriva acestei entități:

\begin{enumerate}[leftmargin=*, label=\arabic*.)]
    \item publicarea unei postari platite pe Facebook (ID \href{https://www.facebook.com/ads/library/?id=1014236510508683}{1014236510508683}) dupa ora 18:00 pe 30.11.2024, cu un buget de 100-199 RON, care contine un numar CMF (CMF 11240015),  si promoveaza indirect USR printr-un mesaj care, desi nu mentioneaza candidati anumiti,  foloseste limbaj emotional si referinte istorice pentru a crea o asociere pozitiva cu valorile partidului, avand ca efect electoral mobilizarea electoratului in favoarea USR.  Postarea, cu o estimare a audientei de 50.000-100.000 de persoane,  nu respecta standardele jurnalismului obiectiv, ci se prezinta ca o forma de propaganda electorala, avand ca scop influentarea votului in favoarea USR.  Contact: maramures@usrmm.ro, +40751926939.
    \item publicarea unei postari platite pe Facebook (ID \href{https://www.facebook.com/ads/library/?id=1043264764266897}{1043264764266897}) dupa ora 18:00 pe 30.11.2024, care promoveaza indirect candidatura lui Brian Cristian la alegerile parlamentare din 1 decembrie, folosind statistici pentru a-i evidentia realizarile si indemnand alegatorii sa voteze candidati care le vor reprezenta interesele in Parlament.  Postarea, cu un buget de RON 400-499 si o audienta de 6.000-7.000 de impresii, are un efect electoral clar, avand ca obiectiv influentarea votului in favoarea USR. Prezenta numarului CMF 11240015 confirma caracterul de propaganda electorala.  Fraze precum "vocea voastra in Parlament" si "sa avem o Romania pentru toti, nu doar pentru unii!" constituie indemnuri directe la vot, incalcand prevederile legale privind perioada de campanie electorala.
    \item publicarea unei postari pe Facebook (ID: \href{https://www.facebook.com/ads/library/?id=1113515196939145}{1113515196939145}) dupa ora 18:00 pe 30.11.2024, care contine propaganda electorala pentru Brian Cristian, candidat USR la alegerile parlamentare. Postarea, care a costat sub 100 RON,  foloseste un limbaj persuasiv ("Alegeti oameni care muncesc pentru voi"),  mentioneaza explicit candidatul si contine hashtag-uri promotionale (\#briancristian2024, \#briancristiandeputat),  avand ca efect electoral influentarea voturilor in favoarea lui Brian Cristian. Prezenta numarului CMF 11240015 confirma caracterul de material electoral.  Datele din Ad Library arata o estimare a impactului de 10.000-50.000 de persoane,  tinand cont de locatia vizata (judetul Maramures) si grupele de varsta (18-50 ani).  Contact: maramures@usrmm.ro, +40751926939.
    \item publicarea unei postari pe Facebook dupa ora 18:00 pe 30.11.2024, cu ID-ul \href{https://www.facebook.com/ads/library/?id=1279326216711414}{1279326216711414}, care promoveaza USR prin mesaje persuasive precum "Hai la vot!", "Votati echipa USR" si avertismente despre consecintele votarii altor partide, avand un efect electoral clar de influentare a voturilor in favoarea USR. Postarea, care a costat intre 1000 si 1500 RON, a avut o acoperire de 15.000-20.000 de impresii, conform datelor din Ad Library, si a vizat utilizatorii din judetul Maramures, cu varste intre 18 si 65 de ani.  Prezenta numarului CMF 11240015 confirma natura electorala a postarii, demonstrand intentia de a influenta votul in favoarea USR, in clara incalcare a prevederilor legale privind perioada de campanie electorala.  Contact: maramures@usrmm.ro, +40751926939.
    \item publicarea unei postari platite pe Facebook (ID: \href{https://www.facebook.com/ads/library/?id=1326594138703691}{1326594138703691}) dupa ora 18:00 pe 30.11.2024, care promoveaza candidatura lui Brian Cristian si, indirect, pe cea a Elenei Lasconi,  prin utilizarea de hashtag-uri precum \#briancristian, \#briancristian2024, \#briancristiandeputat, \#elenalasconi si \#lasconipresedinte,  si prin acuzatii nefondate privind nereguli electorale, cu scopul de a influenta votul alegatorilor. Postarea, cu un buget de 100-199 RON, a atins o audienta estimata de 50.000-100.000 de persoane in judetul Maramures,  continand un apel la actiune ("s\textbackslash{}u0103 nu uita\textbackslash{}u021bi s\textbackslash{}u0103 \textbackslash{}u00eei taxa\textbackslash{}u021bi la vot!"),  constituie propaganda electorala interzisa dupa incheierea campaniei electorale.  Contact: maramures@usrmm.ro, +40751926939.
    \item publicarea unei postari pe Facebook (ID \href{https://www.facebook.com/ads/library/?id=1508451479867376}{1508451479867376}) dupa ora 18:00 pe 30.11.2024, care contine propaganda electorala,  mentionand explicit "\#ElenaLasconi" intr-un context negativ si promovandu-l pe Teodor Lazar intr-un mod pozitiv, cu scopul clar de a influenta votul alegatorilor. Postarea, platita intre 100 si 199 RON, a avut o estimare a vizualizarilor intre 15.000 si 20.000,  tinand cont de numarul CMF 11240015,  si a fost directionata catre populatia din judetul Bacau, cu varsta cuprinsa intre 18 si 65 de ani.  Textul postarii, care face referire la munca grea a mamei sale, contrastand-o cu "pensii speciale, sinecuri si cadouri de lux", este o tactica evidenta de manipulare a opiniei publice in favoarea lui Teodor Lazar.  Contact: teodor.lazar@gmail.com, +40740269432, https://www.facebook.com/teodor.lazar.usr/.
    \item publicarea, dupa ora 18:00 pe 30.11.2024, a unei postari pe Facebook (ID: \href{https://www.facebook.com/ads/library/?id=1665040587725701}{1665040587725701}) cu scopul de a promova candidatura lui Andrei Plujar la alegerile parlamentare, continand elemente de propaganda electorala, inclusiv numarul CMF 11240015,  adresandu-se publicului larg printr-o postare platita pe Facebook, cu o cheltuiala estimata intre 1000 si 1500 RON si o estimare a numarului de impresii intre 30.000 si 35.000, avand ca efect electoral influentarea votului in favoarea candidatului mentionat.  Postarea, desi initial difuzata in perioada campaniei electorale, continua sa fie activa dupa incheierea acesteia, constituind o incalcare clara a legislatiei.  Datele de contact ale USR Caras-Severin sunt disponibile pe pagina lor de Facebook si includ adresa de email caras-severin.judet@usr.ro si numarul de telefon +40759017677.
    \item publicarea, dupa ora 18:00 pe 30.11.2024, a unei postari pe Facebook (ID: \href{https://www.facebook.com/ads/library/?id=2234871730229952}{2234871730229952}) cu caracter de propaganda electorala,  continand un program politic detaliat si un numar CMF (CMF:11240015),  avand drept scop influentarea votului in favoarea USR,  prin promovarea indirecta a candidatului Lorand Toth. Postarea, cu o cheltuiala estimata intre 1000 si 1500 RON, a generat intre 100.000 si 125.000 de impresii,  tinand cont de datele din Ad Library,  si a vizat in principal judetul Bistrita-Nasaud.  Lipsa mentionarii explicite a unui candidat nu diminueaza caracterul de propaganda electorala, dat fiind contextul, continutul si obiectivele postarii.  Contact: secretariat@usrbn.ro, +40744579499.
    \item publicarea, dupa ora 18:00 pe 30.11.2024, a unei postari pe Facebook (ID: \href{https://www.facebook.com/ads/library/?id=2299342510466202}{2299342510466202}), cu un buget de RON 300-399, care promoveaza explicit candidatii USR pentru alegerile parlamentare din Bistrita-Nasaud, inclusiv prin mentionarea numelor lor (Lorand Toth, Rus Marinel, Orban Alexandra, Cicai Andreea, Rus Gheorghe, Sangeorzan Nadia, Bidian Paul, Sangeorzan Artene, Muresan Liana, Popescu Dumitru, Dumitriu-Muresan Dragos) si prezenta unui cod CMF (11240015), avand un efect electoral clar prin promovarea acestor candidati si influentarea votului, atingand 35K-40K impresii.  Postarea, desi pretinde a fi informativa, are ca scop principal influentarea votului in favoarea candidatilor USR, reprezentand astfel propaganda electorala interzisa dupa incheierea perioadei legale de campanie. Contact: stan\_calyn@yahoo.com, +40741641443, https://www.facebook.com/bistritanasaudusr/.
    \item publicarea unei postari platite pe Facebook (ID \href{https://www.facebook.com/ads/library/?id=2308499799504251}{2308499799504251}) dupa ora 18:00 pe 30.11.2024, care promoveaza candidatul Brian Cristian la alegerile parlamentare,  folosind un ton promotional si auto-laudativ, evidentiind realizarile sale si indemnand alegatorii sa se informeze corect inainte de a vota. Postarea contine un numar CMF (11240015), specific materialelor de campanie electorala, si are ca efect electoral cresterea numarului de voturi pentru Brian Cristian.  Textul postarii, "Cand multi parlamentari au mimat munca, eu am ajuns pe primul loc si la initiative legislative: 277! Am deja 10 legi promulgate...", este clar orientat spre influentarea votului.  Avand in vedere cheltuielile de sub 100 RON, conform datelor din Ad Library, si o estimare a audientei de 50.000-100.000 persoane, impactul acestei postari este semnificativ.  Contactul USR Maramures este maramures@usrmm.ro si +40751926939.
    \item publicarea unei postari platite pe Facebook (ID: \href{https://www.facebook.com/ads/library/?id=2898751980281073}{2898751980281073}), dupa ora 18:00 pe 30.11.2024, cu scopul de a influenta votul in alegerile parlamentare. Postarea, desi incepe cu o declaratie personala, culmineaza cu o indemnare directa la vot pentru candidatii considerati demni de incredere,  "oferind increderea celor care credeti de cuviinta ca merita sa va reprezinte in Parlament".  Acest apel direct la vot, combinat cu bugetul alocat publicitatii (RON 200-RON 299),  ajungand la 10.000-15.000 de impresii, constituie propaganda electorala interzisa dupa incheierea perioadei legale de campanie.  Contact: caras-severin.judet@usr.ro, +40759017677.
    \item publicarea unei postari pe Facebook (ID: \href{https://www.facebook.com/ads/library/?id=3869633213284210}{3869633213284210}) dupa ora 18:00 pe 30.11.2024, care constituie propaganda electorala, avand ca efect influentarea votului in favoarea Elenei Lasconi si a partidului USR. Postarea, cu un buget de RON200-RON299, a atins o audienta estimata la 100.000-500.000 de persoane, continand mesaje explicite de sustinere a Elenei Lasconi si USR, prezentand un contrast negativ fata de partidele oponente si indemnand la vot pentru USR. Prezenta numarului CMF 11240015 confirma caracterul de propaganda electorala.  Aceasta postare incalca prevederile legale privind incetarea campaniei electorale, avand in vedere ca este o forma de propaganda electorala dupa incheierea perioadei legale.  Contact: teodor.lazar@gmail.com, +40740269432, https://www.facebook.com/teodor.lazar.usr/.
    \item publicarea unei postari platite pe Facebook (ID postare: \href{https://www.facebook.com/ads/library/?id=3944305055814200}{3944305055814200}), dupa ora 18:00 pe 30.11.2024, cu mesajul "Voteaza USR! Voteaza echipa Elena Lasconi! Pozitia 6 pe buletinul de vot. \#ElenaLasconiPresedinte \#USRlaguvernare CMF 11240015",  avand ca efect electoral influentarea voturilor in favoarea candidatei Elena Lasconi si a partidului USR.  Postarea, care a beneficiat de o investitie financiara intre 100 si 199 RON, a atins o audienta estimata intre 50.000 si 100.000 de persoane, conform datelor din Ad Library.  Mesajul direct, indemnul explicit la vot si prezenta CMF-ului demonstreaza clar intentia de a face propaganda electorala dupa incheierea perioadei legale, constituind o incalcare grava a legislatiei electorale.  Datele de contact ale USR Maramures sunt disponibile la comunicare@usr.ro si +40743564061.
    \item publicarea unei postari platite pe Facebook (ID \href{https://www.facebook.com/ads/library/?id=4123748011229965}{4123748011229965}), dupa ora 18:00 pe 30.11.2024, care promoveaza candidatul Elena Lasconi la presedintie, indemnand la vot cu fraza "Voteaza USR! Voteaza echipa Elena Lasconi!",  avand un efect electoral clar pozitiv asupra acesteia. Postarea, care a avut o estimare a vizualizarilor intre 3000 si 4000,  si o estimare a audientei de 100.000 - 500.000 de persoane,  contine un numar CMF (11240015), indicand material de campanie electorala.  Cheltuielile pentru aceasta postare au fost intre 100 si 199 RON.  Aceste actiuni constituie propaganda electorala dupa incheierea perioadei legale, incalcand prevederile legale.  Informatiile de contact ale USR Maramures sunt comunicare@usr.ro si +40743564061.
    \item publicarea unei postari platite pe Facebook (ID \href{https://www.facebook.com/ads/library/?id=435012952985112}{435012952985112}) dupa ora 18:00 pe 30.11.2024, cu mesajul "Voteaza USR! Voteaza echipa Elena Lasconi! Pozitia 6 pe buletinul de vot.",  avand ca efect electoral influentarea voturilor in favoarea candidatei Elena Lasconi si a partidului USR.  Postarea, care contine si numarul CMF 11240015, indeplineste toate criteriile materialului de propaganda electorala conform art. 36 (7) din Legea nr. 334/2006, depasind limitele activitatii jurnalistice si avand ca scop clar influentarea votului.  Cheltuielile pentru aceasta postare se situeaza intre 100 si 199 RON, conform Ad Library, ajungand la 2000-3000 de impresii, cu o estimare a acoperirii de 10.000-50.000 de persoane.  Informatiile de contact ale USR sunt disponibile la comunicare@usr.ro si +40743564061.
    \item publicarea, dupa ora 18:00 pe 30.11.2024, a unei postari pe Facebook (ID \href{https://www.facebook.com/ads/library/?id=472301982091367}{472301982091367}) cu scopul de a promova candidatura lui Brian Cristian la alegerile parlamentare,  postare ce contine un apel explicit la vot ("Votati reprezentanti care muncesc pentru voi", "Candidez pentru un nou mandat", "Va indemn sa ma sustineti") si un numar CMF (CMF 11240015),  avand un efect electoral pozitiv asupra candidatului.  Postarea, desi include unele date factuale, este preponderent de natura promotionala, lipsita de obiectivitate jurnalistica si se incadreaza in definitia materialului de propaganda electorala conform art. 36 (7) din LEGE nr. 334 din 17 iulie 2006.  Avand in vedere ca postarea a fost difuzata dupa incheierea campaniei electorale, aceasta constituie o incalcare a prevederilor legale.  Postarea a avut o estimare a acoperirii de 50.000-100.000 de persoane, iar cheltuielile de publicitate au fost sub 100 RON, conform informatiilor din Ad Library.  Persoana de contact este maramures@usrmm.ro, +40751926939.
    \item publicarea unei postari pe Facebook (ID \href{https://www.facebook.com/ads/library/?id=544653405387551}{544653405387551}) dupa ora 18:00 pe 30.11.2024, care promoveaza candidata Elena Lasconi si Partidul USR pentru alegerile parlamentare, folosind un limbaj puternic si apeluri directe la vot. Postarea, cu un buget de 200-299 RON, a generat 8.000-9.000 de impresii si o estimare a acoperirii de 100.000-500.000 de persoane, avand ca efect electoral influentarea votului in favoarea candidatei mentionate. Prezenta numarului CMF 11240015 confirma caracterul de propaganda electorala.  Continutul postarii, care face referire la miza alegerilor, viitorul democratic al Romaniei si indeamna la vot pentru USR, constituie o incalcare clara a prevederilor legale.  Contact: teodor.lazar@gmail.com, +40740269432.
    \item publicarea unei reclame platite pe Facebook (ID postare: \href{https://www.facebook.com/ads/library/?id=786314463636029}{786314463636029}), dupa ora 18:00 pe 30.11.2024, care promoveaza candidatura lui Brian Cristian la alegerile parlamentare,  utilizand un limbaj persuasiv si apeluri directe la vot ("La alegerile parlamentare votati reprezentanti care muncesc pentru voi", "Candidez pentru un nou mandat de deputat", "Va indemn sa ma sustineti").  Reclama, care a generat intre 15.000 si 20.000 de impresii si a avut o estimare a acoperirii de 50.000 - 100.000 de persoane,  contine un numar CMF (CMF 11240015), indicand clar natura sa de propaganda electorala.  Cheltuielile pentru aceasta reclama s-au situat intre 500 si 599 RON.  Informatiile de contact ale USR Maramures sunt disponibile la maramures@usrmm.ro si +40751926939.  Efectul electoral urmarit este influentarea votului in favoarea lui Brian Cristian, prin prezentarea sa intr-o lumina favorabila si prin solicitarea directa a voturilor.
    \item publicarea unei postari platite pe Facebook (ID postare: \href{https://www.facebook.com/ads/library/?id=853336236997296}{853336236997296}), dupa ora 18:00 pe 30.11.2024, care promoveaza candidatura Elenei Lasconi la presedintie si indeamna la vot pentru aceasta si pentru USR, folosind sloganuri precum "Voteaza USR! Voteaza echipa Elena Lasconi!",  precum si numarul CMF 11240015,  avand un efect electoral clar pozitiv asupra candidatei mentionate. Postarea, avand in vedere cheltuielile declarate intre 100 si 199 RON, a atins o audienta estimata intre 100.000 si 500.000 de persoane, conform datelor din Ad Library,  si a fost publicata de catre USR Maramures, cu datele de contact comunicate ca fiind comunicare@usr.ro si +40743564061,  constituie propaganda electorala interzisa dupa incheierea perioadei legale de campanie.  Aceasta postare indeplineste toate criteriile articolului 36 (7) din LEGE nr. 334 din 17 iulie 2006, referindu-se direct la un candidat, fiind utilizata dupa incheierea campaniei electorale, avand un obiectiv electoral clar si adresandu-se publicului larg, depasind limitele activitatii jurnalistice.
    \item publicarea unei postari pe Facebook (ID \href{https://www.facebook.com/ads/library/?id=8547083375402612}{8547083375402612}) dupa ora 18:00 pe 30.11.2024, care constituie propaganda electorala interzisa. Postarea, care a beneficiat de o cheltuiala de 300-399 RON si a atins 10.000-15.000 de impresii, promoveaza explicit candidata Elena Lasconi, folosind fraza "In turul 2 votam Elena Lasconi!",  constituind un apel direct la vot.  Prezenta numarului CMF 11240015 indica material electoral oficial.  Desi postarea mentioneaza alegerile din Republica Moldova, mesajul principal este o indemnare clara la vot pentru Elena Lasconi in alegerile prezidentiale romanesti, avand un efect electoral pozitiv asupra acesteia.  Contactul advertiserului este teodor.lazar@gmail.com si +40740269432.
    \item publicarea unei reclame platite pe Facebook (ID postare: \href{https://www.facebook.com/ads/library/?id=875948338037913}{875948338037913}), dupa ora 18:00 pe 30.11.2024, cu mesajul "Voteaza USR! Voteaza echipa Elena Lasconi! Pozitia 6 pe buletinul de vot.",  avand ca efect electoral influentarea voturilor in favoarea candidatei Elena Lasconi si a partidului USR.  Reclama, care contine numarul CMF 11240015, se incadreaza in definitia materialului de propaganda electorala conform art. 36 (7) din Legea nr. 334/2006, depasind limitele jurnalismului si avand ca scop explicit influentarea votului.  Cheltuielile de publicitate se situeaza intre 100 si 199 RON, cu o estimare a audientei de 100.000 - 500.000 persoane, conform datelor din Ad Library.  Datele de contact ale USR Maramures sunt disponibile la comunicare@usr.ro si +40743564061.
    \item publicarea unei postari platite pe Facebook (ID \href{https://www.facebook.com/ads/library/?id=889472632963010}{889472632963010}) dupa ora 18:00 pe 30.11.2024, care promoveaza implicit candidatura lui Brian Cristian la alegerile parlamentare. Postarea, care contine numarul CMF 11240015, prezinta o imagine favorabila a activitatii sale parlamentare si indeamna alegatorii sa il voteze, folosind fraze precum Alegeti oameni care muncesc pentru voi.  Acest lucru constituie propaganda electorala interzisa dupa incheierea perioadei de campanie.  Avand in vedere cheltuielile estimate intre 500 si 599 RON si o estimare a audientei de 100.000 - 500.000 persoane, impactul electoral este semnificativ.  Contactul USR Maramures este maramures@usrmm.ro si +40751926939.
    \item publicarea unei postari pe Facebook (ID: \href{https://www.facebook.com/ads/library/?id=890958233105596}{890958233105596}) dupa ora 18:00 pe 30.11.2024, care constituie propaganda electorala, avand in vedere ca promoveaza implicit candidatul Brian Cristian prin evidentierea activitatii sale parlamentare si afilierea la USR, cu scopul clar de a influenta votul alegatorilor. Postarea, platita cu suma de 800-899 RON, a atins o audienta estimata la 50.000-100.000 persoane, conform datelor din Ad Library.  Prezenta CMF-ului 11240015 confirma caracterul de propaganda electorala.  Continutul nu este jurnalistic, ci promotional, lipsit de obiectivitate.  Aceasta actiune contravine prevederilor legale privind incetarea campaniei electorale si interzicerea propagarii materialelor electorale dupa incheierea acesteia.  Datele de contact ale advertiserului sunt maramures@usrmm.ro si +40751926939.
\end{enumerate}

\vspace{0.5cm}

\subsection{USR Maramures}
Următoarele fapte contravenționale sunt sesizate împotriva acestei entități:

\begin{enumerate}[leftmargin=*, label=\arabic*.)]
    \item publicarea dupa ora 18:00 pe 30.11.2024 a unei postari pe Facebook (ID \href{https://www.facebook.com/ads/library/?id=1244156820201582}{1244156820201582}), cu continut electoral care promoveaza candidatul Brian Cristian, evidentiind realizarile sale legislative (277 initiative, 10 legi promulgate) si indemnand alegatorii sa se informeze corect inainte de a-si alege parlamentarii.  Postarea, care a beneficiat de o investitie financiara de sub 100 RON, a avut o estimare a audientei de 100.000 - 500.000 de persoane,  si contine numarul CMF 11240015, fapt ce confirma caracterul sau de material electoral.  Efectul electoral al postarii este clar pozitiv pentru Brian Cristian, avand ca obiectiv influentarea votului in favoarea sa.  Informatiile de contact ale USR Maramures sunt disponibile la maramures@usrmm.ro si +40751926939.
\end{enumerate}

\vspace{0.5cm}

\subsection{Uniunea Elena din Romania}
Următoarele fapte contravenționale sunt sesizate împotriva acestei entități:

\begin{enumerate}[leftmargin=*, label=\arabic*.)]
    \item publicarea unei reclame platite pe Facebook dupa ora 18:00 pe 30.11.2024, cu ID-ul postarii \href{https://www.facebook.com/ads/library/?id=383031788155206}{383031788155206}, care promoveaza candidatul Dragos-Gabriel Zisopol pentru alegerile parlamentare, avand ca efect electoral influentarea votului in favoarea acestuia. Postarea, desi prezentata sub forma unui ghid, contine un CMF (13240044), indicand clar natura sa de propaganda electorala.  Textul si imaginea postarii fac referire explicita la candidatul Zisopol si la Uniunea Elena din Romania, avand ca obiectiv cresterea numarului de voturi pentru acesta.  Reclama a avut o estimare a impactului de peste 1 milion de oameni, conform datelor din Ad Library, cu o cheltuiala de sub 100 RON.  Contactul advertiserului este electorale@uniunea-elena.ro si +40760673009.
\end{enumerate}

\vspace{0.5cm}

\subsection{Uniunea Elenă din România}
Următoarele fapte contravenționale sunt sesizate împotriva acestei entități:

\begin{enumerate}[leftmargin=*, label=\arabic*.)]
    \item publicarea unei reclame electorale platite pe Facebook dupa ora 18:00 pe 30.11.2024 (ID postare: \href{https://www.facebook.com/ads/library/?id=919000556854153}{919000556854153}), care promoveaza candidatul Dragos-Gabriel Zisopol pentru Camera Deputatilor, cu un efect electoral clar pozitiv, prin indemnul explicit la vot ("va invit sa votati candidatul Uniunii Elene din Romania") si prezentarea acestuia ca o solutie de incredere si stabilitate.  Reclama, care a avut o estimare a acoperirii de 100.000 - 500.000 de persoane,  contine un numar CMF (13240044), indicand o cheltuiala electorala, si se adreseaza publicului larg din Sibiu, cu o cheltuiala sub 100 RON, conform datelor din Ad Library.  Contact: electorale@uniunea-elena.ro, +40760673009.  Aceasta actiune constituie propaganda electorala interzisa dupa incheierea perioadei legale de campanie.
\end{enumerate}

\vspace{0.5cm}

\subsection{VESTEA MEDIA}
Următoarele fapte contravenționale sunt sesizate împotriva acestei entități:

\begin{enumerate}[leftmargin=*, label=\arabic*.)]
    \item publicarea pe Facebook, dupa ora 18:00 pe 30.11.2024, a unei postari (ID \href{https://www.facebook.com/ads/library/?id=1221511728912592}{1221511728912592}) cu continut defaimator la adresa Ancai Dragu, prin asocierea acesteia cu activitati infractionale ale lui Gabriel Voiculescu, avand ca efect electoral o influentare negativa a opiniei publice asupra acesteia. Postarea, cu o cheltuiala de sub 100 RON, a atins o audienta de 1000-2000 de impresii, vizand judetul Neamt, si contine afirmatii precum Doamna cu pardesiu rosu si cu sot condamnat pentru complicitate la coruptie cu prejudiciu de 60 milioane euro. Sotul, infractor pentru Voiculescu, ea, salariata de lux a mogulului...,  lipsita de obiectivitate si echilibru jurnalistic, constituind astfel propaganda electorala negativa, in ciuda lipsei unui apel explicit la vot.  Contact: office@vestea.net, +40756539876.
\end{enumerate}

\vspace{0.5cm}

\subsection{Vincze Loránt}
Următoarele fapte contravenționale sunt sesizate împotriva acestei entități:

\begin{enumerate}[leftmargin=*, label=\arabic*.)]
    \item publicarea unei reclame platite pe Facebook (ID postare: \href{https://www.facebook.com/ads/library/?id=1084383169604194}{1084383169604194}), dupa ora 18:00 pe 30.11.2024, care promoveaza explicit partidul RMDSZ, indemnand publicul la vot in favoarea acestuia cu fraza "Vasarnap vegyunk reszt a valasztason es tamogassuk az RMDSZ-t!", avand un efect electoral clar de influentare a votului in favoarea RMDSZ. Reclama, cu o cheltuiala estimata intre 100 si 199 RON, a atins o audienta de peste 1 milion de persoane, conform datelor din Ad Library, si utilizeaza un video pentru a-si transmite mesajul.  Contactul advertiserului este office@vinczelorant.eu si +40720616363.  Aceasta actiune constituie propaganda electorala interzisa dupa incheierea perioadei legale de campanie.
    \item publicarea unei postari platite pe Facebook (ID: \href{https://www.facebook.com/ads/library/?id=557108376952221}{557108376952221}) dupa ora 18:00 pe 30.11.2024, care promoveaza direct RMDSZ, un partid politic participant la alegerile parlamentare. Postarea, cu o cheltuiala de 100-199 RON, a atins peste 8000-9000 de impresii si peste 1 milion de persoane, utilizand un limbaj persuasiv pentru a influenta votul in favoarea RMDSZ.  Textul postarii, care subliniaza nevoia de reprezentare parlamentara puternica pentru comunitatea maghiara din Romania si intentia RMDSZ de a participa la coalitia de guvernare pentru a utiliza resursele in dezvoltarea comunitatii, constituie propaganda electorala clara, avand ca efect electoral cresterea numarului de voturi pentru RMDSZ.  Lipsa unui numar CMF accentueaza lipsa de transparenta.  Contact: office@vinczelorant.eu, +40720616363.
\end{enumerate}

\vspace{0.5cm}

\subsection{Vocea - Botosani}
Următoarele fapte contravenționale sunt sesizate împotriva acestei entități:

\begin{enumerate}[leftmargin=*, label=\arabic*.)]
    \item publicarea unei postari pe Facebook (ID: \href{https://www.facebook.com/ads/library/?id=1000778341856558}{1000778341856558}) dupa ora 18:00 pe 30.11.2024, cu continut electoral care promoveaza schimbarea fata de PSD printr-un mesaj direct si persuasiv ("PSD te-a vrut tacut... A venit vremea schimbarii! NOI suntem VOCEA! NOI decidem pe 1 decembrie!"), avand ca efect electoral influentarea negativa a voturilor pentru PSD. Postarea, cu o cheltuiala de sub 100 RON, a atins o audienta estimata intre 100.000 si 500.000 de persoane, conform datelor din Ad Library, si a vizat mai multe localitati din judetele Botosani, Iasi si Suceava.  Contactul este office@voceabotosani.ro si +40759112233.  Aceasta postare, desi publicata de un site de stiri, depaseste limitele jurnalismului obiectiv, constituind propaganda electorala interzisa dupa incheierea campaniei electorale.
    \item publicarea unei postari pe Facebook (ID: \href{https://www.facebook.com/ads/library/?id=2178796235854689}{2178796235854689}), dupa ora 18:00 pe 30.11.2024, cu un buget de sub 100 RON, care prezinta o imagine extrem de pozitiva a primarului Ciprian Ivanov din comuna Copalau, fara a mentiona vreun candidat parlamentar, dar avand un efect electoral indirect prin influentarea perceptiei electoratului asupra sa si, implicit, asupra partidului sau.  Postarea, vizibila unui public larg din judetul Botosani,  foloseste o strategie de marketing pozitiva, lipsita de obiectivitate jurnalistica, pentru a promova indirect un partid politic.  Aceasta actiune, desi nu solicita explicit votul pentru un candidat parlamentar, constituie propaganda electorala ascunsa, incalcand prevederile legale.  Contact: office@voceabotosani.ro, +407591112233.
\end{enumerate}

\vspace{0.5cm}

\subsection{Vocea - Botoșani}
Următoarele fapte contravenționale sunt sesizate împotriva acestei entități:

\begin{enumerate}[leftmargin=*, label=\arabic*.)]
    \item publicarea unei postari platite pe Facebook (ID \href{https://www.facebook.com/ads/library/?id=1118428286643213}{1118428286643213}), dupa ora 18:00 pe 30.11.2024, care prezinta o comparatie tendentioasa intre performanta primarului Maria Hutu (PSD) si primarul PNL Domunco Dumitru-Ovidiu, in contextul alegerilor din iunie 2024, cu scopul evident de a influenta votul alegatorilor.  Intrebarea sugestiva Cat de increzatori sunt locuitorii \#Comuna\_Varful\_Campului ca au votat ce trebuie la alegerile din iunie 2024?  si vizarea unor multiple localitati din judetul Botosani, conform datelor din Ad Library, demonstreaza clar intentia de a influenta votul.  Postarea, cu o estimare a bugetului sub RON 100 si o potentiala audienta de 100.000-500.000 persoane, constituie propaganda electorala interzisa dupa incheierea perioadei de campanie.  Contactul Vocea - Botosani este office@voceabotosani.ro si +40759112233.
\end{enumerate}

\vspace{0.5cm}

\subsection{Vocea Botosani}
Următoarele fapte contravenționale sunt sesizate împotriva acestei entități:

\begin{enumerate}[leftmargin=*, label=\arabic*.)]
    \item publicarea unei reclame platite pe Facebook, dupa ora 18:00 pe 30.11.2024, cu ID-ul \href{https://www.facebook.com/ads/library/?id=1106326714453008}{1106326714453008}, care, desi nu mentioneaza explicit un candidat, solicita opinia publica cu privire la echipa lui Valeriu Iftime si aspiratiile lor parlamentare, avand un efect electoral clar prin influentarea perceptiei alegatorilor si potential influentand votul.  Reclama, vizand zone geografice specifice din judetul Botosani si avand un buget de sub 100 RON, conform Ad Library, a atins o audienta estimata intre 7.000 si 8.000 de impresii, cu o estimare a acoperirii intre 100.000 si 500.000 de persoane.  Intrebarile retorice din postare, cum ar fi Ce mesaj transmit botosanenii \#echipeiValeriuIftime pentru Parlament? si Pe cine vor botosanenii in Parlamentul Romaniei pe candidatii echipei \#ValeriuIftime pentru a-i reprezenta?, sunt concepute pentru a influenta votul, constituind astfel propaganda electorala interzisa dupa incheierea perioadei de campanie.  Datele de contact ale advertiserului sunt office@voceabotosani.ro si +40759112233.
    \item publicarea unei postari platite pe Facebook (ID: \href{https://www.facebook.com/ads/library/?id=444817985328121}{444817985328121}), dupa ora 18:00 pe 30.11.2024, care promoveaza candidatura lui Valeriu Iftime la alegerile parlamentare. Postarea, cu o cheltuiala de 400-499 RON,  adresata publicului larg din judetul Botosani si zonele limitrofe, contine intrebarea directa: Au incredere botosanenii in echipa Valeriu Iftime pentru Parlamentul Romaniei?, avand ca efect electoral clar influentarea votului in favoarea candidatului mentionat.  Lipsa unui numar CMF nu diminueaza caracterul de propaganda electorala, avand in vedere contextul clar al alegerilor si intentia evidenta de a influenta votul.  Postarea, vizualizata de 70.000-80.000 de persoane,  constituie o incalcare a legii, fiind propaganda electorala dupa incheierea perioadei legale de campanie.  Datele de contact ale advertiserului sunt office@voceabotosani.ro si +40759112233.
    \item publicarea unei postari platite pe Facebook (ID \href{https://www.facebook.com/ads/library/?id=485935913915547}{485935913915547}), dupa ora 18:00 pe 30.11.2024, care promoveaza, intr-un mod evident, entitatea politica sau candidatul asociat cu Valeriu Iftime, utilizand limbaj pozitiv si hashtag-uri (\#echipa Valeriu Iftime) pentru a influenta electoratul din Botosani.  Postarea, cu un buget de sub RON 100,  se adreseaza unui public larg, avand ca obiectiv clar influentarea votului in favoarea lui Valeriu Iftime.  Lipsa obiectivitatii jurnalistice si intentia evidenta de promovare constituie propaganda electorala, conform articolului 36 (7) din LEGE nr. 334 din 17 iulie 2006.  Textul "Ei spun ca \#echipa Valeriu Iftime este o solutie sanatoasa pentru viitorul judetului nostru" este o afirmatie clara de sustinere, lipsita de neutralitate.  Datele de contact ale Vocea Botosani sunt disponibile pe Facebook (office@voceabotosani.ro, +40759112233, https://voceabotosani.ro/).
\end{enumerate}

\vspace{0.5cm}

\subsection{Vocea Dorohoi}
Următoarele fapte contravenționale sunt sesizate împotriva acestei entități:

\begin{enumerate}[leftmargin=*, label=\arabic*.)]
    \item publicarea unei postari pe Facebook (ID: \href{https://www.facebook.com/ads/library/?id=1824225495050389}{1824225495050389}) dupa ora 18:00 pe 30.11.2024, cu un buget sub 100 RON, care contine propaganda electorala negativa indreptata impotriva Partidului Social Democrat (PSD). Postarea, vizibila unui public larg in judetul Botosani si zonele limitrofe, utilizeaza un limbaj puternic si o chemare explicita la actiune ("NOI decidem pe 1 decembrie!"), avand ca efect electoral influentarea negativa a voturilor pentru PSD.  Continutul ("PSD te-a vrut tacut... A venit vremea schimbarii!") este clar orientat politic si nu se incadreaza in jurnalismul obiectiv.  Datele din Ad Library indica o estimare a impactului de 3000-4000 de impresii si o raza de actiune de 100.000-500.000 de persoane, amplificand efectul propagandistic. Contact: office@voceadorohoi.ro, +40759112233.
    \item publicarea unei postari platite pe Facebook dupa ora 18:00 pe 30.11.2024, cu ID-ul \href{https://www.facebook.com/ads/library/?id=949234727252215}{949234727252215}, care promoveaza indirect echipa lui Valeriu Iftime, avand un efect electoral pozitiv asupra acestuia. Postarea, desi nu mentioneaza explicit un partid, indeplineste toate criteriile materialului de propaganda electorala conform articolului 36 (7) din LEGE nr. 334 din 17 iulie 2006, prin referirea directa la o echipa politica, adresarea catre o audienta larga din judetul Botosani si obiectivul clar de a influenta votul in favoarea acesteia.  Textul "o solutie sanatoasa pentru viitorul judetului nostru" evidentiaza intentia de a influenta electoratul.  Chiar daca nu exista un numar CMF, caracterul promotional al postarii, vizibilitatea sa extinsa (3K-4K impresii, 100K-500K reach estimat) si cheltuielile financiare (sub 100 RON) confirma intentia de a influenta alegerile.  Contactul poate fi stabilit prin office@voceadorohoi.ro sau +40759112233.
\end{enumerate}

\vspace{0.5cm}

\subsection{Ziarul Tricolorul}
Următoarele fapte contravenționale sunt sesizate împotriva acestei entități:

\begin{enumerate}[leftmargin=*, label=\arabic*.)]
    \item publicarea unei postari pe Facebook (ID: \href{https://www.facebook.com/ads/library/?id=912542110533708}{912542110533708}), dupa ora 18:00 pe 30.11.2024, cu continut electoral negativ ce vizeaza candidata Dumitrita Gliga si Partidul Social Democrat (PSD). Postarea, prezentata sub forma unui articol de stiri, contine acuzatii grave si lipseste de obiectivitate, avand drept scop influentarea negativa a electoratului prin asocierea candidatei cu actiuni reprobabile ale tatalui sau, fost lider PSD.  Textul, difuzat printr-o campanie platita pe Facebook cu o estimare a audientei de peste 1 milion de persoane si un buget de sub 100 RON,  foloseste un limbaj acuzator si lipsit de neutralitate jurnalistica, depasind limitele activitatii jurnalistice de informare a publicului.  Efectul electoral este clar negativ pentru candidata Dumitrita Gliga si PSD, avand ca obiectiv influentarea votului.  Informatiile prezentate, fara dovezi concrete si fara perspective contrare, constituie o forma de propaganda electorala interzisa dupa incheierea perioadei de campanie.  Contact: asociatiakogaionon@gmail.com, +40744343554, adresa: strada migdalilor nr 61, Pantelimon, Ilfov 077145, RO.
\end{enumerate}

\vspace{0.5cm}

\subsection{info-sud-est.ro}
Următoarele fapte contravenționale sunt sesizate împotriva acestei entități:

\begin{enumerate}[leftmargin=*, label=\arabic*.)]
    \item publicarea unei postari pe Facebook (ID \href{https://www.facebook.com/ads/library/?id=587821263743282}{587821263743282}), dupa ora 18:00 pe 30.11.2024, cu continut ce constituie propaganda electorala,  avand ca efect influentarea negativa a opiniei publice asupra unor candidati la alegerile parlamentare. Postarea, desi prezentata sub forma unui articol de stiri, utilizeaza un limbaj inflamator si selectiv, concentrandu-se pe aspecte negative ale comportamentului unor candidati, fara a oferi o perspectiva echilibrata. Titlul senzationalist si utilizarea unor expresii jignitoare ("Printeso, sugi p***!") demonstreaza intentia de a influenta votul.  Campania a avut o estimare a razei de 500.000 - 1.000.000 de persoane, cu cheltuieli sub 100 RON, conform datelor din Ad Library.  Contact: redactia@info-sud-est.ro, +40740134925.
\end{enumerate}

\vspace{0.5cm}

\section{Solicitări}

Față de cele de mai sus, solicit:

\begin{enumerate}[leftmargin=*, label=\arabic*.]
    \item Constatarea contravențiilor săvârșite;
    \item Identificarea persoanelor vinovate;
    \item Aplicarea sancțiunilor prevăzute de lege.
\end{enumerate}

\section{Anexe}

Anexez prezentei plângeri următoarele dovezi:

\begin{enumerate}[leftmargin=*, label=\arabic*.]
    \item Capturi de ecran ale postărilor care fac obiectul sesizării;
    \item Dovada calității de observator electoral.
\end{enumerate}

\vspace{1cm}
\noindent Data: \today

\vspace{1.5cm}
\noindent Observator electoral,\\[0.3cm]
Deleanu Ștefan-Lucian

\vspace{1cm}
\noindent Semnătura: [SEMNAT ELECTRONIC]

\end{document}
